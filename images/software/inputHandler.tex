\begin{tikzpicture}[%
        align=center,
        text=solarized-base02,
        draw=solarized-base02,
        remember picture,
        overlay,
    ]
    \small
    \ttfamily

    \begin{scope}[
        every node/.style = draw,
        terminal/.append style={
            rounded rectangle,
            fill=solarized-violet,
            text=solarized-base3,
            inner sep=2mm,
        }, % data packages
        sign/.style={
            inner sep=2mm,
            rounded corners=1mm,
            fill=solarized-magenta,
            text=solarized-base3,
        },         % custom signal style
        circ/.style={
            inner sep=2mm,
            rounded corners=1mm,
            double,
            fill=solarized-base02,
            draw=solarized-base02,
            text=solarized-base2,
        }, % circuitry
        proc/.style={
            inner sep=2mm,
            rounded corners=1mm,
            fill=solarized-cyan,
            text=solarized-base3,
        },       % process/activity
        stor/.style={
            fill=cyan!30
        },         % storage
        dbtable/.style={
            text=solarized-base3,
            draw=solarized-base3,
            rounded corners=1mm,
            inner sep=2mm,
        } % database tables
    ]
        %\node (stringcurrents-request) [
        %    sign,
        %    signal,
        %    signal to=east
        %] at (10mm,0mm) {stringcurrents\_request};

        \node (stringcurrents-request) [
            proc,
        ] at (20mm,-20mm) {stringcurrents\_request};

        %\node (string-support) [
        %    right=30mm of stringcurrents-request,
        %    draw=white,
        %] { };

        \node (stringnumber) [
            sign,
            signal,
            signal to=east,
            signal from=west,
            above right=20mm of stringcurrents-request,
        ] {stringnumber};

        \node (stringcurrentget) [
            sign,
            signal,
            signal to=west,
            signal from=east,
            right=20mm of stringcurrents-request,
        ] {stringcurrent};

        \node (read-stringcurrents-i2c) [
            proc,
            right=of stringcurrentget,
        ] {read\_stringcurrents\_i2c};

        \node (write-string-into-database) [
            proc,
            below=of read-stringcurrents-i2c,
        ] {write\_string\_into\_database};

        \node (stringcurrentput) [
            sign,
            signal,
            signal to=east,
            signal from=west,
            left=of write-string-into-database,
        ] {stringcurrent};

        %\node (string-compare) [
        %    terminal,
        %    below=20mm of stringcurrents-request,
        %] {string\_compare};

        \node (string-compare) [
            proc,
            below=20mm of stringcurrents-request,
        ] {string\_compare};

        \node (modulevoltage-request) [
            proc,
            below=20mm of string-compare,
        ] {modulevoltage\_request};

        \node (stringNo-serialNo) [
            sign,
            signal,
            signal to=east,
            signal from=west,
            right=20mm of modulevoltage-request,
            align=center,
        ] {stringnumber\\serialnumber};

        \node (serialNo-voltage) [
            sign,
            signal,
            signal to=west,
            signal from=east,
            below=of stringNo-serialNo,
            align=center,
        ] {serialnumber\\voltage};

        \node (read-modulevoltage-i2c) [
            proc,
            right=of stringNo-serialNo,
        ] {read\_modulevoltage\_i2c};

        \node (serialNo-stringNo-voltage) [
            sign,
            signal,
            signal to=south,
            signal from=north,
            below=60mm of modulevoltage-request,
            align=center,
        ] {serialnumber\\stringnumber\\voltage};

        \node (support1) [
            fill=white,
            draw=white,
            above=of serialNo-stringNo-voltage,
        ] { };

        \node (write-panel-into-database) [
            proc,
            below=of serialNo-stringNo-voltage,
        ] {write\_panel\_into\_database};

        \node (serialNo1) [
            sign,
            signal,
            signal to=east,
            signal from=west,
            right=of write-panel-into-database,
        ] {serialnumber};

        \node (write-string1-into-database) [
            proc,
            above right=of serialNo1,
        ] {write\_string1\_into\_database};

        \node (write-string2-into-database) [
            proc,
            below=of write-string1-into-database,
        ] {write\_string2\_into\_database};

        \node (write-string3-into-database) [
            proc,
            below=of write-string2-into-database,
        ] {write\_string3\_into\_database};
    \end{scope}

    \begin{scope}[on background layer]
        \node (getStringCurrents) [%
            draw,
            double,
            rounded corners=5mm,
            fill=solarized-base3,
            inner sep=1.75em,
            fit=(read-stringcurrents-i2c) (write-string-into-database) (stringnumber) (stringcurrentget) (stringcurrentput),
            text height=-0.5em,
            align=right,
        ] {\sffamily wird pro String $1\times$ ausgef\"uhrt, also total $3\times$};

        \node (getModuleVoltages) [%
            draw,
            double,
            rounded corners=5mm,
            fill=solarized-base3,
            inner sep=1.75em,
            fit=(read-modulevoltage-i2c) (stringNo-serialNo) (serialNo-voltage),
            text height=-0.5em,
            align=right,
        ] {\sffamily mit Schlaufe f\"ur jedes Modul ausgef\"uhrt};

        \node (checkSN) [%
            draw,
            double,
            rounded corners=5mm,
            fill=solarized-base3,
            inner sep=1.75em,
            fit=(serialNo-stringNo-voltage) (write-panel-into-database) (serialNo1) (write-string1-into-database) (write-string2-into-database) (write-string3-into-database) (support1),
            text height=-0.5em,
            align=right,
        ] {\sffamily SN wird \"uberpr\"uft und in das entsprechende Table\\eingetragen, falls noch kein Eintrag vorhanden ist.};

        \node (mainProcess) [%
            draw,
            double,
            rounded corners=5mm,
            fill=solarized-base2,
            inner sep=1.5em,
            fit=(stringcurrents-request) (string-compare) (modulevoltage-request),
            text height=-0.5em,
            align=right,
            text=solarized-base02,
        ] {\sffamily Hauptablauf des Files};
    \end{scope}


    \begin{scope}[
            rounded corners,
            every path/.append style={draw=solarized-base03,},
    ]
        \draw[-latex] (stringcurrents-request) |- (stringnumber);
        \draw[-latex] (stringnumber) -| (read-stringcurrents-i2c);
        \draw[-latex] (read-stringcurrents-i2c) -- (stringcurrentget);
        \draw[-latex] (stringcurrentget) -- (stringcurrents-request);
        \draw[-latex] (stringcurrents-request) |- (stringcurrentput);
        \draw[-latex] (stringcurrentput) -- (write-string-into-database);

        \draw[-latex] (stringcurrents-request) -- (string-compare);
        \draw[-latex] (string-compare) -- (modulevoltage-request);

        \draw[-latex] (modulevoltage-request) -- (stringNo-serialNo);
        \draw[-latex] (stringNo-serialNo) -- (read-modulevoltage-i2c);
        \draw[-latex] (read-modulevoltage-i2c) |- (serialNo-voltage);
        \draw[-latex] (serialNo-voltage) -| (modulevoltage-request);


        \draw[-latex] (modulevoltage-request) -- (serialNo-stringNo-voltage);
        \draw[-latex] (serialNo-stringNo-voltage) -- (write-panel-into-database);
        \draw[-latex] (write-panel-into-database) -- (serialNo1);
        \draw[dashed,-latex] (serialNo1) |- (write-string1-into-database);
        \draw[dashed,-latex] (serialNo1) -- (write-string2-into-database);
        \draw[dashed,-latex] (serialNo1) |- (write-string3-into-database);
    \end{scope}

\end{tikzpicture}
