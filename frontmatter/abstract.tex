% **************************************************************************** %
\chapter*{Abstract}
\label{chap:abstract}
% **************************************************************************** %
\enlargethispage{2em}

This project's  aim was to develop  a system for real-time  monitoring of each
panel  of a  photovoltaic facility.   The  system must  be cost-effective  and
should scale  from small single-household solutions  to large industrial-scale
solar  farms. Panels which  are not  operating at  full capacity  for whatever
reason (dirt, shade,  defects) must be detected and the  user informed so that
appropriate measures (cleaning, replacement) can be taken.

Monitoring  of  the  individual  photovoltaic modules  allows  the  facility's
operator to  optimally run  their solar  plant, thus  reducing losses  in both
power output and profit.

Current solutions for per-module monitoring  of solar facilities are expensive
and often left  out to save costs.   As alternative sources of  energy such as
wind and  solar power  grow in  importance, the overall  losses in  the energy
industry of power  and money incurred due to insufficient  monitoring of solar
facilities will become unsustainable.

Our  system  has  two  primary components: A  controller  which  is  installed
centrally  near  the  inverter  and  a  sensor  on  each  photovoltaic  panel.
Communication between the sensors and the  master is routed through the direct
current power  transmission line;  no additional wiring  is needed. A  coil is
used to couple the signal to the power line. In case of an error (e.g. a dirty
panel), an error message is sent to a user-configurable phone number.

Simulations for  various coupling methods  for a string  of 20 PV  panels have
been performed. Inductive  coupling at  non-resonance conditions results  in a
signal  level  of roughly  \SI{6}{\milli\volt}  peak-to-peak  at the  receiver
without  amplification. Operating  the  circuit  at resonance  yields  a  much
improved peak-to-peak  voltage of \SI{250}{\milli\volt} at  the receiver (also
without amplification), which is sufficient for our purposes.

A  frequency sweep  for  the coupling  coil has  been  measured and  inductive
behavior up to  \SI{20}{\mega\hertz} verified, thus ensuring that  the coil is
adaptable enough to allow the use of a vast range of frequencies as needed.

The system's components have been implemented. A  signal can be coupled to the
DC transmission line  and the sensor can perform  correct voltage measurements
and return that  data. Both the graphical user interface on  the master device
and its database back-end are functioning.

\vspace{2em}
\textbf{Key  words:}  photovoltaic  technology,  photovoltaic  module,  remote
monitoring, solar technology, PV cell, power efficiency, alternative energy,
powerline, communication, inductive coupling, capacitive coupling
