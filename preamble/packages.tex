% -------------------------------------------------------- %
% NOTE: There   are  two   kinds  of   packages  in   this %
% document. On ond hand, there  are those which are needed %
% for this  template to function at  as intended. It might %
% still  compile without  them,  but it  will probably  no %
% longer look as it should.                                %
%                                                          %
% On the other hand, there are those which I have found to %
% be useful over the years  and tend to commonly use.  You %
% may or may  not need those.  Feel free  to disable those %
% you do not need.                                         %
%                                                          %
% By default,  I recommend leaving the  following packages %
% enabled:                                                 %
% - fontenc: for output font encoding                      %
% - inputenc: for using  non-standard characters in input, %
%   such as Umlauts or other accented characters           %
% - graphicx:  needed for  the  chapter  style (based  on  %
%   veelo)                                                 %
%                                                          %
% Feel free to  disable and enable the  rest as needed. If %
% the document no longer compiles or breaks aesthetically, %
% you will notice soon enough...                           %
% -------------------------------------------------------- %


% -------------------------------------------------------- %
% General Packages                                         %
% -------------------------------------------------------- %
\usepackage[T1]{fontenc}     % output encoding
\usepackage[utf8]{inputenc}  % input encoding
\usepackage[ngerman]{babel}
\usepackage{lipsum}          % filler text
\usepackage{graphicx}
\usepackage{amsmath}         % for reasonable math typesetting
\usepackage{pdfpages}        % include pdf documents
%\usepackage{amsfonts}        % not sure yet if we need this
\usepackage{adjustbox}       % helps w/ minipage alignmant
\usepackage{pbox}            % boxes w/ line breaks in tables
\usepackage[textsize=footnotesize, textwidth = 37mm, german, colorinlistoftodos]{todonotes}
\usepackage{calc}            % used for calculating margins and widths for A3 pages
%\usepackage{caption}         % captions outside float environments, overrides memoir's caption facilities
\usepackage[separate-uncertainty=true]{siunitx}
\usepackage[light]{kpfonts}
\usepackage{counttexruns}
\usepackage[european,siunitx,cuteinductors]{circuitikz}
%\usepackage{datetime2}


% -------------------------------------------------------- %
% Draft Watermark                                          %
%                                                          %
% Prints  a watermark  across the  page, marking  it as  a %
% draft.                                                   %
% -------------------------------------------------------- %
\usepackage{draftwatermark}
\SetWatermarkText{Entwurf, \today}
\SetWatermarkScale{0.5}
%\usepackage[final]{draftwatermark} % removes watermark



% -------------------------------------------------------- %
% TIKZ and PGF                                             %
%                                                          %
% TODO: See if this  should be put in a  separate file, or %
% if  having  it  here makes  sense. Alternatively,  these %
% documents might not need to be set globally at all.      %
% -------------------------------------------------------- %
\usepackage{tikz}
\usetikzlibrary{arrows}
\usetikzlibrary{decorations.pathmorphing}
\usepackage{pgfplots}
%\usepgfplotslibrary{units}
\usetikzlibrary{pgfplots.units}
\pgfplotsset{compat=newest}
\pgfplotsset{max space between ticks=80pt}
\pgfplotsset{max space between ticks=80pt}
\pgfplotsset{try min ticks=5}
\pgfplotsset{
    tick label style={font=\small},
    label style={font=\small},
    legend style={font=\footnotesize}
}
\pgfplotsset{every axis plot/.append style={%
    line width=0.5pt}}
    %thick}}


% -------------------------------------------------------- %
% Conditionals                                             %
% -------------------------------------------------------- %
% http://tex.stackexchange.com/questions/5894/latex-conditional-expression
\usepackage{etoolbox}

% -------------------------------------------------------- %
% Packages which might be used under certain circumstances %
% -------------------------------------------------------- %
%\usepackage{geometry}
%\usepackage[english]{babel}
%\usepackage{kpfonts}

% -------------------------------------------------------- %
% Set link  colors and  all that good  stuff. See hyperref %
% manual for more info and options if you wish.            %
% -------------------------------------------------------- %
\newtoggle{paper}
%\toggletrue{paper} % we're printing on paper
\togglefalse{paper} % we're making an electronic version

\iftoggle{paper}{%
    % ---------------------------------------------------- %
    % If  we're  printing on  paper,  don't  do any  fancy %
    % coloring for links and such.                         %
    % ---------------------------------------------------- %
    \usepackage[%
        bookmarksnumbered=true,
        colorlinks=true,
        linkcolor=black,
        citecolor=black,
        urlcolor=black,
        %hidelinks=false,
    ]{hyperref}
}{%
    % ---------------------------------------------------- %
    % If  we're  creating  an electronic  version  of  our %
    % document, color links as follows.                    %
    % ---------------------------------------------------- %
    \usepackage[%
        bookmarksnumbered=true,
        colorlinks=true,
        linkcolor=blue,
        citecolor=blue,
        urlcolor=magenta,
        %hidelinks=false,
    ]{hyperref}
}


% -------------------------------------------------------- %
% xcolor  and kvoptions  are  loaded by  xcolor-solarized. %
% If   you  require   special   options   for  these   two %
% packages,  uncomment  these  two lines  and  pass  those %
% options. Otherwise  leave them  commented out  since the %
% packages are loaded anyway.                              %
% -------------------------------------------------------- %
%\usepackage{xcolor}
%\usepackage{kvoptions}
\usepackage[prefix=sol]{xcolor-solarized}
