% **************************************************************************** %
\chapter{L\"osungsans\"atze und Simulationen}
\label{chap:simu}
% **************************************************************************** %

In  diesem Abschnitt  werden Schaltungen  f\"ur drei  L\"osungsans\"atze (zwei
zur  FSK, eine  zur  ASK)  vorgestellt, mit  \code{LTspice  IV} simuliert  und
beurteilt.\todo{LTspice: source}

Es werden  jeweils die  Schaltung f\"ur  den Sender,  den Empf\"anger  und das
Gesamtsystem untersucht.


% ---------------------------------------------------------------------------- %
\section{Frequenzumtastung: Kapazitive Einkopplung}
\label{sec:simu:fsk:capacitive}
% ---------------------------------------------------------------------------- %

% ---------------------------------------------------------------------------- %
\subsection{Sender}
\label{sec:simu:fsk:capacitive:transmitter}
% ---------------------------------------------------------------------------- %

% ---------------------------------------------------------------------------- %
\subsection{Empf\"anger}
\label{sec:simu:fsk:capacitive:receiver}
% ---------------------------------------------------------------------------- %

% ---------------------------------------------------------------------------- %
\subsection{Gesamtsystem}
\label{sec:simu:fsk:capacitive:system}
% ---------------------------------------------------------------------------- %

% ---------------------------------------------------------------------------- %
\section{Frequenzumtastung: Induktive Einkopplung}
\label{sec:simu:fsk:inductive}
% ---------------------------------------------------------------------------- %

Eine induktive Einkopplung legt eine  Spule um die DC-Leitung. Auf diese Spule
wird von der FSK-Schaltung das zu  \"ubertragende Signal gegeben und die Spule
induziert in  der DC-Leitung  entsprechende Spannungs-Rippel, die  vom \Master
ausgewertet werden k\"onnen. Der entsprechende  Schaltkreis ist schematisch in
Abbildung \ref{fig:circ:coupling:inductive} dargestellt.

Verglichen mit  Kondensatoren sind Spulen relativ  gross und teuer. Allerdings
ist das Prinzip  der induktiven Einkopplung gut dokumentiert  und die Aussicht
auf Erfolg (bei vern\"unftigem Aufwand) somit gut.

\begin{figure}[h!tb]
    \centering
    \begin{circuitikz}
    \draw
    (-1,0) to[empty photodiode,o-,l_=PV-Modul] (1,0) -- (2,0) to[L,l_=L2] (4,0) -- (7,0) to[L,l_=L2] (9,0) to[short] (10,0)
    (2.5,0.30) -- (3.5,0.30)
    (2.25,0.35) node {K}
    (2.5,0.40) -- (3.5,0.40)
    (7.5,0.30) -- (8.5,0.30)
    (7.25,0.35) node {K}
    (7.5,0.40) -- (8.5,0.40)
    (4,2) -- (4,0.7) to[L,l_=L1] (2,0.7) -- (2,2) to[sinusoidal voltage source,l^=$U_{\mathrm{Sender}}$] (4,2)
    (9,2) -- (9,0.7) to[L,l_=L1] (7,0.7) -- (7,2) to[sinusoidal voltage source,l^=$U_{\mathrm{Empf\text{\"a}nger}}$] (9,2)
    ;
\end{circuitikz}

    \caption{Induktive Einkopplung}
    \label{fig:circ:coupling:inductive}
\end{figure}

% ---------------------------------------------------------------------------- %
\subsection{Sender}
\label{sec:simu:fsk:inductive:transmitter}
% ---------------------------------------------------------------------------- %

% ---------------------------------------------------------------------------- %
\subsection{Empf\"anger}
\label{sec:simu:fsk:inductive:receiver}
% ---------------------------------------------------------------------------- %

% ---------------------------------------------------------------------------- %
\subsection{Gesamtsystem}
\label{sec:simu:fsk:inductive:inductive}
% ---------------------------------------------------------------------------- %


% ---------------------------------------------------------------------------- %
\section{Amplitudenumtastung}
\label{sec:simu:ask}
% ---------------------------------------------------------------------------- %

Wie in Abschnitt \todo{reference} erw\"ahnt, wird bei dieser L\"osungsvariante
jeweils   ein   Modul   gesteuert   kurzgeschlossen. Dies   verursacht   kurze
Spannungseinbr\"uche auf  der DC-Leitung,  welche vom  Empf\"anger ausgewertet
werden k\"onnen, wie in Abbildung \todo{reference}vereinfact dargestellt.

Potentielle Probleme sind bei  dieser L\"osungsvariante in folgenden Bereichen
zu suchen:

\begin{symbols}
    \firmlist
    \item[\textbf{Induktivit\"at der Leitung:}]
        Der  Spannungsabfall   auf  der  DC-Leitung  wird   bei  geschlossenem
        Stromkreis zu Strom\"anderungen auf der DC-Leitung f\"uhren. Dies wird
        eine Spannungs\"anderung auf der DC-Leitung bewirken\footnotemark.
        \footnotetext{%
            Spannung in Abh\"angigkeit der Strom\"anderung:
            $v = L \cdot \frac{\mathrm{d}i}{\mathrm{d}t}$%
        }

        Gem\"ass    Lenz'scher     Regel    \todo{reference}     wird    diese
        Spannungs\"anderung  so gerichtet  sein,  dass  sich der  zugeh\"orige
        Strom der aufgezwungenen \"Anderung widersetzt.

        Es  kann also  sein, dass  die  Induktivit\"at der  DC-Leitung das  zu
        \"ubertragende  Signal  so  stark  kompensiert,  dass  es  nicht  mehr
        detektierbar  ist. Je   h\"oher  die  Frequenz,  mit   der  das  Modul
        kurzgeschlossen  und  wieder  ge\"offnet   wird,  um  so  h\"oher  die
        zugeh\"origen Strom\"anderungen
        $\frac{\mathrm{d}i}{\mathrm{d}{t}}$
        und somit Impedanz der Induktivit\"at (gem\"ass
        $Z_L = j \omega L$).
    \item[\textbf{Kapazit\"at des Solarmoduls:}]
        Dem  Solarmodul   wird  bei   diesem  Vorgehen  das   Verhalten  einer
        Wechselstromquelle   aufgezwungen. Besitzt  es   interne  parasit\"are
        Kapazit\"aten,  k\"onnen  diese  bei  den  abrupten  \"Anderungen  der
        Spannung hohe Str\"ome im  kurzgeschlossenen Pfad und seinen Bauteilen
        verursachen.\footnotemark
        \footnotetext{%
            Strom\"anderung in Abh\"angigkeit der Spannungs\"anderung:
            $I(t) = C \cdot \frac{\mathrm{V(t)}}{\mathrm{d}t}$%
        }

        Besitzen    diese   Bauteile    Ohm'sche   Widerst\"ande,    entstehen
        entsprechende thermische  Verluste, welche die  Bauteile besch\"adigen
        k\"onnen.
\end{symbols}


% ---------------------------------------------------------------------------- %
\subsection{Sender}
\label{subsec:simu:ask:sensor}
% ---------------------------------------------------------------------------- %

Das  gesteuerte Kurzschliessen  des Moduls  wird mit  einem MOSFET  umgesetzt,
welcher zwischen  Eingang und  Ausgang des Moduls  durchschalten kann  und vom
Microcontroller auf dem Sensor gesteuert wird. \fref{fig:module:mosfet:simple}
zeigt diesen Aufbau schematisch.

\begin{figure}[h!tb]
    \centering
    % Pro memoriam:
%
%            |  D
%      | |---+
%      |
%      | |<--+  B
%      |     |
% G ---+ |---+
%            |  S
%(mos.B) node[anchor=west] {B}
%(mos.G) node[anchor=east] {G}
%(mos.D) node[anchor=north] {D}
%(mos.S) node[anchor=south] {S}

\begin{circuitikz}
    \small
    \draw
    (4.5,1) node[nigfete] (mos) {MOSFET}

    (0,0) to[empty photodiode,l_=PV-Modul] (0,2) -- (4.5,2) -- (mos.D)
    %(0,0) to[dcisource,l_=PV-Modul] (0,4) -- (8,4) -- (mos.D)
    (mos.S) -- (4.5,0) -- (0,0)

    (2.25,0.725) to[sinusoidal voltage source,l^=Controller] (mos.G)
    (2.25,0.725) -- (2.25,0) node[circ] { }
    ;
\end{circuitikz}

    \caption{%
        Gesteuerter     Kurzschluss     eines    Solarmoduls     mit     einem
        microcontroller-gesteuerten       Transistor. Die      vollst\"andigen
        \code{LTspice}-Schaltungen  sind  in Anhang  \ref{app:simu:module}  ab
        Seite \pageref{app:simu:module} dokumentiert.%
    }
    \label{fig:module:mosfet:simple}
\end{figure}

%TODO: line width
\begin{figure}
    \begin{tikzpicture}
       \begin{scope}[x={(0mm,0mm)},y={(120mm,\textwidth)}]
           \begin{axis}[%
                   height=40mm,
                   width=\textwidth,
                   at={(0,70mm)},
                   grid=both,
                   xlabel=Zeit (\si{\micro\second}),
                   ylabel=Strom (\si{\ampere}),
                   %x unit=u,
                   change x base=true,
                   %line width = 1pt,
                   thick,
                   x SI prefix=micro,
               ]
               \addplot[-,color=blue] table {data/module-72cells--I-MOSFET--0005u.dat};
           \end{axis}
           \begin{axis}[%
                   height=40mm,
                   width=\textwidth,
                   at={(0,35mm)},
                   grid=both,
                   xlabel=Zeit (\si{\micro\second}),
                   ylabel=Spannung (\si{\volt}),
                   %x unit=u,
                   change x base=true,
                   x SI prefix=micro,
               ]
               \addplot[-,color=magenta] table {data/module-72cells--U-MOSFET--0005u.dat};
           \end{axis}
           \begin{axis}[%
                   height=40mm,
                   width=\textwidth,
                   at={(0,0)},
                   grid=both,
                   xlabel=Zeit (\si{\micro\second}),
                   ylabel=Leistung (\si{\watt}),
                   %x unit=u,
                   change x base=true,
                   x SI prefix=micro,
               ]
               \addplot[-,color=red] table {data/module-72cells--P-MOSFET--0005u.dat};
           \end{axis}
       \end{scope}
   \end{tikzpicture}
   \caption{%
       Verlauf    von    Strom,    Spannung    und    Leistung    am    MOSFET
       bei   einer   Schaltfrequenz   von   \SI{10}{\kilo\hertz}   bei   einer
       Modulkonfiguration    von    $36     \times    2$    Zellen    gem\"ass
       Schema    in    \fref{fig:ltspice:module:cellBased:36x2}   auf    Seite
       \ref{fig:ltspice:module:cellBased:36x2}. Der    MOSFET     wurde    mit
       \SI{3.3}{\volt} angesteuert, da dies  die gr\"osste vom Microcontroller
       zur Verf\"ugung stehende Spannung ist.\
       Anhang       \ref{app:sec:simu:complementary:36x2}      auf       Seite
       \pageref{app:sec:simu:complementary:36x2} beinhaltet zum Vergleich noch
       Simulationen f\"ur den Zeitraum von einer Millisekunde.%
   }
   \label{fig:simu:results:36x2:3u}
\end{figure}

\begin{figure}
    \begin{tikzpicture}
       \begin{scope}[x={(0mm,0mm)},y={(120mm,\textwidth)}]
           \begin{axis}[%
                   height=40mm,
                   width=\textwidth,
                   at={(0,70mm)},
                   grid=both,
                   xlabel=Zeit (\si{\micro\second}),
                   ylabel=Strom (\si{\ampere}),
                   %x unit=u,
                   change x base=true,
                   x SI prefix=micro,
               ]
               \addplot[-,color=blue] table {data/module-72cells-series--I-MOSFET--0003u.dat};
           \end{axis}
           \begin{axis}[%
                   height=40mm,
                   width=\textwidth,
                   at={(0,35mm)},
                   grid=both,
                   xlabel=Zeit (\si{\micro\second}),
                   ylabel=Spannung (\si{\volt}),
                   %x unit=u,
                   change x base=true,
                   x SI prefix=micro,
               ]
               \addplot[-,color=magenta] table {data/module-72cells-series--U-MOSFET--0003u.dat};
           \end{axis}
           \begin{axis}[%
                   height=40mm,
                   width=\textwidth,
                   at={(0,0)},
                   grid=both,
                   xlabel=Zeit (\si{\micro\second}),
                   ylabel=Leistung (\si{\watt}),
                   %x unit=u,
                   change x base=true,
                   x SI prefix=micro,
               ]
               \addplot[-,color=red] table {data/module-72cells-series--P-MOSFET--0003u.dat};
           \end{axis}
       \end{scope}
   \end{tikzpicture}
   \caption{%
       Verlauf    von    Strom,    Spannung    und    Leistung    am    MOSFET
       bei   einer   Schaltfrequenz   von   \SI{10}{\kilo\hertz}   bei   einer
       Modulkonfiguration    von    $72     \times    1$    Zellen    gem\"ass
       Schema    in    \fref{fig:ltspice:module:cellBased:36x2}   auf    Seite
       \ref{fig:ltspice:module:cellBased:72x1}. Der    MOSFET     wurde    mit
       \SI{3.3}{\volt} angesteuert, da dies  die gr\"osste vom Microcontroller
       zur Verf\"ugung stehende Spannung ist.\protect\\
       Anhang     \ref{app:sec:simu:complementary:72x1}     auf     Seite
       \pageref{app:sec:simu:complementary:72x1} beinhaltet zum Vergleich noch
       Simulationen f\"ur den Zeitraum von einer Millisekunde.%
   }
   \label{fig:simu:results:72x1:3u}
\end{figure}

Die    Resultate   der    Simulation   f\"ir    ein   $36    \times   2$-Modul
sind   in   \fref{fig:simu:results:36x2:3u}   f\"r   einen   Zeitbereich   von
\SI{3}{\micro\second}  dargestellt,  die  Ergebnisse   f\"ur  das  $72  \times
1$-Modul  in  \fref{fig:simu:results:72x1:3u}.   Tabellen  \ref{tab:36x2:heat}
\ref{tab:72x1:heat} fassen die wichtigsten Eckdaten der Simulationen zusammen,
inklusive Durchschnittswerte f\"ur Strom und Leistung.

\begin{table}[h!tb]
    \centering
    \caption{%
        Eckdaten zur Simulation des Kurzschlussverfahrens f\"ur ein Solarpanel
        mit  $36   \times  2$  Zellen  und   ein  Panel  mit  $72   \times  1$
        Zellen.\protect\\
        Bei     den     Durchschnittswerten     sind     sowohl     Ergebnisse
        f\"ur      \SI{1}{\milli\second}     wie      auch     \SI{1}{\second}
        angegeben,    um   zu    zeigen,   dass    sich   die    Konfiguration
        bereits    bei    einer    Millisekunde   stabilisiert    hat    (auch
        zu     sehen    in     Anhang    \ref{app:sec:simu:complementary:36x2}
        auf      Seite      \pageref{app:sec:simu:complementary:36x2}      und
        Anhang      \ref{app:sec:simu:complementary:72x1}       auf      Seite
        \pageref{app:sec:simu:complementary:72x1}).\protect\\
        Die     Dauer      der     Spitze     ist     beim      $36     \times
        2$-Modul     \SI{2}{\micro\second}    und     \SI{1.78}{\micro\second}
        beim    $72    \times    1$-Modul    (siehe    unterste    Plots    in
        den     Abbildungen      \ref{fig:simu:results:36x2:3u}     respektive
        \ref{fig:simu:results:72x1:3u}).%
    }
    \label{tab:36x2:heat}
    \begin{tabular}{lrr}

    \toprule
    Kriterium                                      & $36 \times 2$-Modul & $72 \times 1$-Modul \\
    \midrule
     Spitzenwert Strom:                            & \SI{32}{\ampere}    & \SI{24}{\ampere}    \\
     Spitzenwert Leistung:                         & \SI{310}{\watt}     & \SI{535}{\watt}     \\
     $\overline{P}$ f\"ur Dauer der Spitze         & \SI{158.77}{\watt}  & \SI{267.46}{\watt}  \\
     $\overline{P}$ f\"ur \SI{1}{\milli\second}    & \SI{4.312}{\watt}   & \SI{6.0577}{\watt}  \\
     $\overline{P}$ f\"ur \SI{1}{\second}          & \SI{4.3047}{\watt}  & \SI{6.0433}{\watt}  \\
    \bottomrule
    \end{tabular}
\end{table}

Aus  \fref{fig:simu:results:36x2:5u}   und  \tref{tab:36x2:heat}   ziehen  wir
folgende  Schlussfolgerungen:

\begin{enumerate}
    \firmlist
    \item
        Der  Spitzenwert f\"ur  den  Strom ist  hoch,  aber Transistoren,  die
        solche Str\"ome  verkraften k\"onnen,  sind zu  vern\"unftigen Preisen
        erh\"altlich.
    \item
        Die    Spitzenwerte    der    Leistungen   zwar    nur    sehr    kurz
        (etwa   \SI{2}{\micro\second}),   aber   sehr  hoch. Auch   wenn   der
        Durchschnittliche Leistungswert  weit unter  der Grenze  von g\"unstig
        erh\"altlichen  MOSFETs   liegt,  k\"onnte  die   Leistungsspitze  den
        Transistor irreversibel sch\"adigen.
    \item
        Der  durchschnittliche  Leistungsverbrauch  ist  weit  \"uber  dem  im
        Pflichtenheft angestrebten Wert von \SI{100}{\milli\watt}.
\end{enumerate}
\todo{korrekte Schlussfolgerungen?}


% ---------------------------------------------------------------------------- %
\clearpage
\subsection{\Master (Empf\"anger)}
\label{subsec:simu:ask:recv}
% ---------------------------------------------------------------------------- %


% ---------------------------------------------------------------------------- %
\subsection{Gesamtsystem}
\label{subsec:simu:ask:total}
% ---------------------------------------------------------------------------- %
