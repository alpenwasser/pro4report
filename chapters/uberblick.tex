% **************************************************************************** %
\chapter{\"Uberblick}
\label{chap:uberblick}
% **************************************************************************** %
Dieses Kapitel  beschreibt zuerst die grobe  Idee unsers L\"osungskonzepts. Es
wird  dargelegt, wie  unser System  in  eine Solaranlage  (bestehend oder  neu
aufgebaut) integriert wird, wie das System mit seiner Umgebung interagiert und
wie es zu bedienen ist.


% ---------------------------------------------------------------------------- %
\section{Aufbau einer Solaranlage}
\label{sec:solaranlage:aufbau}
% ---------------------------------------------------------------------------- %

Grunds\"atzlicher Aufbau einer Solaranlage: Zelle -> Modul -> Strings -> Anlage


% ---------------------------------------------------------------------------- %
\section{Einbettung in Umwelt}
\label{sec:einbettung}
% ---------------------------------------------------------------------------- %

Hier  wird  beschrieben,  wie  unser System  physisch  mit  einer  Solaranlage
integriert wird und welche Schnittstellen es zu welchem Zweck zur Anlage hat.
\todo{kein eigener Abschnitt mehr n\"otig}


% ---------------------------------------------------------------------------- %
\section{Kernproblem: Kommunikation \"uber DC-Leitung}
\label{sec:commDCLine}
% ---------------------------------------------------------------------------- %

\todo{Referenz auf Konzept aus obigem Kapitel}
Es sind  zwei L\"osungsans\"atze untersucht  worden, um ein Signal  \"uber die
Gleichstromleitung zu senden:

\textbf{Frequency-shift keying}: Bei  der FSK (Frequenzumtastung  auf Deutsch)
wird  dem in  der Leitung  fliessenden Gleichstrom  ein (verh\"altnism\"assig)
kleines  Signal aufmoduliert,  welches  die  zu \"ubertragenden  Informationen
enth\"alt. Die  Frequenz   des  aufmodulierten   Anteils  wird   in  diskreten
Schritten variiert  und jeweils  einem Symbol zugeordnet. Bei  einer bin\"aren
Umsetzung  werden  zwei  Frequenzen  benutzt; eine  f\"ur  \code{0}  und  eine
f\"ur  \code{1}.   Das  Verfahren ist  schematisch  in  \fref{fig:fsk:concept}
dargestellt.

\begin{figure}[h!tb]
    \centering
    %% Creator: Matplotlib, PGF backend
%%
%% To include the figure in your LaTeX document, write
%%   \input{<filename>.pgf}
%%
%% Make sure the required packages are loaded in your preamble
%%   \usepackage{pgf}
%%
%% Figures using additional raster images can only be included by \input if
%% they are in the same directory as the main LaTeX file. For loading figures
%% from other directories you can use the `import` package
%%   \usepackage{import}
%% and then include the figures with
%%   \import{<path to file>}{<filename>.pgf}
%%
%% Matplotlib used the following preamble
%%   \usepackage{fontspec}
%%   \setmainfont{Bitstream Vera Serif}
%%   \setsansfont{Bitstream Vera Sans}
%%   \setmonofont{Bitstream Vera Sans Mono}
%%
\begingroup%
\makeatletter%
\begin{pgfpicture}%
\pgfpathrectangle{\pgfpointorigin}{\pgfqpoint{4.500000in}{3.500000in}}%
\pgfusepath{use as bounding box, clip}%
\begin{pgfscope}%
\pgfsetbuttcap%
\pgfsetmiterjoin%
\pgfsetlinewidth{0.000000pt}%
\definecolor{currentstroke}{rgb}{0.000000,0.000000,0.000000}%
\pgfsetstrokecolor{currentstroke}%
\pgfsetstrokeopacity{0.000000}%
\pgfsetdash{}{0pt}%
\pgfpathmoveto{\pgfqpoint{0.000000in}{0.000000in}}%
\pgfpathlineto{\pgfqpoint{4.500000in}{0.000000in}}%
\pgfpathlineto{\pgfqpoint{4.500000in}{3.500000in}}%
\pgfpathlineto{\pgfqpoint{0.000000in}{3.500000in}}%
\pgfpathclose%
\pgfusepath{}%
\end{pgfscope}%
\begin{pgfscope}%
\pgfsetbuttcap%
\pgfsetmiterjoin%
\pgfsetlinewidth{0.000000pt}%
\definecolor{currentstroke}{rgb}{0.000000,0.000000,0.000000}%
\pgfsetstrokecolor{currentstroke}%
\pgfsetstrokeopacity{0.000000}%
\pgfsetdash{}{0pt}%
\pgfpathmoveto{\pgfqpoint{0.675000in}{2.268396in}}%
\pgfpathlineto{\pgfqpoint{4.275000in}{2.268396in}}%
\pgfpathlineto{\pgfqpoint{4.275000in}{3.325000in}}%
\pgfpathlineto{\pgfqpoint{0.675000in}{3.325000in}}%
\pgfpathclose%
\pgfusepath{}%
\end{pgfscope}%
\begin{pgfscope}%
\pgfpathrectangle{\pgfqpoint{0.675000in}{2.268396in}}{\pgfqpoint{3.600000in}{1.056604in}} %
\pgfusepath{clip}%
\pgfsetrectcap%
\pgfsetroundjoin%
\pgfsetlinewidth{0.501875pt}%
\definecolor{currentstroke}{rgb}{0.000000,0.000000,1.000000}%
\pgfsetstrokecolor{currentstroke}%
\pgfsetdash{}{0pt}%
\pgfpathmoveto{\pgfqpoint{0.675000in}{2.356447in}}%
\pgfpathlineto{\pgfqpoint{1.531798in}{2.356447in}}%
\pgfusepath{stroke}%
\end{pgfscope}%
\begin{pgfscope}%
\pgfpathrectangle{\pgfqpoint{0.675000in}{2.268396in}}{\pgfqpoint{3.600000in}{1.056604in}} %
\pgfusepath{clip}%
\pgfsetrectcap%
\pgfsetroundjoin%
\pgfsetlinewidth{0.501875pt}%
\definecolor{currentstroke}{rgb}{0.501961,0.501961,0.501961}%
\pgfsetstrokecolor{currentstroke}%
\pgfsetdash{}{0pt}%
\pgfpathmoveto{\pgfqpoint{1.531798in}{2.356447in}}%
\pgfpathlineto{\pgfqpoint{1.531798in}{3.236950in}}%
\pgfusepath{stroke}%
\end{pgfscope}%
\begin{pgfscope}%
\pgfpathrectangle{\pgfqpoint{0.675000in}{2.268396in}}{\pgfqpoint{3.600000in}{1.056604in}} %
\pgfusepath{clip}%
\pgfsetrectcap%
\pgfsetroundjoin%
\pgfsetlinewidth{0.501875pt}%
\definecolor{currentstroke}{rgb}{1.000000,0.000000,1.000000}%
\pgfsetstrokecolor{currentstroke}%
\pgfsetdash{}{0pt}%
\pgfpathmoveto{\pgfqpoint{1.531798in}{3.236950in}}%
\pgfpathlineto{\pgfqpoint{2.388596in}{3.236950in}}%
\pgfusepath{stroke}%
\end{pgfscope}%
\begin{pgfscope}%
\pgfpathrectangle{\pgfqpoint{0.675000in}{2.268396in}}{\pgfqpoint{3.600000in}{1.056604in}} %
\pgfusepath{clip}%
\pgfsetrectcap%
\pgfsetroundjoin%
\pgfsetlinewidth{0.501875pt}%
\definecolor{currentstroke}{rgb}{0.501961,0.501961,0.501961}%
\pgfsetstrokecolor{currentstroke}%
\pgfsetdash{}{0pt}%
\pgfpathmoveto{\pgfqpoint{2.388596in}{3.236950in}}%
\pgfpathlineto{\pgfqpoint{2.388596in}{2.356447in}}%
\pgfusepath{stroke}%
\end{pgfscope}%
\begin{pgfscope}%
\pgfpathrectangle{\pgfqpoint{0.675000in}{2.268396in}}{\pgfqpoint{3.600000in}{1.056604in}} %
\pgfusepath{clip}%
\pgfsetrectcap%
\pgfsetroundjoin%
\pgfsetlinewidth{0.501875pt}%
\definecolor{currentstroke}{rgb}{0.000000,0.000000,1.000000}%
\pgfsetstrokecolor{currentstroke}%
\pgfsetdash{}{0pt}%
\pgfpathmoveto{\pgfqpoint{2.388596in}{2.356447in}}%
\pgfpathlineto{\pgfqpoint{3.245394in}{2.356447in}}%
\pgfusepath{stroke}%
\end{pgfscope}%
\begin{pgfscope}%
\pgfpathrectangle{\pgfqpoint{0.675000in}{2.268396in}}{\pgfqpoint{3.600000in}{1.056604in}} %
\pgfusepath{clip}%
\pgfsetrectcap%
\pgfsetroundjoin%
\pgfsetlinewidth{0.501875pt}%
\definecolor{currentstroke}{rgb}{0.501961,0.501961,0.501961}%
\pgfsetstrokecolor{currentstroke}%
\pgfsetdash{}{0pt}%
\pgfpathmoveto{\pgfqpoint{3.245394in}{2.356447in}}%
\pgfpathlineto{\pgfqpoint{3.245394in}{3.236950in}}%
\pgfusepath{stroke}%
\end{pgfscope}%
\begin{pgfscope}%
\pgfpathrectangle{\pgfqpoint{0.675000in}{2.268396in}}{\pgfqpoint{3.600000in}{1.056604in}} %
\pgfusepath{clip}%
\pgfsetrectcap%
\pgfsetroundjoin%
\pgfsetlinewidth{0.501875pt}%
\definecolor{currentstroke}{rgb}{1.000000,0.000000,1.000000}%
\pgfsetstrokecolor{currentstroke}%
\pgfsetdash{}{0pt}%
\pgfpathmoveto{\pgfqpoint{3.245394in}{3.236950in}}%
\pgfpathlineto{\pgfqpoint{4.102192in}{3.236950in}}%
\pgfusepath{stroke}%
\end{pgfscope}%
\begin{pgfscope}%
\pgfsetrectcap%
\pgfsetmiterjoin%
\pgfsetlinewidth{0.501875pt}%
\definecolor{currentstroke}{rgb}{0.000000,0.000000,0.000000}%
\pgfsetstrokecolor{currentstroke}%
\pgfsetdash{}{0pt}%
\pgfpathmoveto{\pgfqpoint{0.675000in}{3.325000in}}%
\pgfpathlineto{\pgfqpoint{4.275000in}{3.325000in}}%
\pgfusepath{stroke}%
\end{pgfscope}%
\begin{pgfscope}%
\pgfsetrectcap%
\pgfsetmiterjoin%
\pgfsetlinewidth{0.501875pt}%
\definecolor{currentstroke}{rgb}{0.000000,0.000000,0.000000}%
\pgfsetstrokecolor{currentstroke}%
\pgfsetdash{}{0pt}%
\pgfpathmoveto{\pgfqpoint{0.675000in}{2.268396in}}%
\pgfpathlineto{\pgfqpoint{0.675000in}{3.325000in}}%
\pgfusepath{stroke}%
\end{pgfscope}%
\begin{pgfscope}%
\pgfsetrectcap%
\pgfsetmiterjoin%
\pgfsetlinewidth{0.501875pt}%
\definecolor{currentstroke}{rgb}{0.000000,0.000000,0.000000}%
\pgfsetstrokecolor{currentstroke}%
\pgfsetdash{}{0pt}%
\pgfpathmoveto{\pgfqpoint{4.275000in}{2.268396in}}%
\pgfpathlineto{\pgfqpoint{4.275000in}{3.325000in}}%
\pgfusepath{stroke}%
\end{pgfscope}%
\begin{pgfscope}%
\pgfsetrectcap%
\pgfsetmiterjoin%
\pgfsetlinewidth{0.501875pt}%
\definecolor{currentstroke}{rgb}{0.000000,0.000000,0.000000}%
\pgfsetstrokecolor{currentstroke}%
\pgfsetdash{}{0pt}%
\pgfpathmoveto{\pgfqpoint{0.675000in}{2.268396in}}%
\pgfpathlineto{\pgfqpoint{4.275000in}{2.268396in}}%
\pgfusepath{stroke}%
\end{pgfscope}%
\begin{pgfscope}%
\pgfsetbuttcap%
\pgfsetroundjoin%
\definecolor{currentfill}{rgb}{0.000000,0.000000,0.000000}%
\pgfsetfillcolor{currentfill}%
\pgfsetlinewidth{0.501875pt}%
\definecolor{currentstroke}{rgb}{0.000000,0.000000,0.000000}%
\pgfsetstrokecolor{currentstroke}%
\pgfsetdash{}{0pt}%
\pgfsys@defobject{currentmarker}{\pgfqpoint{0.000000in}{0.000000in}}{\pgfqpoint{0.000000in}{0.055556in}}{%
\pgfpathmoveto{\pgfqpoint{0.000000in}{0.000000in}}%
\pgfpathlineto{\pgfqpoint{0.000000in}{0.055556in}}%
\pgfusepath{stroke,fill}%
}%
\begin{pgfscope}%
\pgfsys@transformshift{0.675000in}{2.268396in}%
\pgfsys@useobject{currentmarker}{}%
\end{pgfscope}%
\end{pgfscope}%
\begin{pgfscope}%
\pgfsetbuttcap%
\pgfsetroundjoin%
\definecolor{currentfill}{rgb}{0.000000,0.000000,0.000000}%
\pgfsetfillcolor{currentfill}%
\pgfsetlinewidth{0.501875pt}%
\definecolor{currentstroke}{rgb}{0.000000,0.000000,0.000000}%
\pgfsetstrokecolor{currentstroke}%
\pgfsetdash{}{0pt}%
\pgfsys@defobject{currentmarker}{\pgfqpoint{0.000000in}{-0.055556in}}{\pgfqpoint{0.000000in}{0.000000in}}{%
\pgfpathmoveto{\pgfqpoint{0.000000in}{0.000000in}}%
\pgfpathlineto{\pgfqpoint{0.000000in}{-0.055556in}}%
\pgfusepath{stroke,fill}%
}%
\begin{pgfscope}%
\pgfsys@transformshift{0.675000in}{3.325000in}%
\pgfsys@useobject{currentmarker}{}%
\end{pgfscope}%
\end{pgfscope}%
\begin{pgfscope}%
\pgftext[x=0.675000in,y=2.212841in,,top]{\rmfamily\fontsize{9.000000}{10.800000}\selectfont \(\displaystyle 0.0000\)}%
\end{pgfscope}%
\begin{pgfscope}%
\pgfsetbuttcap%
\pgfsetroundjoin%
\definecolor{currentfill}{rgb}{0.000000,0.000000,0.000000}%
\pgfsetfillcolor{currentfill}%
\pgfsetlinewidth{0.501875pt}%
\definecolor{currentstroke}{rgb}{0.000000,0.000000,0.000000}%
\pgfsetstrokecolor{currentstroke}%
\pgfsetdash{}{0pt}%
\pgfsys@defobject{currentmarker}{\pgfqpoint{0.000000in}{0.000000in}}{\pgfqpoint{0.000000in}{0.055556in}}{%
\pgfpathmoveto{\pgfqpoint{0.000000in}{0.000000in}}%
\pgfpathlineto{\pgfqpoint{0.000000in}{0.055556in}}%
\pgfusepath{stroke,fill}%
}%
\begin{pgfscope}%
\pgfsys@transformshift{1.125000in}{2.268396in}%
\pgfsys@useobject{currentmarker}{}%
\end{pgfscope}%
\end{pgfscope}%
\begin{pgfscope}%
\pgfsetbuttcap%
\pgfsetroundjoin%
\definecolor{currentfill}{rgb}{0.000000,0.000000,0.000000}%
\pgfsetfillcolor{currentfill}%
\pgfsetlinewidth{0.501875pt}%
\definecolor{currentstroke}{rgb}{0.000000,0.000000,0.000000}%
\pgfsetstrokecolor{currentstroke}%
\pgfsetdash{}{0pt}%
\pgfsys@defobject{currentmarker}{\pgfqpoint{0.000000in}{-0.055556in}}{\pgfqpoint{0.000000in}{0.000000in}}{%
\pgfpathmoveto{\pgfqpoint{0.000000in}{0.000000in}}%
\pgfpathlineto{\pgfqpoint{0.000000in}{-0.055556in}}%
\pgfusepath{stroke,fill}%
}%
\begin{pgfscope}%
\pgfsys@transformshift{1.125000in}{3.325000in}%
\pgfsys@useobject{currentmarker}{}%
\end{pgfscope}%
\end{pgfscope}%
\begin{pgfscope}%
\pgftext[x=1.125000in,y=2.212841in,,top]{\rmfamily\fontsize{9.000000}{10.800000}\selectfont \(\displaystyle 0.0002\)}%
\end{pgfscope}%
\begin{pgfscope}%
\pgfsetbuttcap%
\pgfsetroundjoin%
\definecolor{currentfill}{rgb}{0.000000,0.000000,0.000000}%
\pgfsetfillcolor{currentfill}%
\pgfsetlinewidth{0.501875pt}%
\definecolor{currentstroke}{rgb}{0.000000,0.000000,0.000000}%
\pgfsetstrokecolor{currentstroke}%
\pgfsetdash{}{0pt}%
\pgfsys@defobject{currentmarker}{\pgfqpoint{0.000000in}{0.000000in}}{\pgfqpoint{0.000000in}{0.055556in}}{%
\pgfpathmoveto{\pgfqpoint{0.000000in}{0.000000in}}%
\pgfpathlineto{\pgfqpoint{0.000000in}{0.055556in}}%
\pgfusepath{stroke,fill}%
}%
\begin{pgfscope}%
\pgfsys@transformshift{1.575000in}{2.268396in}%
\pgfsys@useobject{currentmarker}{}%
\end{pgfscope}%
\end{pgfscope}%
\begin{pgfscope}%
\pgfsetbuttcap%
\pgfsetroundjoin%
\definecolor{currentfill}{rgb}{0.000000,0.000000,0.000000}%
\pgfsetfillcolor{currentfill}%
\pgfsetlinewidth{0.501875pt}%
\definecolor{currentstroke}{rgb}{0.000000,0.000000,0.000000}%
\pgfsetstrokecolor{currentstroke}%
\pgfsetdash{}{0pt}%
\pgfsys@defobject{currentmarker}{\pgfqpoint{0.000000in}{-0.055556in}}{\pgfqpoint{0.000000in}{0.000000in}}{%
\pgfpathmoveto{\pgfqpoint{0.000000in}{0.000000in}}%
\pgfpathlineto{\pgfqpoint{0.000000in}{-0.055556in}}%
\pgfusepath{stroke,fill}%
}%
\begin{pgfscope}%
\pgfsys@transformshift{1.575000in}{3.325000in}%
\pgfsys@useobject{currentmarker}{}%
\end{pgfscope}%
\end{pgfscope}%
\begin{pgfscope}%
\pgftext[x=1.575000in,y=2.212841in,,top]{\rmfamily\fontsize{9.000000}{10.800000}\selectfont \(\displaystyle 0.0004\)}%
\end{pgfscope}%
\begin{pgfscope}%
\pgfsetbuttcap%
\pgfsetroundjoin%
\definecolor{currentfill}{rgb}{0.000000,0.000000,0.000000}%
\pgfsetfillcolor{currentfill}%
\pgfsetlinewidth{0.501875pt}%
\definecolor{currentstroke}{rgb}{0.000000,0.000000,0.000000}%
\pgfsetstrokecolor{currentstroke}%
\pgfsetdash{}{0pt}%
\pgfsys@defobject{currentmarker}{\pgfqpoint{0.000000in}{0.000000in}}{\pgfqpoint{0.000000in}{0.055556in}}{%
\pgfpathmoveto{\pgfqpoint{0.000000in}{0.000000in}}%
\pgfpathlineto{\pgfqpoint{0.000000in}{0.055556in}}%
\pgfusepath{stroke,fill}%
}%
\begin{pgfscope}%
\pgfsys@transformshift{2.025000in}{2.268396in}%
\pgfsys@useobject{currentmarker}{}%
\end{pgfscope}%
\end{pgfscope}%
\begin{pgfscope}%
\pgfsetbuttcap%
\pgfsetroundjoin%
\definecolor{currentfill}{rgb}{0.000000,0.000000,0.000000}%
\pgfsetfillcolor{currentfill}%
\pgfsetlinewidth{0.501875pt}%
\definecolor{currentstroke}{rgb}{0.000000,0.000000,0.000000}%
\pgfsetstrokecolor{currentstroke}%
\pgfsetdash{}{0pt}%
\pgfsys@defobject{currentmarker}{\pgfqpoint{0.000000in}{-0.055556in}}{\pgfqpoint{0.000000in}{0.000000in}}{%
\pgfpathmoveto{\pgfqpoint{0.000000in}{0.000000in}}%
\pgfpathlineto{\pgfqpoint{0.000000in}{-0.055556in}}%
\pgfusepath{stroke,fill}%
}%
\begin{pgfscope}%
\pgfsys@transformshift{2.025000in}{3.325000in}%
\pgfsys@useobject{currentmarker}{}%
\end{pgfscope}%
\end{pgfscope}%
\begin{pgfscope}%
\pgftext[x=2.025000in,y=2.212841in,,top]{\rmfamily\fontsize{9.000000}{10.800000}\selectfont \(\displaystyle 0.0006\)}%
\end{pgfscope}%
\begin{pgfscope}%
\pgfsetbuttcap%
\pgfsetroundjoin%
\definecolor{currentfill}{rgb}{0.000000,0.000000,0.000000}%
\pgfsetfillcolor{currentfill}%
\pgfsetlinewidth{0.501875pt}%
\definecolor{currentstroke}{rgb}{0.000000,0.000000,0.000000}%
\pgfsetstrokecolor{currentstroke}%
\pgfsetdash{}{0pt}%
\pgfsys@defobject{currentmarker}{\pgfqpoint{0.000000in}{0.000000in}}{\pgfqpoint{0.000000in}{0.055556in}}{%
\pgfpathmoveto{\pgfqpoint{0.000000in}{0.000000in}}%
\pgfpathlineto{\pgfqpoint{0.000000in}{0.055556in}}%
\pgfusepath{stroke,fill}%
}%
\begin{pgfscope}%
\pgfsys@transformshift{2.475000in}{2.268396in}%
\pgfsys@useobject{currentmarker}{}%
\end{pgfscope}%
\end{pgfscope}%
\begin{pgfscope}%
\pgfsetbuttcap%
\pgfsetroundjoin%
\definecolor{currentfill}{rgb}{0.000000,0.000000,0.000000}%
\pgfsetfillcolor{currentfill}%
\pgfsetlinewidth{0.501875pt}%
\definecolor{currentstroke}{rgb}{0.000000,0.000000,0.000000}%
\pgfsetstrokecolor{currentstroke}%
\pgfsetdash{}{0pt}%
\pgfsys@defobject{currentmarker}{\pgfqpoint{0.000000in}{-0.055556in}}{\pgfqpoint{0.000000in}{0.000000in}}{%
\pgfpathmoveto{\pgfqpoint{0.000000in}{0.000000in}}%
\pgfpathlineto{\pgfqpoint{0.000000in}{-0.055556in}}%
\pgfusepath{stroke,fill}%
}%
\begin{pgfscope}%
\pgfsys@transformshift{2.475000in}{3.325000in}%
\pgfsys@useobject{currentmarker}{}%
\end{pgfscope}%
\end{pgfscope}%
\begin{pgfscope}%
\pgftext[x=2.475000in,y=2.212841in,,top]{\rmfamily\fontsize{9.000000}{10.800000}\selectfont \(\displaystyle 0.0008\)}%
\end{pgfscope}%
\begin{pgfscope}%
\pgfsetbuttcap%
\pgfsetroundjoin%
\definecolor{currentfill}{rgb}{0.000000,0.000000,0.000000}%
\pgfsetfillcolor{currentfill}%
\pgfsetlinewidth{0.501875pt}%
\definecolor{currentstroke}{rgb}{0.000000,0.000000,0.000000}%
\pgfsetstrokecolor{currentstroke}%
\pgfsetdash{}{0pt}%
\pgfsys@defobject{currentmarker}{\pgfqpoint{0.000000in}{0.000000in}}{\pgfqpoint{0.000000in}{0.055556in}}{%
\pgfpathmoveto{\pgfqpoint{0.000000in}{0.000000in}}%
\pgfpathlineto{\pgfqpoint{0.000000in}{0.055556in}}%
\pgfusepath{stroke,fill}%
}%
\begin{pgfscope}%
\pgfsys@transformshift{2.925000in}{2.268396in}%
\pgfsys@useobject{currentmarker}{}%
\end{pgfscope}%
\end{pgfscope}%
\begin{pgfscope}%
\pgfsetbuttcap%
\pgfsetroundjoin%
\definecolor{currentfill}{rgb}{0.000000,0.000000,0.000000}%
\pgfsetfillcolor{currentfill}%
\pgfsetlinewidth{0.501875pt}%
\definecolor{currentstroke}{rgb}{0.000000,0.000000,0.000000}%
\pgfsetstrokecolor{currentstroke}%
\pgfsetdash{}{0pt}%
\pgfsys@defobject{currentmarker}{\pgfqpoint{0.000000in}{-0.055556in}}{\pgfqpoint{0.000000in}{0.000000in}}{%
\pgfpathmoveto{\pgfqpoint{0.000000in}{0.000000in}}%
\pgfpathlineto{\pgfqpoint{0.000000in}{-0.055556in}}%
\pgfusepath{stroke,fill}%
}%
\begin{pgfscope}%
\pgfsys@transformshift{2.925000in}{3.325000in}%
\pgfsys@useobject{currentmarker}{}%
\end{pgfscope}%
\end{pgfscope}%
\begin{pgfscope}%
\pgftext[x=2.925000in,y=2.212841in,,top]{\rmfamily\fontsize{9.000000}{10.800000}\selectfont \(\displaystyle 0.0010\)}%
\end{pgfscope}%
\begin{pgfscope}%
\pgfsetbuttcap%
\pgfsetroundjoin%
\definecolor{currentfill}{rgb}{0.000000,0.000000,0.000000}%
\pgfsetfillcolor{currentfill}%
\pgfsetlinewidth{0.501875pt}%
\definecolor{currentstroke}{rgb}{0.000000,0.000000,0.000000}%
\pgfsetstrokecolor{currentstroke}%
\pgfsetdash{}{0pt}%
\pgfsys@defobject{currentmarker}{\pgfqpoint{0.000000in}{0.000000in}}{\pgfqpoint{0.000000in}{0.055556in}}{%
\pgfpathmoveto{\pgfqpoint{0.000000in}{0.000000in}}%
\pgfpathlineto{\pgfqpoint{0.000000in}{0.055556in}}%
\pgfusepath{stroke,fill}%
}%
\begin{pgfscope}%
\pgfsys@transformshift{3.375000in}{2.268396in}%
\pgfsys@useobject{currentmarker}{}%
\end{pgfscope}%
\end{pgfscope}%
\begin{pgfscope}%
\pgfsetbuttcap%
\pgfsetroundjoin%
\definecolor{currentfill}{rgb}{0.000000,0.000000,0.000000}%
\pgfsetfillcolor{currentfill}%
\pgfsetlinewidth{0.501875pt}%
\definecolor{currentstroke}{rgb}{0.000000,0.000000,0.000000}%
\pgfsetstrokecolor{currentstroke}%
\pgfsetdash{}{0pt}%
\pgfsys@defobject{currentmarker}{\pgfqpoint{0.000000in}{-0.055556in}}{\pgfqpoint{0.000000in}{0.000000in}}{%
\pgfpathmoveto{\pgfqpoint{0.000000in}{0.000000in}}%
\pgfpathlineto{\pgfqpoint{0.000000in}{-0.055556in}}%
\pgfusepath{stroke,fill}%
}%
\begin{pgfscope}%
\pgfsys@transformshift{3.375000in}{3.325000in}%
\pgfsys@useobject{currentmarker}{}%
\end{pgfscope}%
\end{pgfscope}%
\begin{pgfscope}%
\pgftext[x=3.375000in,y=2.212841in,,top]{\rmfamily\fontsize{9.000000}{10.800000}\selectfont \(\displaystyle 0.0012\)}%
\end{pgfscope}%
\begin{pgfscope}%
\pgfsetbuttcap%
\pgfsetroundjoin%
\definecolor{currentfill}{rgb}{0.000000,0.000000,0.000000}%
\pgfsetfillcolor{currentfill}%
\pgfsetlinewidth{0.501875pt}%
\definecolor{currentstroke}{rgb}{0.000000,0.000000,0.000000}%
\pgfsetstrokecolor{currentstroke}%
\pgfsetdash{}{0pt}%
\pgfsys@defobject{currentmarker}{\pgfqpoint{0.000000in}{0.000000in}}{\pgfqpoint{0.000000in}{0.055556in}}{%
\pgfpathmoveto{\pgfqpoint{0.000000in}{0.000000in}}%
\pgfpathlineto{\pgfqpoint{0.000000in}{0.055556in}}%
\pgfusepath{stroke,fill}%
}%
\begin{pgfscope}%
\pgfsys@transformshift{3.825000in}{2.268396in}%
\pgfsys@useobject{currentmarker}{}%
\end{pgfscope}%
\end{pgfscope}%
\begin{pgfscope}%
\pgfsetbuttcap%
\pgfsetroundjoin%
\definecolor{currentfill}{rgb}{0.000000,0.000000,0.000000}%
\pgfsetfillcolor{currentfill}%
\pgfsetlinewidth{0.501875pt}%
\definecolor{currentstroke}{rgb}{0.000000,0.000000,0.000000}%
\pgfsetstrokecolor{currentstroke}%
\pgfsetdash{}{0pt}%
\pgfsys@defobject{currentmarker}{\pgfqpoint{0.000000in}{-0.055556in}}{\pgfqpoint{0.000000in}{0.000000in}}{%
\pgfpathmoveto{\pgfqpoint{0.000000in}{0.000000in}}%
\pgfpathlineto{\pgfqpoint{0.000000in}{-0.055556in}}%
\pgfusepath{stroke,fill}%
}%
\begin{pgfscope}%
\pgfsys@transformshift{3.825000in}{3.325000in}%
\pgfsys@useobject{currentmarker}{}%
\end{pgfscope}%
\end{pgfscope}%
\begin{pgfscope}%
\pgftext[x=3.825000in,y=2.212841in,,top]{\rmfamily\fontsize{9.000000}{10.800000}\selectfont \(\displaystyle 0.0014\)}%
\end{pgfscope}%
\begin{pgfscope}%
\pgfsetbuttcap%
\pgfsetroundjoin%
\definecolor{currentfill}{rgb}{0.000000,0.000000,0.000000}%
\pgfsetfillcolor{currentfill}%
\pgfsetlinewidth{0.501875pt}%
\definecolor{currentstroke}{rgb}{0.000000,0.000000,0.000000}%
\pgfsetstrokecolor{currentstroke}%
\pgfsetdash{}{0pt}%
\pgfsys@defobject{currentmarker}{\pgfqpoint{0.000000in}{0.000000in}}{\pgfqpoint{0.000000in}{0.055556in}}{%
\pgfpathmoveto{\pgfqpoint{0.000000in}{0.000000in}}%
\pgfpathlineto{\pgfqpoint{0.000000in}{0.055556in}}%
\pgfusepath{stroke,fill}%
}%
\begin{pgfscope}%
\pgfsys@transformshift{4.275000in}{2.268396in}%
\pgfsys@useobject{currentmarker}{}%
\end{pgfscope}%
\end{pgfscope}%
\begin{pgfscope}%
\pgfsetbuttcap%
\pgfsetroundjoin%
\definecolor{currentfill}{rgb}{0.000000,0.000000,0.000000}%
\pgfsetfillcolor{currentfill}%
\pgfsetlinewidth{0.501875pt}%
\definecolor{currentstroke}{rgb}{0.000000,0.000000,0.000000}%
\pgfsetstrokecolor{currentstroke}%
\pgfsetdash{}{0pt}%
\pgfsys@defobject{currentmarker}{\pgfqpoint{0.000000in}{-0.055556in}}{\pgfqpoint{0.000000in}{0.000000in}}{%
\pgfpathmoveto{\pgfqpoint{0.000000in}{0.000000in}}%
\pgfpathlineto{\pgfqpoint{0.000000in}{-0.055556in}}%
\pgfusepath{stroke,fill}%
}%
\begin{pgfscope}%
\pgfsys@transformshift{4.275000in}{3.325000in}%
\pgfsys@useobject{currentmarker}{}%
\end{pgfscope}%
\end{pgfscope}%
\begin{pgfscope}%
\pgftext[x=4.275000in,y=2.212841in,,top]{\rmfamily\fontsize{9.000000}{10.800000}\selectfont \(\displaystyle 0.0016\)}%
\end{pgfscope}%
\begin{pgfscope}%
\pgftext[x=2.475000in,y=2.022425in,,top]{\rmfamily\fontsize{9.000000}{10.800000}\selectfont Zeit (s)}%
\end{pgfscope}%
\begin{pgfscope}%
\pgfsetbuttcap%
\pgfsetroundjoin%
\definecolor{currentfill}{rgb}{0.000000,0.000000,0.000000}%
\pgfsetfillcolor{currentfill}%
\pgfsetlinewidth{0.501875pt}%
\definecolor{currentstroke}{rgb}{0.000000,0.000000,0.000000}%
\pgfsetstrokecolor{currentstroke}%
\pgfsetdash{}{0pt}%
\pgfsys@defobject{currentmarker}{\pgfqpoint{0.000000in}{0.000000in}}{\pgfqpoint{0.055556in}{0.000000in}}{%
\pgfpathmoveto{\pgfqpoint{0.000000in}{0.000000in}}%
\pgfpathlineto{\pgfqpoint{0.055556in}{0.000000in}}%
\pgfusepath{stroke,fill}%
}%
\begin{pgfscope}%
\pgfsys@transformshift{0.675000in}{2.356447in}%
\pgfsys@useobject{currentmarker}{}%
\end{pgfscope}%
\end{pgfscope}%
\begin{pgfscope}%
\pgfsetbuttcap%
\pgfsetroundjoin%
\definecolor{currentfill}{rgb}{0.000000,0.000000,0.000000}%
\pgfsetfillcolor{currentfill}%
\pgfsetlinewidth{0.501875pt}%
\definecolor{currentstroke}{rgb}{0.000000,0.000000,0.000000}%
\pgfsetstrokecolor{currentstroke}%
\pgfsetdash{}{0pt}%
\pgfsys@defobject{currentmarker}{\pgfqpoint{-0.055556in}{0.000000in}}{\pgfqpoint{0.000000in}{0.000000in}}{%
\pgfpathmoveto{\pgfqpoint{0.000000in}{0.000000in}}%
\pgfpathlineto{\pgfqpoint{-0.055556in}{0.000000in}}%
\pgfusepath{stroke,fill}%
}%
\begin{pgfscope}%
\pgfsys@transformshift{4.275000in}{2.356447in}%
\pgfsys@useobject{currentmarker}{}%
\end{pgfscope}%
\end{pgfscope}%
\begin{pgfscope}%
\pgftext[x=0.619444in,y=2.356447in,right,]{\rmfamily\fontsize{9.000000}{10.800000}\selectfont \(\displaystyle 0.0\)}%
\end{pgfscope}%
\begin{pgfscope}%
\pgfsetbuttcap%
\pgfsetroundjoin%
\definecolor{currentfill}{rgb}{0.000000,0.000000,0.000000}%
\pgfsetfillcolor{currentfill}%
\pgfsetlinewidth{0.501875pt}%
\definecolor{currentstroke}{rgb}{0.000000,0.000000,0.000000}%
\pgfsetstrokecolor{currentstroke}%
\pgfsetdash{}{0pt}%
\pgfsys@defobject{currentmarker}{\pgfqpoint{0.000000in}{0.000000in}}{\pgfqpoint{0.055556in}{0.000000in}}{%
\pgfpathmoveto{\pgfqpoint{0.000000in}{0.000000in}}%
\pgfpathlineto{\pgfqpoint{0.055556in}{0.000000in}}%
\pgfusepath{stroke,fill}%
}%
\begin{pgfscope}%
\pgfsys@transformshift{0.675000in}{2.532547in}%
\pgfsys@useobject{currentmarker}{}%
\end{pgfscope}%
\end{pgfscope}%
\begin{pgfscope}%
\pgfsetbuttcap%
\pgfsetroundjoin%
\definecolor{currentfill}{rgb}{0.000000,0.000000,0.000000}%
\pgfsetfillcolor{currentfill}%
\pgfsetlinewidth{0.501875pt}%
\definecolor{currentstroke}{rgb}{0.000000,0.000000,0.000000}%
\pgfsetstrokecolor{currentstroke}%
\pgfsetdash{}{0pt}%
\pgfsys@defobject{currentmarker}{\pgfqpoint{-0.055556in}{0.000000in}}{\pgfqpoint{0.000000in}{0.000000in}}{%
\pgfpathmoveto{\pgfqpoint{0.000000in}{0.000000in}}%
\pgfpathlineto{\pgfqpoint{-0.055556in}{0.000000in}}%
\pgfusepath{stroke,fill}%
}%
\begin{pgfscope}%
\pgfsys@transformshift{4.275000in}{2.532547in}%
\pgfsys@useobject{currentmarker}{}%
\end{pgfscope}%
\end{pgfscope}%
\begin{pgfscope}%
\pgftext[x=0.619444in,y=2.532547in,right,]{\rmfamily\fontsize{9.000000}{10.800000}\selectfont \(\displaystyle 0.2\)}%
\end{pgfscope}%
\begin{pgfscope}%
\pgfsetbuttcap%
\pgfsetroundjoin%
\definecolor{currentfill}{rgb}{0.000000,0.000000,0.000000}%
\pgfsetfillcolor{currentfill}%
\pgfsetlinewidth{0.501875pt}%
\definecolor{currentstroke}{rgb}{0.000000,0.000000,0.000000}%
\pgfsetstrokecolor{currentstroke}%
\pgfsetdash{}{0pt}%
\pgfsys@defobject{currentmarker}{\pgfqpoint{0.000000in}{0.000000in}}{\pgfqpoint{0.055556in}{0.000000in}}{%
\pgfpathmoveto{\pgfqpoint{0.000000in}{0.000000in}}%
\pgfpathlineto{\pgfqpoint{0.055556in}{0.000000in}}%
\pgfusepath{stroke,fill}%
}%
\begin{pgfscope}%
\pgfsys@transformshift{0.675000in}{2.708648in}%
\pgfsys@useobject{currentmarker}{}%
\end{pgfscope}%
\end{pgfscope}%
\begin{pgfscope}%
\pgfsetbuttcap%
\pgfsetroundjoin%
\definecolor{currentfill}{rgb}{0.000000,0.000000,0.000000}%
\pgfsetfillcolor{currentfill}%
\pgfsetlinewidth{0.501875pt}%
\definecolor{currentstroke}{rgb}{0.000000,0.000000,0.000000}%
\pgfsetstrokecolor{currentstroke}%
\pgfsetdash{}{0pt}%
\pgfsys@defobject{currentmarker}{\pgfqpoint{-0.055556in}{0.000000in}}{\pgfqpoint{0.000000in}{0.000000in}}{%
\pgfpathmoveto{\pgfqpoint{0.000000in}{0.000000in}}%
\pgfpathlineto{\pgfqpoint{-0.055556in}{0.000000in}}%
\pgfusepath{stroke,fill}%
}%
\begin{pgfscope}%
\pgfsys@transformshift{4.275000in}{2.708648in}%
\pgfsys@useobject{currentmarker}{}%
\end{pgfscope}%
\end{pgfscope}%
\begin{pgfscope}%
\pgftext[x=0.619444in,y=2.708648in,right,]{\rmfamily\fontsize{9.000000}{10.800000}\selectfont \(\displaystyle 0.4\)}%
\end{pgfscope}%
\begin{pgfscope}%
\pgfsetbuttcap%
\pgfsetroundjoin%
\definecolor{currentfill}{rgb}{0.000000,0.000000,0.000000}%
\pgfsetfillcolor{currentfill}%
\pgfsetlinewidth{0.501875pt}%
\definecolor{currentstroke}{rgb}{0.000000,0.000000,0.000000}%
\pgfsetstrokecolor{currentstroke}%
\pgfsetdash{}{0pt}%
\pgfsys@defobject{currentmarker}{\pgfqpoint{0.000000in}{0.000000in}}{\pgfqpoint{0.055556in}{0.000000in}}{%
\pgfpathmoveto{\pgfqpoint{0.000000in}{0.000000in}}%
\pgfpathlineto{\pgfqpoint{0.055556in}{0.000000in}}%
\pgfusepath{stroke,fill}%
}%
\begin{pgfscope}%
\pgfsys@transformshift{0.675000in}{2.884748in}%
\pgfsys@useobject{currentmarker}{}%
\end{pgfscope}%
\end{pgfscope}%
\begin{pgfscope}%
\pgfsetbuttcap%
\pgfsetroundjoin%
\definecolor{currentfill}{rgb}{0.000000,0.000000,0.000000}%
\pgfsetfillcolor{currentfill}%
\pgfsetlinewidth{0.501875pt}%
\definecolor{currentstroke}{rgb}{0.000000,0.000000,0.000000}%
\pgfsetstrokecolor{currentstroke}%
\pgfsetdash{}{0pt}%
\pgfsys@defobject{currentmarker}{\pgfqpoint{-0.055556in}{0.000000in}}{\pgfqpoint{0.000000in}{0.000000in}}{%
\pgfpathmoveto{\pgfqpoint{0.000000in}{0.000000in}}%
\pgfpathlineto{\pgfqpoint{-0.055556in}{0.000000in}}%
\pgfusepath{stroke,fill}%
}%
\begin{pgfscope}%
\pgfsys@transformshift{4.275000in}{2.884748in}%
\pgfsys@useobject{currentmarker}{}%
\end{pgfscope}%
\end{pgfscope}%
\begin{pgfscope}%
\pgftext[x=0.619444in,y=2.884748in,right,]{\rmfamily\fontsize{9.000000}{10.800000}\selectfont \(\displaystyle 0.6\)}%
\end{pgfscope}%
\begin{pgfscope}%
\pgfsetbuttcap%
\pgfsetroundjoin%
\definecolor{currentfill}{rgb}{0.000000,0.000000,0.000000}%
\pgfsetfillcolor{currentfill}%
\pgfsetlinewidth{0.501875pt}%
\definecolor{currentstroke}{rgb}{0.000000,0.000000,0.000000}%
\pgfsetstrokecolor{currentstroke}%
\pgfsetdash{}{0pt}%
\pgfsys@defobject{currentmarker}{\pgfqpoint{0.000000in}{0.000000in}}{\pgfqpoint{0.055556in}{0.000000in}}{%
\pgfpathmoveto{\pgfqpoint{0.000000in}{0.000000in}}%
\pgfpathlineto{\pgfqpoint{0.055556in}{0.000000in}}%
\pgfusepath{stroke,fill}%
}%
\begin{pgfscope}%
\pgfsys@transformshift{0.675000in}{3.060849in}%
\pgfsys@useobject{currentmarker}{}%
\end{pgfscope}%
\end{pgfscope}%
\begin{pgfscope}%
\pgfsetbuttcap%
\pgfsetroundjoin%
\definecolor{currentfill}{rgb}{0.000000,0.000000,0.000000}%
\pgfsetfillcolor{currentfill}%
\pgfsetlinewidth{0.501875pt}%
\definecolor{currentstroke}{rgb}{0.000000,0.000000,0.000000}%
\pgfsetstrokecolor{currentstroke}%
\pgfsetdash{}{0pt}%
\pgfsys@defobject{currentmarker}{\pgfqpoint{-0.055556in}{0.000000in}}{\pgfqpoint{0.000000in}{0.000000in}}{%
\pgfpathmoveto{\pgfqpoint{0.000000in}{0.000000in}}%
\pgfpathlineto{\pgfqpoint{-0.055556in}{0.000000in}}%
\pgfusepath{stroke,fill}%
}%
\begin{pgfscope}%
\pgfsys@transformshift{4.275000in}{3.060849in}%
\pgfsys@useobject{currentmarker}{}%
\end{pgfscope}%
\end{pgfscope}%
\begin{pgfscope}%
\pgftext[x=0.619444in,y=3.060849in,right,]{\rmfamily\fontsize{9.000000}{10.800000}\selectfont \(\displaystyle 0.8\)}%
\end{pgfscope}%
\begin{pgfscope}%
\pgfsetbuttcap%
\pgfsetroundjoin%
\definecolor{currentfill}{rgb}{0.000000,0.000000,0.000000}%
\pgfsetfillcolor{currentfill}%
\pgfsetlinewidth{0.501875pt}%
\definecolor{currentstroke}{rgb}{0.000000,0.000000,0.000000}%
\pgfsetstrokecolor{currentstroke}%
\pgfsetdash{}{0pt}%
\pgfsys@defobject{currentmarker}{\pgfqpoint{0.000000in}{0.000000in}}{\pgfqpoint{0.055556in}{0.000000in}}{%
\pgfpathmoveto{\pgfqpoint{0.000000in}{0.000000in}}%
\pgfpathlineto{\pgfqpoint{0.055556in}{0.000000in}}%
\pgfusepath{stroke,fill}%
}%
\begin{pgfscope}%
\pgfsys@transformshift{0.675000in}{3.236950in}%
\pgfsys@useobject{currentmarker}{}%
\end{pgfscope}%
\end{pgfscope}%
\begin{pgfscope}%
\pgfsetbuttcap%
\pgfsetroundjoin%
\definecolor{currentfill}{rgb}{0.000000,0.000000,0.000000}%
\pgfsetfillcolor{currentfill}%
\pgfsetlinewidth{0.501875pt}%
\definecolor{currentstroke}{rgb}{0.000000,0.000000,0.000000}%
\pgfsetstrokecolor{currentstroke}%
\pgfsetdash{}{0pt}%
\pgfsys@defobject{currentmarker}{\pgfqpoint{-0.055556in}{0.000000in}}{\pgfqpoint{0.000000in}{0.000000in}}{%
\pgfpathmoveto{\pgfqpoint{0.000000in}{0.000000in}}%
\pgfpathlineto{\pgfqpoint{-0.055556in}{0.000000in}}%
\pgfusepath{stroke,fill}%
}%
\begin{pgfscope}%
\pgfsys@transformshift{4.275000in}{3.236950in}%
\pgfsys@useobject{currentmarker}{}%
\end{pgfscope}%
\end{pgfscope}%
\begin{pgfscope}%
\pgftext[x=0.619444in,y=3.236950in,right,]{\rmfamily\fontsize{9.000000}{10.800000}\selectfont \(\displaystyle 1.0\)}%
\end{pgfscope}%
\begin{pgfscope}%
\pgftext[x=0.385842in,y=2.796698in,,bottom,rotate=90.000000]{\rmfamily\fontsize{9.000000}{10.800000}\selectfont Symbol}%
\end{pgfscope}%
\begin{pgfscope}%
\pgftext[x=2.475000in,y=3.394444in,,base]{\rmfamily\fontsize{11.000000}{13.200000}\selectfont Daten}%
\end{pgfscope}%
\begin{pgfscope}%
\pgfsetbuttcap%
\pgfsetmiterjoin%
\pgfsetlinewidth{0.000000pt}%
\definecolor{currentstroke}{rgb}{0.000000,0.000000,0.000000}%
\pgfsetstrokecolor{currentstroke}%
\pgfsetstrokeopacity{0.000000}%
\pgfsetdash{}{0pt}%
\pgfpathmoveto{\pgfqpoint{0.675000in}{0.525000in}}%
\pgfpathlineto{\pgfqpoint{4.275000in}{0.525000in}}%
\pgfpathlineto{\pgfqpoint{4.275000in}{1.581604in}}%
\pgfpathlineto{\pgfqpoint{0.675000in}{1.581604in}}%
\pgfpathclose%
\pgfusepath{}%
\end{pgfscope}%
\begin{pgfscope}%
\pgfpathrectangle{\pgfqpoint{0.675000in}{0.525000in}}{\pgfqpoint{3.600000in}{1.056604in}} %
\pgfusepath{clip}%
\pgfsetrectcap%
\pgfsetroundjoin%
\pgfsetlinewidth{0.501875pt}%
\definecolor{currentstroke}{rgb}{0.000000,0.000000,1.000000}%
\pgfsetstrokecolor{currentstroke}%
\pgfsetdash{}{0pt}%
\pgfpathmoveto{\pgfqpoint{0.675000in}{0.573027in}}%
\pgfpathlineto{\pgfqpoint{0.679288in}{0.573977in}}%
\pgfpathlineto{\pgfqpoint{0.684434in}{0.577618in}}%
\pgfpathlineto{\pgfqpoint{0.690438in}{0.585286in}}%
\pgfpathlineto{\pgfqpoint{0.697299in}{0.598485in}}%
\pgfpathlineto{\pgfqpoint{0.705018in}{0.618827in}}%
\pgfpathlineto{\pgfqpoint{0.713595in}{0.647938in}}%
\pgfpathlineto{\pgfqpoint{0.723886in}{0.691280in}}%
\pgfpathlineto{\pgfqpoint{0.735894in}{0.752172in}}%
\pgfpathlineto{\pgfqpoint{0.750474in}{0.838430in}}%
\pgfpathlineto{\pgfqpoint{0.770200in}{0.969903in}}%
\pgfpathlineto{\pgfqpoint{0.818228in}{1.296050in}}%
\pgfpathlineto{\pgfqpoint{0.833666in}{1.382886in}}%
\pgfpathlineto{\pgfqpoint{0.845673in}{1.438992in}}%
\pgfpathlineto{\pgfqpoint{0.855965in}{1.477642in}}%
\pgfpathlineto{\pgfqpoint{0.865400in}{1.504612in}}%
\pgfpathlineto{\pgfqpoint{0.873118in}{1.520280in}}%
\pgfpathlineto{\pgfqpoint{0.879980in}{1.529192in}}%
\pgfpathlineto{\pgfqpoint{0.885126in}{1.532719in}}%
\pgfpathlineto{\pgfqpoint{0.890272in}{1.533517in}}%
\pgfpathlineto{\pgfqpoint{0.894560in}{1.532093in}}%
\pgfpathlineto{\pgfqpoint{0.899706in}{1.527886in}}%
\pgfpathlineto{\pgfqpoint{0.905709in}{1.519565in}}%
\pgfpathlineto{\pgfqpoint{0.912571in}{1.505636in}}%
\pgfpathlineto{\pgfqpoint{0.920290in}{1.484505in}}%
\pgfpathlineto{\pgfqpoint{0.929724in}{1.451214in}}%
\pgfpathlineto{\pgfqpoint{0.940016in}{1.406247in}}%
\pgfpathlineto{\pgfqpoint{0.952023in}{1.343724in}}%
\pgfpathlineto{\pgfqpoint{0.967461in}{1.250439in}}%
\pgfpathlineto{\pgfqpoint{0.988902in}{1.105306in}}%
\pgfpathlineto{\pgfqpoint{1.029212in}{0.830364in}}%
\pgfpathlineto{\pgfqpoint{1.045507in}{0.735997in}}%
\pgfpathlineto{\pgfqpoint{1.058372in}{0.674005in}}%
\pgfpathlineto{\pgfqpoint{1.069522in}{0.631102in}}%
\pgfpathlineto{\pgfqpoint{1.078956in}{0.603561in}}%
\pgfpathlineto{\pgfqpoint{1.086675in}{0.587403in}}%
\pgfpathlineto{\pgfqpoint{1.093536in}{0.578044in}}%
\pgfpathlineto{\pgfqpoint{1.098682in}{0.574176in}}%
\pgfpathlineto{\pgfqpoint{1.103828in}{0.573037in}}%
\pgfpathlineto{\pgfqpoint{1.108116in}{0.574176in}}%
\pgfpathlineto{\pgfqpoint{1.113262in}{0.578044in}}%
\pgfpathlineto{\pgfqpoint{1.119266in}{0.585973in}}%
\pgfpathlineto{\pgfqpoint{1.126127in}{0.599465in}}%
\pgfpathlineto{\pgfqpoint{1.133846in}{0.620123in}}%
\pgfpathlineto{\pgfqpoint{1.142422in}{0.649567in}}%
\pgfpathlineto{\pgfqpoint{1.152714in}{0.693273in}}%
\pgfpathlineto{\pgfqpoint{1.164721in}{0.754531in}}%
\pgfpathlineto{\pgfqpoint{1.179302in}{0.841136in}}%
\pgfpathlineto{\pgfqpoint{1.199028in}{0.972880in}}%
\pgfpathlineto{\pgfqpoint{1.246199in}{1.293439in}}%
\pgfpathlineto{\pgfqpoint{1.261636in}{1.380683in}}%
\pgfpathlineto{\pgfqpoint{1.273644in}{1.437184in}}%
\pgfpathlineto{\pgfqpoint{1.283936in}{1.476219in}}%
\pgfpathlineto{\pgfqpoint{1.293370in}{1.503570in}}%
\pgfpathlineto{\pgfqpoint{1.301089in}{1.519565in}}%
\pgfpathlineto{\pgfqpoint{1.307950in}{1.528775in}}%
\pgfpathlineto{\pgfqpoint{1.313096in}{1.532529in}}%
\pgfpathlineto{\pgfqpoint{1.318242in}{1.533555in}}%
\pgfpathlineto{\pgfqpoint{1.322530in}{1.532321in}}%
\pgfpathlineto{\pgfqpoint{1.327676in}{1.528340in}}%
\pgfpathlineto{\pgfqpoint{1.333680in}{1.520280in}}%
\pgfpathlineto{\pgfqpoint{1.340541in}{1.506643in}}%
\pgfpathlineto{\pgfqpoint{1.348260in}{1.485827in}}%
\pgfpathlineto{\pgfqpoint{1.356836in}{1.456217in}}%
\pgfpathlineto{\pgfqpoint{1.367128in}{1.412330in}}%
\pgfpathlineto{\pgfqpoint{1.379135in}{1.350889in}}%
\pgfpathlineto{\pgfqpoint{1.393715in}{1.264112in}}%
\pgfpathlineto{\pgfqpoint{1.413442in}{1.132235in}}%
\pgfpathlineto{\pgfqpoint{1.460613in}{0.811858in}}%
\pgfpathlineto{\pgfqpoint{1.476050in}{0.724818in}}%
\pgfpathlineto{\pgfqpoint{1.488058in}{0.668514in}}%
\pgfpathlineto{\pgfqpoint{1.498349in}{0.629671in}}%
\pgfpathlineto{\pgfqpoint{1.507784in}{0.602510in}}%
\pgfpathlineto{\pgfqpoint{1.515503in}{0.586679in}}%
\pgfpathlineto{\pgfqpoint{1.522364in}{0.577618in}}%
\pgfpathlineto{\pgfqpoint{1.527510in}{0.573977in}}%
\pgfpathlineto{\pgfqpoint{1.531798in}{0.573027in}}%
\pgfpathlineto{\pgfqpoint{1.531798in}{0.573027in}}%
\pgfusepath{stroke}%
\end{pgfscope}%
\begin{pgfscope}%
\pgfpathrectangle{\pgfqpoint{0.675000in}{0.525000in}}{\pgfqpoint{3.600000in}{1.056604in}} %
\pgfusepath{clip}%
\pgfsetrectcap%
\pgfsetroundjoin%
\pgfsetlinewidth{0.501875pt}%
\definecolor{currentstroke}{rgb}{1.000000,0.000000,1.000000}%
\pgfsetstrokecolor{currentstroke}%
\pgfsetdash{}{0pt}%
\pgfpathmoveto{\pgfqpoint{1.531798in}{0.573027in}}%
\pgfpathlineto{\pgfqpoint{1.533513in}{0.574395in}}%
\pgfpathlineto{\pgfqpoint{1.536086in}{0.581551in}}%
\pgfpathlineto{\pgfqpoint{1.539517in}{0.600462in}}%
\pgfpathlineto{\pgfqpoint{1.543805in}{0.638509in}}%
\pgfpathlineto{\pgfqpoint{1.549809in}{0.716108in}}%
\pgfpathlineto{\pgfqpoint{1.557528in}{0.849303in}}%
\pgfpathlineto{\pgfqpoint{1.572965in}{1.167731in}}%
\pgfpathlineto{\pgfqpoint{1.584115in}{1.373994in}}%
\pgfpathlineto{\pgfqpoint{1.590976in}{1.465792in}}%
\pgfpathlineto{\pgfqpoint{1.596122in}{1.510488in}}%
\pgfpathlineto{\pgfqpoint{1.600410in}{1.529969in}}%
\pgfpathlineto{\pgfqpoint{1.602983in}{1.533555in}}%
\pgfpathlineto{\pgfqpoint{1.604699in}{1.532529in}}%
\pgfpathlineto{\pgfqpoint{1.607272in}{1.525881in}}%
\pgfpathlineto{\pgfqpoint{1.610702in}{1.507631in}}%
\pgfpathlineto{\pgfqpoint{1.614991in}{1.470360in}}%
\pgfpathlineto{\pgfqpoint{1.620994in}{1.393707in}}%
\pgfpathlineto{\pgfqpoint{1.628713in}{1.261394in}}%
\pgfpathlineto{\pgfqpoint{1.643293in}{0.960994in}}%
\pgfpathlineto{\pgfqpoint{1.654443in}{0.749825in}}%
\pgfpathlineto{\pgfqpoint{1.662162in}{0.643151in}}%
\pgfpathlineto{\pgfqpoint{1.667308in}{0.597524in}}%
\pgfpathlineto{\pgfqpoint{1.671596in}{0.577211in}}%
\pgfpathlineto{\pgfqpoint{1.674169in}{0.573113in}}%
\pgfpathlineto{\pgfqpoint{1.675884in}{0.573797in}}%
\pgfpathlineto{\pgfqpoint{1.678457in}{0.579936in}}%
\pgfpathlineto{\pgfqpoint{1.681888in}{0.597524in}}%
\pgfpathlineto{\pgfqpoint{1.686176in}{0.634016in}}%
\pgfpathlineto{\pgfqpoint{1.692180in}{0.709715in}}%
\pgfpathlineto{\pgfqpoint{1.699899in}{0.841136in}}%
\pgfpathlineto{\pgfqpoint{1.714479in}{1.141159in}}%
\pgfpathlineto{\pgfqpoint{1.726486in}{1.367192in}}%
\pgfpathlineto{\pgfqpoint{1.733347in}{1.461077in}}%
\pgfpathlineto{\pgfqpoint{1.738493in}{1.507631in}}%
\pgfpathlineto{\pgfqpoint{1.742781in}{1.528775in}}%
\pgfpathlineto{\pgfqpoint{1.745354in}{1.533384in}}%
\pgfpathlineto{\pgfqpoint{1.747070in}{1.533042in}}%
\pgfpathlineto{\pgfqpoint{1.748785in}{1.529969in}}%
\pgfpathlineto{\pgfqpoint{1.751358in}{1.520280in}}%
\pgfpathlineto{\pgfqpoint{1.754788in}{1.498094in}}%
\pgfpathlineto{\pgfqpoint{1.759934in}{1.446069in}}%
\pgfpathlineto{\pgfqpoint{1.765938in}{1.360277in}}%
\pgfpathlineto{\pgfqpoint{1.774515in}{1.202576in}}%
\pgfpathlineto{\pgfqpoint{1.800244in}{0.703445in}}%
\pgfpathlineto{\pgfqpoint{1.807105in}{0.621436in}}%
\pgfpathlineto{\pgfqpoint{1.812251in}{0.585286in}}%
\pgfpathlineto{\pgfqpoint{1.815682in}{0.574395in}}%
\pgfpathlineto{\pgfqpoint{1.817397in}{0.573027in}}%
\pgfpathlineto{\pgfqpoint{1.819113in}{0.574395in}}%
\pgfpathlineto{\pgfqpoint{1.821686in}{0.581551in}}%
\pgfpathlineto{\pgfqpoint{1.825116in}{0.600462in}}%
\pgfpathlineto{\pgfqpoint{1.829405in}{0.638509in}}%
\pgfpathlineto{\pgfqpoint{1.835408in}{0.716108in}}%
\pgfpathlineto{\pgfqpoint{1.843127in}{0.849303in}}%
\pgfpathlineto{\pgfqpoint{1.858565in}{1.167731in}}%
\pgfpathlineto{\pgfqpoint{1.869714in}{1.373994in}}%
\pgfpathlineto{\pgfqpoint{1.876576in}{1.465792in}}%
\pgfpathlineto{\pgfqpoint{1.881722in}{1.510488in}}%
\pgfpathlineto{\pgfqpoint{1.886010in}{1.529969in}}%
\pgfpathlineto{\pgfqpoint{1.888583in}{1.533555in}}%
\pgfpathlineto{\pgfqpoint{1.890298in}{1.532529in}}%
\pgfpathlineto{\pgfqpoint{1.892871in}{1.525881in}}%
\pgfpathlineto{\pgfqpoint{1.896302in}{1.507631in}}%
\pgfpathlineto{\pgfqpoint{1.900590in}{1.470360in}}%
\pgfpathlineto{\pgfqpoint{1.906594in}{1.393707in}}%
\pgfpathlineto{\pgfqpoint{1.914312in}{1.261394in}}%
\pgfpathlineto{\pgfqpoint{1.928893in}{0.960994in}}%
\pgfpathlineto{\pgfqpoint{1.940042in}{0.749825in}}%
\pgfpathlineto{\pgfqpoint{1.947761in}{0.643151in}}%
\pgfpathlineto{\pgfqpoint{1.952907in}{0.597524in}}%
\pgfpathlineto{\pgfqpoint{1.957195in}{0.577211in}}%
\pgfpathlineto{\pgfqpoint{1.959768in}{0.573113in}}%
\pgfpathlineto{\pgfqpoint{1.961483in}{0.573797in}}%
\pgfpathlineto{\pgfqpoint{1.964056in}{0.579936in}}%
\pgfpathlineto{\pgfqpoint{1.967487in}{0.597524in}}%
\pgfpathlineto{\pgfqpoint{1.971775in}{0.634016in}}%
\pgfpathlineto{\pgfqpoint{1.977779in}{0.709715in}}%
\pgfpathlineto{\pgfqpoint{1.985498in}{0.841136in}}%
\pgfpathlineto{\pgfqpoint{2.000078in}{1.141159in}}%
\pgfpathlineto{\pgfqpoint{2.012085in}{1.367192in}}%
\pgfpathlineto{\pgfqpoint{2.018946in}{1.461077in}}%
\pgfpathlineto{\pgfqpoint{2.024092in}{1.507631in}}%
\pgfpathlineto{\pgfqpoint{2.028381in}{1.528775in}}%
\pgfpathlineto{\pgfqpoint{2.030954in}{1.533384in}}%
\pgfpathlineto{\pgfqpoint{2.032669in}{1.533042in}}%
\pgfpathlineto{\pgfqpoint{2.034384in}{1.529969in}}%
\pgfpathlineto{\pgfqpoint{2.036957in}{1.520280in}}%
\pgfpathlineto{\pgfqpoint{2.040388in}{1.498094in}}%
\pgfpathlineto{\pgfqpoint{2.045534in}{1.446069in}}%
\pgfpathlineto{\pgfqpoint{2.051537in}{1.360277in}}%
\pgfpathlineto{\pgfqpoint{2.060114in}{1.202576in}}%
\pgfpathlineto{\pgfqpoint{2.085844in}{0.703445in}}%
\pgfpathlineto{\pgfqpoint{2.092705in}{0.621436in}}%
\pgfpathlineto{\pgfqpoint{2.097851in}{0.585286in}}%
\pgfpathlineto{\pgfqpoint{2.101281in}{0.574395in}}%
\pgfpathlineto{\pgfqpoint{2.102997in}{0.573027in}}%
\pgfpathlineto{\pgfqpoint{2.104712in}{0.574395in}}%
\pgfpathlineto{\pgfqpoint{2.107285in}{0.581551in}}%
\pgfpathlineto{\pgfqpoint{2.110716in}{0.600462in}}%
\pgfpathlineto{\pgfqpoint{2.115004in}{0.638509in}}%
\pgfpathlineto{\pgfqpoint{2.121007in}{0.716108in}}%
\pgfpathlineto{\pgfqpoint{2.128726in}{0.849303in}}%
\pgfpathlineto{\pgfqpoint{2.144164in}{1.167731in}}%
\pgfpathlineto{\pgfqpoint{2.155314in}{1.373994in}}%
\pgfpathlineto{\pgfqpoint{2.162175in}{1.465792in}}%
\pgfpathlineto{\pgfqpoint{2.167321in}{1.510488in}}%
\pgfpathlineto{\pgfqpoint{2.171609in}{1.529969in}}%
\pgfpathlineto{\pgfqpoint{2.174182in}{1.533555in}}%
\pgfpathlineto{\pgfqpoint{2.175897in}{1.532529in}}%
\pgfpathlineto{\pgfqpoint{2.178470in}{1.525881in}}%
\pgfpathlineto{\pgfqpoint{2.181901in}{1.507631in}}%
\pgfpathlineto{\pgfqpoint{2.186189in}{1.470360in}}%
\pgfpathlineto{\pgfqpoint{2.192193in}{1.393707in}}%
\pgfpathlineto{\pgfqpoint{2.199912in}{1.261394in}}%
\pgfpathlineto{\pgfqpoint{2.214492in}{0.960994in}}%
\pgfpathlineto{\pgfqpoint{2.225641in}{0.749825in}}%
\pgfpathlineto{\pgfqpoint{2.233360in}{0.643151in}}%
\pgfpathlineto{\pgfqpoint{2.238506in}{0.597524in}}%
\pgfpathlineto{\pgfqpoint{2.242795in}{0.577211in}}%
\pgfpathlineto{\pgfqpoint{2.245367in}{0.573113in}}%
\pgfpathlineto{\pgfqpoint{2.247083in}{0.573797in}}%
\pgfpathlineto{\pgfqpoint{2.249656in}{0.579936in}}%
\pgfpathlineto{\pgfqpoint{2.253086in}{0.597524in}}%
\pgfpathlineto{\pgfqpoint{2.257375in}{0.634016in}}%
\pgfpathlineto{\pgfqpoint{2.263378in}{0.709715in}}%
\pgfpathlineto{\pgfqpoint{2.271097in}{0.841136in}}%
\pgfpathlineto{\pgfqpoint{2.285677in}{1.141159in}}%
\pgfpathlineto{\pgfqpoint{2.297684in}{1.367192in}}%
\pgfpathlineto{\pgfqpoint{2.304546in}{1.461077in}}%
\pgfpathlineto{\pgfqpoint{2.309692in}{1.507631in}}%
\pgfpathlineto{\pgfqpoint{2.313980in}{1.528775in}}%
\pgfpathlineto{\pgfqpoint{2.316553in}{1.533384in}}%
\pgfpathlineto{\pgfqpoint{2.318268in}{1.533042in}}%
\pgfpathlineto{\pgfqpoint{2.319984in}{1.529969in}}%
\pgfpathlineto{\pgfqpoint{2.322557in}{1.520280in}}%
\pgfpathlineto{\pgfqpoint{2.325987in}{1.498094in}}%
\pgfpathlineto{\pgfqpoint{2.331133in}{1.446069in}}%
\pgfpathlineto{\pgfqpoint{2.337137in}{1.360277in}}%
\pgfpathlineto{\pgfqpoint{2.345713in}{1.202576in}}%
\pgfpathlineto{\pgfqpoint{2.371443in}{0.703445in}}%
\pgfpathlineto{\pgfqpoint{2.378304in}{0.621436in}}%
\pgfpathlineto{\pgfqpoint{2.383450in}{0.585286in}}%
\pgfpathlineto{\pgfqpoint{2.386881in}{0.574395in}}%
\pgfpathlineto{\pgfqpoint{2.388596in}{0.573027in}}%
\pgfpathlineto{\pgfqpoint{2.388596in}{0.573027in}}%
\pgfusepath{stroke}%
\end{pgfscope}%
\begin{pgfscope}%
\pgfpathrectangle{\pgfqpoint{0.675000in}{0.525000in}}{\pgfqpoint{3.600000in}{1.056604in}} %
\pgfusepath{clip}%
\pgfsetrectcap%
\pgfsetroundjoin%
\pgfsetlinewidth{0.501875pt}%
\definecolor{currentstroke}{rgb}{0.000000,0.000000,1.000000}%
\pgfsetstrokecolor{currentstroke}%
\pgfsetdash{}{0pt}%
\pgfpathmoveto{\pgfqpoint{2.388596in}{0.573027in}}%
\pgfpathlineto{\pgfqpoint{2.392884in}{0.573977in}}%
\pgfpathlineto{\pgfqpoint{2.398030in}{0.577618in}}%
\pgfpathlineto{\pgfqpoint{2.404034in}{0.585286in}}%
\pgfpathlineto{\pgfqpoint{2.410895in}{0.598485in}}%
\pgfpathlineto{\pgfqpoint{2.418614in}{0.618827in}}%
\pgfpathlineto{\pgfqpoint{2.427190in}{0.647938in}}%
\pgfpathlineto{\pgfqpoint{2.437482in}{0.691280in}}%
\pgfpathlineto{\pgfqpoint{2.449490in}{0.752172in}}%
\pgfpathlineto{\pgfqpoint{2.464070in}{0.838430in}}%
\pgfpathlineto{\pgfqpoint{2.483796in}{0.969903in}}%
\pgfpathlineto{\pgfqpoint{2.531824in}{1.296050in}}%
\pgfpathlineto{\pgfqpoint{2.547262in}{1.382886in}}%
\pgfpathlineto{\pgfqpoint{2.559269in}{1.438992in}}%
\pgfpathlineto{\pgfqpoint{2.569561in}{1.477642in}}%
\pgfpathlineto{\pgfqpoint{2.578996in}{1.504612in}}%
\pgfpathlineto{\pgfqpoint{2.586714in}{1.520280in}}%
\pgfpathlineto{\pgfqpoint{2.593576in}{1.529192in}}%
\pgfpathlineto{\pgfqpoint{2.598722in}{1.532719in}}%
\pgfpathlineto{\pgfqpoint{2.603868in}{1.533517in}}%
\pgfpathlineto{\pgfqpoint{2.608156in}{1.532093in}}%
\pgfpathlineto{\pgfqpoint{2.613302in}{1.527886in}}%
\pgfpathlineto{\pgfqpoint{2.619305in}{1.519565in}}%
\pgfpathlineto{\pgfqpoint{2.626167in}{1.505636in}}%
\pgfpathlineto{\pgfqpoint{2.633886in}{1.484505in}}%
\pgfpathlineto{\pgfqpoint{2.643320in}{1.451214in}}%
\pgfpathlineto{\pgfqpoint{2.653612in}{1.406247in}}%
\pgfpathlineto{\pgfqpoint{2.665619in}{1.343724in}}%
\pgfpathlineto{\pgfqpoint{2.681057in}{1.250439in}}%
\pgfpathlineto{\pgfqpoint{2.702498in}{1.105306in}}%
\pgfpathlineto{\pgfqpoint{2.742808in}{0.830364in}}%
\pgfpathlineto{\pgfqpoint{2.759103in}{0.735997in}}%
\pgfpathlineto{\pgfqpoint{2.771968in}{0.674005in}}%
\pgfpathlineto{\pgfqpoint{2.783118in}{0.631102in}}%
\pgfpathlineto{\pgfqpoint{2.792552in}{0.603561in}}%
\pgfpathlineto{\pgfqpoint{2.800271in}{0.587403in}}%
\pgfpathlineto{\pgfqpoint{2.807132in}{0.578044in}}%
\pgfpathlineto{\pgfqpoint{2.812278in}{0.574176in}}%
\pgfpathlineto{\pgfqpoint{2.817424in}{0.573037in}}%
\pgfpathlineto{\pgfqpoint{2.821712in}{0.574176in}}%
\pgfpathlineto{\pgfqpoint{2.826858in}{0.578044in}}%
\pgfpathlineto{\pgfqpoint{2.832862in}{0.585973in}}%
\pgfpathlineto{\pgfqpoint{2.839723in}{0.599465in}}%
\pgfpathlineto{\pgfqpoint{2.847442in}{0.620123in}}%
\pgfpathlineto{\pgfqpoint{2.856018in}{0.649567in}}%
\pgfpathlineto{\pgfqpoint{2.866310in}{0.693273in}}%
\pgfpathlineto{\pgfqpoint{2.878317in}{0.754531in}}%
\pgfpathlineto{\pgfqpoint{2.892898in}{0.841136in}}%
\pgfpathlineto{\pgfqpoint{2.912624in}{0.972880in}}%
\pgfpathlineto{\pgfqpoint{2.959795in}{1.293439in}}%
\pgfpathlineto{\pgfqpoint{2.975232in}{1.380683in}}%
\pgfpathlineto{\pgfqpoint{2.987240in}{1.437184in}}%
\pgfpathlineto{\pgfqpoint{2.997532in}{1.476219in}}%
\pgfpathlineto{\pgfqpoint{3.006966in}{1.503570in}}%
\pgfpathlineto{\pgfqpoint{3.014685in}{1.519565in}}%
\pgfpathlineto{\pgfqpoint{3.021546in}{1.528775in}}%
\pgfpathlineto{\pgfqpoint{3.026692in}{1.532529in}}%
\pgfpathlineto{\pgfqpoint{3.031838in}{1.533555in}}%
\pgfpathlineto{\pgfqpoint{3.036126in}{1.532321in}}%
\pgfpathlineto{\pgfqpoint{3.041272in}{1.528340in}}%
\pgfpathlineto{\pgfqpoint{3.047276in}{1.520280in}}%
\pgfpathlineto{\pgfqpoint{3.054137in}{1.506643in}}%
\pgfpathlineto{\pgfqpoint{3.061856in}{1.485827in}}%
\pgfpathlineto{\pgfqpoint{3.070432in}{1.456217in}}%
\pgfpathlineto{\pgfqpoint{3.080724in}{1.412330in}}%
\pgfpathlineto{\pgfqpoint{3.092731in}{1.350889in}}%
\pgfpathlineto{\pgfqpoint{3.107311in}{1.264112in}}%
\pgfpathlineto{\pgfqpoint{3.127038in}{1.132235in}}%
\pgfpathlineto{\pgfqpoint{3.174209in}{0.811858in}}%
\pgfpathlineto{\pgfqpoint{3.189646in}{0.724818in}}%
\pgfpathlineto{\pgfqpoint{3.201654in}{0.668514in}}%
\pgfpathlineto{\pgfqpoint{3.211945in}{0.629671in}}%
\pgfpathlineto{\pgfqpoint{3.221380in}{0.602510in}}%
\pgfpathlineto{\pgfqpoint{3.229099in}{0.586679in}}%
\pgfpathlineto{\pgfqpoint{3.235960in}{0.577618in}}%
\pgfpathlineto{\pgfqpoint{3.241106in}{0.573977in}}%
\pgfpathlineto{\pgfqpoint{3.245394in}{0.573027in}}%
\pgfpathlineto{\pgfqpoint{3.245394in}{0.573027in}}%
\pgfusepath{stroke}%
\end{pgfscope}%
\begin{pgfscope}%
\pgfpathrectangle{\pgfqpoint{0.675000in}{0.525000in}}{\pgfqpoint{3.600000in}{1.056604in}} %
\pgfusepath{clip}%
\pgfsetrectcap%
\pgfsetroundjoin%
\pgfsetlinewidth{0.501875pt}%
\definecolor{currentstroke}{rgb}{1.000000,0.000000,1.000000}%
\pgfsetstrokecolor{currentstroke}%
\pgfsetdash{}{0pt}%
\pgfpathmoveto{\pgfqpoint{3.245394in}{0.573027in}}%
\pgfpathlineto{\pgfqpoint{3.247109in}{0.574395in}}%
\pgfpathlineto{\pgfqpoint{3.249682in}{0.581551in}}%
\pgfpathlineto{\pgfqpoint{3.253113in}{0.600462in}}%
\pgfpathlineto{\pgfqpoint{3.257401in}{0.638509in}}%
\pgfpathlineto{\pgfqpoint{3.263405in}{0.716108in}}%
\pgfpathlineto{\pgfqpoint{3.271124in}{0.849303in}}%
\pgfpathlineto{\pgfqpoint{3.286561in}{1.167731in}}%
\pgfpathlineto{\pgfqpoint{3.297711in}{1.373994in}}%
\pgfpathlineto{\pgfqpoint{3.304572in}{1.465792in}}%
\pgfpathlineto{\pgfqpoint{3.309718in}{1.510488in}}%
\pgfpathlineto{\pgfqpoint{3.314006in}{1.529969in}}%
\pgfpathlineto{\pgfqpoint{3.316579in}{1.533555in}}%
\pgfpathlineto{\pgfqpoint{3.318295in}{1.532529in}}%
\pgfpathlineto{\pgfqpoint{3.320868in}{1.525881in}}%
\pgfpathlineto{\pgfqpoint{3.324298in}{1.507631in}}%
\pgfpathlineto{\pgfqpoint{3.328587in}{1.470360in}}%
\pgfpathlineto{\pgfqpoint{3.334590in}{1.393707in}}%
\pgfpathlineto{\pgfqpoint{3.342309in}{1.261394in}}%
\pgfpathlineto{\pgfqpoint{3.356889in}{0.960994in}}%
\pgfpathlineto{\pgfqpoint{3.368039in}{0.749825in}}%
\pgfpathlineto{\pgfqpoint{3.375758in}{0.643151in}}%
\pgfpathlineto{\pgfqpoint{3.380904in}{0.597524in}}%
\pgfpathlineto{\pgfqpoint{3.385192in}{0.577211in}}%
\pgfpathlineto{\pgfqpoint{3.387765in}{0.573113in}}%
\pgfpathlineto{\pgfqpoint{3.389480in}{0.573797in}}%
\pgfpathlineto{\pgfqpoint{3.392053in}{0.579936in}}%
\pgfpathlineto{\pgfqpoint{3.395484in}{0.597524in}}%
\pgfpathlineto{\pgfqpoint{3.399772in}{0.634016in}}%
\pgfpathlineto{\pgfqpoint{3.405776in}{0.709715in}}%
\pgfpathlineto{\pgfqpoint{3.413494in}{0.841136in}}%
\pgfpathlineto{\pgfqpoint{3.428075in}{1.141159in}}%
\pgfpathlineto{\pgfqpoint{3.440082in}{1.367192in}}%
\pgfpathlineto{\pgfqpoint{3.446943in}{1.461077in}}%
\pgfpathlineto{\pgfqpoint{3.452089in}{1.507631in}}%
\pgfpathlineto{\pgfqpoint{3.456377in}{1.528775in}}%
\pgfpathlineto{\pgfqpoint{3.458950in}{1.533384in}}%
\pgfpathlineto{\pgfqpoint{3.460666in}{1.533042in}}%
\pgfpathlineto{\pgfqpoint{3.462381in}{1.529969in}}%
\pgfpathlineto{\pgfqpoint{3.464954in}{1.520280in}}%
\pgfpathlineto{\pgfqpoint{3.468384in}{1.498094in}}%
\pgfpathlineto{\pgfqpoint{3.473530in}{1.446069in}}%
\pgfpathlineto{\pgfqpoint{3.479534in}{1.360277in}}%
\pgfpathlineto{\pgfqpoint{3.488111in}{1.202576in}}%
\pgfpathlineto{\pgfqpoint{3.513840in}{0.703445in}}%
\pgfpathlineto{\pgfqpoint{3.520701in}{0.621436in}}%
\pgfpathlineto{\pgfqpoint{3.525847in}{0.585286in}}%
\pgfpathlineto{\pgfqpoint{3.529278in}{0.574395in}}%
\pgfpathlineto{\pgfqpoint{3.530993in}{0.573027in}}%
\pgfpathlineto{\pgfqpoint{3.532709in}{0.574395in}}%
\pgfpathlineto{\pgfqpoint{3.535282in}{0.581551in}}%
\pgfpathlineto{\pgfqpoint{3.538712in}{0.600462in}}%
\pgfpathlineto{\pgfqpoint{3.543001in}{0.638509in}}%
\pgfpathlineto{\pgfqpoint{3.549004in}{0.716108in}}%
\pgfpathlineto{\pgfqpoint{3.556723in}{0.849303in}}%
\pgfpathlineto{\pgfqpoint{3.572161in}{1.167731in}}%
\pgfpathlineto{\pgfqpoint{3.583310in}{1.373994in}}%
\pgfpathlineto{\pgfqpoint{3.590172in}{1.465792in}}%
\pgfpathlineto{\pgfqpoint{3.595317in}{1.510488in}}%
\pgfpathlineto{\pgfqpoint{3.599606in}{1.529969in}}%
\pgfpathlineto{\pgfqpoint{3.602179in}{1.533555in}}%
\pgfpathlineto{\pgfqpoint{3.603894in}{1.532529in}}%
\pgfpathlineto{\pgfqpoint{3.606467in}{1.525881in}}%
\pgfpathlineto{\pgfqpoint{3.609898in}{1.507631in}}%
\pgfpathlineto{\pgfqpoint{3.614186in}{1.470360in}}%
\pgfpathlineto{\pgfqpoint{3.620190in}{1.393707in}}%
\pgfpathlineto{\pgfqpoint{3.627908in}{1.261394in}}%
\pgfpathlineto{\pgfqpoint{3.642489in}{0.960994in}}%
\pgfpathlineto{\pgfqpoint{3.653638in}{0.749825in}}%
\pgfpathlineto{\pgfqpoint{3.661357in}{0.643151in}}%
\pgfpathlineto{\pgfqpoint{3.666503in}{0.597524in}}%
\pgfpathlineto{\pgfqpoint{3.670791in}{0.577211in}}%
\pgfpathlineto{\pgfqpoint{3.673364in}{0.573113in}}%
\pgfpathlineto{\pgfqpoint{3.675079in}{0.573797in}}%
\pgfpathlineto{\pgfqpoint{3.677652in}{0.579936in}}%
\pgfpathlineto{\pgfqpoint{3.681083in}{0.597524in}}%
\pgfpathlineto{\pgfqpoint{3.685371in}{0.634016in}}%
\pgfpathlineto{\pgfqpoint{3.691375in}{0.709715in}}%
\pgfpathlineto{\pgfqpoint{3.699094in}{0.841136in}}%
\pgfpathlineto{\pgfqpoint{3.713674in}{1.141159in}}%
\pgfpathlineto{\pgfqpoint{3.725681in}{1.367192in}}%
\pgfpathlineto{\pgfqpoint{3.732542in}{1.461077in}}%
\pgfpathlineto{\pgfqpoint{3.737688in}{1.507631in}}%
\pgfpathlineto{\pgfqpoint{3.741977in}{1.528775in}}%
\pgfpathlineto{\pgfqpoint{3.744550in}{1.533384in}}%
\pgfpathlineto{\pgfqpoint{3.746265in}{1.533042in}}%
\pgfpathlineto{\pgfqpoint{3.747980in}{1.529969in}}%
\pgfpathlineto{\pgfqpoint{3.750553in}{1.520280in}}%
\pgfpathlineto{\pgfqpoint{3.753984in}{1.498094in}}%
\pgfpathlineto{\pgfqpoint{3.759130in}{1.446069in}}%
\pgfpathlineto{\pgfqpoint{3.765133in}{1.360277in}}%
\pgfpathlineto{\pgfqpoint{3.773710in}{1.202576in}}%
\pgfpathlineto{\pgfqpoint{3.799440in}{0.703445in}}%
\pgfpathlineto{\pgfqpoint{3.806301in}{0.621436in}}%
\pgfpathlineto{\pgfqpoint{3.811447in}{0.585286in}}%
\pgfpathlineto{\pgfqpoint{3.814877in}{0.574395in}}%
\pgfpathlineto{\pgfqpoint{3.816593in}{0.573027in}}%
\pgfpathlineto{\pgfqpoint{3.818308in}{0.574395in}}%
\pgfpathlineto{\pgfqpoint{3.820881in}{0.581551in}}%
\pgfpathlineto{\pgfqpoint{3.824312in}{0.600462in}}%
\pgfpathlineto{\pgfqpoint{3.828600in}{0.638509in}}%
\pgfpathlineto{\pgfqpoint{3.834603in}{0.716108in}}%
\pgfpathlineto{\pgfqpoint{3.842322in}{0.849303in}}%
\pgfpathlineto{\pgfqpoint{3.857760in}{1.167731in}}%
\pgfpathlineto{\pgfqpoint{3.868910in}{1.373994in}}%
\pgfpathlineto{\pgfqpoint{3.875771in}{1.465792in}}%
\pgfpathlineto{\pgfqpoint{3.880917in}{1.510488in}}%
\pgfpathlineto{\pgfqpoint{3.885205in}{1.529969in}}%
\pgfpathlineto{\pgfqpoint{3.887778in}{1.533555in}}%
\pgfpathlineto{\pgfqpoint{3.889493in}{1.532529in}}%
\pgfpathlineto{\pgfqpoint{3.892066in}{1.525881in}}%
\pgfpathlineto{\pgfqpoint{3.895497in}{1.507631in}}%
\pgfpathlineto{\pgfqpoint{3.899785in}{1.470360in}}%
\pgfpathlineto{\pgfqpoint{3.905789in}{1.393707in}}%
\pgfpathlineto{\pgfqpoint{3.913508in}{1.261394in}}%
\pgfpathlineto{\pgfqpoint{3.928088in}{0.960994in}}%
\pgfpathlineto{\pgfqpoint{3.939237in}{0.749825in}}%
\pgfpathlineto{\pgfqpoint{3.946956in}{0.643151in}}%
\pgfpathlineto{\pgfqpoint{3.952102in}{0.597524in}}%
\pgfpathlineto{\pgfqpoint{3.956391in}{0.577211in}}%
\pgfpathlineto{\pgfqpoint{3.958963in}{0.573113in}}%
\pgfpathlineto{\pgfqpoint{3.960679in}{0.573797in}}%
\pgfpathlineto{\pgfqpoint{3.963252in}{0.579936in}}%
\pgfpathlineto{\pgfqpoint{3.966682in}{0.597524in}}%
\pgfpathlineto{\pgfqpoint{3.970971in}{0.634016in}}%
\pgfpathlineto{\pgfqpoint{3.976974in}{0.709715in}}%
\pgfpathlineto{\pgfqpoint{3.984693in}{0.841136in}}%
\pgfpathlineto{\pgfqpoint{3.999273in}{1.141159in}}%
\pgfpathlineto{\pgfqpoint{4.011280in}{1.367192in}}%
\pgfpathlineto{\pgfqpoint{4.018142in}{1.461077in}}%
\pgfpathlineto{\pgfqpoint{4.023288in}{1.507631in}}%
\pgfpathlineto{\pgfqpoint{4.027576in}{1.528775in}}%
\pgfpathlineto{\pgfqpoint{4.030149in}{1.533384in}}%
\pgfpathlineto{\pgfqpoint{4.031864in}{1.533042in}}%
\pgfpathlineto{\pgfqpoint{4.033580in}{1.529969in}}%
\pgfpathlineto{\pgfqpoint{4.036153in}{1.520280in}}%
\pgfpathlineto{\pgfqpoint{4.039583in}{1.498094in}}%
\pgfpathlineto{\pgfqpoint{4.044729in}{1.446069in}}%
\pgfpathlineto{\pgfqpoint{4.050733in}{1.360277in}}%
\pgfpathlineto{\pgfqpoint{4.059309in}{1.202576in}}%
\pgfpathlineto{\pgfqpoint{4.085039in}{0.703445in}}%
\pgfpathlineto{\pgfqpoint{4.091900in}{0.621436in}}%
\pgfpathlineto{\pgfqpoint{4.097046in}{0.585286in}}%
\pgfpathlineto{\pgfqpoint{4.100477in}{0.574395in}}%
\pgfpathlineto{\pgfqpoint{4.102192in}{0.573027in}}%
\pgfpathlineto{\pgfqpoint{4.102192in}{0.573027in}}%
\pgfusepath{stroke}%
\end{pgfscope}%
\begin{pgfscope}%
\pgfsetrectcap%
\pgfsetmiterjoin%
\pgfsetlinewidth{0.501875pt}%
\definecolor{currentstroke}{rgb}{0.000000,0.000000,0.000000}%
\pgfsetstrokecolor{currentstroke}%
\pgfsetdash{}{0pt}%
\pgfpathmoveto{\pgfqpoint{0.675000in}{1.581604in}}%
\pgfpathlineto{\pgfqpoint{4.275000in}{1.581604in}}%
\pgfusepath{stroke}%
\end{pgfscope}%
\begin{pgfscope}%
\pgfsetrectcap%
\pgfsetmiterjoin%
\pgfsetlinewidth{0.501875pt}%
\definecolor{currentstroke}{rgb}{0.000000,0.000000,0.000000}%
\pgfsetstrokecolor{currentstroke}%
\pgfsetdash{}{0pt}%
\pgfpathmoveto{\pgfqpoint{0.675000in}{0.525000in}}%
\pgfpathlineto{\pgfqpoint{0.675000in}{1.581604in}}%
\pgfusepath{stroke}%
\end{pgfscope}%
\begin{pgfscope}%
\pgfsetrectcap%
\pgfsetmiterjoin%
\pgfsetlinewidth{0.501875pt}%
\definecolor{currentstroke}{rgb}{0.000000,0.000000,0.000000}%
\pgfsetstrokecolor{currentstroke}%
\pgfsetdash{}{0pt}%
\pgfpathmoveto{\pgfqpoint{4.275000in}{0.525000in}}%
\pgfpathlineto{\pgfqpoint{4.275000in}{1.581604in}}%
\pgfusepath{stroke}%
\end{pgfscope}%
\begin{pgfscope}%
\pgfsetrectcap%
\pgfsetmiterjoin%
\pgfsetlinewidth{0.501875pt}%
\definecolor{currentstroke}{rgb}{0.000000,0.000000,0.000000}%
\pgfsetstrokecolor{currentstroke}%
\pgfsetdash{}{0pt}%
\pgfpathmoveto{\pgfqpoint{0.675000in}{0.525000in}}%
\pgfpathlineto{\pgfqpoint{4.275000in}{0.525000in}}%
\pgfusepath{stroke}%
\end{pgfscope}%
\begin{pgfscope}%
\pgfsetbuttcap%
\pgfsetroundjoin%
\definecolor{currentfill}{rgb}{0.000000,0.000000,0.000000}%
\pgfsetfillcolor{currentfill}%
\pgfsetlinewidth{0.501875pt}%
\definecolor{currentstroke}{rgb}{0.000000,0.000000,0.000000}%
\pgfsetstrokecolor{currentstroke}%
\pgfsetdash{}{0pt}%
\pgfsys@defobject{currentmarker}{\pgfqpoint{0.000000in}{0.000000in}}{\pgfqpoint{0.000000in}{0.055556in}}{%
\pgfpathmoveto{\pgfqpoint{0.000000in}{0.000000in}}%
\pgfpathlineto{\pgfqpoint{0.000000in}{0.055556in}}%
\pgfusepath{stroke,fill}%
}%
\begin{pgfscope}%
\pgfsys@transformshift{0.675000in}{0.525000in}%
\pgfsys@useobject{currentmarker}{}%
\end{pgfscope}%
\end{pgfscope}%
\begin{pgfscope}%
\pgfsetbuttcap%
\pgfsetroundjoin%
\definecolor{currentfill}{rgb}{0.000000,0.000000,0.000000}%
\pgfsetfillcolor{currentfill}%
\pgfsetlinewidth{0.501875pt}%
\definecolor{currentstroke}{rgb}{0.000000,0.000000,0.000000}%
\pgfsetstrokecolor{currentstroke}%
\pgfsetdash{}{0pt}%
\pgfsys@defobject{currentmarker}{\pgfqpoint{0.000000in}{-0.055556in}}{\pgfqpoint{0.000000in}{0.000000in}}{%
\pgfpathmoveto{\pgfqpoint{0.000000in}{0.000000in}}%
\pgfpathlineto{\pgfqpoint{0.000000in}{-0.055556in}}%
\pgfusepath{stroke,fill}%
}%
\begin{pgfscope}%
\pgfsys@transformshift{0.675000in}{1.581604in}%
\pgfsys@useobject{currentmarker}{}%
\end{pgfscope}%
\end{pgfscope}%
\begin{pgfscope}%
\pgftext[x=0.675000in,y=0.469444in,,top]{\rmfamily\fontsize{9.000000}{10.800000}\selectfont \(\displaystyle 0.0000\)}%
\end{pgfscope}%
\begin{pgfscope}%
\pgfsetbuttcap%
\pgfsetroundjoin%
\definecolor{currentfill}{rgb}{0.000000,0.000000,0.000000}%
\pgfsetfillcolor{currentfill}%
\pgfsetlinewidth{0.501875pt}%
\definecolor{currentstroke}{rgb}{0.000000,0.000000,0.000000}%
\pgfsetstrokecolor{currentstroke}%
\pgfsetdash{}{0pt}%
\pgfsys@defobject{currentmarker}{\pgfqpoint{0.000000in}{0.000000in}}{\pgfqpoint{0.000000in}{0.055556in}}{%
\pgfpathmoveto{\pgfqpoint{0.000000in}{0.000000in}}%
\pgfpathlineto{\pgfqpoint{0.000000in}{0.055556in}}%
\pgfusepath{stroke,fill}%
}%
\begin{pgfscope}%
\pgfsys@transformshift{1.125000in}{0.525000in}%
\pgfsys@useobject{currentmarker}{}%
\end{pgfscope}%
\end{pgfscope}%
\begin{pgfscope}%
\pgfsetbuttcap%
\pgfsetroundjoin%
\definecolor{currentfill}{rgb}{0.000000,0.000000,0.000000}%
\pgfsetfillcolor{currentfill}%
\pgfsetlinewidth{0.501875pt}%
\definecolor{currentstroke}{rgb}{0.000000,0.000000,0.000000}%
\pgfsetstrokecolor{currentstroke}%
\pgfsetdash{}{0pt}%
\pgfsys@defobject{currentmarker}{\pgfqpoint{0.000000in}{-0.055556in}}{\pgfqpoint{0.000000in}{0.000000in}}{%
\pgfpathmoveto{\pgfqpoint{0.000000in}{0.000000in}}%
\pgfpathlineto{\pgfqpoint{0.000000in}{-0.055556in}}%
\pgfusepath{stroke,fill}%
}%
\begin{pgfscope}%
\pgfsys@transformshift{1.125000in}{1.581604in}%
\pgfsys@useobject{currentmarker}{}%
\end{pgfscope}%
\end{pgfscope}%
\begin{pgfscope}%
\pgftext[x=1.125000in,y=0.469444in,,top]{\rmfamily\fontsize{9.000000}{10.800000}\selectfont \(\displaystyle 0.0002\)}%
\end{pgfscope}%
\begin{pgfscope}%
\pgfsetbuttcap%
\pgfsetroundjoin%
\definecolor{currentfill}{rgb}{0.000000,0.000000,0.000000}%
\pgfsetfillcolor{currentfill}%
\pgfsetlinewidth{0.501875pt}%
\definecolor{currentstroke}{rgb}{0.000000,0.000000,0.000000}%
\pgfsetstrokecolor{currentstroke}%
\pgfsetdash{}{0pt}%
\pgfsys@defobject{currentmarker}{\pgfqpoint{0.000000in}{0.000000in}}{\pgfqpoint{0.000000in}{0.055556in}}{%
\pgfpathmoveto{\pgfqpoint{0.000000in}{0.000000in}}%
\pgfpathlineto{\pgfqpoint{0.000000in}{0.055556in}}%
\pgfusepath{stroke,fill}%
}%
\begin{pgfscope}%
\pgfsys@transformshift{1.575000in}{0.525000in}%
\pgfsys@useobject{currentmarker}{}%
\end{pgfscope}%
\end{pgfscope}%
\begin{pgfscope}%
\pgfsetbuttcap%
\pgfsetroundjoin%
\definecolor{currentfill}{rgb}{0.000000,0.000000,0.000000}%
\pgfsetfillcolor{currentfill}%
\pgfsetlinewidth{0.501875pt}%
\definecolor{currentstroke}{rgb}{0.000000,0.000000,0.000000}%
\pgfsetstrokecolor{currentstroke}%
\pgfsetdash{}{0pt}%
\pgfsys@defobject{currentmarker}{\pgfqpoint{0.000000in}{-0.055556in}}{\pgfqpoint{0.000000in}{0.000000in}}{%
\pgfpathmoveto{\pgfqpoint{0.000000in}{0.000000in}}%
\pgfpathlineto{\pgfqpoint{0.000000in}{-0.055556in}}%
\pgfusepath{stroke,fill}%
}%
\begin{pgfscope}%
\pgfsys@transformshift{1.575000in}{1.581604in}%
\pgfsys@useobject{currentmarker}{}%
\end{pgfscope}%
\end{pgfscope}%
\begin{pgfscope}%
\pgftext[x=1.575000in,y=0.469444in,,top]{\rmfamily\fontsize{9.000000}{10.800000}\selectfont \(\displaystyle 0.0004\)}%
\end{pgfscope}%
\begin{pgfscope}%
\pgfsetbuttcap%
\pgfsetroundjoin%
\definecolor{currentfill}{rgb}{0.000000,0.000000,0.000000}%
\pgfsetfillcolor{currentfill}%
\pgfsetlinewidth{0.501875pt}%
\definecolor{currentstroke}{rgb}{0.000000,0.000000,0.000000}%
\pgfsetstrokecolor{currentstroke}%
\pgfsetdash{}{0pt}%
\pgfsys@defobject{currentmarker}{\pgfqpoint{0.000000in}{0.000000in}}{\pgfqpoint{0.000000in}{0.055556in}}{%
\pgfpathmoveto{\pgfqpoint{0.000000in}{0.000000in}}%
\pgfpathlineto{\pgfqpoint{0.000000in}{0.055556in}}%
\pgfusepath{stroke,fill}%
}%
\begin{pgfscope}%
\pgfsys@transformshift{2.025000in}{0.525000in}%
\pgfsys@useobject{currentmarker}{}%
\end{pgfscope}%
\end{pgfscope}%
\begin{pgfscope}%
\pgfsetbuttcap%
\pgfsetroundjoin%
\definecolor{currentfill}{rgb}{0.000000,0.000000,0.000000}%
\pgfsetfillcolor{currentfill}%
\pgfsetlinewidth{0.501875pt}%
\definecolor{currentstroke}{rgb}{0.000000,0.000000,0.000000}%
\pgfsetstrokecolor{currentstroke}%
\pgfsetdash{}{0pt}%
\pgfsys@defobject{currentmarker}{\pgfqpoint{0.000000in}{-0.055556in}}{\pgfqpoint{0.000000in}{0.000000in}}{%
\pgfpathmoveto{\pgfqpoint{0.000000in}{0.000000in}}%
\pgfpathlineto{\pgfqpoint{0.000000in}{-0.055556in}}%
\pgfusepath{stroke,fill}%
}%
\begin{pgfscope}%
\pgfsys@transformshift{2.025000in}{1.581604in}%
\pgfsys@useobject{currentmarker}{}%
\end{pgfscope}%
\end{pgfscope}%
\begin{pgfscope}%
\pgftext[x=2.025000in,y=0.469444in,,top]{\rmfamily\fontsize{9.000000}{10.800000}\selectfont \(\displaystyle 0.0006\)}%
\end{pgfscope}%
\begin{pgfscope}%
\pgfsetbuttcap%
\pgfsetroundjoin%
\definecolor{currentfill}{rgb}{0.000000,0.000000,0.000000}%
\pgfsetfillcolor{currentfill}%
\pgfsetlinewidth{0.501875pt}%
\definecolor{currentstroke}{rgb}{0.000000,0.000000,0.000000}%
\pgfsetstrokecolor{currentstroke}%
\pgfsetdash{}{0pt}%
\pgfsys@defobject{currentmarker}{\pgfqpoint{0.000000in}{0.000000in}}{\pgfqpoint{0.000000in}{0.055556in}}{%
\pgfpathmoveto{\pgfqpoint{0.000000in}{0.000000in}}%
\pgfpathlineto{\pgfqpoint{0.000000in}{0.055556in}}%
\pgfusepath{stroke,fill}%
}%
\begin{pgfscope}%
\pgfsys@transformshift{2.475000in}{0.525000in}%
\pgfsys@useobject{currentmarker}{}%
\end{pgfscope}%
\end{pgfscope}%
\begin{pgfscope}%
\pgfsetbuttcap%
\pgfsetroundjoin%
\definecolor{currentfill}{rgb}{0.000000,0.000000,0.000000}%
\pgfsetfillcolor{currentfill}%
\pgfsetlinewidth{0.501875pt}%
\definecolor{currentstroke}{rgb}{0.000000,0.000000,0.000000}%
\pgfsetstrokecolor{currentstroke}%
\pgfsetdash{}{0pt}%
\pgfsys@defobject{currentmarker}{\pgfqpoint{0.000000in}{-0.055556in}}{\pgfqpoint{0.000000in}{0.000000in}}{%
\pgfpathmoveto{\pgfqpoint{0.000000in}{0.000000in}}%
\pgfpathlineto{\pgfqpoint{0.000000in}{-0.055556in}}%
\pgfusepath{stroke,fill}%
}%
\begin{pgfscope}%
\pgfsys@transformshift{2.475000in}{1.581604in}%
\pgfsys@useobject{currentmarker}{}%
\end{pgfscope}%
\end{pgfscope}%
\begin{pgfscope}%
\pgftext[x=2.475000in,y=0.469444in,,top]{\rmfamily\fontsize{9.000000}{10.800000}\selectfont \(\displaystyle 0.0008\)}%
\end{pgfscope}%
\begin{pgfscope}%
\pgfsetbuttcap%
\pgfsetroundjoin%
\definecolor{currentfill}{rgb}{0.000000,0.000000,0.000000}%
\pgfsetfillcolor{currentfill}%
\pgfsetlinewidth{0.501875pt}%
\definecolor{currentstroke}{rgb}{0.000000,0.000000,0.000000}%
\pgfsetstrokecolor{currentstroke}%
\pgfsetdash{}{0pt}%
\pgfsys@defobject{currentmarker}{\pgfqpoint{0.000000in}{0.000000in}}{\pgfqpoint{0.000000in}{0.055556in}}{%
\pgfpathmoveto{\pgfqpoint{0.000000in}{0.000000in}}%
\pgfpathlineto{\pgfqpoint{0.000000in}{0.055556in}}%
\pgfusepath{stroke,fill}%
}%
\begin{pgfscope}%
\pgfsys@transformshift{2.925000in}{0.525000in}%
\pgfsys@useobject{currentmarker}{}%
\end{pgfscope}%
\end{pgfscope}%
\begin{pgfscope}%
\pgfsetbuttcap%
\pgfsetroundjoin%
\definecolor{currentfill}{rgb}{0.000000,0.000000,0.000000}%
\pgfsetfillcolor{currentfill}%
\pgfsetlinewidth{0.501875pt}%
\definecolor{currentstroke}{rgb}{0.000000,0.000000,0.000000}%
\pgfsetstrokecolor{currentstroke}%
\pgfsetdash{}{0pt}%
\pgfsys@defobject{currentmarker}{\pgfqpoint{0.000000in}{-0.055556in}}{\pgfqpoint{0.000000in}{0.000000in}}{%
\pgfpathmoveto{\pgfqpoint{0.000000in}{0.000000in}}%
\pgfpathlineto{\pgfqpoint{0.000000in}{-0.055556in}}%
\pgfusepath{stroke,fill}%
}%
\begin{pgfscope}%
\pgfsys@transformshift{2.925000in}{1.581604in}%
\pgfsys@useobject{currentmarker}{}%
\end{pgfscope}%
\end{pgfscope}%
\begin{pgfscope}%
\pgftext[x=2.925000in,y=0.469444in,,top]{\rmfamily\fontsize{9.000000}{10.800000}\selectfont \(\displaystyle 0.0010\)}%
\end{pgfscope}%
\begin{pgfscope}%
\pgfsetbuttcap%
\pgfsetroundjoin%
\definecolor{currentfill}{rgb}{0.000000,0.000000,0.000000}%
\pgfsetfillcolor{currentfill}%
\pgfsetlinewidth{0.501875pt}%
\definecolor{currentstroke}{rgb}{0.000000,0.000000,0.000000}%
\pgfsetstrokecolor{currentstroke}%
\pgfsetdash{}{0pt}%
\pgfsys@defobject{currentmarker}{\pgfqpoint{0.000000in}{0.000000in}}{\pgfqpoint{0.000000in}{0.055556in}}{%
\pgfpathmoveto{\pgfqpoint{0.000000in}{0.000000in}}%
\pgfpathlineto{\pgfqpoint{0.000000in}{0.055556in}}%
\pgfusepath{stroke,fill}%
}%
\begin{pgfscope}%
\pgfsys@transformshift{3.375000in}{0.525000in}%
\pgfsys@useobject{currentmarker}{}%
\end{pgfscope}%
\end{pgfscope}%
\begin{pgfscope}%
\pgfsetbuttcap%
\pgfsetroundjoin%
\definecolor{currentfill}{rgb}{0.000000,0.000000,0.000000}%
\pgfsetfillcolor{currentfill}%
\pgfsetlinewidth{0.501875pt}%
\definecolor{currentstroke}{rgb}{0.000000,0.000000,0.000000}%
\pgfsetstrokecolor{currentstroke}%
\pgfsetdash{}{0pt}%
\pgfsys@defobject{currentmarker}{\pgfqpoint{0.000000in}{-0.055556in}}{\pgfqpoint{0.000000in}{0.000000in}}{%
\pgfpathmoveto{\pgfqpoint{0.000000in}{0.000000in}}%
\pgfpathlineto{\pgfqpoint{0.000000in}{-0.055556in}}%
\pgfusepath{stroke,fill}%
}%
\begin{pgfscope}%
\pgfsys@transformshift{3.375000in}{1.581604in}%
\pgfsys@useobject{currentmarker}{}%
\end{pgfscope}%
\end{pgfscope}%
\begin{pgfscope}%
\pgftext[x=3.375000in,y=0.469444in,,top]{\rmfamily\fontsize{9.000000}{10.800000}\selectfont \(\displaystyle 0.0012\)}%
\end{pgfscope}%
\begin{pgfscope}%
\pgfsetbuttcap%
\pgfsetroundjoin%
\definecolor{currentfill}{rgb}{0.000000,0.000000,0.000000}%
\pgfsetfillcolor{currentfill}%
\pgfsetlinewidth{0.501875pt}%
\definecolor{currentstroke}{rgb}{0.000000,0.000000,0.000000}%
\pgfsetstrokecolor{currentstroke}%
\pgfsetdash{}{0pt}%
\pgfsys@defobject{currentmarker}{\pgfqpoint{0.000000in}{0.000000in}}{\pgfqpoint{0.000000in}{0.055556in}}{%
\pgfpathmoveto{\pgfqpoint{0.000000in}{0.000000in}}%
\pgfpathlineto{\pgfqpoint{0.000000in}{0.055556in}}%
\pgfusepath{stroke,fill}%
}%
\begin{pgfscope}%
\pgfsys@transformshift{3.825000in}{0.525000in}%
\pgfsys@useobject{currentmarker}{}%
\end{pgfscope}%
\end{pgfscope}%
\begin{pgfscope}%
\pgfsetbuttcap%
\pgfsetroundjoin%
\definecolor{currentfill}{rgb}{0.000000,0.000000,0.000000}%
\pgfsetfillcolor{currentfill}%
\pgfsetlinewidth{0.501875pt}%
\definecolor{currentstroke}{rgb}{0.000000,0.000000,0.000000}%
\pgfsetstrokecolor{currentstroke}%
\pgfsetdash{}{0pt}%
\pgfsys@defobject{currentmarker}{\pgfqpoint{0.000000in}{-0.055556in}}{\pgfqpoint{0.000000in}{0.000000in}}{%
\pgfpathmoveto{\pgfqpoint{0.000000in}{0.000000in}}%
\pgfpathlineto{\pgfqpoint{0.000000in}{-0.055556in}}%
\pgfusepath{stroke,fill}%
}%
\begin{pgfscope}%
\pgfsys@transformshift{3.825000in}{1.581604in}%
\pgfsys@useobject{currentmarker}{}%
\end{pgfscope}%
\end{pgfscope}%
\begin{pgfscope}%
\pgftext[x=3.825000in,y=0.469444in,,top]{\rmfamily\fontsize{9.000000}{10.800000}\selectfont \(\displaystyle 0.0014\)}%
\end{pgfscope}%
\begin{pgfscope}%
\pgfsetbuttcap%
\pgfsetroundjoin%
\definecolor{currentfill}{rgb}{0.000000,0.000000,0.000000}%
\pgfsetfillcolor{currentfill}%
\pgfsetlinewidth{0.501875pt}%
\definecolor{currentstroke}{rgb}{0.000000,0.000000,0.000000}%
\pgfsetstrokecolor{currentstroke}%
\pgfsetdash{}{0pt}%
\pgfsys@defobject{currentmarker}{\pgfqpoint{0.000000in}{0.000000in}}{\pgfqpoint{0.000000in}{0.055556in}}{%
\pgfpathmoveto{\pgfqpoint{0.000000in}{0.000000in}}%
\pgfpathlineto{\pgfqpoint{0.000000in}{0.055556in}}%
\pgfusepath{stroke,fill}%
}%
\begin{pgfscope}%
\pgfsys@transformshift{4.275000in}{0.525000in}%
\pgfsys@useobject{currentmarker}{}%
\end{pgfscope}%
\end{pgfscope}%
\begin{pgfscope}%
\pgfsetbuttcap%
\pgfsetroundjoin%
\definecolor{currentfill}{rgb}{0.000000,0.000000,0.000000}%
\pgfsetfillcolor{currentfill}%
\pgfsetlinewidth{0.501875pt}%
\definecolor{currentstroke}{rgb}{0.000000,0.000000,0.000000}%
\pgfsetstrokecolor{currentstroke}%
\pgfsetdash{}{0pt}%
\pgfsys@defobject{currentmarker}{\pgfqpoint{0.000000in}{-0.055556in}}{\pgfqpoint{0.000000in}{0.000000in}}{%
\pgfpathmoveto{\pgfqpoint{0.000000in}{0.000000in}}%
\pgfpathlineto{\pgfqpoint{0.000000in}{-0.055556in}}%
\pgfusepath{stroke,fill}%
}%
\begin{pgfscope}%
\pgfsys@transformshift{4.275000in}{1.581604in}%
\pgfsys@useobject{currentmarker}{}%
\end{pgfscope}%
\end{pgfscope}%
\begin{pgfscope}%
\pgftext[x=4.275000in,y=0.469444in,,top]{\rmfamily\fontsize{9.000000}{10.800000}\selectfont \(\displaystyle 0.0016\)}%
\end{pgfscope}%
\begin{pgfscope}%
\pgftext[x=2.475000in,y=0.279028in,,top]{\rmfamily\fontsize{9.000000}{10.800000}\selectfont Zeit (s)}%
\end{pgfscope}%
\begin{pgfscope}%
\pgfsetbuttcap%
\pgfsetroundjoin%
\definecolor{currentfill}{rgb}{0.000000,0.000000,0.000000}%
\pgfsetfillcolor{currentfill}%
\pgfsetlinewidth{0.501875pt}%
\definecolor{currentstroke}{rgb}{0.000000,0.000000,0.000000}%
\pgfsetstrokecolor{currentstroke}%
\pgfsetdash{}{0pt}%
\pgfsys@defobject{currentmarker}{\pgfqpoint{0.000000in}{0.000000in}}{\pgfqpoint{0.055556in}{0.000000in}}{%
\pgfpathmoveto{\pgfqpoint{0.000000in}{0.000000in}}%
\pgfpathlineto{\pgfqpoint{0.055556in}{0.000000in}}%
\pgfusepath{stroke,fill}%
}%
\begin{pgfscope}%
\pgfsys@transformshift{0.675000in}{0.573027in}%
\pgfsys@useobject{currentmarker}{}%
\end{pgfscope}%
\end{pgfscope}%
\begin{pgfscope}%
\pgfsetbuttcap%
\pgfsetroundjoin%
\definecolor{currentfill}{rgb}{0.000000,0.000000,0.000000}%
\pgfsetfillcolor{currentfill}%
\pgfsetlinewidth{0.501875pt}%
\definecolor{currentstroke}{rgb}{0.000000,0.000000,0.000000}%
\pgfsetstrokecolor{currentstroke}%
\pgfsetdash{}{0pt}%
\pgfsys@defobject{currentmarker}{\pgfqpoint{-0.055556in}{0.000000in}}{\pgfqpoint{0.000000in}{0.000000in}}{%
\pgfpathmoveto{\pgfqpoint{0.000000in}{0.000000in}}%
\pgfpathlineto{\pgfqpoint{-0.055556in}{0.000000in}}%
\pgfusepath{stroke,fill}%
}%
\begin{pgfscope}%
\pgfsys@transformshift{4.275000in}{0.573027in}%
\pgfsys@useobject{currentmarker}{}%
\end{pgfscope}%
\end{pgfscope}%
\begin{pgfscope}%
\pgftext[x=0.619444in,y=0.573027in,right,]{\rmfamily\fontsize{9.000000}{10.800000}\selectfont \(\displaystyle -1.5\)}%
\end{pgfscope}%
\begin{pgfscope}%
\pgfsetbuttcap%
\pgfsetroundjoin%
\definecolor{currentfill}{rgb}{0.000000,0.000000,0.000000}%
\pgfsetfillcolor{currentfill}%
\pgfsetlinewidth{0.501875pt}%
\definecolor{currentstroke}{rgb}{0.000000,0.000000,0.000000}%
\pgfsetstrokecolor{currentstroke}%
\pgfsetdash{}{0pt}%
\pgfsys@defobject{currentmarker}{\pgfqpoint{0.000000in}{0.000000in}}{\pgfqpoint{0.055556in}{0.000000in}}{%
\pgfpathmoveto{\pgfqpoint{0.000000in}{0.000000in}}%
\pgfpathlineto{\pgfqpoint{0.055556in}{0.000000in}}%
\pgfusepath{stroke,fill}%
}%
\begin{pgfscope}%
\pgfsys@transformshift{0.675000in}{0.733119in}%
\pgfsys@useobject{currentmarker}{}%
\end{pgfscope}%
\end{pgfscope}%
\begin{pgfscope}%
\pgfsetbuttcap%
\pgfsetroundjoin%
\definecolor{currentfill}{rgb}{0.000000,0.000000,0.000000}%
\pgfsetfillcolor{currentfill}%
\pgfsetlinewidth{0.501875pt}%
\definecolor{currentstroke}{rgb}{0.000000,0.000000,0.000000}%
\pgfsetstrokecolor{currentstroke}%
\pgfsetdash{}{0pt}%
\pgfsys@defobject{currentmarker}{\pgfqpoint{-0.055556in}{0.000000in}}{\pgfqpoint{0.000000in}{0.000000in}}{%
\pgfpathmoveto{\pgfqpoint{0.000000in}{0.000000in}}%
\pgfpathlineto{\pgfqpoint{-0.055556in}{0.000000in}}%
\pgfusepath{stroke,fill}%
}%
\begin{pgfscope}%
\pgfsys@transformshift{4.275000in}{0.733119in}%
\pgfsys@useobject{currentmarker}{}%
\end{pgfscope}%
\end{pgfscope}%
\begin{pgfscope}%
\pgftext[x=0.619444in,y=0.733119in,right,]{\rmfamily\fontsize{9.000000}{10.800000}\selectfont \(\displaystyle -1.0\)}%
\end{pgfscope}%
\begin{pgfscope}%
\pgfsetbuttcap%
\pgfsetroundjoin%
\definecolor{currentfill}{rgb}{0.000000,0.000000,0.000000}%
\pgfsetfillcolor{currentfill}%
\pgfsetlinewidth{0.501875pt}%
\definecolor{currentstroke}{rgb}{0.000000,0.000000,0.000000}%
\pgfsetstrokecolor{currentstroke}%
\pgfsetdash{}{0pt}%
\pgfsys@defobject{currentmarker}{\pgfqpoint{0.000000in}{0.000000in}}{\pgfqpoint{0.055556in}{0.000000in}}{%
\pgfpathmoveto{\pgfqpoint{0.000000in}{0.000000in}}%
\pgfpathlineto{\pgfqpoint{0.055556in}{0.000000in}}%
\pgfusepath{stroke,fill}%
}%
\begin{pgfscope}%
\pgfsys@transformshift{0.675000in}{0.893210in}%
\pgfsys@useobject{currentmarker}{}%
\end{pgfscope}%
\end{pgfscope}%
\begin{pgfscope}%
\pgfsetbuttcap%
\pgfsetroundjoin%
\definecolor{currentfill}{rgb}{0.000000,0.000000,0.000000}%
\pgfsetfillcolor{currentfill}%
\pgfsetlinewidth{0.501875pt}%
\definecolor{currentstroke}{rgb}{0.000000,0.000000,0.000000}%
\pgfsetstrokecolor{currentstroke}%
\pgfsetdash{}{0pt}%
\pgfsys@defobject{currentmarker}{\pgfqpoint{-0.055556in}{0.000000in}}{\pgfqpoint{0.000000in}{0.000000in}}{%
\pgfpathmoveto{\pgfqpoint{0.000000in}{0.000000in}}%
\pgfpathlineto{\pgfqpoint{-0.055556in}{0.000000in}}%
\pgfusepath{stroke,fill}%
}%
\begin{pgfscope}%
\pgfsys@transformshift{4.275000in}{0.893210in}%
\pgfsys@useobject{currentmarker}{}%
\end{pgfscope}%
\end{pgfscope}%
\begin{pgfscope}%
\pgftext[x=0.619444in,y=0.893210in,right,]{\rmfamily\fontsize{9.000000}{10.800000}\selectfont \(\displaystyle -0.5\)}%
\end{pgfscope}%
\begin{pgfscope}%
\pgfsetbuttcap%
\pgfsetroundjoin%
\definecolor{currentfill}{rgb}{0.000000,0.000000,0.000000}%
\pgfsetfillcolor{currentfill}%
\pgfsetlinewidth{0.501875pt}%
\definecolor{currentstroke}{rgb}{0.000000,0.000000,0.000000}%
\pgfsetstrokecolor{currentstroke}%
\pgfsetdash{}{0pt}%
\pgfsys@defobject{currentmarker}{\pgfqpoint{0.000000in}{0.000000in}}{\pgfqpoint{0.055556in}{0.000000in}}{%
\pgfpathmoveto{\pgfqpoint{0.000000in}{0.000000in}}%
\pgfpathlineto{\pgfqpoint{0.055556in}{0.000000in}}%
\pgfusepath{stroke,fill}%
}%
\begin{pgfscope}%
\pgfsys@transformshift{0.675000in}{1.053302in}%
\pgfsys@useobject{currentmarker}{}%
\end{pgfscope}%
\end{pgfscope}%
\begin{pgfscope}%
\pgfsetbuttcap%
\pgfsetroundjoin%
\definecolor{currentfill}{rgb}{0.000000,0.000000,0.000000}%
\pgfsetfillcolor{currentfill}%
\pgfsetlinewidth{0.501875pt}%
\definecolor{currentstroke}{rgb}{0.000000,0.000000,0.000000}%
\pgfsetstrokecolor{currentstroke}%
\pgfsetdash{}{0pt}%
\pgfsys@defobject{currentmarker}{\pgfqpoint{-0.055556in}{0.000000in}}{\pgfqpoint{0.000000in}{0.000000in}}{%
\pgfpathmoveto{\pgfqpoint{0.000000in}{0.000000in}}%
\pgfpathlineto{\pgfqpoint{-0.055556in}{0.000000in}}%
\pgfusepath{stroke,fill}%
}%
\begin{pgfscope}%
\pgfsys@transformshift{4.275000in}{1.053302in}%
\pgfsys@useobject{currentmarker}{}%
\end{pgfscope}%
\end{pgfscope}%
\begin{pgfscope}%
\pgftext[x=0.619444in,y=1.053302in,right,]{\rmfamily\fontsize{9.000000}{10.800000}\selectfont \(\displaystyle 0.0\)}%
\end{pgfscope}%
\begin{pgfscope}%
\pgfsetbuttcap%
\pgfsetroundjoin%
\definecolor{currentfill}{rgb}{0.000000,0.000000,0.000000}%
\pgfsetfillcolor{currentfill}%
\pgfsetlinewidth{0.501875pt}%
\definecolor{currentstroke}{rgb}{0.000000,0.000000,0.000000}%
\pgfsetstrokecolor{currentstroke}%
\pgfsetdash{}{0pt}%
\pgfsys@defobject{currentmarker}{\pgfqpoint{0.000000in}{0.000000in}}{\pgfqpoint{0.055556in}{0.000000in}}{%
\pgfpathmoveto{\pgfqpoint{0.000000in}{0.000000in}}%
\pgfpathlineto{\pgfqpoint{0.055556in}{0.000000in}}%
\pgfusepath{stroke,fill}%
}%
\begin{pgfscope}%
\pgfsys@transformshift{0.675000in}{1.213393in}%
\pgfsys@useobject{currentmarker}{}%
\end{pgfscope}%
\end{pgfscope}%
\begin{pgfscope}%
\pgfsetbuttcap%
\pgfsetroundjoin%
\definecolor{currentfill}{rgb}{0.000000,0.000000,0.000000}%
\pgfsetfillcolor{currentfill}%
\pgfsetlinewidth{0.501875pt}%
\definecolor{currentstroke}{rgb}{0.000000,0.000000,0.000000}%
\pgfsetstrokecolor{currentstroke}%
\pgfsetdash{}{0pt}%
\pgfsys@defobject{currentmarker}{\pgfqpoint{-0.055556in}{0.000000in}}{\pgfqpoint{0.000000in}{0.000000in}}{%
\pgfpathmoveto{\pgfqpoint{0.000000in}{0.000000in}}%
\pgfpathlineto{\pgfqpoint{-0.055556in}{0.000000in}}%
\pgfusepath{stroke,fill}%
}%
\begin{pgfscope}%
\pgfsys@transformshift{4.275000in}{1.213393in}%
\pgfsys@useobject{currentmarker}{}%
\end{pgfscope}%
\end{pgfscope}%
\begin{pgfscope}%
\pgftext[x=0.619444in,y=1.213393in,right,]{\rmfamily\fontsize{9.000000}{10.800000}\selectfont \(\displaystyle 0.5\)}%
\end{pgfscope}%
\begin{pgfscope}%
\pgfsetbuttcap%
\pgfsetroundjoin%
\definecolor{currentfill}{rgb}{0.000000,0.000000,0.000000}%
\pgfsetfillcolor{currentfill}%
\pgfsetlinewidth{0.501875pt}%
\definecolor{currentstroke}{rgb}{0.000000,0.000000,0.000000}%
\pgfsetstrokecolor{currentstroke}%
\pgfsetdash{}{0pt}%
\pgfsys@defobject{currentmarker}{\pgfqpoint{0.000000in}{0.000000in}}{\pgfqpoint{0.055556in}{0.000000in}}{%
\pgfpathmoveto{\pgfqpoint{0.000000in}{0.000000in}}%
\pgfpathlineto{\pgfqpoint{0.055556in}{0.000000in}}%
\pgfusepath{stroke,fill}%
}%
\begin{pgfscope}%
\pgfsys@transformshift{0.675000in}{1.373485in}%
\pgfsys@useobject{currentmarker}{}%
\end{pgfscope}%
\end{pgfscope}%
\begin{pgfscope}%
\pgfsetbuttcap%
\pgfsetroundjoin%
\definecolor{currentfill}{rgb}{0.000000,0.000000,0.000000}%
\pgfsetfillcolor{currentfill}%
\pgfsetlinewidth{0.501875pt}%
\definecolor{currentstroke}{rgb}{0.000000,0.000000,0.000000}%
\pgfsetstrokecolor{currentstroke}%
\pgfsetdash{}{0pt}%
\pgfsys@defobject{currentmarker}{\pgfqpoint{-0.055556in}{0.000000in}}{\pgfqpoint{0.000000in}{0.000000in}}{%
\pgfpathmoveto{\pgfqpoint{0.000000in}{0.000000in}}%
\pgfpathlineto{\pgfqpoint{-0.055556in}{0.000000in}}%
\pgfusepath{stroke,fill}%
}%
\begin{pgfscope}%
\pgfsys@transformshift{4.275000in}{1.373485in}%
\pgfsys@useobject{currentmarker}{}%
\end{pgfscope}%
\end{pgfscope}%
\begin{pgfscope}%
\pgftext[x=0.619444in,y=1.373485in,right,]{\rmfamily\fontsize{9.000000}{10.800000}\selectfont \(\displaystyle 1.0\)}%
\end{pgfscope}%
\begin{pgfscope}%
\pgfsetbuttcap%
\pgfsetroundjoin%
\definecolor{currentfill}{rgb}{0.000000,0.000000,0.000000}%
\pgfsetfillcolor{currentfill}%
\pgfsetlinewidth{0.501875pt}%
\definecolor{currentstroke}{rgb}{0.000000,0.000000,0.000000}%
\pgfsetstrokecolor{currentstroke}%
\pgfsetdash{}{0pt}%
\pgfsys@defobject{currentmarker}{\pgfqpoint{0.000000in}{0.000000in}}{\pgfqpoint{0.055556in}{0.000000in}}{%
\pgfpathmoveto{\pgfqpoint{0.000000in}{0.000000in}}%
\pgfpathlineto{\pgfqpoint{0.055556in}{0.000000in}}%
\pgfusepath{stroke,fill}%
}%
\begin{pgfscope}%
\pgfsys@transformshift{0.675000in}{1.533576in}%
\pgfsys@useobject{currentmarker}{}%
\end{pgfscope}%
\end{pgfscope}%
\begin{pgfscope}%
\pgfsetbuttcap%
\pgfsetroundjoin%
\definecolor{currentfill}{rgb}{0.000000,0.000000,0.000000}%
\pgfsetfillcolor{currentfill}%
\pgfsetlinewidth{0.501875pt}%
\definecolor{currentstroke}{rgb}{0.000000,0.000000,0.000000}%
\pgfsetstrokecolor{currentstroke}%
\pgfsetdash{}{0pt}%
\pgfsys@defobject{currentmarker}{\pgfqpoint{-0.055556in}{0.000000in}}{\pgfqpoint{0.000000in}{0.000000in}}{%
\pgfpathmoveto{\pgfqpoint{0.000000in}{0.000000in}}%
\pgfpathlineto{\pgfqpoint{-0.055556in}{0.000000in}}%
\pgfusepath{stroke,fill}%
}%
\begin{pgfscope}%
\pgfsys@transformshift{4.275000in}{1.533576in}%
\pgfsys@useobject{currentmarker}{}%
\end{pgfscope}%
\end{pgfscope}%
\begin{pgfscope}%
\pgftext[x=0.619444in,y=1.533576in,right,]{\rmfamily\fontsize{9.000000}{10.800000}\selectfont \(\displaystyle 1.5\)}%
\end{pgfscope}%
\begin{pgfscope}%
\pgftext[x=0.285920in,y=1.053302in,,bottom,rotate=90.000000]{\rmfamily\fontsize{9.000000}{10.800000}\selectfont Spannung (V)}%
\end{pgfscope}%
\begin{pgfscope}%
\pgftext[x=2.475000in,y=1.651048in,,base]{\rmfamily\fontsize{11.000000}{13.200000}\selectfont Moduliertes Signal}%
\end{pgfscope}%
\end{pgfpicture}%
\makeatother%
\endgroup%

    \caption{%
        \emph{Frequency-shift  keying}: Oben   sind  die   zu  \"ubertragenden
        digitalen  Daten  als  \code{1}  und \code{0}  abgebildet,  unten  das
        zugeh\"orige Verhalten des  modulierten Signals.%
    }
    \label{fig:fsk:concept}
\end{figure}


\textbf{Amplitude-shift  keying}: Die  ASK (Amplitudenumtastung  auf  Deutsch)
benutzt statt verschiedenen Frequenzen unterschiedliche Amplituden, um Symbole
zu  codieren.  \fref{fig:ask:concept}  stellt  das  grundlegende  Konzept  des
Verfahrens schematisch dar.

Unsere Umsetzung w\"urde einen Transistor  benutzen, um jeweils ein Solarmodul
bei  einer Tr\"agerfrequenz  von  etwa \SI{10}{\kilo\hertz}  kurzzuschliessen.
Dadurch bricht die Spannung auf  einem Strang von seriell geschalteten Modulen
kurz ein, was als Signal  codiert und ausgewertet werden kann. Grunds\"atzlich
handelt  es sich  bei  diesem Prinzip  um  eine ASK  mit  den zwei  Amplituden
\SI{0}{\volt} und  dem Betrag des Spannungsabfalls  durch Kurzschliessen eines
Moduls.
\todo{Zahlen in Ticks auf Achsen weglassen?}

\begin{figure}[h!tb]
    \centering
    %% Creator: Matplotlib, PGF backend
%%
%% To include the figure in your LaTeX document, write
%%   \input{<filename>.pgf}
%%
%% Make sure the required packages are loaded in your preamble
%%   \usepackage{pgf}
%%
%% Figures using additional raster images can only be included by \input if
%% they are in the same directory as the main LaTeX file. For loading figures
%% from other directories you can use the `import` package
%%   \usepackage{import}
%% and then include the figures with
%%   \import{<path to file>}{<filename>.pgf}
%%
%% Matplotlib used the following preamble
%%   \usepackage{fontspec}
%%   \setmainfont{Bitstream Vera Serif}
%%   \setsansfont{Bitstream Vera Sans}
%%   \setmonofont{Bitstream Vera Sans Mono}
%%
\begingroup%
\makeatletter%
\begin{pgfpicture}%
\pgfpathrectangle{\pgfpointorigin}{\pgfqpoint{5.000000in}{3.500000in}}%
\pgfusepath{use as bounding box, clip}%
\begin{pgfscope}%
\pgfsetbuttcap%
\pgfsetmiterjoin%
\pgfsetlinewidth{0.000000pt}%
\definecolor{currentstroke}{rgb}{0.000000,0.000000,0.000000}%
\pgfsetstrokecolor{currentstroke}%
\pgfsetstrokeopacity{0.000000}%
\pgfsetdash{}{0pt}%
\pgfpathmoveto{\pgfqpoint{0.000000in}{0.000000in}}%
\pgfpathlineto{\pgfqpoint{5.000000in}{0.000000in}}%
\pgfpathlineto{\pgfqpoint{5.000000in}{3.500000in}}%
\pgfpathlineto{\pgfqpoint{0.000000in}{3.500000in}}%
\pgfpathclose%
\pgfusepath{}%
\end{pgfscope}%
\begin{pgfscope}%
\pgfsetbuttcap%
\pgfsetmiterjoin%
\pgfsetlinewidth{0.000000pt}%
\definecolor{currentstroke}{rgb}{0.000000,0.000000,0.000000}%
\pgfsetstrokecolor{currentstroke}%
\pgfsetstrokeopacity{0.000000}%
\pgfsetdash{}{0pt}%
\pgfpathmoveto{\pgfqpoint{0.750000in}{2.268396in}}%
\pgfpathlineto{\pgfqpoint{4.750000in}{2.268396in}}%
\pgfpathlineto{\pgfqpoint{4.750000in}{3.325000in}}%
\pgfpathlineto{\pgfqpoint{0.750000in}{3.325000in}}%
\pgfpathclose%
\pgfusepath{}%
\end{pgfscope}%
\begin{pgfscope}%
\pgfpathrectangle{\pgfqpoint{0.750000in}{2.268396in}}{\pgfqpoint{4.000000in}{1.056604in}} %
\pgfusepath{clip}%
\pgfsetrectcap%
\pgfsetroundjoin%
\pgfsetlinewidth{0.501875pt}%
\definecolor{currentstroke}{rgb}{0.000000,0.000000,1.000000}%
\pgfsetstrokecolor{currentstroke}%
\pgfsetdash{}{0pt}%
\pgfpathmoveto{\pgfqpoint{0.750000in}{2.356447in}}%
\pgfpathlineto{\pgfqpoint{1.701998in}{2.356447in}}%
\pgfusepath{stroke}%
\end{pgfscope}%
\begin{pgfscope}%
\pgfpathrectangle{\pgfqpoint{0.750000in}{2.268396in}}{\pgfqpoint{4.000000in}{1.056604in}} %
\pgfusepath{clip}%
\pgfsetrectcap%
\pgfsetroundjoin%
\pgfsetlinewidth{0.501875pt}%
\definecolor{currentstroke}{rgb}{0.501961,0.501961,0.501961}%
\pgfsetstrokecolor{currentstroke}%
\pgfsetdash{}{0pt}%
\pgfpathmoveto{\pgfqpoint{1.701998in}{2.356447in}}%
\pgfpathlineto{\pgfqpoint{1.701998in}{3.236950in}}%
\pgfusepath{stroke}%
\end{pgfscope}%
\begin{pgfscope}%
\pgfpathrectangle{\pgfqpoint{0.750000in}{2.268396in}}{\pgfqpoint{4.000000in}{1.056604in}} %
\pgfusepath{clip}%
\pgfsetrectcap%
\pgfsetroundjoin%
\pgfsetlinewidth{0.501875pt}%
\definecolor{currentstroke}{rgb}{1.000000,0.000000,1.000000}%
\pgfsetstrokecolor{currentstroke}%
\pgfsetdash{}{0pt}%
\pgfpathmoveto{\pgfqpoint{1.701998in}{3.236950in}}%
\pgfpathlineto{\pgfqpoint{2.653996in}{3.236950in}}%
\pgfusepath{stroke}%
\end{pgfscope}%
\begin{pgfscope}%
\pgfpathrectangle{\pgfqpoint{0.750000in}{2.268396in}}{\pgfqpoint{4.000000in}{1.056604in}} %
\pgfusepath{clip}%
\pgfsetrectcap%
\pgfsetroundjoin%
\pgfsetlinewidth{0.501875pt}%
\definecolor{currentstroke}{rgb}{0.501961,0.501961,0.501961}%
\pgfsetstrokecolor{currentstroke}%
\pgfsetdash{}{0pt}%
\pgfpathmoveto{\pgfqpoint{2.653996in}{3.236950in}}%
\pgfpathlineto{\pgfqpoint{2.653996in}{2.356447in}}%
\pgfusepath{stroke}%
\end{pgfscope}%
\begin{pgfscope}%
\pgfpathrectangle{\pgfqpoint{0.750000in}{2.268396in}}{\pgfqpoint{4.000000in}{1.056604in}} %
\pgfusepath{clip}%
\pgfsetrectcap%
\pgfsetroundjoin%
\pgfsetlinewidth{0.501875pt}%
\definecolor{currentstroke}{rgb}{0.000000,0.000000,1.000000}%
\pgfsetstrokecolor{currentstroke}%
\pgfsetdash{}{0pt}%
\pgfpathmoveto{\pgfqpoint{2.653996in}{2.356447in}}%
\pgfpathlineto{\pgfqpoint{3.605993in}{2.356447in}}%
\pgfusepath{stroke}%
\end{pgfscope}%
\begin{pgfscope}%
\pgfpathrectangle{\pgfqpoint{0.750000in}{2.268396in}}{\pgfqpoint{4.000000in}{1.056604in}} %
\pgfusepath{clip}%
\pgfsetrectcap%
\pgfsetroundjoin%
\pgfsetlinewidth{0.501875pt}%
\definecolor{currentstroke}{rgb}{0.501961,0.501961,0.501961}%
\pgfsetstrokecolor{currentstroke}%
\pgfsetdash{}{0pt}%
\pgfpathmoveto{\pgfqpoint{3.605993in}{2.356447in}}%
\pgfpathlineto{\pgfqpoint{3.605993in}{3.236950in}}%
\pgfusepath{stroke}%
\end{pgfscope}%
\begin{pgfscope}%
\pgfpathrectangle{\pgfqpoint{0.750000in}{2.268396in}}{\pgfqpoint{4.000000in}{1.056604in}} %
\pgfusepath{clip}%
\pgfsetrectcap%
\pgfsetroundjoin%
\pgfsetlinewidth{0.501875pt}%
\definecolor{currentstroke}{rgb}{1.000000,0.000000,1.000000}%
\pgfsetstrokecolor{currentstroke}%
\pgfsetdash{}{0pt}%
\pgfpathmoveto{\pgfqpoint{3.605993in}{3.236950in}}%
\pgfpathlineto{\pgfqpoint{4.557991in}{3.236950in}}%
\pgfusepath{stroke}%
\end{pgfscope}%
\begin{pgfscope}%
\pgfsetrectcap%
\pgfsetmiterjoin%
\pgfsetlinewidth{0.501875pt}%
\definecolor{currentstroke}{rgb}{0.000000,0.000000,0.000000}%
\pgfsetstrokecolor{currentstroke}%
\pgfsetdash{}{0pt}%
\pgfpathmoveto{\pgfqpoint{4.750000in}{2.268396in}}%
\pgfpathlineto{\pgfqpoint{4.750000in}{3.325000in}}%
\pgfusepath{stroke}%
\end{pgfscope}%
\begin{pgfscope}%
\pgfsetrectcap%
\pgfsetmiterjoin%
\pgfsetlinewidth{0.501875pt}%
\definecolor{currentstroke}{rgb}{0.000000,0.000000,0.000000}%
\pgfsetstrokecolor{currentstroke}%
\pgfsetdash{}{0pt}%
\pgfpathmoveto{\pgfqpoint{0.750000in}{2.268396in}}%
\pgfpathlineto{\pgfqpoint{0.750000in}{3.325000in}}%
\pgfusepath{stroke}%
\end{pgfscope}%
\begin{pgfscope}%
\pgfsetrectcap%
\pgfsetmiterjoin%
\pgfsetlinewidth{0.501875pt}%
\definecolor{currentstroke}{rgb}{0.000000,0.000000,0.000000}%
\pgfsetstrokecolor{currentstroke}%
\pgfsetdash{}{0pt}%
\pgfpathmoveto{\pgfqpoint{0.750000in}{2.268396in}}%
\pgfpathlineto{\pgfqpoint{4.750000in}{2.268396in}}%
\pgfusepath{stroke}%
\end{pgfscope}%
\begin{pgfscope}%
\pgfsetrectcap%
\pgfsetmiterjoin%
\pgfsetlinewidth{0.501875pt}%
\definecolor{currentstroke}{rgb}{0.000000,0.000000,0.000000}%
\pgfsetstrokecolor{currentstroke}%
\pgfsetdash{}{0pt}%
\pgfpathmoveto{\pgfqpoint{0.750000in}{3.325000in}}%
\pgfpathlineto{\pgfqpoint{4.750000in}{3.325000in}}%
\pgfusepath{stroke}%
\end{pgfscope}%
\begin{pgfscope}%
\pgftext[x=2.750000in,y=2.198952in,,top]{\rmfamily\fontsize{9.000000}{10.800000}\selectfont Zeit}%
\end{pgfscope}%
\begin{pgfscope}%
\pgfsetbuttcap%
\pgfsetroundjoin%
\definecolor{currentfill}{rgb}{0.000000,0.000000,0.000000}%
\pgfsetfillcolor{currentfill}%
\pgfsetlinewidth{0.501875pt}%
\definecolor{currentstroke}{rgb}{0.000000,0.000000,0.000000}%
\pgfsetstrokecolor{currentstroke}%
\pgfsetdash{}{0pt}%
\pgfsys@defobject{currentmarker}{\pgfqpoint{0.000000in}{0.000000in}}{\pgfqpoint{0.055556in}{0.000000in}}{%
\pgfpathmoveto{\pgfqpoint{0.000000in}{0.000000in}}%
\pgfpathlineto{\pgfqpoint{0.055556in}{0.000000in}}%
\pgfusepath{stroke,fill}%
}%
\begin{pgfscope}%
\pgfsys@transformshift{0.750000in}{2.356447in}%
\pgfsys@useobject{currentmarker}{}%
\end{pgfscope}%
\end{pgfscope}%
\begin{pgfscope}%
\pgfsetbuttcap%
\pgfsetroundjoin%
\definecolor{currentfill}{rgb}{0.000000,0.000000,0.000000}%
\pgfsetfillcolor{currentfill}%
\pgfsetlinewidth{0.501875pt}%
\definecolor{currentstroke}{rgb}{0.000000,0.000000,0.000000}%
\pgfsetstrokecolor{currentstroke}%
\pgfsetdash{}{0pt}%
\pgfsys@defobject{currentmarker}{\pgfqpoint{-0.055556in}{0.000000in}}{\pgfqpoint{0.000000in}{0.000000in}}{%
\pgfpathmoveto{\pgfqpoint{0.000000in}{0.000000in}}%
\pgfpathlineto{\pgfqpoint{-0.055556in}{0.000000in}}%
\pgfusepath{stroke,fill}%
}%
\begin{pgfscope}%
\pgfsys@transformshift{4.750000in}{2.356447in}%
\pgfsys@useobject{currentmarker}{}%
\end{pgfscope}%
\end{pgfscope}%
\begin{pgfscope}%
\pgftext[x=0.694444in,y=2.356447in,right,]{\rmfamily\fontsize{9.000000}{10.800000}\selectfont \(\displaystyle 0\)}%
\end{pgfscope}%
\begin{pgfscope}%
\pgfsetbuttcap%
\pgfsetroundjoin%
\definecolor{currentfill}{rgb}{0.000000,0.000000,0.000000}%
\pgfsetfillcolor{currentfill}%
\pgfsetlinewidth{0.501875pt}%
\definecolor{currentstroke}{rgb}{0.000000,0.000000,0.000000}%
\pgfsetstrokecolor{currentstroke}%
\pgfsetdash{}{0pt}%
\pgfsys@defobject{currentmarker}{\pgfqpoint{0.000000in}{0.000000in}}{\pgfqpoint{0.055556in}{0.000000in}}{%
\pgfpathmoveto{\pgfqpoint{0.000000in}{0.000000in}}%
\pgfpathlineto{\pgfqpoint{0.055556in}{0.000000in}}%
\pgfusepath{stroke,fill}%
}%
\begin{pgfscope}%
\pgfsys@transformshift{0.750000in}{3.236950in}%
\pgfsys@useobject{currentmarker}{}%
\end{pgfscope}%
\end{pgfscope}%
\begin{pgfscope}%
\pgfsetbuttcap%
\pgfsetroundjoin%
\definecolor{currentfill}{rgb}{0.000000,0.000000,0.000000}%
\pgfsetfillcolor{currentfill}%
\pgfsetlinewidth{0.501875pt}%
\definecolor{currentstroke}{rgb}{0.000000,0.000000,0.000000}%
\pgfsetstrokecolor{currentstroke}%
\pgfsetdash{}{0pt}%
\pgfsys@defobject{currentmarker}{\pgfqpoint{-0.055556in}{0.000000in}}{\pgfqpoint{0.000000in}{0.000000in}}{%
\pgfpathmoveto{\pgfqpoint{0.000000in}{0.000000in}}%
\pgfpathlineto{\pgfqpoint{-0.055556in}{0.000000in}}%
\pgfusepath{stroke,fill}%
}%
\begin{pgfscope}%
\pgfsys@transformshift{4.750000in}{3.236950in}%
\pgfsys@useobject{currentmarker}{}%
\end{pgfscope}%
\end{pgfscope}%
\begin{pgfscope}%
\pgftext[x=0.694444in,y=3.236950in,right,]{\rmfamily\fontsize{9.000000}{10.800000}\selectfont \(\displaystyle 1\)}%
\end{pgfscope}%
\begin{pgfscope}%
\pgftext[x=0.560764in,y=2.796698in,,bottom,rotate=90.000000]{\rmfamily\fontsize{9.000000}{10.800000}\selectfont Symbol}%
\end{pgfscope}%
\begin{pgfscope}%
\pgftext[x=2.750000in,y=3.394444in,,base]{\rmfamily\fontsize{11.000000}{13.200000}\selectfont Daten}%
\end{pgfscope}%
\begin{pgfscope}%
\pgfsetbuttcap%
\pgfsetmiterjoin%
\pgfsetlinewidth{0.000000pt}%
\definecolor{currentstroke}{rgb}{0.000000,0.000000,0.000000}%
\pgfsetstrokecolor{currentstroke}%
\pgfsetstrokeopacity{0.000000}%
\pgfsetdash{}{0pt}%
\pgfpathmoveto{\pgfqpoint{0.750000in}{0.525000in}}%
\pgfpathlineto{\pgfqpoint{4.750000in}{0.525000in}}%
\pgfpathlineto{\pgfqpoint{4.750000in}{1.581604in}}%
\pgfpathlineto{\pgfqpoint{0.750000in}{1.581604in}}%
\pgfpathclose%
\pgfusepath{}%
\end{pgfscope}%
\begin{pgfscope}%
\pgfpathrectangle{\pgfqpoint{0.750000in}{0.525000in}}{\pgfqpoint{4.000000in}{1.056604in}} %
\pgfusepath{clip}%
\pgfsetrectcap%
\pgfsetroundjoin%
\pgfsetlinewidth{0.501875pt}%
\definecolor{currentstroke}{rgb}{0.000000,0.000000,1.000000}%
\pgfsetstrokecolor{currentstroke}%
\pgfsetdash{}{0pt}%
\pgfpathmoveto{\pgfqpoint{0.750000in}{1.053302in}}%
\pgfpathlineto{\pgfqpoint{0.771918in}{1.097077in}}%
\pgfpathlineto{\pgfqpoint{0.783353in}{1.115023in}}%
\pgfpathlineto{\pgfqpoint{0.791930in}{1.124890in}}%
\pgfpathlineto{\pgfqpoint{0.799553in}{1.130604in}}%
\pgfpathlineto{\pgfqpoint{0.806224in}{1.133048in}}%
\pgfpathlineto{\pgfqpoint{0.812895in}{1.133026in}}%
\pgfpathlineto{\pgfqpoint{0.819565in}{1.130538in}}%
\pgfpathlineto{\pgfqpoint{0.826236in}{1.125661in}}%
\pgfpathlineto{\pgfqpoint{0.833860in}{1.117358in}}%
\pgfpathlineto{\pgfqpoint{0.843389in}{1.103392in}}%
\pgfpathlineto{\pgfqpoint{0.855778in}{1.080679in}}%
\pgfpathlineto{\pgfqpoint{0.899613in}{0.995425in}}%
\pgfpathlineto{\pgfqpoint{0.909143in}{0.983482in}}%
\pgfpathlineto{\pgfqpoint{0.916766in}{0.977066in}}%
\pgfpathlineto{\pgfqpoint{0.923437in}{0.973970in}}%
\pgfpathlineto{\pgfqpoint{0.930108in}{0.973328in}}%
\pgfpathlineto{\pgfqpoint{0.936778in}{0.975160in}}%
\pgfpathlineto{\pgfqpoint{0.943449in}{0.979410in}}%
\pgfpathlineto{\pgfqpoint{0.951073in}{0.987054in}}%
\pgfpathlineto{\pgfqpoint{0.959649in}{0.998831in}}%
\pgfpathlineto{\pgfqpoint{0.971085in}{1.018733in}}%
\pgfpathlineto{\pgfqpoint{0.989191in}{1.055819in}}%
\pgfpathlineto{\pgfqpoint{1.010156in}{1.097498in}}%
\pgfpathlineto{\pgfqpoint{1.021591in}{1.115342in}}%
\pgfpathlineto{\pgfqpoint{1.030168in}{1.125114in}}%
\pgfpathlineto{\pgfqpoint{1.037791in}{1.130733in}}%
\pgfpathlineto{\pgfqpoint{1.044462in}{1.133090in}}%
\pgfpathlineto{\pgfqpoint{1.051132in}{1.132980in}}%
\pgfpathlineto{\pgfqpoint{1.057803in}{1.130404in}}%
\pgfpathlineto{\pgfqpoint{1.064474in}{1.125444in}}%
\pgfpathlineto{\pgfqpoint{1.072097in}{1.117055in}}%
\pgfpathlineto{\pgfqpoint{1.081627in}{1.102998in}}%
\pgfpathlineto{\pgfqpoint{1.094015in}{1.080206in}}%
\pgfpathlineto{\pgfqpoint{1.136898in}{0.996479in}}%
\pgfpathlineto{\pgfqpoint{1.146428in}{0.984233in}}%
\pgfpathlineto{\pgfqpoint{1.154051in}{0.977540in}}%
\pgfpathlineto{\pgfqpoint{1.160722in}{0.974185in}}%
\pgfpathlineto{\pgfqpoint{1.167392in}{0.973278in}}%
\pgfpathlineto{\pgfqpoint{1.174063in}{0.974847in}}%
\pgfpathlineto{\pgfqpoint{1.180734in}{0.978842in}}%
\pgfpathlineto{\pgfqpoint{1.188357in}{0.986218in}}%
\pgfpathlineto{\pgfqpoint{1.196934in}{0.997734in}}%
\pgfpathlineto{\pgfqpoint{1.207416in}{1.015589in}}%
\pgfpathlineto{\pgfqpoint{1.223617in}{1.048271in}}%
\pgfpathlineto{\pgfqpoint{1.247440in}{1.096231in}}%
\pgfpathlineto{\pgfqpoint{1.258876in}{1.114377in}}%
\pgfpathlineto{\pgfqpoint{1.267452in}{1.124434in}}%
\pgfpathlineto{\pgfqpoint{1.275076in}{1.130336in}}%
\pgfpathlineto{\pgfqpoint{1.281747in}{1.132955in}}%
\pgfpathlineto{\pgfqpoint{1.288417in}{1.133110in}}%
\pgfpathlineto{\pgfqpoint{1.295088in}{1.130796in}}%
\pgfpathlineto{\pgfqpoint{1.301758in}{1.126086in}}%
\pgfpathlineto{\pgfqpoint{1.309382in}{1.117957in}}%
\pgfpathlineto{\pgfqpoint{1.318912in}{1.104174in}}%
\pgfpathlineto{\pgfqpoint{1.331300in}{1.081623in}}%
\pgfpathlineto{\pgfqpoint{1.376089in}{0.994734in}}%
\pgfpathlineto{\pgfqpoint{1.385618in}{0.982995in}}%
\pgfpathlineto{\pgfqpoint{1.393242in}{0.976765in}}%
\pgfpathlineto{\pgfqpoint{1.399912in}{0.973842in}}%
\pgfpathlineto{\pgfqpoint{1.406583in}{0.973377in}}%
\pgfpathlineto{\pgfqpoint{1.413254in}{0.975385in}}%
\pgfpathlineto{\pgfqpoint{1.419924in}{0.979803in}}%
\pgfpathlineto{\pgfqpoint{1.427548in}{0.987624in}}%
\pgfpathlineto{\pgfqpoint{1.437077in}{1.001082in}}%
\pgfpathlineto{\pgfqpoint{1.448513in}{1.021482in}}%
\pgfpathlineto{\pgfqpoint{1.468525in}{1.062845in}}%
\pgfpathlineto{\pgfqpoint{1.486631in}{1.098334in}}%
\pgfpathlineto{\pgfqpoint{1.498066in}{1.115974in}}%
\pgfpathlineto{\pgfqpoint{1.506643in}{1.125553in}}%
\pgfpathlineto{\pgfqpoint{1.514266in}{1.130982in}}%
\pgfpathlineto{\pgfqpoint{1.520937in}{1.133165in}}%
\pgfpathlineto{\pgfqpoint{1.527608in}{1.132877in}}%
\pgfpathlineto{\pgfqpoint{1.534278in}{1.130128in}}%
\pgfpathlineto{\pgfqpoint{1.540949in}{1.125002in}}%
\pgfpathlineto{\pgfqpoint{1.548573in}{1.116441in}}%
\pgfpathlineto{\pgfqpoint{1.558102in}{1.102205in}}%
\pgfpathlineto{\pgfqpoint{1.570491in}{1.079255in}}%
\pgfpathlineto{\pgfqpoint{1.612420in}{0.997193in}}%
\pgfpathlineto{\pgfqpoint{1.621950in}{0.984748in}}%
\pgfpathlineto{\pgfqpoint{1.629574in}{0.977871in}}%
\pgfpathlineto{\pgfqpoint{1.636244in}{0.974345in}}%
\pgfpathlineto{\pgfqpoint{1.642915in}{0.973261in}}%
\pgfpathlineto{\pgfqpoint{1.649585in}{0.974653in}}%
\pgfpathlineto{\pgfqpoint{1.656256in}{0.978478in}}%
\pgfpathlineto{\pgfqpoint{1.663880in}{0.985674in}}%
\pgfpathlineto{\pgfqpoint{1.672456in}{0.997013in}}%
\pgfpathlineto{\pgfqpoint{1.682939in}{1.014704in}}%
\pgfpathlineto{\pgfqpoint{1.698186in}{1.045260in}}%
\pgfpathlineto{\pgfqpoint{1.701998in}{1.053302in}}%
\pgfpathlineto{\pgfqpoint{1.701998in}{1.053302in}}%
\pgfusepath{stroke}%
\end{pgfscope}%
\begin{pgfscope}%
\pgfpathrectangle{\pgfqpoint{0.750000in}{0.525000in}}{\pgfqpoint{4.000000in}{1.056604in}} %
\pgfusepath{clip}%
\pgfsetrectcap%
\pgfsetroundjoin%
\pgfsetlinewidth{0.501875pt}%
\definecolor{currentstroke}{rgb}{1.000000,0.000000,1.000000}%
\pgfsetstrokecolor{currentstroke}%
\pgfsetdash{}{0pt}%
\pgfpathmoveto{\pgfqpoint{1.701998in}{1.053302in}}%
\pgfpathlineto{\pgfqpoint{1.722963in}{1.305756in}}%
\pgfpathlineto{\pgfqpoint{1.733445in}{1.407780in}}%
\pgfpathlineto{\pgfqpoint{1.742022in}{1.471479in}}%
\pgfpathlineto{\pgfqpoint{1.748692in}{1.506393in}}%
\pgfpathlineto{\pgfqpoint{1.754410in}{1.525193in}}%
\pgfpathlineto{\pgfqpoint{1.758222in}{1.531782in}}%
\pgfpathlineto{\pgfqpoint{1.761081in}{1.533547in}}%
\pgfpathlineto{\pgfqpoint{1.763940in}{1.532579in}}%
\pgfpathlineto{\pgfqpoint{1.766798in}{1.528881in}}%
\pgfpathlineto{\pgfqpoint{1.770610in}{1.519745in}}%
\pgfpathlineto{\pgfqpoint{1.775375in}{1.501704in}}%
\pgfpathlineto{\pgfqpoint{1.781093in}{1.470734in}}%
\pgfpathlineto{\pgfqpoint{1.788716in}{1.414827in}}%
\pgfpathlineto{\pgfqpoint{1.798246in}{1.324741in}}%
\pgfpathlineto{\pgfqpoint{1.810634in}{1.183068in}}%
\pgfpathlineto{\pgfqpoint{1.847799in}{0.741130in}}%
\pgfpathlineto{\pgfqpoint{1.857329in}{0.660101in}}%
\pgfpathlineto{\pgfqpoint{1.864952in}{0.612907in}}%
\pgfpathlineto{\pgfqpoint{1.870670in}{0.589101in}}%
\pgfpathlineto{\pgfqpoint{1.875435in}{0.577311in}}%
\pgfpathlineto{\pgfqpoint{1.879247in}{0.573289in}}%
\pgfpathlineto{\pgfqpoint{1.882105in}{0.573460in}}%
\pgfpathlineto{\pgfqpoint{1.884964in}{0.576363in}}%
\pgfpathlineto{\pgfqpoint{1.888776in}{0.584453in}}%
\pgfpathlineto{\pgfqpoint{1.893541in}{0.601222in}}%
\pgfpathlineto{\pgfqpoint{1.899259in}{0.630743in}}%
\pgfpathlineto{\pgfqpoint{1.905929in}{0.677271in}}%
\pgfpathlineto{\pgfqpoint{1.914506in}{0.753940in}}%
\pgfpathlineto{\pgfqpoint{1.925941in}{0.879144in}}%
\pgfpathlineto{\pgfqpoint{1.947859in}{1.152270in}}%
\pgfpathlineto{\pgfqpoint{1.963106in}{1.328467in}}%
\pgfpathlineto{\pgfqpoint{1.973589in}{1.425545in}}%
\pgfpathlineto{\pgfqpoint{1.981212in}{1.478699in}}%
\pgfpathlineto{\pgfqpoint{1.987883in}{1.511177in}}%
\pgfpathlineto{\pgfqpoint{1.992648in}{1.525746in}}%
\pgfpathlineto{\pgfqpoint{1.996460in}{1.532033in}}%
\pgfpathlineto{\pgfqpoint{1.999318in}{1.533571in}}%
\pgfpathlineto{\pgfqpoint{2.002177in}{1.532375in}}%
\pgfpathlineto{\pgfqpoint{2.005036in}{1.528451in}}%
\pgfpathlineto{\pgfqpoint{2.008848in}{1.519016in}}%
\pgfpathlineto{\pgfqpoint{2.013613in}{1.500613in}}%
\pgfpathlineto{\pgfqpoint{2.019330in}{1.469232in}}%
\pgfpathlineto{\pgfqpoint{2.026954in}{1.412831in}}%
\pgfpathlineto{\pgfqpoint{2.036483in}{1.322243in}}%
\pgfpathlineto{\pgfqpoint{2.048872in}{1.180157in}}%
\pgfpathlineto{\pgfqpoint{2.085084in}{0.748072in}}%
\pgfpathlineto{\pgfqpoint{2.094613in}{0.665374in}}%
\pgfpathlineto{\pgfqpoint{2.102237in}{0.616600in}}%
\pgfpathlineto{\pgfqpoint{2.107955in}{0.591508in}}%
\pgfpathlineto{\pgfqpoint{2.112720in}{0.578603in}}%
\pgfpathlineto{\pgfqpoint{2.116531in}{0.573674in}}%
\pgfpathlineto{\pgfqpoint{2.119390in}{0.573161in}}%
\pgfpathlineto{\pgfqpoint{2.122249in}{0.575382in}}%
\pgfpathlineto{\pgfqpoint{2.126061in}{0.582571in}}%
\pgfpathlineto{\pgfqpoint{2.130826in}{0.598243in}}%
\pgfpathlineto{\pgfqpoint{2.136543in}{0.626511in}}%
\pgfpathlineto{\pgfqpoint{2.143214in}{0.671701in}}%
\pgfpathlineto{\pgfqpoint{2.151791in}{0.746907in}}%
\pgfpathlineto{\pgfqpoint{2.163226in}{0.870730in}}%
\pgfpathlineto{\pgfqpoint{2.183238in}{1.119545in}}%
\pgfpathlineto{\pgfqpoint{2.200391in}{1.320991in}}%
\pgfpathlineto{\pgfqpoint{2.210873in}{1.419753in}}%
\pgfpathlineto{\pgfqpoint{2.219450in}{1.480092in}}%
\pgfpathlineto{\pgfqpoint{2.226121in}{1.512080in}}%
\pgfpathlineto{\pgfqpoint{2.230885in}{1.526280in}}%
\pgfpathlineto{\pgfqpoint{2.234697in}{1.532265in}}%
\pgfpathlineto{\pgfqpoint{2.237556in}{1.533576in}}%
\pgfpathlineto{\pgfqpoint{2.240415in}{1.532152in}}%
\pgfpathlineto{\pgfqpoint{2.243274in}{1.528001in}}%
\pgfpathlineto{\pgfqpoint{2.247086in}{1.518269in}}%
\pgfpathlineto{\pgfqpoint{2.251850in}{1.499504in}}%
\pgfpathlineto{\pgfqpoint{2.257568in}{1.467713in}}%
\pgfpathlineto{\pgfqpoint{2.265192in}{1.410821in}}%
\pgfpathlineto{\pgfqpoint{2.274721in}{1.319736in}}%
\pgfpathlineto{\pgfqpoint{2.287110in}{1.177241in}}%
\pgfpathlineto{\pgfqpoint{2.323322in}{0.745746in}}%
\pgfpathlineto{\pgfqpoint{2.332851in}{0.663601in}}%
\pgfpathlineto{\pgfqpoint{2.340475in}{0.615352in}}%
\pgfpathlineto{\pgfqpoint{2.346192in}{0.590687in}}%
\pgfpathlineto{\pgfqpoint{2.350957in}{0.578153in}}%
\pgfpathlineto{\pgfqpoint{2.354769in}{0.573527in}}%
\pgfpathlineto{\pgfqpoint{2.357628in}{0.573242in}}%
\pgfpathlineto{\pgfqpoint{2.360487in}{0.575690in}}%
\pgfpathlineto{\pgfqpoint{2.364299in}{0.583180in}}%
\pgfpathlineto{\pgfqpoint{2.369063in}{0.599218in}}%
\pgfpathlineto{\pgfqpoint{2.374781in}{0.627905in}}%
\pgfpathlineto{\pgfqpoint{2.381452in}{0.673542in}}%
\pgfpathlineto{\pgfqpoint{2.390028in}{0.749240in}}%
\pgfpathlineto{\pgfqpoint{2.401464in}{0.873528in}}%
\pgfpathlineto{\pgfqpoint{2.421476in}{1.122535in}}%
\pgfpathlineto{\pgfqpoint{2.438629in}{1.323493in}}%
\pgfpathlineto{\pgfqpoint{2.449111in}{1.421698in}}%
\pgfpathlineto{\pgfqpoint{2.457688in}{1.481469in}}%
\pgfpathlineto{\pgfqpoint{2.464358in}{1.512964in}}%
\pgfpathlineto{\pgfqpoint{2.469123in}{1.526795in}}%
\pgfpathlineto{\pgfqpoint{2.472935in}{1.532479in}}%
\pgfpathlineto{\pgfqpoint{2.475794in}{1.533561in}}%
\pgfpathlineto{\pgfqpoint{2.478653in}{1.531910in}}%
\pgfpathlineto{\pgfqpoint{2.481511in}{1.527533in}}%
\pgfpathlineto{\pgfqpoint{2.485323in}{1.517503in}}%
\pgfpathlineto{\pgfqpoint{2.490088in}{1.498378in}}%
\pgfpathlineto{\pgfqpoint{2.495806in}{1.466178in}}%
\pgfpathlineto{\pgfqpoint{2.503429in}{1.408797in}}%
\pgfpathlineto{\pgfqpoint{2.512959in}{1.317217in}}%
\pgfpathlineto{\pgfqpoint{2.525347in}{1.174321in}}%
\pgfpathlineto{\pgfqpoint{2.560606in}{0.752760in}}%
\pgfpathlineto{\pgfqpoint{2.570136in}{0.668966in}}%
\pgfpathlineto{\pgfqpoint{2.577759in}{0.619149in}}%
\pgfpathlineto{\pgfqpoint{2.584430in}{0.589885in}}%
\pgfpathlineto{\pgfqpoint{2.589195in}{0.577722in}}%
\pgfpathlineto{\pgfqpoint{2.593007in}{0.573398in}}%
\pgfpathlineto{\pgfqpoint{2.595866in}{0.573341in}}%
\pgfpathlineto{\pgfqpoint{2.598724in}{0.576017in}}%
\pgfpathlineto{\pgfqpoint{2.602536in}{0.583807in}}%
\pgfpathlineto{\pgfqpoint{2.607301in}{0.600211in}}%
\pgfpathlineto{\pgfqpoint{2.613019in}{0.629316in}}%
\pgfpathlineto{\pgfqpoint{2.619689in}{0.675399in}}%
\pgfpathlineto{\pgfqpoint{2.628266in}{0.751584in}}%
\pgfpathlineto{\pgfqpoint{2.639701in}{0.876333in}}%
\pgfpathlineto{\pgfqpoint{2.653996in}{1.053302in}}%
\pgfpathlineto{\pgfqpoint{2.653996in}{1.053302in}}%
\pgfusepath{stroke}%
\end{pgfscope}%
\begin{pgfscope}%
\pgfpathrectangle{\pgfqpoint{0.750000in}{0.525000in}}{\pgfqpoint{4.000000in}{1.056604in}} %
\pgfusepath{clip}%
\pgfsetrectcap%
\pgfsetroundjoin%
\pgfsetlinewidth{0.501875pt}%
\definecolor{currentstroke}{rgb}{0.000000,0.000000,1.000000}%
\pgfsetstrokecolor{currentstroke}%
\pgfsetdash{}{0pt}%
\pgfpathmoveto{\pgfqpoint{2.653996in}{1.053302in}}%
\pgfpathlineto{\pgfqpoint{2.675913in}{1.097077in}}%
\pgfpathlineto{\pgfqpoint{2.687349in}{1.115023in}}%
\pgfpathlineto{\pgfqpoint{2.695925in}{1.124890in}}%
\pgfpathlineto{\pgfqpoint{2.703549in}{1.130604in}}%
\pgfpathlineto{\pgfqpoint{2.710220in}{1.133048in}}%
\pgfpathlineto{\pgfqpoint{2.716890in}{1.133026in}}%
\pgfpathlineto{\pgfqpoint{2.723561in}{1.130538in}}%
\pgfpathlineto{\pgfqpoint{2.730232in}{1.125661in}}%
\pgfpathlineto{\pgfqpoint{2.737855in}{1.117358in}}%
\pgfpathlineto{\pgfqpoint{2.747385in}{1.103392in}}%
\pgfpathlineto{\pgfqpoint{2.759773in}{1.080679in}}%
\pgfpathlineto{\pgfqpoint{2.803609in}{0.995425in}}%
\pgfpathlineto{\pgfqpoint{2.813138in}{0.983482in}}%
\pgfpathlineto{\pgfqpoint{2.820762in}{0.977066in}}%
\pgfpathlineto{\pgfqpoint{2.827433in}{0.973970in}}%
\pgfpathlineto{\pgfqpoint{2.834103in}{0.973328in}}%
\pgfpathlineto{\pgfqpoint{2.840774in}{0.975160in}}%
\pgfpathlineto{\pgfqpoint{2.847445in}{0.979410in}}%
\pgfpathlineto{\pgfqpoint{2.855068in}{0.987054in}}%
\pgfpathlineto{\pgfqpoint{2.863645in}{0.998831in}}%
\pgfpathlineto{\pgfqpoint{2.875080in}{1.018733in}}%
\pgfpathlineto{\pgfqpoint{2.893186in}{1.055819in}}%
\pgfpathlineto{\pgfqpoint{2.914151in}{1.097498in}}%
\pgfpathlineto{\pgfqpoint{2.925587in}{1.115342in}}%
\pgfpathlineto{\pgfqpoint{2.934163in}{1.125114in}}%
\pgfpathlineto{\pgfqpoint{2.941787in}{1.130733in}}%
\pgfpathlineto{\pgfqpoint{2.948457in}{1.133090in}}%
\pgfpathlineto{\pgfqpoint{2.955128in}{1.132980in}}%
\pgfpathlineto{\pgfqpoint{2.961799in}{1.130404in}}%
\pgfpathlineto{\pgfqpoint{2.968469in}{1.125444in}}%
\pgfpathlineto{\pgfqpoint{2.976093in}{1.117055in}}%
\pgfpathlineto{\pgfqpoint{2.985622in}{1.102998in}}%
\pgfpathlineto{\pgfqpoint{2.998011in}{1.080206in}}%
\pgfpathlineto{\pgfqpoint{3.040894in}{0.996479in}}%
\pgfpathlineto{\pgfqpoint{3.050423in}{0.984233in}}%
\pgfpathlineto{\pgfqpoint{3.058047in}{0.977540in}}%
\pgfpathlineto{\pgfqpoint{3.064717in}{0.974185in}}%
\pgfpathlineto{\pgfqpoint{3.071388in}{0.973278in}}%
\pgfpathlineto{\pgfqpoint{3.078059in}{0.974847in}}%
\pgfpathlineto{\pgfqpoint{3.084729in}{0.978842in}}%
\pgfpathlineto{\pgfqpoint{3.092353in}{0.986218in}}%
\pgfpathlineto{\pgfqpoint{3.100929in}{0.997734in}}%
\pgfpathlineto{\pgfqpoint{3.111412in}{1.015589in}}%
\pgfpathlineto{\pgfqpoint{3.127612in}{1.048271in}}%
\pgfpathlineto{\pgfqpoint{3.151436in}{1.096231in}}%
\pgfpathlineto{\pgfqpoint{3.162871in}{1.114377in}}%
\pgfpathlineto{\pgfqpoint{3.171448in}{1.124434in}}%
\pgfpathlineto{\pgfqpoint{3.179071in}{1.130336in}}%
\pgfpathlineto{\pgfqpoint{3.185742in}{1.132955in}}%
\pgfpathlineto{\pgfqpoint{3.192413in}{1.133110in}}%
\pgfpathlineto{\pgfqpoint{3.199083in}{1.130796in}}%
\pgfpathlineto{\pgfqpoint{3.205754in}{1.126086in}}%
\pgfpathlineto{\pgfqpoint{3.213378in}{1.117957in}}%
\pgfpathlineto{\pgfqpoint{3.222907in}{1.104174in}}%
\pgfpathlineto{\pgfqpoint{3.235295in}{1.081623in}}%
\pgfpathlineto{\pgfqpoint{3.280084in}{0.994734in}}%
\pgfpathlineto{\pgfqpoint{3.289614in}{0.982995in}}%
\pgfpathlineto{\pgfqpoint{3.297237in}{0.976765in}}%
\pgfpathlineto{\pgfqpoint{3.303908in}{0.973842in}}%
\pgfpathlineto{\pgfqpoint{3.310579in}{0.973377in}}%
\pgfpathlineto{\pgfqpoint{3.317249in}{0.975385in}}%
\pgfpathlineto{\pgfqpoint{3.323920in}{0.979803in}}%
\pgfpathlineto{\pgfqpoint{3.331544in}{0.987624in}}%
\pgfpathlineto{\pgfqpoint{3.341073in}{1.001082in}}%
\pgfpathlineto{\pgfqpoint{3.352508in}{1.021482in}}%
\pgfpathlineto{\pgfqpoint{3.372520in}{1.062845in}}%
\pgfpathlineto{\pgfqpoint{3.390626in}{1.098334in}}%
\pgfpathlineto{\pgfqpoint{3.402062in}{1.115974in}}%
\pgfpathlineto{\pgfqpoint{3.410638in}{1.125553in}}%
\pgfpathlineto{\pgfqpoint{3.418262in}{1.130982in}}%
\pgfpathlineto{\pgfqpoint{3.424933in}{1.133165in}}%
\pgfpathlineto{\pgfqpoint{3.431603in}{1.132877in}}%
\pgfpathlineto{\pgfqpoint{3.438274in}{1.130128in}}%
\pgfpathlineto{\pgfqpoint{3.444945in}{1.125002in}}%
\pgfpathlineto{\pgfqpoint{3.452568in}{1.116441in}}%
\pgfpathlineto{\pgfqpoint{3.462098in}{1.102205in}}%
\pgfpathlineto{\pgfqpoint{3.474486in}{1.079255in}}%
\pgfpathlineto{\pgfqpoint{3.516416in}{0.997193in}}%
\pgfpathlineto{\pgfqpoint{3.525945in}{0.984748in}}%
\pgfpathlineto{\pgfqpoint{3.533569in}{0.977871in}}%
\pgfpathlineto{\pgfqpoint{3.540240in}{0.974345in}}%
\pgfpathlineto{\pgfqpoint{3.546910in}{0.973261in}}%
\pgfpathlineto{\pgfqpoint{3.553581in}{0.974653in}}%
\pgfpathlineto{\pgfqpoint{3.560252in}{0.978478in}}%
\pgfpathlineto{\pgfqpoint{3.567875in}{0.985674in}}%
\pgfpathlineto{\pgfqpoint{3.576452in}{0.997013in}}%
\pgfpathlineto{\pgfqpoint{3.586934in}{1.014704in}}%
\pgfpathlineto{\pgfqpoint{3.602182in}{1.045260in}}%
\pgfpathlineto{\pgfqpoint{3.605993in}{1.053302in}}%
\pgfpathlineto{\pgfqpoint{3.605993in}{1.053302in}}%
\pgfusepath{stroke}%
\end{pgfscope}%
\begin{pgfscope}%
\pgfpathrectangle{\pgfqpoint{0.750000in}{0.525000in}}{\pgfqpoint{4.000000in}{1.056604in}} %
\pgfusepath{clip}%
\pgfsetrectcap%
\pgfsetroundjoin%
\pgfsetlinewidth{0.501875pt}%
\definecolor{currentstroke}{rgb}{1.000000,0.000000,1.000000}%
\pgfsetstrokecolor{currentstroke}%
\pgfsetdash{}{0pt}%
\pgfpathmoveto{\pgfqpoint{3.605993in}{1.053302in}}%
\pgfpathlineto{\pgfqpoint{3.626958in}{1.305756in}}%
\pgfpathlineto{\pgfqpoint{3.637441in}{1.407780in}}%
\pgfpathlineto{\pgfqpoint{3.646017in}{1.471479in}}%
\pgfpathlineto{\pgfqpoint{3.652688in}{1.506393in}}%
\pgfpathlineto{\pgfqpoint{3.658406in}{1.525193in}}%
\pgfpathlineto{\pgfqpoint{3.662217in}{1.531782in}}%
\pgfpathlineto{\pgfqpoint{3.665076in}{1.533547in}}%
\pgfpathlineto{\pgfqpoint{3.667935in}{1.532579in}}%
\pgfpathlineto{\pgfqpoint{3.670794in}{1.528881in}}%
\pgfpathlineto{\pgfqpoint{3.674606in}{1.519745in}}%
\pgfpathlineto{\pgfqpoint{3.679371in}{1.501704in}}%
\pgfpathlineto{\pgfqpoint{3.685088in}{1.470734in}}%
\pgfpathlineto{\pgfqpoint{3.692712in}{1.414827in}}%
\pgfpathlineto{\pgfqpoint{3.702241in}{1.324741in}}%
\pgfpathlineto{\pgfqpoint{3.714630in}{1.183068in}}%
\pgfpathlineto{\pgfqpoint{3.751795in}{0.741130in}}%
\pgfpathlineto{\pgfqpoint{3.761324in}{0.660101in}}%
\pgfpathlineto{\pgfqpoint{3.768948in}{0.612907in}}%
\pgfpathlineto{\pgfqpoint{3.774666in}{0.589101in}}%
\pgfpathlineto{\pgfqpoint{3.779430in}{0.577311in}}%
\pgfpathlineto{\pgfqpoint{3.783242in}{0.573289in}}%
\pgfpathlineto{\pgfqpoint{3.786101in}{0.573460in}}%
\pgfpathlineto{\pgfqpoint{3.788960in}{0.576363in}}%
\pgfpathlineto{\pgfqpoint{3.792772in}{0.584453in}}%
\pgfpathlineto{\pgfqpoint{3.797536in}{0.601222in}}%
\pgfpathlineto{\pgfqpoint{3.803254in}{0.630743in}}%
\pgfpathlineto{\pgfqpoint{3.809925in}{0.677271in}}%
\pgfpathlineto{\pgfqpoint{3.818501in}{0.753940in}}%
\pgfpathlineto{\pgfqpoint{3.829937in}{0.879144in}}%
\pgfpathlineto{\pgfqpoint{3.851855in}{1.152270in}}%
\pgfpathlineto{\pgfqpoint{3.867102in}{1.328467in}}%
\pgfpathlineto{\pgfqpoint{3.877584in}{1.425545in}}%
\pgfpathlineto{\pgfqpoint{3.885208in}{1.478699in}}%
\pgfpathlineto{\pgfqpoint{3.891879in}{1.511177in}}%
\pgfpathlineto{\pgfqpoint{3.896643in}{1.525746in}}%
\pgfpathlineto{\pgfqpoint{3.900455in}{1.532033in}}%
\pgfpathlineto{\pgfqpoint{3.903314in}{1.533571in}}%
\pgfpathlineto{\pgfqpoint{3.906173in}{1.532375in}}%
\pgfpathlineto{\pgfqpoint{3.909032in}{1.528451in}}%
\pgfpathlineto{\pgfqpoint{3.912843in}{1.519016in}}%
\pgfpathlineto{\pgfqpoint{3.917608in}{1.500613in}}%
\pgfpathlineto{\pgfqpoint{3.923326in}{1.469232in}}%
\pgfpathlineto{\pgfqpoint{3.930950in}{1.412831in}}%
\pgfpathlineto{\pgfqpoint{3.940479in}{1.322243in}}%
\pgfpathlineto{\pgfqpoint{3.952867in}{1.180157in}}%
\pgfpathlineto{\pgfqpoint{3.989080in}{0.748072in}}%
\pgfpathlineto{\pgfqpoint{3.998609in}{0.665374in}}%
\pgfpathlineto{\pgfqpoint{4.006233in}{0.616600in}}%
\pgfpathlineto{\pgfqpoint{4.011950in}{0.591508in}}%
\pgfpathlineto{\pgfqpoint{4.016715in}{0.578603in}}%
\pgfpathlineto{\pgfqpoint{4.020527in}{0.573674in}}%
\pgfpathlineto{\pgfqpoint{4.023386in}{0.573161in}}%
\pgfpathlineto{\pgfqpoint{4.026245in}{0.575382in}}%
\pgfpathlineto{\pgfqpoint{4.030056in}{0.582571in}}%
\pgfpathlineto{\pgfqpoint{4.034821in}{0.598243in}}%
\pgfpathlineto{\pgfqpoint{4.040539in}{0.626511in}}%
\pgfpathlineto{\pgfqpoint{4.047210in}{0.671701in}}%
\pgfpathlineto{\pgfqpoint{4.055786in}{0.746907in}}%
\pgfpathlineto{\pgfqpoint{4.067221in}{0.870730in}}%
\pgfpathlineto{\pgfqpoint{4.087233in}{1.119545in}}%
\pgfpathlineto{\pgfqpoint{4.104387in}{1.320991in}}%
\pgfpathlineto{\pgfqpoint{4.114869in}{1.419753in}}%
\pgfpathlineto{\pgfqpoint{4.123446in}{1.480092in}}%
\pgfpathlineto{\pgfqpoint{4.130116in}{1.512080in}}%
\pgfpathlineto{\pgfqpoint{4.134881in}{1.526280in}}%
\pgfpathlineto{\pgfqpoint{4.138693in}{1.532265in}}%
\pgfpathlineto{\pgfqpoint{4.141552in}{1.533576in}}%
\pgfpathlineto{\pgfqpoint{4.144410in}{1.532152in}}%
\pgfpathlineto{\pgfqpoint{4.147269in}{1.528001in}}%
\pgfpathlineto{\pgfqpoint{4.151081in}{1.518269in}}%
\pgfpathlineto{\pgfqpoint{4.155846in}{1.499504in}}%
\pgfpathlineto{\pgfqpoint{4.161564in}{1.467713in}}%
\pgfpathlineto{\pgfqpoint{4.169187in}{1.410821in}}%
\pgfpathlineto{\pgfqpoint{4.178717in}{1.319736in}}%
\pgfpathlineto{\pgfqpoint{4.191105in}{1.177241in}}%
\pgfpathlineto{\pgfqpoint{4.227317in}{0.745746in}}%
\pgfpathlineto{\pgfqpoint{4.236847in}{0.663601in}}%
\pgfpathlineto{\pgfqpoint{4.244470in}{0.615352in}}%
\pgfpathlineto{\pgfqpoint{4.250188in}{0.590687in}}%
\pgfpathlineto{\pgfqpoint{4.254953in}{0.578153in}}%
\pgfpathlineto{\pgfqpoint{4.258765in}{0.573527in}}%
\pgfpathlineto{\pgfqpoint{4.261623in}{0.573242in}}%
\pgfpathlineto{\pgfqpoint{4.264482in}{0.575690in}}%
\pgfpathlineto{\pgfqpoint{4.268294in}{0.583180in}}%
\pgfpathlineto{\pgfqpoint{4.273059in}{0.599218in}}%
\pgfpathlineto{\pgfqpoint{4.278777in}{0.627905in}}%
\pgfpathlineto{\pgfqpoint{4.285447in}{0.673542in}}%
\pgfpathlineto{\pgfqpoint{4.294024in}{0.749240in}}%
\pgfpathlineto{\pgfqpoint{4.305459in}{0.873528in}}%
\pgfpathlineto{\pgfqpoint{4.325471in}{1.122535in}}%
\pgfpathlineto{\pgfqpoint{4.342624in}{1.323493in}}%
\pgfpathlineto{\pgfqpoint{4.353107in}{1.421698in}}%
\pgfpathlineto{\pgfqpoint{4.361683in}{1.481469in}}%
\pgfpathlineto{\pgfqpoint{4.368354in}{1.512964in}}%
\pgfpathlineto{\pgfqpoint{4.373119in}{1.526795in}}%
\pgfpathlineto{\pgfqpoint{4.376930in}{1.532479in}}%
\pgfpathlineto{\pgfqpoint{4.379789in}{1.533561in}}%
\pgfpathlineto{\pgfqpoint{4.382648in}{1.531910in}}%
\pgfpathlineto{\pgfqpoint{4.385507in}{1.527533in}}%
\pgfpathlineto{\pgfqpoint{4.389319in}{1.517503in}}%
\pgfpathlineto{\pgfqpoint{4.394084in}{1.498378in}}%
\pgfpathlineto{\pgfqpoint{4.399801in}{1.466178in}}%
\pgfpathlineto{\pgfqpoint{4.407425in}{1.408797in}}%
\pgfpathlineto{\pgfqpoint{4.416954in}{1.317217in}}%
\pgfpathlineto{\pgfqpoint{4.429343in}{1.174321in}}%
\pgfpathlineto{\pgfqpoint{4.464602in}{0.752760in}}%
\pgfpathlineto{\pgfqpoint{4.474131in}{0.668966in}}%
\pgfpathlineto{\pgfqpoint{4.481755in}{0.619149in}}%
\pgfpathlineto{\pgfqpoint{4.488426in}{0.589885in}}%
\pgfpathlineto{\pgfqpoint{4.493190in}{0.577722in}}%
\pgfpathlineto{\pgfqpoint{4.497002in}{0.573398in}}%
\pgfpathlineto{\pgfqpoint{4.499861in}{0.573341in}}%
\pgfpathlineto{\pgfqpoint{4.502720in}{0.576017in}}%
\pgfpathlineto{\pgfqpoint{4.506532in}{0.583807in}}%
\pgfpathlineto{\pgfqpoint{4.511297in}{0.600211in}}%
\pgfpathlineto{\pgfqpoint{4.517014in}{0.629316in}}%
\pgfpathlineto{\pgfqpoint{4.523685in}{0.675399in}}%
\pgfpathlineto{\pgfqpoint{4.532261in}{0.751584in}}%
\pgfpathlineto{\pgfqpoint{4.543697in}{0.876333in}}%
\pgfpathlineto{\pgfqpoint{4.557991in}{1.053302in}}%
\pgfpathlineto{\pgfqpoint{4.557991in}{1.053302in}}%
\pgfusepath{stroke}%
\end{pgfscope}%
\begin{pgfscope}%
\pgfsetrectcap%
\pgfsetmiterjoin%
\pgfsetlinewidth{0.501875pt}%
\definecolor{currentstroke}{rgb}{0.000000,0.000000,0.000000}%
\pgfsetstrokecolor{currentstroke}%
\pgfsetdash{}{0pt}%
\pgfpathmoveto{\pgfqpoint{4.750000in}{0.525000in}}%
\pgfpathlineto{\pgfqpoint{4.750000in}{1.581604in}}%
\pgfusepath{stroke}%
\end{pgfscope}%
\begin{pgfscope}%
\pgfsetrectcap%
\pgfsetmiterjoin%
\pgfsetlinewidth{0.501875pt}%
\definecolor{currentstroke}{rgb}{0.000000,0.000000,0.000000}%
\pgfsetstrokecolor{currentstroke}%
\pgfsetdash{}{0pt}%
\pgfpathmoveto{\pgfqpoint{0.750000in}{0.525000in}}%
\pgfpathlineto{\pgfqpoint{0.750000in}{1.581604in}}%
\pgfusepath{stroke}%
\end{pgfscope}%
\begin{pgfscope}%
\pgfsetrectcap%
\pgfsetmiterjoin%
\pgfsetlinewidth{0.501875pt}%
\definecolor{currentstroke}{rgb}{0.000000,0.000000,0.000000}%
\pgfsetstrokecolor{currentstroke}%
\pgfsetdash{}{0pt}%
\pgfpathmoveto{\pgfqpoint{0.750000in}{0.525000in}}%
\pgfpathlineto{\pgfqpoint{4.750000in}{0.525000in}}%
\pgfusepath{stroke}%
\end{pgfscope}%
\begin{pgfscope}%
\pgfsetrectcap%
\pgfsetmiterjoin%
\pgfsetlinewidth{0.501875pt}%
\definecolor{currentstroke}{rgb}{0.000000,0.000000,0.000000}%
\pgfsetstrokecolor{currentstroke}%
\pgfsetdash{}{0pt}%
\pgfpathmoveto{\pgfqpoint{0.750000in}{1.581604in}}%
\pgfpathlineto{\pgfqpoint{4.750000in}{1.581604in}}%
\pgfusepath{stroke}%
\end{pgfscope}%
\begin{pgfscope}%
\pgftext[x=2.750000in,y=0.455556in,,top]{\rmfamily\fontsize{9.000000}{10.800000}\selectfont Zeit}%
\end{pgfscope}%
\begin{pgfscope}%
\pgftext[x=0.680556in,y=1.053302in,,bottom,rotate=90.000000]{\rmfamily\fontsize{9.000000}{10.800000}\selectfont Spannung}%
\end{pgfscope}%
\begin{pgfscope}%
\pgftext[x=2.750000in,y=1.651048in,,base]{\rmfamily\fontsize{11.000000}{13.200000}\selectfont Moduliertes Signal}%
\end{pgfscope}%
\end{pgfpicture}%
\makeatother%
\endgroup%

    \caption{%
        \emph{Amplitude-shift  keying}: Oben   sind  die   zu  \"ubertragenden
        digitalen  Daten als  \code{1} und  \code{0} abgebildet. Die  mittlere
        Abbildung stellt eine typische Umsetzung des Konzepts mit harmonischer
        Tr\"agerschwingung und zwei verschiedenen Amplituden dar.\protect\\
        Das  unterste   Signal  ist  eine  vereinfachte   Darstellung  unserer
        Variante.  Die  Gleichspannung an  der Leitung  ist etwas  weniger als
        \SI{1}{\kilo\volt}; beim Kurzschluss eines Moduls erfolgt eine Abfolge
        von Spannungseinbr\"uchen auf der Leitung, welche die Daten kodiert.%
    }
    \label{fig:ask:concept}
\end{figure}

