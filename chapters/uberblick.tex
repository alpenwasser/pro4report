% **************************************************************************** %
\chapter{\"Uberblick}
\label{chap:uberblick}
% **************************************************************************** %
Dieses Kapitel  beschreibt zuerst die grobe  Idee unsers L\"osungskonzepts. Es
wird  dargelegt, wie  unser System  in  eine Solaranlage  (bestehend oder  neu
aufgebaut) integriert wird, wie das System mit seiner Umgebung interagiert und
wie es zu bedienen ist.


% ---------------------------------------------------------------------------- %
\section{Aufbau einer Solaranlage}
\label{sec:solaranlage:aufbau}
% ---------------------------------------------------------------------------- %

Grunds\"atzlicher Aufbau einer Solaranlage: Zelle -> Modul -> Strings -> Anlage


% ---------------------------------------------------------------------------- %
\section{Einbettung in Umwelt}
\label{sec:einbettung}
% ---------------------------------------------------------------------------- %

Hier  wird  beschrieben,  wie  unser System  physisch  mit  einer  Solaranlage
integriert wird und welche Schnittstellen es zu welchem Zweck zur Anlage hat.
\todo{kein eigener Abschnitt mehr n\"otig}


% ---------------------------------------------------------------------------- %
\section{Kernproblem: Kommunikation \"uber DC-Leitung}
\label{sec:commDCLine}
% ---------------------------------------------------------------------------- %

\todo{Referenz auf Konzept aus obigem Kapitel}
Es sind  zwei L\"osungsans\"atze untersucht  worden, um ein Signal  \"uber die
Gleichstromleitung zu senden:

\textbf{Frequency-shift keying}: Bei  der FSK (Frequenzumtastung  auf Deutsch)
wird  dem in  der Leitung  fliessenden Gleichstrom  ein (verh\"altnism\"assig)
kleines  Signal aufmoduliert,  welches  die  zu \"ubertragenden  Informationen
enth\"alt. Die  Frequenz   des  aufmodulierten   Anteils  wird   in  diskreten
Schritten variiert  und jeweils  einem Symbol zugeordnet. Bei  einer bin\"aren
Umsetzung  werden  zwei  Frequenzen  benutzt; eine  f\"ur  \code{0}  und  eine
f\"ur  \code{1}.   Das  Verfahren ist  schematisch  in  \fref{fig:fsk:concept}
dargestellt.

\begin{figure}[h!tb]
    \centering
    %% Creator: Matplotlib, PGF backend
%%
%% To include the figure in your LaTeX document, write
%%   \input{<filename>.pgf}
%%
%% Make sure the required packages are loaded in your preamble
%%   \usepackage{pgf}
%%
%% Figures using additional raster images can only be included by \input if
%% they are in the same directory as the main LaTeX file. For loading figures
%% from other directories you can use the `import` package
%%   \usepackage{import}
%% and then include the figures with
%%   \import{<path to file>}{<filename>.pgf}
%%
%% Matplotlib used the following preamble
%%   \usepackage{fontspec}
%%   \setmainfont{Bitstream Vera Serif}
%%   \setsansfont{Bitstream Vera Sans}
%%   \setmonofont{Bitstream Vera Sans Mono}
%%
\begingroup%
\makeatletter%
\begin{pgfpicture}%
\pgfpathrectangle{\pgfpointorigin}{\pgfqpoint{4.500000in}{3.500000in}}%
\pgfusepath{use as bounding box, clip}%
\begin{pgfscope}%
\pgfsetbuttcap%
\pgfsetmiterjoin%
\pgfsetlinewidth{0.000000pt}%
\definecolor{currentstroke}{rgb}{0.000000,0.000000,0.000000}%
\pgfsetstrokecolor{currentstroke}%
\pgfsetstrokeopacity{0.000000}%
\pgfsetdash{}{0pt}%
\pgfpathmoveto{\pgfqpoint{0.000000in}{0.000000in}}%
\pgfpathlineto{\pgfqpoint{4.500000in}{0.000000in}}%
\pgfpathlineto{\pgfqpoint{4.500000in}{3.500000in}}%
\pgfpathlineto{\pgfqpoint{0.000000in}{3.500000in}}%
\pgfpathclose%
\pgfusepath{}%
\end{pgfscope}%
\begin{pgfscope}%
\pgfsetbuttcap%
\pgfsetmiterjoin%
\pgfsetlinewidth{0.000000pt}%
\definecolor{currentstroke}{rgb}{0.000000,0.000000,0.000000}%
\pgfsetstrokecolor{currentstroke}%
\pgfsetstrokeopacity{0.000000}%
\pgfsetdash{}{0pt}%
\pgfpathmoveto{\pgfqpoint{0.675000in}{2.268396in}}%
\pgfpathlineto{\pgfqpoint{4.275000in}{2.268396in}}%
\pgfpathlineto{\pgfqpoint{4.275000in}{3.325000in}}%
\pgfpathlineto{\pgfqpoint{0.675000in}{3.325000in}}%
\pgfpathclose%
\pgfusepath{}%
\end{pgfscope}%
\begin{pgfscope}%
\pgfpathrectangle{\pgfqpoint{0.675000in}{2.268396in}}{\pgfqpoint{3.600000in}{1.056604in}} %
\pgfusepath{clip}%
\pgfsetrectcap%
\pgfsetroundjoin%
\pgfsetlinewidth{0.501875pt}%
\definecolor{currentstroke}{rgb}{0.000000,0.000000,1.000000}%
\pgfsetstrokecolor{currentstroke}%
\pgfsetdash{}{0pt}%
\pgfpathmoveto{\pgfqpoint{0.675000in}{2.356447in}}%
\pgfpathlineto{\pgfqpoint{1.531798in}{2.356447in}}%
\pgfusepath{stroke}%
\end{pgfscope}%
\begin{pgfscope}%
\pgfpathrectangle{\pgfqpoint{0.675000in}{2.268396in}}{\pgfqpoint{3.600000in}{1.056604in}} %
\pgfusepath{clip}%
\pgfsetrectcap%
\pgfsetroundjoin%
\pgfsetlinewidth{0.501875pt}%
\definecolor{currentstroke}{rgb}{0.501961,0.501961,0.501961}%
\pgfsetstrokecolor{currentstroke}%
\pgfsetdash{}{0pt}%
\pgfpathmoveto{\pgfqpoint{1.531798in}{2.356447in}}%
\pgfpathlineto{\pgfqpoint{1.531798in}{3.236950in}}%
\pgfusepath{stroke}%
\end{pgfscope}%
\begin{pgfscope}%
\pgfpathrectangle{\pgfqpoint{0.675000in}{2.268396in}}{\pgfqpoint{3.600000in}{1.056604in}} %
\pgfusepath{clip}%
\pgfsetrectcap%
\pgfsetroundjoin%
\pgfsetlinewidth{0.501875pt}%
\definecolor{currentstroke}{rgb}{1.000000,0.000000,1.000000}%
\pgfsetstrokecolor{currentstroke}%
\pgfsetdash{}{0pt}%
\pgfpathmoveto{\pgfqpoint{1.531798in}{3.236950in}}%
\pgfpathlineto{\pgfqpoint{2.388596in}{3.236950in}}%
\pgfusepath{stroke}%
\end{pgfscope}%
\begin{pgfscope}%
\pgfpathrectangle{\pgfqpoint{0.675000in}{2.268396in}}{\pgfqpoint{3.600000in}{1.056604in}} %
\pgfusepath{clip}%
\pgfsetrectcap%
\pgfsetroundjoin%
\pgfsetlinewidth{0.501875pt}%
\definecolor{currentstroke}{rgb}{0.501961,0.501961,0.501961}%
\pgfsetstrokecolor{currentstroke}%
\pgfsetdash{}{0pt}%
\pgfpathmoveto{\pgfqpoint{2.388596in}{3.236950in}}%
\pgfpathlineto{\pgfqpoint{2.388596in}{2.356447in}}%
\pgfusepath{stroke}%
\end{pgfscope}%
\begin{pgfscope}%
\pgfpathrectangle{\pgfqpoint{0.675000in}{2.268396in}}{\pgfqpoint{3.600000in}{1.056604in}} %
\pgfusepath{clip}%
\pgfsetrectcap%
\pgfsetroundjoin%
\pgfsetlinewidth{0.501875pt}%
\definecolor{currentstroke}{rgb}{0.000000,0.000000,1.000000}%
\pgfsetstrokecolor{currentstroke}%
\pgfsetdash{}{0pt}%
\pgfpathmoveto{\pgfqpoint{2.388596in}{2.356447in}}%
\pgfpathlineto{\pgfqpoint{3.245394in}{2.356447in}}%
\pgfusepath{stroke}%
\end{pgfscope}%
\begin{pgfscope}%
\pgfpathrectangle{\pgfqpoint{0.675000in}{2.268396in}}{\pgfqpoint{3.600000in}{1.056604in}} %
\pgfusepath{clip}%
\pgfsetrectcap%
\pgfsetroundjoin%
\pgfsetlinewidth{0.501875pt}%
\definecolor{currentstroke}{rgb}{0.501961,0.501961,0.501961}%
\pgfsetstrokecolor{currentstroke}%
\pgfsetdash{}{0pt}%
\pgfpathmoveto{\pgfqpoint{3.245394in}{2.356447in}}%
\pgfpathlineto{\pgfqpoint{3.245394in}{3.236950in}}%
\pgfusepath{stroke}%
\end{pgfscope}%
\begin{pgfscope}%
\pgfpathrectangle{\pgfqpoint{0.675000in}{2.268396in}}{\pgfqpoint{3.600000in}{1.056604in}} %
\pgfusepath{clip}%
\pgfsetrectcap%
\pgfsetroundjoin%
\pgfsetlinewidth{0.501875pt}%
\definecolor{currentstroke}{rgb}{1.000000,0.000000,1.000000}%
\pgfsetstrokecolor{currentstroke}%
\pgfsetdash{}{0pt}%
\pgfpathmoveto{\pgfqpoint{3.245394in}{3.236950in}}%
\pgfpathlineto{\pgfqpoint{4.102192in}{3.236950in}}%
\pgfusepath{stroke}%
\end{pgfscope}%
\begin{pgfscope}%
\pgfsetrectcap%
\pgfsetmiterjoin%
\pgfsetlinewidth{0.501875pt}%
\definecolor{currentstroke}{rgb}{0.000000,0.000000,0.000000}%
\pgfsetstrokecolor{currentstroke}%
\pgfsetdash{}{0pt}%
\pgfpathmoveto{\pgfqpoint{0.675000in}{3.325000in}}%
\pgfpathlineto{\pgfqpoint{4.275000in}{3.325000in}}%
\pgfusepath{stroke}%
\end{pgfscope}%
\begin{pgfscope}%
\pgfsetrectcap%
\pgfsetmiterjoin%
\pgfsetlinewidth{0.501875pt}%
\definecolor{currentstroke}{rgb}{0.000000,0.000000,0.000000}%
\pgfsetstrokecolor{currentstroke}%
\pgfsetdash{}{0pt}%
\pgfpathmoveto{\pgfqpoint{0.675000in}{2.268396in}}%
\pgfpathlineto{\pgfqpoint{0.675000in}{3.325000in}}%
\pgfusepath{stroke}%
\end{pgfscope}%
\begin{pgfscope}%
\pgfsetrectcap%
\pgfsetmiterjoin%
\pgfsetlinewidth{0.501875pt}%
\definecolor{currentstroke}{rgb}{0.000000,0.000000,0.000000}%
\pgfsetstrokecolor{currentstroke}%
\pgfsetdash{}{0pt}%
\pgfpathmoveto{\pgfqpoint{4.275000in}{2.268396in}}%
\pgfpathlineto{\pgfqpoint{4.275000in}{3.325000in}}%
\pgfusepath{stroke}%
\end{pgfscope}%
\begin{pgfscope}%
\pgfsetrectcap%
\pgfsetmiterjoin%
\pgfsetlinewidth{0.501875pt}%
\definecolor{currentstroke}{rgb}{0.000000,0.000000,0.000000}%
\pgfsetstrokecolor{currentstroke}%
\pgfsetdash{}{0pt}%
\pgfpathmoveto{\pgfqpoint{0.675000in}{2.268396in}}%
\pgfpathlineto{\pgfqpoint{4.275000in}{2.268396in}}%
\pgfusepath{stroke}%
\end{pgfscope}%
\begin{pgfscope}%
\pgfsetbuttcap%
\pgfsetroundjoin%
\definecolor{currentfill}{rgb}{0.000000,0.000000,0.000000}%
\pgfsetfillcolor{currentfill}%
\pgfsetlinewidth{0.501875pt}%
\definecolor{currentstroke}{rgb}{0.000000,0.000000,0.000000}%
\pgfsetstrokecolor{currentstroke}%
\pgfsetdash{}{0pt}%
\pgfsys@defobject{currentmarker}{\pgfqpoint{0.000000in}{0.000000in}}{\pgfqpoint{0.000000in}{0.055556in}}{%
\pgfpathmoveto{\pgfqpoint{0.000000in}{0.000000in}}%
\pgfpathlineto{\pgfqpoint{0.000000in}{0.055556in}}%
\pgfusepath{stroke,fill}%
}%
\begin{pgfscope}%
\pgfsys@transformshift{0.675000in}{2.268396in}%
\pgfsys@useobject{currentmarker}{}%
\end{pgfscope}%
\end{pgfscope}%
\begin{pgfscope}%
\pgfsetbuttcap%
\pgfsetroundjoin%
\definecolor{currentfill}{rgb}{0.000000,0.000000,0.000000}%
\pgfsetfillcolor{currentfill}%
\pgfsetlinewidth{0.501875pt}%
\definecolor{currentstroke}{rgb}{0.000000,0.000000,0.000000}%
\pgfsetstrokecolor{currentstroke}%
\pgfsetdash{}{0pt}%
\pgfsys@defobject{currentmarker}{\pgfqpoint{0.000000in}{-0.055556in}}{\pgfqpoint{0.000000in}{0.000000in}}{%
\pgfpathmoveto{\pgfqpoint{0.000000in}{0.000000in}}%
\pgfpathlineto{\pgfqpoint{0.000000in}{-0.055556in}}%
\pgfusepath{stroke,fill}%
}%
\begin{pgfscope}%
\pgfsys@transformshift{0.675000in}{3.325000in}%
\pgfsys@useobject{currentmarker}{}%
\end{pgfscope}%
\end{pgfscope}%
\begin{pgfscope}%
\pgftext[x=0.675000in,y=2.212841in,,top]{\rmfamily\fontsize{9.000000}{10.800000}\selectfont \(\displaystyle 0.0000\)}%
\end{pgfscope}%
\begin{pgfscope}%
\pgfsetbuttcap%
\pgfsetroundjoin%
\definecolor{currentfill}{rgb}{0.000000,0.000000,0.000000}%
\pgfsetfillcolor{currentfill}%
\pgfsetlinewidth{0.501875pt}%
\definecolor{currentstroke}{rgb}{0.000000,0.000000,0.000000}%
\pgfsetstrokecolor{currentstroke}%
\pgfsetdash{}{0pt}%
\pgfsys@defobject{currentmarker}{\pgfqpoint{0.000000in}{0.000000in}}{\pgfqpoint{0.000000in}{0.055556in}}{%
\pgfpathmoveto{\pgfqpoint{0.000000in}{0.000000in}}%
\pgfpathlineto{\pgfqpoint{0.000000in}{0.055556in}}%
\pgfusepath{stroke,fill}%
}%
\begin{pgfscope}%
\pgfsys@transformshift{1.125000in}{2.268396in}%
\pgfsys@useobject{currentmarker}{}%
\end{pgfscope}%
\end{pgfscope}%
\begin{pgfscope}%
\pgfsetbuttcap%
\pgfsetroundjoin%
\definecolor{currentfill}{rgb}{0.000000,0.000000,0.000000}%
\pgfsetfillcolor{currentfill}%
\pgfsetlinewidth{0.501875pt}%
\definecolor{currentstroke}{rgb}{0.000000,0.000000,0.000000}%
\pgfsetstrokecolor{currentstroke}%
\pgfsetdash{}{0pt}%
\pgfsys@defobject{currentmarker}{\pgfqpoint{0.000000in}{-0.055556in}}{\pgfqpoint{0.000000in}{0.000000in}}{%
\pgfpathmoveto{\pgfqpoint{0.000000in}{0.000000in}}%
\pgfpathlineto{\pgfqpoint{0.000000in}{-0.055556in}}%
\pgfusepath{stroke,fill}%
}%
\begin{pgfscope}%
\pgfsys@transformshift{1.125000in}{3.325000in}%
\pgfsys@useobject{currentmarker}{}%
\end{pgfscope}%
\end{pgfscope}%
\begin{pgfscope}%
\pgftext[x=1.125000in,y=2.212841in,,top]{\rmfamily\fontsize{9.000000}{10.800000}\selectfont \(\displaystyle 0.0002\)}%
\end{pgfscope}%
\begin{pgfscope}%
\pgfsetbuttcap%
\pgfsetroundjoin%
\definecolor{currentfill}{rgb}{0.000000,0.000000,0.000000}%
\pgfsetfillcolor{currentfill}%
\pgfsetlinewidth{0.501875pt}%
\definecolor{currentstroke}{rgb}{0.000000,0.000000,0.000000}%
\pgfsetstrokecolor{currentstroke}%
\pgfsetdash{}{0pt}%
\pgfsys@defobject{currentmarker}{\pgfqpoint{0.000000in}{0.000000in}}{\pgfqpoint{0.000000in}{0.055556in}}{%
\pgfpathmoveto{\pgfqpoint{0.000000in}{0.000000in}}%
\pgfpathlineto{\pgfqpoint{0.000000in}{0.055556in}}%
\pgfusepath{stroke,fill}%
}%
\begin{pgfscope}%
\pgfsys@transformshift{1.575000in}{2.268396in}%
\pgfsys@useobject{currentmarker}{}%
\end{pgfscope}%
\end{pgfscope}%
\begin{pgfscope}%
\pgfsetbuttcap%
\pgfsetroundjoin%
\definecolor{currentfill}{rgb}{0.000000,0.000000,0.000000}%
\pgfsetfillcolor{currentfill}%
\pgfsetlinewidth{0.501875pt}%
\definecolor{currentstroke}{rgb}{0.000000,0.000000,0.000000}%
\pgfsetstrokecolor{currentstroke}%
\pgfsetdash{}{0pt}%
\pgfsys@defobject{currentmarker}{\pgfqpoint{0.000000in}{-0.055556in}}{\pgfqpoint{0.000000in}{0.000000in}}{%
\pgfpathmoveto{\pgfqpoint{0.000000in}{0.000000in}}%
\pgfpathlineto{\pgfqpoint{0.000000in}{-0.055556in}}%
\pgfusepath{stroke,fill}%
}%
\begin{pgfscope}%
\pgfsys@transformshift{1.575000in}{3.325000in}%
\pgfsys@useobject{currentmarker}{}%
\end{pgfscope}%
\end{pgfscope}%
\begin{pgfscope}%
\pgftext[x=1.575000in,y=2.212841in,,top]{\rmfamily\fontsize{9.000000}{10.800000}\selectfont \(\displaystyle 0.0004\)}%
\end{pgfscope}%
\begin{pgfscope}%
\pgfsetbuttcap%
\pgfsetroundjoin%
\definecolor{currentfill}{rgb}{0.000000,0.000000,0.000000}%
\pgfsetfillcolor{currentfill}%
\pgfsetlinewidth{0.501875pt}%
\definecolor{currentstroke}{rgb}{0.000000,0.000000,0.000000}%
\pgfsetstrokecolor{currentstroke}%
\pgfsetdash{}{0pt}%
\pgfsys@defobject{currentmarker}{\pgfqpoint{0.000000in}{0.000000in}}{\pgfqpoint{0.000000in}{0.055556in}}{%
\pgfpathmoveto{\pgfqpoint{0.000000in}{0.000000in}}%
\pgfpathlineto{\pgfqpoint{0.000000in}{0.055556in}}%
\pgfusepath{stroke,fill}%
}%
\begin{pgfscope}%
\pgfsys@transformshift{2.025000in}{2.268396in}%
\pgfsys@useobject{currentmarker}{}%
\end{pgfscope}%
\end{pgfscope}%
\begin{pgfscope}%
\pgfsetbuttcap%
\pgfsetroundjoin%
\definecolor{currentfill}{rgb}{0.000000,0.000000,0.000000}%
\pgfsetfillcolor{currentfill}%
\pgfsetlinewidth{0.501875pt}%
\definecolor{currentstroke}{rgb}{0.000000,0.000000,0.000000}%
\pgfsetstrokecolor{currentstroke}%
\pgfsetdash{}{0pt}%
\pgfsys@defobject{currentmarker}{\pgfqpoint{0.000000in}{-0.055556in}}{\pgfqpoint{0.000000in}{0.000000in}}{%
\pgfpathmoveto{\pgfqpoint{0.000000in}{0.000000in}}%
\pgfpathlineto{\pgfqpoint{0.000000in}{-0.055556in}}%
\pgfusepath{stroke,fill}%
}%
\begin{pgfscope}%
\pgfsys@transformshift{2.025000in}{3.325000in}%
\pgfsys@useobject{currentmarker}{}%
\end{pgfscope}%
\end{pgfscope}%
\begin{pgfscope}%
\pgftext[x=2.025000in,y=2.212841in,,top]{\rmfamily\fontsize{9.000000}{10.800000}\selectfont \(\displaystyle 0.0006\)}%
\end{pgfscope}%
\begin{pgfscope}%
\pgfsetbuttcap%
\pgfsetroundjoin%
\definecolor{currentfill}{rgb}{0.000000,0.000000,0.000000}%
\pgfsetfillcolor{currentfill}%
\pgfsetlinewidth{0.501875pt}%
\definecolor{currentstroke}{rgb}{0.000000,0.000000,0.000000}%
\pgfsetstrokecolor{currentstroke}%
\pgfsetdash{}{0pt}%
\pgfsys@defobject{currentmarker}{\pgfqpoint{0.000000in}{0.000000in}}{\pgfqpoint{0.000000in}{0.055556in}}{%
\pgfpathmoveto{\pgfqpoint{0.000000in}{0.000000in}}%
\pgfpathlineto{\pgfqpoint{0.000000in}{0.055556in}}%
\pgfusepath{stroke,fill}%
}%
\begin{pgfscope}%
\pgfsys@transformshift{2.475000in}{2.268396in}%
\pgfsys@useobject{currentmarker}{}%
\end{pgfscope}%
\end{pgfscope}%
\begin{pgfscope}%
\pgfsetbuttcap%
\pgfsetroundjoin%
\definecolor{currentfill}{rgb}{0.000000,0.000000,0.000000}%
\pgfsetfillcolor{currentfill}%
\pgfsetlinewidth{0.501875pt}%
\definecolor{currentstroke}{rgb}{0.000000,0.000000,0.000000}%
\pgfsetstrokecolor{currentstroke}%
\pgfsetdash{}{0pt}%
\pgfsys@defobject{currentmarker}{\pgfqpoint{0.000000in}{-0.055556in}}{\pgfqpoint{0.000000in}{0.000000in}}{%
\pgfpathmoveto{\pgfqpoint{0.000000in}{0.000000in}}%
\pgfpathlineto{\pgfqpoint{0.000000in}{-0.055556in}}%
\pgfusepath{stroke,fill}%
}%
\begin{pgfscope}%
\pgfsys@transformshift{2.475000in}{3.325000in}%
\pgfsys@useobject{currentmarker}{}%
\end{pgfscope}%
\end{pgfscope}%
\begin{pgfscope}%
\pgftext[x=2.475000in,y=2.212841in,,top]{\rmfamily\fontsize{9.000000}{10.800000}\selectfont \(\displaystyle 0.0008\)}%
\end{pgfscope}%
\begin{pgfscope}%
\pgfsetbuttcap%
\pgfsetroundjoin%
\definecolor{currentfill}{rgb}{0.000000,0.000000,0.000000}%
\pgfsetfillcolor{currentfill}%
\pgfsetlinewidth{0.501875pt}%
\definecolor{currentstroke}{rgb}{0.000000,0.000000,0.000000}%
\pgfsetstrokecolor{currentstroke}%
\pgfsetdash{}{0pt}%
\pgfsys@defobject{currentmarker}{\pgfqpoint{0.000000in}{0.000000in}}{\pgfqpoint{0.000000in}{0.055556in}}{%
\pgfpathmoveto{\pgfqpoint{0.000000in}{0.000000in}}%
\pgfpathlineto{\pgfqpoint{0.000000in}{0.055556in}}%
\pgfusepath{stroke,fill}%
}%
\begin{pgfscope}%
\pgfsys@transformshift{2.925000in}{2.268396in}%
\pgfsys@useobject{currentmarker}{}%
\end{pgfscope}%
\end{pgfscope}%
\begin{pgfscope}%
\pgfsetbuttcap%
\pgfsetroundjoin%
\definecolor{currentfill}{rgb}{0.000000,0.000000,0.000000}%
\pgfsetfillcolor{currentfill}%
\pgfsetlinewidth{0.501875pt}%
\definecolor{currentstroke}{rgb}{0.000000,0.000000,0.000000}%
\pgfsetstrokecolor{currentstroke}%
\pgfsetdash{}{0pt}%
\pgfsys@defobject{currentmarker}{\pgfqpoint{0.000000in}{-0.055556in}}{\pgfqpoint{0.000000in}{0.000000in}}{%
\pgfpathmoveto{\pgfqpoint{0.000000in}{0.000000in}}%
\pgfpathlineto{\pgfqpoint{0.000000in}{-0.055556in}}%
\pgfusepath{stroke,fill}%
}%
\begin{pgfscope}%
\pgfsys@transformshift{2.925000in}{3.325000in}%
\pgfsys@useobject{currentmarker}{}%
\end{pgfscope}%
\end{pgfscope}%
\begin{pgfscope}%
\pgftext[x=2.925000in,y=2.212841in,,top]{\rmfamily\fontsize{9.000000}{10.800000}\selectfont \(\displaystyle 0.0010\)}%
\end{pgfscope}%
\begin{pgfscope}%
\pgfsetbuttcap%
\pgfsetroundjoin%
\definecolor{currentfill}{rgb}{0.000000,0.000000,0.000000}%
\pgfsetfillcolor{currentfill}%
\pgfsetlinewidth{0.501875pt}%
\definecolor{currentstroke}{rgb}{0.000000,0.000000,0.000000}%
\pgfsetstrokecolor{currentstroke}%
\pgfsetdash{}{0pt}%
\pgfsys@defobject{currentmarker}{\pgfqpoint{0.000000in}{0.000000in}}{\pgfqpoint{0.000000in}{0.055556in}}{%
\pgfpathmoveto{\pgfqpoint{0.000000in}{0.000000in}}%
\pgfpathlineto{\pgfqpoint{0.000000in}{0.055556in}}%
\pgfusepath{stroke,fill}%
}%
\begin{pgfscope}%
\pgfsys@transformshift{3.375000in}{2.268396in}%
\pgfsys@useobject{currentmarker}{}%
\end{pgfscope}%
\end{pgfscope}%
\begin{pgfscope}%
\pgfsetbuttcap%
\pgfsetroundjoin%
\definecolor{currentfill}{rgb}{0.000000,0.000000,0.000000}%
\pgfsetfillcolor{currentfill}%
\pgfsetlinewidth{0.501875pt}%
\definecolor{currentstroke}{rgb}{0.000000,0.000000,0.000000}%
\pgfsetstrokecolor{currentstroke}%
\pgfsetdash{}{0pt}%
\pgfsys@defobject{currentmarker}{\pgfqpoint{0.000000in}{-0.055556in}}{\pgfqpoint{0.000000in}{0.000000in}}{%
\pgfpathmoveto{\pgfqpoint{0.000000in}{0.000000in}}%
\pgfpathlineto{\pgfqpoint{0.000000in}{-0.055556in}}%
\pgfusepath{stroke,fill}%
}%
\begin{pgfscope}%
\pgfsys@transformshift{3.375000in}{3.325000in}%
\pgfsys@useobject{currentmarker}{}%
\end{pgfscope}%
\end{pgfscope}%
\begin{pgfscope}%
\pgftext[x=3.375000in,y=2.212841in,,top]{\rmfamily\fontsize{9.000000}{10.800000}\selectfont \(\displaystyle 0.0012\)}%
\end{pgfscope}%
\begin{pgfscope}%
\pgfsetbuttcap%
\pgfsetroundjoin%
\definecolor{currentfill}{rgb}{0.000000,0.000000,0.000000}%
\pgfsetfillcolor{currentfill}%
\pgfsetlinewidth{0.501875pt}%
\definecolor{currentstroke}{rgb}{0.000000,0.000000,0.000000}%
\pgfsetstrokecolor{currentstroke}%
\pgfsetdash{}{0pt}%
\pgfsys@defobject{currentmarker}{\pgfqpoint{0.000000in}{0.000000in}}{\pgfqpoint{0.000000in}{0.055556in}}{%
\pgfpathmoveto{\pgfqpoint{0.000000in}{0.000000in}}%
\pgfpathlineto{\pgfqpoint{0.000000in}{0.055556in}}%
\pgfusepath{stroke,fill}%
}%
\begin{pgfscope}%
\pgfsys@transformshift{3.825000in}{2.268396in}%
\pgfsys@useobject{currentmarker}{}%
\end{pgfscope}%
\end{pgfscope}%
\begin{pgfscope}%
\pgfsetbuttcap%
\pgfsetroundjoin%
\definecolor{currentfill}{rgb}{0.000000,0.000000,0.000000}%
\pgfsetfillcolor{currentfill}%
\pgfsetlinewidth{0.501875pt}%
\definecolor{currentstroke}{rgb}{0.000000,0.000000,0.000000}%
\pgfsetstrokecolor{currentstroke}%
\pgfsetdash{}{0pt}%
\pgfsys@defobject{currentmarker}{\pgfqpoint{0.000000in}{-0.055556in}}{\pgfqpoint{0.000000in}{0.000000in}}{%
\pgfpathmoveto{\pgfqpoint{0.000000in}{0.000000in}}%
\pgfpathlineto{\pgfqpoint{0.000000in}{-0.055556in}}%
\pgfusepath{stroke,fill}%
}%
\begin{pgfscope}%
\pgfsys@transformshift{3.825000in}{3.325000in}%
\pgfsys@useobject{currentmarker}{}%
\end{pgfscope}%
\end{pgfscope}%
\begin{pgfscope}%
\pgftext[x=3.825000in,y=2.212841in,,top]{\rmfamily\fontsize{9.000000}{10.800000}\selectfont \(\displaystyle 0.0014\)}%
\end{pgfscope}%
\begin{pgfscope}%
\pgfsetbuttcap%
\pgfsetroundjoin%
\definecolor{currentfill}{rgb}{0.000000,0.000000,0.000000}%
\pgfsetfillcolor{currentfill}%
\pgfsetlinewidth{0.501875pt}%
\definecolor{currentstroke}{rgb}{0.000000,0.000000,0.000000}%
\pgfsetstrokecolor{currentstroke}%
\pgfsetdash{}{0pt}%
\pgfsys@defobject{currentmarker}{\pgfqpoint{0.000000in}{0.000000in}}{\pgfqpoint{0.000000in}{0.055556in}}{%
\pgfpathmoveto{\pgfqpoint{0.000000in}{0.000000in}}%
\pgfpathlineto{\pgfqpoint{0.000000in}{0.055556in}}%
\pgfusepath{stroke,fill}%
}%
\begin{pgfscope}%
\pgfsys@transformshift{4.275000in}{2.268396in}%
\pgfsys@useobject{currentmarker}{}%
\end{pgfscope}%
\end{pgfscope}%
\begin{pgfscope}%
\pgfsetbuttcap%
\pgfsetroundjoin%
\definecolor{currentfill}{rgb}{0.000000,0.000000,0.000000}%
\pgfsetfillcolor{currentfill}%
\pgfsetlinewidth{0.501875pt}%
\definecolor{currentstroke}{rgb}{0.000000,0.000000,0.000000}%
\pgfsetstrokecolor{currentstroke}%
\pgfsetdash{}{0pt}%
\pgfsys@defobject{currentmarker}{\pgfqpoint{0.000000in}{-0.055556in}}{\pgfqpoint{0.000000in}{0.000000in}}{%
\pgfpathmoveto{\pgfqpoint{0.000000in}{0.000000in}}%
\pgfpathlineto{\pgfqpoint{0.000000in}{-0.055556in}}%
\pgfusepath{stroke,fill}%
}%
\begin{pgfscope}%
\pgfsys@transformshift{4.275000in}{3.325000in}%
\pgfsys@useobject{currentmarker}{}%
\end{pgfscope}%
\end{pgfscope}%
\begin{pgfscope}%
\pgftext[x=4.275000in,y=2.212841in,,top]{\rmfamily\fontsize{9.000000}{10.800000}\selectfont \(\displaystyle 0.0016\)}%
\end{pgfscope}%
\begin{pgfscope}%
\pgftext[x=2.475000in,y=2.022425in,,top]{\rmfamily\fontsize{9.000000}{10.800000}\selectfont Zeit (s)}%
\end{pgfscope}%
\begin{pgfscope}%
\pgfsetbuttcap%
\pgfsetroundjoin%
\definecolor{currentfill}{rgb}{0.000000,0.000000,0.000000}%
\pgfsetfillcolor{currentfill}%
\pgfsetlinewidth{0.501875pt}%
\definecolor{currentstroke}{rgb}{0.000000,0.000000,0.000000}%
\pgfsetstrokecolor{currentstroke}%
\pgfsetdash{}{0pt}%
\pgfsys@defobject{currentmarker}{\pgfqpoint{0.000000in}{0.000000in}}{\pgfqpoint{0.055556in}{0.000000in}}{%
\pgfpathmoveto{\pgfqpoint{0.000000in}{0.000000in}}%
\pgfpathlineto{\pgfqpoint{0.055556in}{0.000000in}}%
\pgfusepath{stroke,fill}%
}%
\begin{pgfscope}%
\pgfsys@transformshift{0.675000in}{2.356447in}%
\pgfsys@useobject{currentmarker}{}%
\end{pgfscope}%
\end{pgfscope}%
\begin{pgfscope}%
\pgfsetbuttcap%
\pgfsetroundjoin%
\definecolor{currentfill}{rgb}{0.000000,0.000000,0.000000}%
\pgfsetfillcolor{currentfill}%
\pgfsetlinewidth{0.501875pt}%
\definecolor{currentstroke}{rgb}{0.000000,0.000000,0.000000}%
\pgfsetstrokecolor{currentstroke}%
\pgfsetdash{}{0pt}%
\pgfsys@defobject{currentmarker}{\pgfqpoint{-0.055556in}{0.000000in}}{\pgfqpoint{0.000000in}{0.000000in}}{%
\pgfpathmoveto{\pgfqpoint{0.000000in}{0.000000in}}%
\pgfpathlineto{\pgfqpoint{-0.055556in}{0.000000in}}%
\pgfusepath{stroke,fill}%
}%
\begin{pgfscope}%
\pgfsys@transformshift{4.275000in}{2.356447in}%
\pgfsys@useobject{currentmarker}{}%
\end{pgfscope}%
\end{pgfscope}%
\begin{pgfscope}%
\pgftext[x=0.619444in,y=2.356447in,right,]{\rmfamily\fontsize{9.000000}{10.800000}\selectfont \(\displaystyle 0.0\)}%
\end{pgfscope}%
\begin{pgfscope}%
\pgfsetbuttcap%
\pgfsetroundjoin%
\definecolor{currentfill}{rgb}{0.000000,0.000000,0.000000}%
\pgfsetfillcolor{currentfill}%
\pgfsetlinewidth{0.501875pt}%
\definecolor{currentstroke}{rgb}{0.000000,0.000000,0.000000}%
\pgfsetstrokecolor{currentstroke}%
\pgfsetdash{}{0pt}%
\pgfsys@defobject{currentmarker}{\pgfqpoint{0.000000in}{0.000000in}}{\pgfqpoint{0.055556in}{0.000000in}}{%
\pgfpathmoveto{\pgfqpoint{0.000000in}{0.000000in}}%
\pgfpathlineto{\pgfqpoint{0.055556in}{0.000000in}}%
\pgfusepath{stroke,fill}%
}%
\begin{pgfscope}%
\pgfsys@transformshift{0.675000in}{2.532547in}%
\pgfsys@useobject{currentmarker}{}%
\end{pgfscope}%
\end{pgfscope}%
\begin{pgfscope}%
\pgfsetbuttcap%
\pgfsetroundjoin%
\definecolor{currentfill}{rgb}{0.000000,0.000000,0.000000}%
\pgfsetfillcolor{currentfill}%
\pgfsetlinewidth{0.501875pt}%
\definecolor{currentstroke}{rgb}{0.000000,0.000000,0.000000}%
\pgfsetstrokecolor{currentstroke}%
\pgfsetdash{}{0pt}%
\pgfsys@defobject{currentmarker}{\pgfqpoint{-0.055556in}{0.000000in}}{\pgfqpoint{0.000000in}{0.000000in}}{%
\pgfpathmoveto{\pgfqpoint{0.000000in}{0.000000in}}%
\pgfpathlineto{\pgfqpoint{-0.055556in}{0.000000in}}%
\pgfusepath{stroke,fill}%
}%
\begin{pgfscope}%
\pgfsys@transformshift{4.275000in}{2.532547in}%
\pgfsys@useobject{currentmarker}{}%
\end{pgfscope}%
\end{pgfscope}%
\begin{pgfscope}%
\pgftext[x=0.619444in,y=2.532547in,right,]{\rmfamily\fontsize{9.000000}{10.800000}\selectfont \(\displaystyle 0.2\)}%
\end{pgfscope}%
\begin{pgfscope}%
\pgfsetbuttcap%
\pgfsetroundjoin%
\definecolor{currentfill}{rgb}{0.000000,0.000000,0.000000}%
\pgfsetfillcolor{currentfill}%
\pgfsetlinewidth{0.501875pt}%
\definecolor{currentstroke}{rgb}{0.000000,0.000000,0.000000}%
\pgfsetstrokecolor{currentstroke}%
\pgfsetdash{}{0pt}%
\pgfsys@defobject{currentmarker}{\pgfqpoint{0.000000in}{0.000000in}}{\pgfqpoint{0.055556in}{0.000000in}}{%
\pgfpathmoveto{\pgfqpoint{0.000000in}{0.000000in}}%
\pgfpathlineto{\pgfqpoint{0.055556in}{0.000000in}}%
\pgfusepath{stroke,fill}%
}%
\begin{pgfscope}%
\pgfsys@transformshift{0.675000in}{2.708648in}%
\pgfsys@useobject{currentmarker}{}%
\end{pgfscope}%
\end{pgfscope}%
\begin{pgfscope}%
\pgfsetbuttcap%
\pgfsetroundjoin%
\definecolor{currentfill}{rgb}{0.000000,0.000000,0.000000}%
\pgfsetfillcolor{currentfill}%
\pgfsetlinewidth{0.501875pt}%
\definecolor{currentstroke}{rgb}{0.000000,0.000000,0.000000}%
\pgfsetstrokecolor{currentstroke}%
\pgfsetdash{}{0pt}%
\pgfsys@defobject{currentmarker}{\pgfqpoint{-0.055556in}{0.000000in}}{\pgfqpoint{0.000000in}{0.000000in}}{%
\pgfpathmoveto{\pgfqpoint{0.000000in}{0.000000in}}%
\pgfpathlineto{\pgfqpoint{-0.055556in}{0.000000in}}%
\pgfusepath{stroke,fill}%
}%
\begin{pgfscope}%
\pgfsys@transformshift{4.275000in}{2.708648in}%
\pgfsys@useobject{currentmarker}{}%
\end{pgfscope}%
\end{pgfscope}%
\begin{pgfscope}%
\pgftext[x=0.619444in,y=2.708648in,right,]{\rmfamily\fontsize{9.000000}{10.800000}\selectfont \(\displaystyle 0.4\)}%
\end{pgfscope}%
\begin{pgfscope}%
\pgfsetbuttcap%
\pgfsetroundjoin%
\definecolor{currentfill}{rgb}{0.000000,0.000000,0.000000}%
\pgfsetfillcolor{currentfill}%
\pgfsetlinewidth{0.501875pt}%
\definecolor{currentstroke}{rgb}{0.000000,0.000000,0.000000}%
\pgfsetstrokecolor{currentstroke}%
\pgfsetdash{}{0pt}%
\pgfsys@defobject{currentmarker}{\pgfqpoint{0.000000in}{0.000000in}}{\pgfqpoint{0.055556in}{0.000000in}}{%
\pgfpathmoveto{\pgfqpoint{0.000000in}{0.000000in}}%
\pgfpathlineto{\pgfqpoint{0.055556in}{0.000000in}}%
\pgfusepath{stroke,fill}%
}%
\begin{pgfscope}%
\pgfsys@transformshift{0.675000in}{2.884748in}%
\pgfsys@useobject{currentmarker}{}%
\end{pgfscope}%
\end{pgfscope}%
\begin{pgfscope}%
\pgfsetbuttcap%
\pgfsetroundjoin%
\definecolor{currentfill}{rgb}{0.000000,0.000000,0.000000}%
\pgfsetfillcolor{currentfill}%
\pgfsetlinewidth{0.501875pt}%
\definecolor{currentstroke}{rgb}{0.000000,0.000000,0.000000}%
\pgfsetstrokecolor{currentstroke}%
\pgfsetdash{}{0pt}%
\pgfsys@defobject{currentmarker}{\pgfqpoint{-0.055556in}{0.000000in}}{\pgfqpoint{0.000000in}{0.000000in}}{%
\pgfpathmoveto{\pgfqpoint{0.000000in}{0.000000in}}%
\pgfpathlineto{\pgfqpoint{-0.055556in}{0.000000in}}%
\pgfusepath{stroke,fill}%
}%
\begin{pgfscope}%
\pgfsys@transformshift{4.275000in}{2.884748in}%
\pgfsys@useobject{currentmarker}{}%
\end{pgfscope}%
\end{pgfscope}%
\begin{pgfscope}%
\pgftext[x=0.619444in,y=2.884748in,right,]{\rmfamily\fontsize{9.000000}{10.800000}\selectfont \(\displaystyle 0.6\)}%
\end{pgfscope}%
\begin{pgfscope}%
\pgfsetbuttcap%
\pgfsetroundjoin%
\definecolor{currentfill}{rgb}{0.000000,0.000000,0.000000}%
\pgfsetfillcolor{currentfill}%
\pgfsetlinewidth{0.501875pt}%
\definecolor{currentstroke}{rgb}{0.000000,0.000000,0.000000}%
\pgfsetstrokecolor{currentstroke}%
\pgfsetdash{}{0pt}%
\pgfsys@defobject{currentmarker}{\pgfqpoint{0.000000in}{0.000000in}}{\pgfqpoint{0.055556in}{0.000000in}}{%
\pgfpathmoveto{\pgfqpoint{0.000000in}{0.000000in}}%
\pgfpathlineto{\pgfqpoint{0.055556in}{0.000000in}}%
\pgfusepath{stroke,fill}%
}%
\begin{pgfscope}%
\pgfsys@transformshift{0.675000in}{3.060849in}%
\pgfsys@useobject{currentmarker}{}%
\end{pgfscope}%
\end{pgfscope}%
\begin{pgfscope}%
\pgfsetbuttcap%
\pgfsetroundjoin%
\definecolor{currentfill}{rgb}{0.000000,0.000000,0.000000}%
\pgfsetfillcolor{currentfill}%
\pgfsetlinewidth{0.501875pt}%
\definecolor{currentstroke}{rgb}{0.000000,0.000000,0.000000}%
\pgfsetstrokecolor{currentstroke}%
\pgfsetdash{}{0pt}%
\pgfsys@defobject{currentmarker}{\pgfqpoint{-0.055556in}{0.000000in}}{\pgfqpoint{0.000000in}{0.000000in}}{%
\pgfpathmoveto{\pgfqpoint{0.000000in}{0.000000in}}%
\pgfpathlineto{\pgfqpoint{-0.055556in}{0.000000in}}%
\pgfusepath{stroke,fill}%
}%
\begin{pgfscope}%
\pgfsys@transformshift{4.275000in}{3.060849in}%
\pgfsys@useobject{currentmarker}{}%
\end{pgfscope}%
\end{pgfscope}%
\begin{pgfscope}%
\pgftext[x=0.619444in,y=3.060849in,right,]{\rmfamily\fontsize{9.000000}{10.800000}\selectfont \(\displaystyle 0.8\)}%
\end{pgfscope}%
\begin{pgfscope}%
\pgfsetbuttcap%
\pgfsetroundjoin%
\definecolor{currentfill}{rgb}{0.000000,0.000000,0.000000}%
\pgfsetfillcolor{currentfill}%
\pgfsetlinewidth{0.501875pt}%
\definecolor{currentstroke}{rgb}{0.000000,0.000000,0.000000}%
\pgfsetstrokecolor{currentstroke}%
\pgfsetdash{}{0pt}%
\pgfsys@defobject{currentmarker}{\pgfqpoint{0.000000in}{0.000000in}}{\pgfqpoint{0.055556in}{0.000000in}}{%
\pgfpathmoveto{\pgfqpoint{0.000000in}{0.000000in}}%
\pgfpathlineto{\pgfqpoint{0.055556in}{0.000000in}}%
\pgfusepath{stroke,fill}%
}%
\begin{pgfscope}%
\pgfsys@transformshift{0.675000in}{3.236950in}%
\pgfsys@useobject{currentmarker}{}%
\end{pgfscope}%
\end{pgfscope}%
\begin{pgfscope}%
\pgfsetbuttcap%
\pgfsetroundjoin%
\definecolor{currentfill}{rgb}{0.000000,0.000000,0.000000}%
\pgfsetfillcolor{currentfill}%
\pgfsetlinewidth{0.501875pt}%
\definecolor{currentstroke}{rgb}{0.000000,0.000000,0.000000}%
\pgfsetstrokecolor{currentstroke}%
\pgfsetdash{}{0pt}%
\pgfsys@defobject{currentmarker}{\pgfqpoint{-0.055556in}{0.000000in}}{\pgfqpoint{0.000000in}{0.000000in}}{%
\pgfpathmoveto{\pgfqpoint{0.000000in}{0.000000in}}%
\pgfpathlineto{\pgfqpoint{-0.055556in}{0.000000in}}%
\pgfusepath{stroke,fill}%
}%
\begin{pgfscope}%
\pgfsys@transformshift{4.275000in}{3.236950in}%
\pgfsys@useobject{currentmarker}{}%
\end{pgfscope}%
\end{pgfscope}%
\begin{pgfscope}%
\pgftext[x=0.619444in,y=3.236950in,right,]{\rmfamily\fontsize{9.000000}{10.800000}\selectfont \(\displaystyle 1.0\)}%
\end{pgfscope}%
\begin{pgfscope}%
\pgftext[x=0.385842in,y=2.796698in,,bottom,rotate=90.000000]{\rmfamily\fontsize{9.000000}{10.800000}\selectfont Symbol}%
\end{pgfscope}%
\begin{pgfscope}%
\pgftext[x=2.475000in,y=3.394444in,,base]{\rmfamily\fontsize{11.000000}{13.200000}\selectfont Daten}%
\end{pgfscope}%
\begin{pgfscope}%
\pgfsetbuttcap%
\pgfsetmiterjoin%
\pgfsetlinewidth{0.000000pt}%
\definecolor{currentstroke}{rgb}{0.000000,0.000000,0.000000}%
\pgfsetstrokecolor{currentstroke}%
\pgfsetstrokeopacity{0.000000}%
\pgfsetdash{}{0pt}%
\pgfpathmoveto{\pgfqpoint{0.675000in}{0.525000in}}%
\pgfpathlineto{\pgfqpoint{4.275000in}{0.525000in}}%
\pgfpathlineto{\pgfqpoint{4.275000in}{1.581604in}}%
\pgfpathlineto{\pgfqpoint{0.675000in}{1.581604in}}%
\pgfpathclose%
\pgfusepath{}%
\end{pgfscope}%
\begin{pgfscope}%
\pgfpathrectangle{\pgfqpoint{0.675000in}{0.525000in}}{\pgfqpoint{3.600000in}{1.056604in}} %
\pgfusepath{clip}%
\pgfsetrectcap%
\pgfsetroundjoin%
\pgfsetlinewidth{0.501875pt}%
\definecolor{currentstroke}{rgb}{0.000000,0.000000,1.000000}%
\pgfsetstrokecolor{currentstroke}%
\pgfsetdash{}{0pt}%
\pgfpathmoveto{\pgfqpoint{0.675000in}{0.573027in}}%
\pgfpathlineto{\pgfqpoint{0.679288in}{0.573977in}}%
\pgfpathlineto{\pgfqpoint{0.684434in}{0.577618in}}%
\pgfpathlineto{\pgfqpoint{0.690438in}{0.585286in}}%
\pgfpathlineto{\pgfqpoint{0.697299in}{0.598485in}}%
\pgfpathlineto{\pgfqpoint{0.705018in}{0.618827in}}%
\pgfpathlineto{\pgfqpoint{0.713595in}{0.647938in}}%
\pgfpathlineto{\pgfqpoint{0.723886in}{0.691280in}}%
\pgfpathlineto{\pgfqpoint{0.735894in}{0.752172in}}%
\pgfpathlineto{\pgfqpoint{0.750474in}{0.838430in}}%
\pgfpathlineto{\pgfqpoint{0.770200in}{0.969903in}}%
\pgfpathlineto{\pgfqpoint{0.818228in}{1.296050in}}%
\pgfpathlineto{\pgfqpoint{0.833666in}{1.382886in}}%
\pgfpathlineto{\pgfqpoint{0.845673in}{1.438992in}}%
\pgfpathlineto{\pgfqpoint{0.855965in}{1.477642in}}%
\pgfpathlineto{\pgfqpoint{0.865400in}{1.504612in}}%
\pgfpathlineto{\pgfqpoint{0.873118in}{1.520280in}}%
\pgfpathlineto{\pgfqpoint{0.879980in}{1.529192in}}%
\pgfpathlineto{\pgfqpoint{0.885126in}{1.532719in}}%
\pgfpathlineto{\pgfqpoint{0.890272in}{1.533517in}}%
\pgfpathlineto{\pgfqpoint{0.894560in}{1.532093in}}%
\pgfpathlineto{\pgfqpoint{0.899706in}{1.527886in}}%
\pgfpathlineto{\pgfqpoint{0.905709in}{1.519565in}}%
\pgfpathlineto{\pgfqpoint{0.912571in}{1.505636in}}%
\pgfpathlineto{\pgfqpoint{0.920290in}{1.484505in}}%
\pgfpathlineto{\pgfqpoint{0.929724in}{1.451214in}}%
\pgfpathlineto{\pgfqpoint{0.940016in}{1.406247in}}%
\pgfpathlineto{\pgfqpoint{0.952023in}{1.343724in}}%
\pgfpathlineto{\pgfqpoint{0.967461in}{1.250439in}}%
\pgfpathlineto{\pgfqpoint{0.988902in}{1.105306in}}%
\pgfpathlineto{\pgfqpoint{1.029212in}{0.830364in}}%
\pgfpathlineto{\pgfqpoint{1.045507in}{0.735997in}}%
\pgfpathlineto{\pgfqpoint{1.058372in}{0.674005in}}%
\pgfpathlineto{\pgfqpoint{1.069522in}{0.631102in}}%
\pgfpathlineto{\pgfqpoint{1.078956in}{0.603561in}}%
\pgfpathlineto{\pgfqpoint{1.086675in}{0.587403in}}%
\pgfpathlineto{\pgfqpoint{1.093536in}{0.578044in}}%
\pgfpathlineto{\pgfqpoint{1.098682in}{0.574176in}}%
\pgfpathlineto{\pgfqpoint{1.103828in}{0.573037in}}%
\pgfpathlineto{\pgfqpoint{1.108116in}{0.574176in}}%
\pgfpathlineto{\pgfqpoint{1.113262in}{0.578044in}}%
\pgfpathlineto{\pgfqpoint{1.119266in}{0.585973in}}%
\pgfpathlineto{\pgfqpoint{1.126127in}{0.599465in}}%
\pgfpathlineto{\pgfqpoint{1.133846in}{0.620123in}}%
\pgfpathlineto{\pgfqpoint{1.142422in}{0.649567in}}%
\pgfpathlineto{\pgfqpoint{1.152714in}{0.693273in}}%
\pgfpathlineto{\pgfqpoint{1.164721in}{0.754531in}}%
\pgfpathlineto{\pgfqpoint{1.179302in}{0.841136in}}%
\pgfpathlineto{\pgfqpoint{1.199028in}{0.972880in}}%
\pgfpathlineto{\pgfqpoint{1.246199in}{1.293439in}}%
\pgfpathlineto{\pgfqpoint{1.261636in}{1.380683in}}%
\pgfpathlineto{\pgfqpoint{1.273644in}{1.437184in}}%
\pgfpathlineto{\pgfqpoint{1.283936in}{1.476219in}}%
\pgfpathlineto{\pgfqpoint{1.293370in}{1.503570in}}%
\pgfpathlineto{\pgfqpoint{1.301089in}{1.519565in}}%
\pgfpathlineto{\pgfqpoint{1.307950in}{1.528775in}}%
\pgfpathlineto{\pgfqpoint{1.313096in}{1.532529in}}%
\pgfpathlineto{\pgfqpoint{1.318242in}{1.533555in}}%
\pgfpathlineto{\pgfqpoint{1.322530in}{1.532321in}}%
\pgfpathlineto{\pgfqpoint{1.327676in}{1.528340in}}%
\pgfpathlineto{\pgfqpoint{1.333680in}{1.520280in}}%
\pgfpathlineto{\pgfqpoint{1.340541in}{1.506643in}}%
\pgfpathlineto{\pgfqpoint{1.348260in}{1.485827in}}%
\pgfpathlineto{\pgfqpoint{1.356836in}{1.456217in}}%
\pgfpathlineto{\pgfqpoint{1.367128in}{1.412330in}}%
\pgfpathlineto{\pgfqpoint{1.379135in}{1.350889in}}%
\pgfpathlineto{\pgfqpoint{1.393715in}{1.264112in}}%
\pgfpathlineto{\pgfqpoint{1.413442in}{1.132235in}}%
\pgfpathlineto{\pgfqpoint{1.460613in}{0.811858in}}%
\pgfpathlineto{\pgfqpoint{1.476050in}{0.724818in}}%
\pgfpathlineto{\pgfqpoint{1.488058in}{0.668514in}}%
\pgfpathlineto{\pgfqpoint{1.498349in}{0.629671in}}%
\pgfpathlineto{\pgfqpoint{1.507784in}{0.602510in}}%
\pgfpathlineto{\pgfqpoint{1.515503in}{0.586679in}}%
\pgfpathlineto{\pgfqpoint{1.522364in}{0.577618in}}%
\pgfpathlineto{\pgfqpoint{1.527510in}{0.573977in}}%
\pgfpathlineto{\pgfqpoint{1.531798in}{0.573027in}}%
\pgfpathlineto{\pgfqpoint{1.531798in}{0.573027in}}%
\pgfusepath{stroke}%
\end{pgfscope}%
\begin{pgfscope}%
\pgfpathrectangle{\pgfqpoint{0.675000in}{0.525000in}}{\pgfqpoint{3.600000in}{1.056604in}} %
\pgfusepath{clip}%
\pgfsetrectcap%
\pgfsetroundjoin%
\pgfsetlinewidth{0.501875pt}%
\definecolor{currentstroke}{rgb}{1.000000,0.000000,1.000000}%
\pgfsetstrokecolor{currentstroke}%
\pgfsetdash{}{0pt}%
\pgfpathmoveto{\pgfqpoint{1.531798in}{0.573027in}}%
\pgfpathlineto{\pgfqpoint{1.533513in}{0.574395in}}%
\pgfpathlineto{\pgfqpoint{1.536086in}{0.581551in}}%
\pgfpathlineto{\pgfqpoint{1.539517in}{0.600462in}}%
\pgfpathlineto{\pgfqpoint{1.543805in}{0.638509in}}%
\pgfpathlineto{\pgfqpoint{1.549809in}{0.716108in}}%
\pgfpathlineto{\pgfqpoint{1.557528in}{0.849303in}}%
\pgfpathlineto{\pgfqpoint{1.572965in}{1.167731in}}%
\pgfpathlineto{\pgfqpoint{1.584115in}{1.373994in}}%
\pgfpathlineto{\pgfqpoint{1.590976in}{1.465792in}}%
\pgfpathlineto{\pgfqpoint{1.596122in}{1.510488in}}%
\pgfpathlineto{\pgfqpoint{1.600410in}{1.529969in}}%
\pgfpathlineto{\pgfqpoint{1.602983in}{1.533555in}}%
\pgfpathlineto{\pgfqpoint{1.604699in}{1.532529in}}%
\pgfpathlineto{\pgfqpoint{1.607272in}{1.525881in}}%
\pgfpathlineto{\pgfqpoint{1.610702in}{1.507631in}}%
\pgfpathlineto{\pgfqpoint{1.614991in}{1.470360in}}%
\pgfpathlineto{\pgfqpoint{1.620994in}{1.393707in}}%
\pgfpathlineto{\pgfqpoint{1.628713in}{1.261394in}}%
\pgfpathlineto{\pgfqpoint{1.643293in}{0.960994in}}%
\pgfpathlineto{\pgfqpoint{1.654443in}{0.749825in}}%
\pgfpathlineto{\pgfqpoint{1.662162in}{0.643151in}}%
\pgfpathlineto{\pgfqpoint{1.667308in}{0.597524in}}%
\pgfpathlineto{\pgfqpoint{1.671596in}{0.577211in}}%
\pgfpathlineto{\pgfqpoint{1.674169in}{0.573113in}}%
\pgfpathlineto{\pgfqpoint{1.675884in}{0.573797in}}%
\pgfpathlineto{\pgfqpoint{1.678457in}{0.579936in}}%
\pgfpathlineto{\pgfqpoint{1.681888in}{0.597524in}}%
\pgfpathlineto{\pgfqpoint{1.686176in}{0.634016in}}%
\pgfpathlineto{\pgfqpoint{1.692180in}{0.709715in}}%
\pgfpathlineto{\pgfqpoint{1.699899in}{0.841136in}}%
\pgfpathlineto{\pgfqpoint{1.714479in}{1.141159in}}%
\pgfpathlineto{\pgfqpoint{1.726486in}{1.367192in}}%
\pgfpathlineto{\pgfqpoint{1.733347in}{1.461077in}}%
\pgfpathlineto{\pgfqpoint{1.738493in}{1.507631in}}%
\pgfpathlineto{\pgfqpoint{1.742781in}{1.528775in}}%
\pgfpathlineto{\pgfqpoint{1.745354in}{1.533384in}}%
\pgfpathlineto{\pgfqpoint{1.747070in}{1.533042in}}%
\pgfpathlineto{\pgfqpoint{1.748785in}{1.529969in}}%
\pgfpathlineto{\pgfqpoint{1.751358in}{1.520280in}}%
\pgfpathlineto{\pgfqpoint{1.754788in}{1.498094in}}%
\pgfpathlineto{\pgfqpoint{1.759934in}{1.446069in}}%
\pgfpathlineto{\pgfqpoint{1.765938in}{1.360277in}}%
\pgfpathlineto{\pgfqpoint{1.774515in}{1.202576in}}%
\pgfpathlineto{\pgfqpoint{1.800244in}{0.703445in}}%
\pgfpathlineto{\pgfqpoint{1.807105in}{0.621436in}}%
\pgfpathlineto{\pgfqpoint{1.812251in}{0.585286in}}%
\pgfpathlineto{\pgfqpoint{1.815682in}{0.574395in}}%
\pgfpathlineto{\pgfqpoint{1.817397in}{0.573027in}}%
\pgfpathlineto{\pgfqpoint{1.819113in}{0.574395in}}%
\pgfpathlineto{\pgfqpoint{1.821686in}{0.581551in}}%
\pgfpathlineto{\pgfqpoint{1.825116in}{0.600462in}}%
\pgfpathlineto{\pgfqpoint{1.829405in}{0.638509in}}%
\pgfpathlineto{\pgfqpoint{1.835408in}{0.716108in}}%
\pgfpathlineto{\pgfqpoint{1.843127in}{0.849303in}}%
\pgfpathlineto{\pgfqpoint{1.858565in}{1.167731in}}%
\pgfpathlineto{\pgfqpoint{1.869714in}{1.373994in}}%
\pgfpathlineto{\pgfqpoint{1.876576in}{1.465792in}}%
\pgfpathlineto{\pgfqpoint{1.881722in}{1.510488in}}%
\pgfpathlineto{\pgfqpoint{1.886010in}{1.529969in}}%
\pgfpathlineto{\pgfqpoint{1.888583in}{1.533555in}}%
\pgfpathlineto{\pgfqpoint{1.890298in}{1.532529in}}%
\pgfpathlineto{\pgfqpoint{1.892871in}{1.525881in}}%
\pgfpathlineto{\pgfqpoint{1.896302in}{1.507631in}}%
\pgfpathlineto{\pgfqpoint{1.900590in}{1.470360in}}%
\pgfpathlineto{\pgfqpoint{1.906594in}{1.393707in}}%
\pgfpathlineto{\pgfqpoint{1.914312in}{1.261394in}}%
\pgfpathlineto{\pgfqpoint{1.928893in}{0.960994in}}%
\pgfpathlineto{\pgfqpoint{1.940042in}{0.749825in}}%
\pgfpathlineto{\pgfqpoint{1.947761in}{0.643151in}}%
\pgfpathlineto{\pgfqpoint{1.952907in}{0.597524in}}%
\pgfpathlineto{\pgfqpoint{1.957195in}{0.577211in}}%
\pgfpathlineto{\pgfqpoint{1.959768in}{0.573113in}}%
\pgfpathlineto{\pgfqpoint{1.961483in}{0.573797in}}%
\pgfpathlineto{\pgfqpoint{1.964056in}{0.579936in}}%
\pgfpathlineto{\pgfqpoint{1.967487in}{0.597524in}}%
\pgfpathlineto{\pgfqpoint{1.971775in}{0.634016in}}%
\pgfpathlineto{\pgfqpoint{1.977779in}{0.709715in}}%
\pgfpathlineto{\pgfqpoint{1.985498in}{0.841136in}}%
\pgfpathlineto{\pgfqpoint{2.000078in}{1.141159in}}%
\pgfpathlineto{\pgfqpoint{2.012085in}{1.367192in}}%
\pgfpathlineto{\pgfqpoint{2.018946in}{1.461077in}}%
\pgfpathlineto{\pgfqpoint{2.024092in}{1.507631in}}%
\pgfpathlineto{\pgfqpoint{2.028381in}{1.528775in}}%
\pgfpathlineto{\pgfqpoint{2.030954in}{1.533384in}}%
\pgfpathlineto{\pgfqpoint{2.032669in}{1.533042in}}%
\pgfpathlineto{\pgfqpoint{2.034384in}{1.529969in}}%
\pgfpathlineto{\pgfqpoint{2.036957in}{1.520280in}}%
\pgfpathlineto{\pgfqpoint{2.040388in}{1.498094in}}%
\pgfpathlineto{\pgfqpoint{2.045534in}{1.446069in}}%
\pgfpathlineto{\pgfqpoint{2.051537in}{1.360277in}}%
\pgfpathlineto{\pgfqpoint{2.060114in}{1.202576in}}%
\pgfpathlineto{\pgfqpoint{2.085844in}{0.703445in}}%
\pgfpathlineto{\pgfqpoint{2.092705in}{0.621436in}}%
\pgfpathlineto{\pgfqpoint{2.097851in}{0.585286in}}%
\pgfpathlineto{\pgfqpoint{2.101281in}{0.574395in}}%
\pgfpathlineto{\pgfqpoint{2.102997in}{0.573027in}}%
\pgfpathlineto{\pgfqpoint{2.104712in}{0.574395in}}%
\pgfpathlineto{\pgfqpoint{2.107285in}{0.581551in}}%
\pgfpathlineto{\pgfqpoint{2.110716in}{0.600462in}}%
\pgfpathlineto{\pgfqpoint{2.115004in}{0.638509in}}%
\pgfpathlineto{\pgfqpoint{2.121007in}{0.716108in}}%
\pgfpathlineto{\pgfqpoint{2.128726in}{0.849303in}}%
\pgfpathlineto{\pgfqpoint{2.144164in}{1.167731in}}%
\pgfpathlineto{\pgfqpoint{2.155314in}{1.373994in}}%
\pgfpathlineto{\pgfqpoint{2.162175in}{1.465792in}}%
\pgfpathlineto{\pgfqpoint{2.167321in}{1.510488in}}%
\pgfpathlineto{\pgfqpoint{2.171609in}{1.529969in}}%
\pgfpathlineto{\pgfqpoint{2.174182in}{1.533555in}}%
\pgfpathlineto{\pgfqpoint{2.175897in}{1.532529in}}%
\pgfpathlineto{\pgfqpoint{2.178470in}{1.525881in}}%
\pgfpathlineto{\pgfqpoint{2.181901in}{1.507631in}}%
\pgfpathlineto{\pgfqpoint{2.186189in}{1.470360in}}%
\pgfpathlineto{\pgfqpoint{2.192193in}{1.393707in}}%
\pgfpathlineto{\pgfqpoint{2.199912in}{1.261394in}}%
\pgfpathlineto{\pgfqpoint{2.214492in}{0.960994in}}%
\pgfpathlineto{\pgfqpoint{2.225641in}{0.749825in}}%
\pgfpathlineto{\pgfqpoint{2.233360in}{0.643151in}}%
\pgfpathlineto{\pgfqpoint{2.238506in}{0.597524in}}%
\pgfpathlineto{\pgfqpoint{2.242795in}{0.577211in}}%
\pgfpathlineto{\pgfqpoint{2.245367in}{0.573113in}}%
\pgfpathlineto{\pgfqpoint{2.247083in}{0.573797in}}%
\pgfpathlineto{\pgfqpoint{2.249656in}{0.579936in}}%
\pgfpathlineto{\pgfqpoint{2.253086in}{0.597524in}}%
\pgfpathlineto{\pgfqpoint{2.257375in}{0.634016in}}%
\pgfpathlineto{\pgfqpoint{2.263378in}{0.709715in}}%
\pgfpathlineto{\pgfqpoint{2.271097in}{0.841136in}}%
\pgfpathlineto{\pgfqpoint{2.285677in}{1.141159in}}%
\pgfpathlineto{\pgfqpoint{2.297684in}{1.367192in}}%
\pgfpathlineto{\pgfqpoint{2.304546in}{1.461077in}}%
\pgfpathlineto{\pgfqpoint{2.309692in}{1.507631in}}%
\pgfpathlineto{\pgfqpoint{2.313980in}{1.528775in}}%
\pgfpathlineto{\pgfqpoint{2.316553in}{1.533384in}}%
\pgfpathlineto{\pgfqpoint{2.318268in}{1.533042in}}%
\pgfpathlineto{\pgfqpoint{2.319984in}{1.529969in}}%
\pgfpathlineto{\pgfqpoint{2.322557in}{1.520280in}}%
\pgfpathlineto{\pgfqpoint{2.325987in}{1.498094in}}%
\pgfpathlineto{\pgfqpoint{2.331133in}{1.446069in}}%
\pgfpathlineto{\pgfqpoint{2.337137in}{1.360277in}}%
\pgfpathlineto{\pgfqpoint{2.345713in}{1.202576in}}%
\pgfpathlineto{\pgfqpoint{2.371443in}{0.703445in}}%
\pgfpathlineto{\pgfqpoint{2.378304in}{0.621436in}}%
\pgfpathlineto{\pgfqpoint{2.383450in}{0.585286in}}%
\pgfpathlineto{\pgfqpoint{2.386881in}{0.574395in}}%
\pgfpathlineto{\pgfqpoint{2.388596in}{0.573027in}}%
\pgfpathlineto{\pgfqpoint{2.388596in}{0.573027in}}%
\pgfusepath{stroke}%
\end{pgfscope}%
\begin{pgfscope}%
\pgfpathrectangle{\pgfqpoint{0.675000in}{0.525000in}}{\pgfqpoint{3.600000in}{1.056604in}} %
\pgfusepath{clip}%
\pgfsetrectcap%
\pgfsetroundjoin%
\pgfsetlinewidth{0.501875pt}%
\definecolor{currentstroke}{rgb}{0.000000,0.000000,1.000000}%
\pgfsetstrokecolor{currentstroke}%
\pgfsetdash{}{0pt}%
\pgfpathmoveto{\pgfqpoint{2.388596in}{0.573027in}}%
\pgfpathlineto{\pgfqpoint{2.392884in}{0.573977in}}%
\pgfpathlineto{\pgfqpoint{2.398030in}{0.577618in}}%
\pgfpathlineto{\pgfqpoint{2.404034in}{0.585286in}}%
\pgfpathlineto{\pgfqpoint{2.410895in}{0.598485in}}%
\pgfpathlineto{\pgfqpoint{2.418614in}{0.618827in}}%
\pgfpathlineto{\pgfqpoint{2.427190in}{0.647938in}}%
\pgfpathlineto{\pgfqpoint{2.437482in}{0.691280in}}%
\pgfpathlineto{\pgfqpoint{2.449490in}{0.752172in}}%
\pgfpathlineto{\pgfqpoint{2.464070in}{0.838430in}}%
\pgfpathlineto{\pgfqpoint{2.483796in}{0.969903in}}%
\pgfpathlineto{\pgfqpoint{2.531824in}{1.296050in}}%
\pgfpathlineto{\pgfqpoint{2.547262in}{1.382886in}}%
\pgfpathlineto{\pgfqpoint{2.559269in}{1.438992in}}%
\pgfpathlineto{\pgfqpoint{2.569561in}{1.477642in}}%
\pgfpathlineto{\pgfqpoint{2.578996in}{1.504612in}}%
\pgfpathlineto{\pgfqpoint{2.586714in}{1.520280in}}%
\pgfpathlineto{\pgfqpoint{2.593576in}{1.529192in}}%
\pgfpathlineto{\pgfqpoint{2.598722in}{1.532719in}}%
\pgfpathlineto{\pgfqpoint{2.603868in}{1.533517in}}%
\pgfpathlineto{\pgfqpoint{2.608156in}{1.532093in}}%
\pgfpathlineto{\pgfqpoint{2.613302in}{1.527886in}}%
\pgfpathlineto{\pgfqpoint{2.619305in}{1.519565in}}%
\pgfpathlineto{\pgfqpoint{2.626167in}{1.505636in}}%
\pgfpathlineto{\pgfqpoint{2.633886in}{1.484505in}}%
\pgfpathlineto{\pgfqpoint{2.643320in}{1.451214in}}%
\pgfpathlineto{\pgfqpoint{2.653612in}{1.406247in}}%
\pgfpathlineto{\pgfqpoint{2.665619in}{1.343724in}}%
\pgfpathlineto{\pgfqpoint{2.681057in}{1.250439in}}%
\pgfpathlineto{\pgfqpoint{2.702498in}{1.105306in}}%
\pgfpathlineto{\pgfqpoint{2.742808in}{0.830364in}}%
\pgfpathlineto{\pgfqpoint{2.759103in}{0.735997in}}%
\pgfpathlineto{\pgfqpoint{2.771968in}{0.674005in}}%
\pgfpathlineto{\pgfqpoint{2.783118in}{0.631102in}}%
\pgfpathlineto{\pgfqpoint{2.792552in}{0.603561in}}%
\pgfpathlineto{\pgfqpoint{2.800271in}{0.587403in}}%
\pgfpathlineto{\pgfqpoint{2.807132in}{0.578044in}}%
\pgfpathlineto{\pgfqpoint{2.812278in}{0.574176in}}%
\pgfpathlineto{\pgfqpoint{2.817424in}{0.573037in}}%
\pgfpathlineto{\pgfqpoint{2.821712in}{0.574176in}}%
\pgfpathlineto{\pgfqpoint{2.826858in}{0.578044in}}%
\pgfpathlineto{\pgfqpoint{2.832862in}{0.585973in}}%
\pgfpathlineto{\pgfqpoint{2.839723in}{0.599465in}}%
\pgfpathlineto{\pgfqpoint{2.847442in}{0.620123in}}%
\pgfpathlineto{\pgfqpoint{2.856018in}{0.649567in}}%
\pgfpathlineto{\pgfqpoint{2.866310in}{0.693273in}}%
\pgfpathlineto{\pgfqpoint{2.878317in}{0.754531in}}%
\pgfpathlineto{\pgfqpoint{2.892898in}{0.841136in}}%
\pgfpathlineto{\pgfqpoint{2.912624in}{0.972880in}}%
\pgfpathlineto{\pgfqpoint{2.959795in}{1.293439in}}%
\pgfpathlineto{\pgfqpoint{2.975232in}{1.380683in}}%
\pgfpathlineto{\pgfqpoint{2.987240in}{1.437184in}}%
\pgfpathlineto{\pgfqpoint{2.997532in}{1.476219in}}%
\pgfpathlineto{\pgfqpoint{3.006966in}{1.503570in}}%
\pgfpathlineto{\pgfqpoint{3.014685in}{1.519565in}}%
\pgfpathlineto{\pgfqpoint{3.021546in}{1.528775in}}%
\pgfpathlineto{\pgfqpoint{3.026692in}{1.532529in}}%
\pgfpathlineto{\pgfqpoint{3.031838in}{1.533555in}}%
\pgfpathlineto{\pgfqpoint{3.036126in}{1.532321in}}%
\pgfpathlineto{\pgfqpoint{3.041272in}{1.528340in}}%
\pgfpathlineto{\pgfqpoint{3.047276in}{1.520280in}}%
\pgfpathlineto{\pgfqpoint{3.054137in}{1.506643in}}%
\pgfpathlineto{\pgfqpoint{3.061856in}{1.485827in}}%
\pgfpathlineto{\pgfqpoint{3.070432in}{1.456217in}}%
\pgfpathlineto{\pgfqpoint{3.080724in}{1.412330in}}%
\pgfpathlineto{\pgfqpoint{3.092731in}{1.350889in}}%
\pgfpathlineto{\pgfqpoint{3.107311in}{1.264112in}}%
\pgfpathlineto{\pgfqpoint{3.127038in}{1.132235in}}%
\pgfpathlineto{\pgfqpoint{3.174209in}{0.811858in}}%
\pgfpathlineto{\pgfqpoint{3.189646in}{0.724818in}}%
\pgfpathlineto{\pgfqpoint{3.201654in}{0.668514in}}%
\pgfpathlineto{\pgfqpoint{3.211945in}{0.629671in}}%
\pgfpathlineto{\pgfqpoint{3.221380in}{0.602510in}}%
\pgfpathlineto{\pgfqpoint{3.229099in}{0.586679in}}%
\pgfpathlineto{\pgfqpoint{3.235960in}{0.577618in}}%
\pgfpathlineto{\pgfqpoint{3.241106in}{0.573977in}}%
\pgfpathlineto{\pgfqpoint{3.245394in}{0.573027in}}%
\pgfpathlineto{\pgfqpoint{3.245394in}{0.573027in}}%
\pgfusepath{stroke}%
\end{pgfscope}%
\begin{pgfscope}%
\pgfpathrectangle{\pgfqpoint{0.675000in}{0.525000in}}{\pgfqpoint{3.600000in}{1.056604in}} %
\pgfusepath{clip}%
\pgfsetrectcap%
\pgfsetroundjoin%
\pgfsetlinewidth{0.501875pt}%
\definecolor{currentstroke}{rgb}{1.000000,0.000000,1.000000}%
\pgfsetstrokecolor{currentstroke}%
\pgfsetdash{}{0pt}%
\pgfpathmoveto{\pgfqpoint{3.245394in}{0.573027in}}%
\pgfpathlineto{\pgfqpoint{3.247109in}{0.574395in}}%
\pgfpathlineto{\pgfqpoint{3.249682in}{0.581551in}}%
\pgfpathlineto{\pgfqpoint{3.253113in}{0.600462in}}%
\pgfpathlineto{\pgfqpoint{3.257401in}{0.638509in}}%
\pgfpathlineto{\pgfqpoint{3.263405in}{0.716108in}}%
\pgfpathlineto{\pgfqpoint{3.271124in}{0.849303in}}%
\pgfpathlineto{\pgfqpoint{3.286561in}{1.167731in}}%
\pgfpathlineto{\pgfqpoint{3.297711in}{1.373994in}}%
\pgfpathlineto{\pgfqpoint{3.304572in}{1.465792in}}%
\pgfpathlineto{\pgfqpoint{3.309718in}{1.510488in}}%
\pgfpathlineto{\pgfqpoint{3.314006in}{1.529969in}}%
\pgfpathlineto{\pgfqpoint{3.316579in}{1.533555in}}%
\pgfpathlineto{\pgfqpoint{3.318295in}{1.532529in}}%
\pgfpathlineto{\pgfqpoint{3.320868in}{1.525881in}}%
\pgfpathlineto{\pgfqpoint{3.324298in}{1.507631in}}%
\pgfpathlineto{\pgfqpoint{3.328587in}{1.470360in}}%
\pgfpathlineto{\pgfqpoint{3.334590in}{1.393707in}}%
\pgfpathlineto{\pgfqpoint{3.342309in}{1.261394in}}%
\pgfpathlineto{\pgfqpoint{3.356889in}{0.960994in}}%
\pgfpathlineto{\pgfqpoint{3.368039in}{0.749825in}}%
\pgfpathlineto{\pgfqpoint{3.375758in}{0.643151in}}%
\pgfpathlineto{\pgfqpoint{3.380904in}{0.597524in}}%
\pgfpathlineto{\pgfqpoint{3.385192in}{0.577211in}}%
\pgfpathlineto{\pgfqpoint{3.387765in}{0.573113in}}%
\pgfpathlineto{\pgfqpoint{3.389480in}{0.573797in}}%
\pgfpathlineto{\pgfqpoint{3.392053in}{0.579936in}}%
\pgfpathlineto{\pgfqpoint{3.395484in}{0.597524in}}%
\pgfpathlineto{\pgfqpoint{3.399772in}{0.634016in}}%
\pgfpathlineto{\pgfqpoint{3.405776in}{0.709715in}}%
\pgfpathlineto{\pgfqpoint{3.413494in}{0.841136in}}%
\pgfpathlineto{\pgfqpoint{3.428075in}{1.141159in}}%
\pgfpathlineto{\pgfqpoint{3.440082in}{1.367192in}}%
\pgfpathlineto{\pgfqpoint{3.446943in}{1.461077in}}%
\pgfpathlineto{\pgfqpoint{3.452089in}{1.507631in}}%
\pgfpathlineto{\pgfqpoint{3.456377in}{1.528775in}}%
\pgfpathlineto{\pgfqpoint{3.458950in}{1.533384in}}%
\pgfpathlineto{\pgfqpoint{3.460666in}{1.533042in}}%
\pgfpathlineto{\pgfqpoint{3.462381in}{1.529969in}}%
\pgfpathlineto{\pgfqpoint{3.464954in}{1.520280in}}%
\pgfpathlineto{\pgfqpoint{3.468384in}{1.498094in}}%
\pgfpathlineto{\pgfqpoint{3.473530in}{1.446069in}}%
\pgfpathlineto{\pgfqpoint{3.479534in}{1.360277in}}%
\pgfpathlineto{\pgfqpoint{3.488111in}{1.202576in}}%
\pgfpathlineto{\pgfqpoint{3.513840in}{0.703445in}}%
\pgfpathlineto{\pgfqpoint{3.520701in}{0.621436in}}%
\pgfpathlineto{\pgfqpoint{3.525847in}{0.585286in}}%
\pgfpathlineto{\pgfqpoint{3.529278in}{0.574395in}}%
\pgfpathlineto{\pgfqpoint{3.530993in}{0.573027in}}%
\pgfpathlineto{\pgfqpoint{3.532709in}{0.574395in}}%
\pgfpathlineto{\pgfqpoint{3.535282in}{0.581551in}}%
\pgfpathlineto{\pgfqpoint{3.538712in}{0.600462in}}%
\pgfpathlineto{\pgfqpoint{3.543001in}{0.638509in}}%
\pgfpathlineto{\pgfqpoint{3.549004in}{0.716108in}}%
\pgfpathlineto{\pgfqpoint{3.556723in}{0.849303in}}%
\pgfpathlineto{\pgfqpoint{3.572161in}{1.167731in}}%
\pgfpathlineto{\pgfqpoint{3.583310in}{1.373994in}}%
\pgfpathlineto{\pgfqpoint{3.590172in}{1.465792in}}%
\pgfpathlineto{\pgfqpoint{3.595317in}{1.510488in}}%
\pgfpathlineto{\pgfqpoint{3.599606in}{1.529969in}}%
\pgfpathlineto{\pgfqpoint{3.602179in}{1.533555in}}%
\pgfpathlineto{\pgfqpoint{3.603894in}{1.532529in}}%
\pgfpathlineto{\pgfqpoint{3.606467in}{1.525881in}}%
\pgfpathlineto{\pgfqpoint{3.609898in}{1.507631in}}%
\pgfpathlineto{\pgfqpoint{3.614186in}{1.470360in}}%
\pgfpathlineto{\pgfqpoint{3.620190in}{1.393707in}}%
\pgfpathlineto{\pgfqpoint{3.627908in}{1.261394in}}%
\pgfpathlineto{\pgfqpoint{3.642489in}{0.960994in}}%
\pgfpathlineto{\pgfqpoint{3.653638in}{0.749825in}}%
\pgfpathlineto{\pgfqpoint{3.661357in}{0.643151in}}%
\pgfpathlineto{\pgfqpoint{3.666503in}{0.597524in}}%
\pgfpathlineto{\pgfqpoint{3.670791in}{0.577211in}}%
\pgfpathlineto{\pgfqpoint{3.673364in}{0.573113in}}%
\pgfpathlineto{\pgfqpoint{3.675079in}{0.573797in}}%
\pgfpathlineto{\pgfqpoint{3.677652in}{0.579936in}}%
\pgfpathlineto{\pgfqpoint{3.681083in}{0.597524in}}%
\pgfpathlineto{\pgfqpoint{3.685371in}{0.634016in}}%
\pgfpathlineto{\pgfqpoint{3.691375in}{0.709715in}}%
\pgfpathlineto{\pgfqpoint{3.699094in}{0.841136in}}%
\pgfpathlineto{\pgfqpoint{3.713674in}{1.141159in}}%
\pgfpathlineto{\pgfqpoint{3.725681in}{1.367192in}}%
\pgfpathlineto{\pgfqpoint{3.732542in}{1.461077in}}%
\pgfpathlineto{\pgfqpoint{3.737688in}{1.507631in}}%
\pgfpathlineto{\pgfqpoint{3.741977in}{1.528775in}}%
\pgfpathlineto{\pgfqpoint{3.744550in}{1.533384in}}%
\pgfpathlineto{\pgfqpoint{3.746265in}{1.533042in}}%
\pgfpathlineto{\pgfqpoint{3.747980in}{1.529969in}}%
\pgfpathlineto{\pgfqpoint{3.750553in}{1.520280in}}%
\pgfpathlineto{\pgfqpoint{3.753984in}{1.498094in}}%
\pgfpathlineto{\pgfqpoint{3.759130in}{1.446069in}}%
\pgfpathlineto{\pgfqpoint{3.765133in}{1.360277in}}%
\pgfpathlineto{\pgfqpoint{3.773710in}{1.202576in}}%
\pgfpathlineto{\pgfqpoint{3.799440in}{0.703445in}}%
\pgfpathlineto{\pgfqpoint{3.806301in}{0.621436in}}%
\pgfpathlineto{\pgfqpoint{3.811447in}{0.585286in}}%
\pgfpathlineto{\pgfqpoint{3.814877in}{0.574395in}}%
\pgfpathlineto{\pgfqpoint{3.816593in}{0.573027in}}%
\pgfpathlineto{\pgfqpoint{3.818308in}{0.574395in}}%
\pgfpathlineto{\pgfqpoint{3.820881in}{0.581551in}}%
\pgfpathlineto{\pgfqpoint{3.824312in}{0.600462in}}%
\pgfpathlineto{\pgfqpoint{3.828600in}{0.638509in}}%
\pgfpathlineto{\pgfqpoint{3.834603in}{0.716108in}}%
\pgfpathlineto{\pgfqpoint{3.842322in}{0.849303in}}%
\pgfpathlineto{\pgfqpoint{3.857760in}{1.167731in}}%
\pgfpathlineto{\pgfqpoint{3.868910in}{1.373994in}}%
\pgfpathlineto{\pgfqpoint{3.875771in}{1.465792in}}%
\pgfpathlineto{\pgfqpoint{3.880917in}{1.510488in}}%
\pgfpathlineto{\pgfqpoint{3.885205in}{1.529969in}}%
\pgfpathlineto{\pgfqpoint{3.887778in}{1.533555in}}%
\pgfpathlineto{\pgfqpoint{3.889493in}{1.532529in}}%
\pgfpathlineto{\pgfqpoint{3.892066in}{1.525881in}}%
\pgfpathlineto{\pgfqpoint{3.895497in}{1.507631in}}%
\pgfpathlineto{\pgfqpoint{3.899785in}{1.470360in}}%
\pgfpathlineto{\pgfqpoint{3.905789in}{1.393707in}}%
\pgfpathlineto{\pgfqpoint{3.913508in}{1.261394in}}%
\pgfpathlineto{\pgfqpoint{3.928088in}{0.960994in}}%
\pgfpathlineto{\pgfqpoint{3.939237in}{0.749825in}}%
\pgfpathlineto{\pgfqpoint{3.946956in}{0.643151in}}%
\pgfpathlineto{\pgfqpoint{3.952102in}{0.597524in}}%
\pgfpathlineto{\pgfqpoint{3.956391in}{0.577211in}}%
\pgfpathlineto{\pgfqpoint{3.958963in}{0.573113in}}%
\pgfpathlineto{\pgfqpoint{3.960679in}{0.573797in}}%
\pgfpathlineto{\pgfqpoint{3.963252in}{0.579936in}}%
\pgfpathlineto{\pgfqpoint{3.966682in}{0.597524in}}%
\pgfpathlineto{\pgfqpoint{3.970971in}{0.634016in}}%
\pgfpathlineto{\pgfqpoint{3.976974in}{0.709715in}}%
\pgfpathlineto{\pgfqpoint{3.984693in}{0.841136in}}%
\pgfpathlineto{\pgfqpoint{3.999273in}{1.141159in}}%
\pgfpathlineto{\pgfqpoint{4.011280in}{1.367192in}}%
\pgfpathlineto{\pgfqpoint{4.018142in}{1.461077in}}%
\pgfpathlineto{\pgfqpoint{4.023288in}{1.507631in}}%
\pgfpathlineto{\pgfqpoint{4.027576in}{1.528775in}}%
\pgfpathlineto{\pgfqpoint{4.030149in}{1.533384in}}%
\pgfpathlineto{\pgfqpoint{4.031864in}{1.533042in}}%
\pgfpathlineto{\pgfqpoint{4.033580in}{1.529969in}}%
\pgfpathlineto{\pgfqpoint{4.036153in}{1.520280in}}%
\pgfpathlineto{\pgfqpoint{4.039583in}{1.498094in}}%
\pgfpathlineto{\pgfqpoint{4.044729in}{1.446069in}}%
\pgfpathlineto{\pgfqpoint{4.050733in}{1.360277in}}%
\pgfpathlineto{\pgfqpoint{4.059309in}{1.202576in}}%
\pgfpathlineto{\pgfqpoint{4.085039in}{0.703445in}}%
\pgfpathlineto{\pgfqpoint{4.091900in}{0.621436in}}%
\pgfpathlineto{\pgfqpoint{4.097046in}{0.585286in}}%
\pgfpathlineto{\pgfqpoint{4.100477in}{0.574395in}}%
\pgfpathlineto{\pgfqpoint{4.102192in}{0.573027in}}%
\pgfpathlineto{\pgfqpoint{4.102192in}{0.573027in}}%
\pgfusepath{stroke}%
\end{pgfscope}%
\begin{pgfscope}%
\pgfsetrectcap%
\pgfsetmiterjoin%
\pgfsetlinewidth{0.501875pt}%
\definecolor{currentstroke}{rgb}{0.000000,0.000000,0.000000}%
\pgfsetstrokecolor{currentstroke}%
\pgfsetdash{}{0pt}%
\pgfpathmoveto{\pgfqpoint{0.675000in}{1.581604in}}%
\pgfpathlineto{\pgfqpoint{4.275000in}{1.581604in}}%
\pgfusepath{stroke}%
\end{pgfscope}%
\begin{pgfscope}%
\pgfsetrectcap%
\pgfsetmiterjoin%
\pgfsetlinewidth{0.501875pt}%
\definecolor{currentstroke}{rgb}{0.000000,0.000000,0.000000}%
\pgfsetstrokecolor{currentstroke}%
\pgfsetdash{}{0pt}%
\pgfpathmoveto{\pgfqpoint{0.675000in}{0.525000in}}%
\pgfpathlineto{\pgfqpoint{0.675000in}{1.581604in}}%
\pgfusepath{stroke}%
\end{pgfscope}%
\begin{pgfscope}%
\pgfsetrectcap%
\pgfsetmiterjoin%
\pgfsetlinewidth{0.501875pt}%
\definecolor{currentstroke}{rgb}{0.000000,0.000000,0.000000}%
\pgfsetstrokecolor{currentstroke}%
\pgfsetdash{}{0pt}%
\pgfpathmoveto{\pgfqpoint{4.275000in}{0.525000in}}%
\pgfpathlineto{\pgfqpoint{4.275000in}{1.581604in}}%
\pgfusepath{stroke}%
\end{pgfscope}%
\begin{pgfscope}%
\pgfsetrectcap%
\pgfsetmiterjoin%
\pgfsetlinewidth{0.501875pt}%
\definecolor{currentstroke}{rgb}{0.000000,0.000000,0.000000}%
\pgfsetstrokecolor{currentstroke}%
\pgfsetdash{}{0pt}%
\pgfpathmoveto{\pgfqpoint{0.675000in}{0.525000in}}%
\pgfpathlineto{\pgfqpoint{4.275000in}{0.525000in}}%
\pgfusepath{stroke}%
\end{pgfscope}%
\begin{pgfscope}%
\pgfsetbuttcap%
\pgfsetroundjoin%
\definecolor{currentfill}{rgb}{0.000000,0.000000,0.000000}%
\pgfsetfillcolor{currentfill}%
\pgfsetlinewidth{0.501875pt}%
\definecolor{currentstroke}{rgb}{0.000000,0.000000,0.000000}%
\pgfsetstrokecolor{currentstroke}%
\pgfsetdash{}{0pt}%
\pgfsys@defobject{currentmarker}{\pgfqpoint{0.000000in}{0.000000in}}{\pgfqpoint{0.000000in}{0.055556in}}{%
\pgfpathmoveto{\pgfqpoint{0.000000in}{0.000000in}}%
\pgfpathlineto{\pgfqpoint{0.000000in}{0.055556in}}%
\pgfusepath{stroke,fill}%
}%
\begin{pgfscope}%
\pgfsys@transformshift{0.675000in}{0.525000in}%
\pgfsys@useobject{currentmarker}{}%
\end{pgfscope}%
\end{pgfscope}%
\begin{pgfscope}%
\pgfsetbuttcap%
\pgfsetroundjoin%
\definecolor{currentfill}{rgb}{0.000000,0.000000,0.000000}%
\pgfsetfillcolor{currentfill}%
\pgfsetlinewidth{0.501875pt}%
\definecolor{currentstroke}{rgb}{0.000000,0.000000,0.000000}%
\pgfsetstrokecolor{currentstroke}%
\pgfsetdash{}{0pt}%
\pgfsys@defobject{currentmarker}{\pgfqpoint{0.000000in}{-0.055556in}}{\pgfqpoint{0.000000in}{0.000000in}}{%
\pgfpathmoveto{\pgfqpoint{0.000000in}{0.000000in}}%
\pgfpathlineto{\pgfqpoint{0.000000in}{-0.055556in}}%
\pgfusepath{stroke,fill}%
}%
\begin{pgfscope}%
\pgfsys@transformshift{0.675000in}{1.581604in}%
\pgfsys@useobject{currentmarker}{}%
\end{pgfscope}%
\end{pgfscope}%
\begin{pgfscope}%
\pgftext[x=0.675000in,y=0.469444in,,top]{\rmfamily\fontsize{9.000000}{10.800000}\selectfont \(\displaystyle 0.0000\)}%
\end{pgfscope}%
\begin{pgfscope}%
\pgfsetbuttcap%
\pgfsetroundjoin%
\definecolor{currentfill}{rgb}{0.000000,0.000000,0.000000}%
\pgfsetfillcolor{currentfill}%
\pgfsetlinewidth{0.501875pt}%
\definecolor{currentstroke}{rgb}{0.000000,0.000000,0.000000}%
\pgfsetstrokecolor{currentstroke}%
\pgfsetdash{}{0pt}%
\pgfsys@defobject{currentmarker}{\pgfqpoint{0.000000in}{0.000000in}}{\pgfqpoint{0.000000in}{0.055556in}}{%
\pgfpathmoveto{\pgfqpoint{0.000000in}{0.000000in}}%
\pgfpathlineto{\pgfqpoint{0.000000in}{0.055556in}}%
\pgfusepath{stroke,fill}%
}%
\begin{pgfscope}%
\pgfsys@transformshift{1.125000in}{0.525000in}%
\pgfsys@useobject{currentmarker}{}%
\end{pgfscope}%
\end{pgfscope}%
\begin{pgfscope}%
\pgfsetbuttcap%
\pgfsetroundjoin%
\definecolor{currentfill}{rgb}{0.000000,0.000000,0.000000}%
\pgfsetfillcolor{currentfill}%
\pgfsetlinewidth{0.501875pt}%
\definecolor{currentstroke}{rgb}{0.000000,0.000000,0.000000}%
\pgfsetstrokecolor{currentstroke}%
\pgfsetdash{}{0pt}%
\pgfsys@defobject{currentmarker}{\pgfqpoint{0.000000in}{-0.055556in}}{\pgfqpoint{0.000000in}{0.000000in}}{%
\pgfpathmoveto{\pgfqpoint{0.000000in}{0.000000in}}%
\pgfpathlineto{\pgfqpoint{0.000000in}{-0.055556in}}%
\pgfusepath{stroke,fill}%
}%
\begin{pgfscope}%
\pgfsys@transformshift{1.125000in}{1.581604in}%
\pgfsys@useobject{currentmarker}{}%
\end{pgfscope}%
\end{pgfscope}%
\begin{pgfscope}%
\pgftext[x=1.125000in,y=0.469444in,,top]{\rmfamily\fontsize{9.000000}{10.800000}\selectfont \(\displaystyle 0.0002\)}%
\end{pgfscope}%
\begin{pgfscope}%
\pgfsetbuttcap%
\pgfsetroundjoin%
\definecolor{currentfill}{rgb}{0.000000,0.000000,0.000000}%
\pgfsetfillcolor{currentfill}%
\pgfsetlinewidth{0.501875pt}%
\definecolor{currentstroke}{rgb}{0.000000,0.000000,0.000000}%
\pgfsetstrokecolor{currentstroke}%
\pgfsetdash{}{0pt}%
\pgfsys@defobject{currentmarker}{\pgfqpoint{0.000000in}{0.000000in}}{\pgfqpoint{0.000000in}{0.055556in}}{%
\pgfpathmoveto{\pgfqpoint{0.000000in}{0.000000in}}%
\pgfpathlineto{\pgfqpoint{0.000000in}{0.055556in}}%
\pgfusepath{stroke,fill}%
}%
\begin{pgfscope}%
\pgfsys@transformshift{1.575000in}{0.525000in}%
\pgfsys@useobject{currentmarker}{}%
\end{pgfscope}%
\end{pgfscope}%
\begin{pgfscope}%
\pgfsetbuttcap%
\pgfsetroundjoin%
\definecolor{currentfill}{rgb}{0.000000,0.000000,0.000000}%
\pgfsetfillcolor{currentfill}%
\pgfsetlinewidth{0.501875pt}%
\definecolor{currentstroke}{rgb}{0.000000,0.000000,0.000000}%
\pgfsetstrokecolor{currentstroke}%
\pgfsetdash{}{0pt}%
\pgfsys@defobject{currentmarker}{\pgfqpoint{0.000000in}{-0.055556in}}{\pgfqpoint{0.000000in}{0.000000in}}{%
\pgfpathmoveto{\pgfqpoint{0.000000in}{0.000000in}}%
\pgfpathlineto{\pgfqpoint{0.000000in}{-0.055556in}}%
\pgfusepath{stroke,fill}%
}%
\begin{pgfscope}%
\pgfsys@transformshift{1.575000in}{1.581604in}%
\pgfsys@useobject{currentmarker}{}%
\end{pgfscope}%
\end{pgfscope}%
\begin{pgfscope}%
\pgftext[x=1.575000in,y=0.469444in,,top]{\rmfamily\fontsize{9.000000}{10.800000}\selectfont \(\displaystyle 0.0004\)}%
\end{pgfscope}%
\begin{pgfscope}%
\pgfsetbuttcap%
\pgfsetroundjoin%
\definecolor{currentfill}{rgb}{0.000000,0.000000,0.000000}%
\pgfsetfillcolor{currentfill}%
\pgfsetlinewidth{0.501875pt}%
\definecolor{currentstroke}{rgb}{0.000000,0.000000,0.000000}%
\pgfsetstrokecolor{currentstroke}%
\pgfsetdash{}{0pt}%
\pgfsys@defobject{currentmarker}{\pgfqpoint{0.000000in}{0.000000in}}{\pgfqpoint{0.000000in}{0.055556in}}{%
\pgfpathmoveto{\pgfqpoint{0.000000in}{0.000000in}}%
\pgfpathlineto{\pgfqpoint{0.000000in}{0.055556in}}%
\pgfusepath{stroke,fill}%
}%
\begin{pgfscope}%
\pgfsys@transformshift{2.025000in}{0.525000in}%
\pgfsys@useobject{currentmarker}{}%
\end{pgfscope}%
\end{pgfscope}%
\begin{pgfscope}%
\pgfsetbuttcap%
\pgfsetroundjoin%
\definecolor{currentfill}{rgb}{0.000000,0.000000,0.000000}%
\pgfsetfillcolor{currentfill}%
\pgfsetlinewidth{0.501875pt}%
\definecolor{currentstroke}{rgb}{0.000000,0.000000,0.000000}%
\pgfsetstrokecolor{currentstroke}%
\pgfsetdash{}{0pt}%
\pgfsys@defobject{currentmarker}{\pgfqpoint{0.000000in}{-0.055556in}}{\pgfqpoint{0.000000in}{0.000000in}}{%
\pgfpathmoveto{\pgfqpoint{0.000000in}{0.000000in}}%
\pgfpathlineto{\pgfqpoint{0.000000in}{-0.055556in}}%
\pgfusepath{stroke,fill}%
}%
\begin{pgfscope}%
\pgfsys@transformshift{2.025000in}{1.581604in}%
\pgfsys@useobject{currentmarker}{}%
\end{pgfscope}%
\end{pgfscope}%
\begin{pgfscope}%
\pgftext[x=2.025000in,y=0.469444in,,top]{\rmfamily\fontsize{9.000000}{10.800000}\selectfont \(\displaystyle 0.0006\)}%
\end{pgfscope}%
\begin{pgfscope}%
\pgfsetbuttcap%
\pgfsetroundjoin%
\definecolor{currentfill}{rgb}{0.000000,0.000000,0.000000}%
\pgfsetfillcolor{currentfill}%
\pgfsetlinewidth{0.501875pt}%
\definecolor{currentstroke}{rgb}{0.000000,0.000000,0.000000}%
\pgfsetstrokecolor{currentstroke}%
\pgfsetdash{}{0pt}%
\pgfsys@defobject{currentmarker}{\pgfqpoint{0.000000in}{0.000000in}}{\pgfqpoint{0.000000in}{0.055556in}}{%
\pgfpathmoveto{\pgfqpoint{0.000000in}{0.000000in}}%
\pgfpathlineto{\pgfqpoint{0.000000in}{0.055556in}}%
\pgfusepath{stroke,fill}%
}%
\begin{pgfscope}%
\pgfsys@transformshift{2.475000in}{0.525000in}%
\pgfsys@useobject{currentmarker}{}%
\end{pgfscope}%
\end{pgfscope}%
\begin{pgfscope}%
\pgfsetbuttcap%
\pgfsetroundjoin%
\definecolor{currentfill}{rgb}{0.000000,0.000000,0.000000}%
\pgfsetfillcolor{currentfill}%
\pgfsetlinewidth{0.501875pt}%
\definecolor{currentstroke}{rgb}{0.000000,0.000000,0.000000}%
\pgfsetstrokecolor{currentstroke}%
\pgfsetdash{}{0pt}%
\pgfsys@defobject{currentmarker}{\pgfqpoint{0.000000in}{-0.055556in}}{\pgfqpoint{0.000000in}{0.000000in}}{%
\pgfpathmoveto{\pgfqpoint{0.000000in}{0.000000in}}%
\pgfpathlineto{\pgfqpoint{0.000000in}{-0.055556in}}%
\pgfusepath{stroke,fill}%
}%
\begin{pgfscope}%
\pgfsys@transformshift{2.475000in}{1.581604in}%
\pgfsys@useobject{currentmarker}{}%
\end{pgfscope}%
\end{pgfscope}%
\begin{pgfscope}%
\pgftext[x=2.475000in,y=0.469444in,,top]{\rmfamily\fontsize{9.000000}{10.800000}\selectfont \(\displaystyle 0.0008\)}%
\end{pgfscope}%
\begin{pgfscope}%
\pgfsetbuttcap%
\pgfsetroundjoin%
\definecolor{currentfill}{rgb}{0.000000,0.000000,0.000000}%
\pgfsetfillcolor{currentfill}%
\pgfsetlinewidth{0.501875pt}%
\definecolor{currentstroke}{rgb}{0.000000,0.000000,0.000000}%
\pgfsetstrokecolor{currentstroke}%
\pgfsetdash{}{0pt}%
\pgfsys@defobject{currentmarker}{\pgfqpoint{0.000000in}{0.000000in}}{\pgfqpoint{0.000000in}{0.055556in}}{%
\pgfpathmoveto{\pgfqpoint{0.000000in}{0.000000in}}%
\pgfpathlineto{\pgfqpoint{0.000000in}{0.055556in}}%
\pgfusepath{stroke,fill}%
}%
\begin{pgfscope}%
\pgfsys@transformshift{2.925000in}{0.525000in}%
\pgfsys@useobject{currentmarker}{}%
\end{pgfscope}%
\end{pgfscope}%
\begin{pgfscope}%
\pgfsetbuttcap%
\pgfsetroundjoin%
\definecolor{currentfill}{rgb}{0.000000,0.000000,0.000000}%
\pgfsetfillcolor{currentfill}%
\pgfsetlinewidth{0.501875pt}%
\definecolor{currentstroke}{rgb}{0.000000,0.000000,0.000000}%
\pgfsetstrokecolor{currentstroke}%
\pgfsetdash{}{0pt}%
\pgfsys@defobject{currentmarker}{\pgfqpoint{0.000000in}{-0.055556in}}{\pgfqpoint{0.000000in}{0.000000in}}{%
\pgfpathmoveto{\pgfqpoint{0.000000in}{0.000000in}}%
\pgfpathlineto{\pgfqpoint{0.000000in}{-0.055556in}}%
\pgfusepath{stroke,fill}%
}%
\begin{pgfscope}%
\pgfsys@transformshift{2.925000in}{1.581604in}%
\pgfsys@useobject{currentmarker}{}%
\end{pgfscope}%
\end{pgfscope}%
\begin{pgfscope}%
\pgftext[x=2.925000in,y=0.469444in,,top]{\rmfamily\fontsize{9.000000}{10.800000}\selectfont \(\displaystyle 0.0010\)}%
\end{pgfscope}%
\begin{pgfscope}%
\pgfsetbuttcap%
\pgfsetroundjoin%
\definecolor{currentfill}{rgb}{0.000000,0.000000,0.000000}%
\pgfsetfillcolor{currentfill}%
\pgfsetlinewidth{0.501875pt}%
\definecolor{currentstroke}{rgb}{0.000000,0.000000,0.000000}%
\pgfsetstrokecolor{currentstroke}%
\pgfsetdash{}{0pt}%
\pgfsys@defobject{currentmarker}{\pgfqpoint{0.000000in}{0.000000in}}{\pgfqpoint{0.000000in}{0.055556in}}{%
\pgfpathmoveto{\pgfqpoint{0.000000in}{0.000000in}}%
\pgfpathlineto{\pgfqpoint{0.000000in}{0.055556in}}%
\pgfusepath{stroke,fill}%
}%
\begin{pgfscope}%
\pgfsys@transformshift{3.375000in}{0.525000in}%
\pgfsys@useobject{currentmarker}{}%
\end{pgfscope}%
\end{pgfscope}%
\begin{pgfscope}%
\pgfsetbuttcap%
\pgfsetroundjoin%
\definecolor{currentfill}{rgb}{0.000000,0.000000,0.000000}%
\pgfsetfillcolor{currentfill}%
\pgfsetlinewidth{0.501875pt}%
\definecolor{currentstroke}{rgb}{0.000000,0.000000,0.000000}%
\pgfsetstrokecolor{currentstroke}%
\pgfsetdash{}{0pt}%
\pgfsys@defobject{currentmarker}{\pgfqpoint{0.000000in}{-0.055556in}}{\pgfqpoint{0.000000in}{0.000000in}}{%
\pgfpathmoveto{\pgfqpoint{0.000000in}{0.000000in}}%
\pgfpathlineto{\pgfqpoint{0.000000in}{-0.055556in}}%
\pgfusepath{stroke,fill}%
}%
\begin{pgfscope}%
\pgfsys@transformshift{3.375000in}{1.581604in}%
\pgfsys@useobject{currentmarker}{}%
\end{pgfscope}%
\end{pgfscope}%
\begin{pgfscope}%
\pgftext[x=3.375000in,y=0.469444in,,top]{\rmfamily\fontsize{9.000000}{10.800000}\selectfont \(\displaystyle 0.0012\)}%
\end{pgfscope}%
\begin{pgfscope}%
\pgfsetbuttcap%
\pgfsetroundjoin%
\definecolor{currentfill}{rgb}{0.000000,0.000000,0.000000}%
\pgfsetfillcolor{currentfill}%
\pgfsetlinewidth{0.501875pt}%
\definecolor{currentstroke}{rgb}{0.000000,0.000000,0.000000}%
\pgfsetstrokecolor{currentstroke}%
\pgfsetdash{}{0pt}%
\pgfsys@defobject{currentmarker}{\pgfqpoint{0.000000in}{0.000000in}}{\pgfqpoint{0.000000in}{0.055556in}}{%
\pgfpathmoveto{\pgfqpoint{0.000000in}{0.000000in}}%
\pgfpathlineto{\pgfqpoint{0.000000in}{0.055556in}}%
\pgfusepath{stroke,fill}%
}%
\begin{pgfscope}%
\pgfsys@transformshift{3.825000in}{0.525000in}%
\pgfsys@useobject{currentmarker}{}%
\end{pgfscope}%
\end{pgfscope}%
\begin{pgfscope}%
\pgfsetbuttcap%
\pgfsetroundjoin%
\definecolor{currentfill}{rgb}{0.000000,0.000000,0.000000}%
\pgfsetfillcolor{currentfill}%
\pgfsetlinewidth{0.501875pt}%
\definecolor{currentstroke}{rgb}{0.000000,0.000000,0.000000}%
\pgfsetstrokecolor{currentstroke}%
\pgfsetdash{}{0pt}%
\pgfsys@defobject{currentmarker}{\pgfqpoint{0.000000in}{-0.055556in}}{\pgfqpoint{0.000000in}{0.000000in}}{%
\pgfpathmoveto{\pgfqpoint{0.000000in}{0.000000in}}%
\pgfpathlineto{\pgfqpoint{0.000000in}{-0.055556in}}%
\pgfusepath{stroke,fill}%
}%
\begin{pgfscope}%
\pgfsys@transformshift{3.825000in}{1.581604in}%
\pgfsys@useobject{currentmarker}{}%
\end{pgfscope}%
\end{pgfscope}%
\begin{pgfscope}%
\pgftext[x=3.825000in,y=0.469444in,,top]{\rmfamily\fontsize{9.000000}{10.800000}\selectfont \(\displaystyle 0.0014\)}%
\end{pgfscope}%
\begin{pgfscope}%
\pgfsetbuttcap%
\pgfsetroundjoin%
\definecolor{currentfill}{rgb}{0.000000,0.000000,0.000000}%
\pgfsetfillcolor{currentfill}%
\pgfsetlinewidth{0.501875pt}%
\definecolor{currentstroke}{rgb}{0.000000,0.000000,0.000000}%
\pgfsetstrokecolor{currentstroke}%
\pgfsetdash{}{0pt}%
\pgfsys@defobject{currentmarker}{\pgfqpoint{0.000000in}{0.000000in}}{\pgfqpoint{0.000000in}{0.055556in}}{%
\pgfpathmoveto{\pgfqpoint{0.000000in}{0.000000in}}%
\pgfpathlineto{\pgfqpoint{0.000000in}{0.055556in}}%
\pgfusepath{stroke,fill}%
}%
\begin{pgfscope}%
\pgfsys@transformshift{4.275000in}{0.525000in}%
\pgfsys@useobject{currentmarker}{}%
\end{pgfscope}%
\end{pgfscope}%
\begin{pgfscope}%
\pgfsetbuttcap%
\pgfsetroundjoin%
\definecolor{currentfill}{rgb}{0.000000,0.000000,0.000000}%
\pgfsetfillcolor{currentfill}%
\pgfsetlinewidth{0.501875pt}%
\definecolor{currentstroke}{rgb}{0.000000,0.000000,0.000000}%
\pgfsetstrokecolor{currentstroke}%
\pgfsetdash{}{0pt}%
\pgfsys@defobject{currentmarker}{\pgfqpoint{0.000000in}{-0.055556in}}{\pgfqpoint{0.000000in}{0.000000in}}{%
\pgfpathmoveto{\pgfqpoint{0.000000in}{0.000000in}}%
\pgfpathlineto{\pgfqpoint{0.000000in}{-0.055556in}}%
\pgfusepath{stroke,fill}%
}%
\begin{pgfscope}%
\pgfsys@transformshift{4.275000in}{1.581604in}%
\pgfsys@useobject{currentmarker}{}%
\end{pgfscope}%
\end{pgfscope}%
\begin{pgfscope}%
\pgftext[x=4.275000in,y=0.469444in,,top]{\rmfamily\fontsize{9.000000}{10.800000}\selectfont \(\displaystyle 0.0016\)}%
\end{pgfscope}%
\begin{pgfscope}%
\pgftext[x=2.475000in,y=0.279028in,,top]{\rmfamily\fontsize{9.000000}{10.800000}\selectfont Zeit (s)}%
\end{pgfscope}%
\begin{pgfscope}%
\pgfsetbuttcap%
\pgfsetroundjoin%
\definecolor{currentfill}{rgb}{0.000000,0.000000,0.000000}%
\pgfsetfillcolor{currentfill}%
\pgfsetlinewidth{0.501875pt}%
\definecolor{currentstroke}{rgb}{0.000000,0.000000,0.000000}%
\pgfsetstrokecolor{currentstroke}%
\pgfsetdash{}{0pt}%
\pgfsys@defobject{currentmarker}{\pgfqpoint{0.000000in}{0.000000in}}{\pgfqpoint{0.055556in}{0.000000in}}{%
\pgfpathmoveto{\pgfqpoint{0.000000in}{0.000000in}}%
\pgfpathlineto{\pgfqpoint{0.055556in}{0.000000in}}%
\pgfusepath{stroke,fill}%
}%
\begin{pgfscope}%
\pgfsys@transformshift{0.675000in}{0.573027in}%
\pgfsys@useobject{currentmarker}{}%
\end{pgfscope}%
\end{pgfscope}%
\begin{pgfscope}%
\pgfsetbuttcap%
\pgfsetroundjoin%
\definecolor{currentfill}{rgb}{0.000000,0.000000,0.000000}%
\pgfsetfillcolor{currentfill}%
\pgfsetlinewidth{0.501875pt}%
\definecolor{currentstroke}{rgb}{0.000000,0.000000,0.000000}%
\pgfsetstrokecolor{currentstroke}%
\pgfsetdash{}{0pt}%
\pgfsys@defobject{currentmarker}{\pgfqpoint{-0.055556in}{0.000000in}}{\pgfqpoint{0.000000in}{0.000000in}}{%
\pgfpathmoveto{\pgfqpoint{0.000000in}{0.000000in}}%
\pgfpathlineto{\pgfqpoint{-0.055556in}{0.000000in}}%
\pgfusepath{stroke,fill}%
}%
\begin{pgfscope}%
\pgfsys@transformshift{4.275000in}{0.573027in}%
\pgfsys@useobject{currentmarker}{}%
\end{pgfscope}%
\end{pgfscope}%
\begin{pgfscope}%
\pgftext[x=0.619444in,y=0.573027in,right,]{\rmfamily\fontsize{9.000000}{10.800000}\selectfont \(\displaystyle -1.5\)}%
\end{pgfscope}%
\begin{pgfscope}%
\pgfsetbuttcap%
\pgfsetroundjoin%
\definecolor{currentfill}{rgb}{0.000000,0.000000,0.000000}%
\pgfsetfillcolor{currentfill}%
\pgfsetlinewidth{0.501875pt}%
\definecolor{currentstroke}{rgb}{0.000000,0.000000,0.000000}%
\pgfsetstrokecolor{currentstroke}%
\pgfsetdash{}{0pt}%
\pgfsys@defobject{currentmarker}{\pgfqpoint{0.000000in}{0.000000in}}{\pgfqpoint{0.055556in}{0.000000in}}{%
\pgfpathmoveto{\pgfqpoint{0.000000in}{0.000000in}}%
\pgfpathlineto{\pgfqpoint{0.055556in}{0.000000in}}%
\pgfusepath{stroke,fill}%
}%
\begin{pgfscope}%
\pgfsys@transformshift{0.675000in}{0.733119in}%
\pgfsys@useobject{currentmarker}{}%
\end{pgfscope}%
\end{pgfscope}%
\begin{pgfscope}%
\pgfsetbuttcap%
\pgfsetroundjoin%
\definecolor{currentfill}{rgb}{0.000000,0.000000,0.000000}%
\pgfsetfillcolor{currentfill}%
\pgfsetlinewidth{0.501875pt}%
\definecolor{currentstroke}{rgb}{0.000000,0.000000,0.000000}%
\pgfsetstrokecolor{currentstroke}%
\pgfsetdash{}{0pt}%
\pgfsys@defobject{currentmarker}{\pgfqpoint{-0.055556in}{0.000000in}}{\pgfqpoint{0.000000in}{0.000000in}}{%
\pgfpathmoveto{\pgfqpoint{0.000000in}{0.000000in}}%
\pgfpathlineto{\pgfqpoint{-0.055556in}{0.000000in}}%
\pgfusepath{stroke,fill}%
}%
\begin{pgfscope}%
\pgfsys@transformshift{4.275000in}{0.733119in}%
\pgfsys@useobject{currentmarker}{}%
\end{pgfscope}%
\end{pgfscope}%
\begin{pgfscope}%
\pgftext[x=0.619444in,y=0.733119in,right,]{\rmfamily\fontsize{9.000000}{10.800000}\selectfont \(\displaystyle -1.0\)}%
\end{pgfscope}%
\begin{pgfscope}%
\pgfsetbuttcap%
\pgfsetroundjoin%
\definecolor{currentfill}{rgb}{0.000000,0.000000,0.000000}%
\pgfsetfillcolor{currentfill}%
\pgfsetlinewidth{0.501875pt}%
\definecolor{currentstroke}{rgb}{0.000000,0.000000,0.000000}%
\pgfsetstrokecolor{currentstroke}%
\pgfsetdash{}{0pt}%
\pgfsys@defobject{currentmarker}{\pgfqpoint{0.000000in}{0.000000in}}{\pgfqpoint{0.055556in}{0.000000in}}{%
\pgfpathmoveto{\pgfqpoint{0.000000in}{0.000000in}}%
\pgfpathlineto{\pgfqpoint{0.055556in}{0.000000in}}%
\pgfusepath{stroke,fill}%
}%
\begin{pgfscope}%
\pgfsys@transformshift{0.675000in}{0.893210in}%
\pgfsys@useobject{currentmarker}{}%
\end{pgfscope}%
\end{pgfscope}%
\begin{pgfscope}%
\pgfsetbuttcap%
\pgfsetroundjoin%
\definecolor{currentfill}{rgb}{0.000000,0.000000,0.000000}%
\pgfsetfillcolor{currentfill}%
\pgfsetlinewidth{0.501875pt}%
\definecolor{currentstroke}{rgb}{0.000000,0.000000,0.000000}%
\pgfsetstrokecolor{currentstroke}%
\pgfsetdash{}{0pt}%
\pgfsys@defobject{currentmarker}{\pgfqpoint{-0.055556in}{0.000000in}}{\pgfqpoint{0.000000in}{0.000000in}}{%
\pgfpathmoveto{\pgfqpoint{0.000000in}{0.000000in}}%
\pgfpathlineto{\pgfqpoint{-0.055556in}{0.000000in}}%
\pgfusepath{stroke,fill}%
}%
\begin{pgfscope}%
\pgfsys@transformshift{4.275000in}{0.893210in}%
\pgfsys@useobject{currentmarker}{}%
\end{pgfscope}%
\end{pgfscope}%
\begin{pgfscope}%
\pgftext[x=0.619444in,y=0.893210in,right,]{\rmfamily\fontsize{9.000000}{10.800000}\selectfont \(\displaystyle -0.5\)}%
\end{pgfscope}%
\begin{pgfscope}%
\pgfsetbuttcap%
\pgfsetroundjoin%
\definecolor{currentfill}{rgb}{0.000000,0.000000,0.000000}%
\pgfsetfillcolor{currentfill}%
\pgfsetlinewidth{0.501875pt}%
\definecolor{currentstroke}{rgb}{0.000000,0.000000,0.000000}%
\pgfsetstrokecolor{currentstroke}%
\pgfsetdash{}{0pt}%
\pgfsys@defobject{currentmarker}{\pgfqpoint{0.000000in}{0.000000in}}{\pgfqpoint{0.055556in}{0.000000in}}{%
\pgfpathmoveto{\pgfqpoint{0.000000in}{0.000000in}}%
\pgfpathlineto{\pgfqpoint{0.055556in}{0.000000in}}%
\pgfusepath{stroke,fill}%
}%
\begin{pgfscope}%
\pgfsys@transformshift{0.675000in}{1.053302in}%
\pgfsys@useobject{currentmarker}{}%
\end{pgfscope}%
\end{pgfscope}%
\begin{pgfscope}%
\pgfsetbuttcap%
\pgfsetroundjoin%
\definecolor{currentfill}{rgb}{0.000000,0.000000,0.000000}%
\pgfsetfillcolor{currentfill}%
\pgfsetlinewidth{0.501875pt}%
\definecolor{currentstroke}{rgb}{0.000000,0.000000,0.000000}%
\pgfsetstrokecolor{currentstroke}%
\pgfsetdash{}{0pt}%
\pgfsys@defobject{currentmarker}{\pgfqpoint{-0.055556in}{0.000000in}}{\pgfqpoint{0.000000in}{0.000000in}}{%
\pgfpathmoveto{\pgfqpoint{0.000000in}{0.000000in}}%
\pgfpathlineto{\pgfqpoint{-0.055556in}{0.000000in}}%
\pgfusepath{stroke,fill}%
}%
\begin{pgfscope}%
\pgfsys@transformshift{4.275000in}{1.053302in}%
\pgfsys@useobject{currentmarker}{}%
\end{pgfscope}%
\end{pgfscope}%
\begin{pgfscope}%
\pgftext[x=0.619444in,y=1.053302in,right,]{\rmfamily\fontsize{9.000000}{10.800000}\selectfont \(\displaystyle 0.0\)}%
\end{pgfscope}%
\begin{pgfscope}%
\pgfsetbuttcap%
\pgfsetroundjoin%
\definecolor{currentfill}{rgb}{0.000000,0.000000,0.000000}%
\pgfsetfillcolor{currentfill}%
\pgfsetlinewidth{0.501875pt}%
\definecolor{currentstroke}{rgb}{0.000000,0.000000,0.000000}%
\pgfsetstrokecolor{currentstroke}%
\pgfsetdash{}{0pt}%
\pgfsys@defobject{currentmarker}{\pgfqpoint{0.000000in}{0.000000in}}{\pgfqpoint{0.055556in}{0.000000in}}{%
\pgfpathmoveto{\pgfqpoint{0.000000in}{0.000000in}}%
\pgfpathlineto{\pgfqpoint{0.055556in}{0.000000in}}%
\pgfusepath{stroke,fill}%
}%
\begin{pgfscope}%
\pgfsys@transformshift{0.675000in}{1.213393in}%
\pgfsys@useobject{currentmarker}{}%
\end{pgfscope}%
\end{pgfscope}%
\begin{pgfscope}%
\pgfsetbuttcap%
\pgfsetroundjoin%
\definecolor{currentfill}{rgb}{0.000000,0.000000,0.000000}%
\pgfsetfillcolor{currentfill}%
\pgfsetlinewidth{0.501875pt}%
\definecolor{currentstroke}{rgb}{0.000000,0.000000,0.000000}%
\pgfsetstrokecolor{currentstroke}%
\pgfsetdash{}{0pt}%
\pgfsys@defobject{currentmarker}{\pgfqpoint{-0.055556in}{0.000000in}}{\pgfqpoint{0.000000in}{0.000000in}}{%
\pgfpathmoveto{\pgfqpoint{0.000000in}{0.000000in}}%
\pgfpathlineto{\pgfqpoint{-0.055556in}{0.000000in}}%
\pgfusepath{stroke,fill}%
}%
\begin{pgfscope}%
\pgfsys@transformshift{4.275000in}{1.213393in}%
\pgfsys@useobject{currentmarker}{}%
\end{pgfscope}%
\end{pgfscope}%
\begin{pgfscope}%
\pgftext[x=0.619444in,y=1.213393in,right,]{\rmfamily\fontsize{9.000000}{10.800000}\selectfont \(\displaystyle 0.5\)}%
\end{pgfscope}%
\begin{pgfscope}%
\pgfsetbuttcap%
\pgfsetroundjoin%
\definecolor{currentfill}{rgb}{0.000000,0.000000,0.000000}%
\pgfsetfillcolor{currentfill}%
\pgfsetlinewidth{0.501875pt}%
\definecolor{currentstroke}{rgb}{0.000000,0.000000,0.000000}%
\pgfsetstrokecolor{currentstroke}%
\pgfsetdash{}{0pt}%
\pgfsys@defobject{currentmarker}{\pgfqpoint{0.000000in}{0.000000in}}{\pgfqpoint{0.055556in}{0.000000in}}{%
\pgfpathmoveto{\pgfqpoint{0.000000in}{0.000000in}}%
\pgfpathlineto{\pgfqpoint{0.055556in}{0.000000in}}%
\pgfusepath{stroke,fill}%
}%
\begin{pgfscope}%
\pgfsys@transformshift{0.675000in}{1.373485in}%
\pgfsys@useobject{currentmarker}{}%
\end{pgfscope}%
\end{pgfscope}%
\begin{pgfscope}%
\pgfsetbuttcap%
\pgfsetroundjoin%
\definecolor{currentfill}{rgb}{0.000000,0.000000,0.000000}%
\pgfsetfillcolor{currentfill}%
\pgfsetlinewidth{0.501875pt}%
\definecolor{currentstroke}{rgb}{0.000000,0.000000,0.000000}%
\pgfsetstrokecolor{currentstroke}%
\pgfsetdash{}{0pt}%
\pgfsys@defobject{currentmarker}{\pgfqpoint{-0.055556in}{0.000000in}}{\pgfqpoint{0.000000in}{0.000000in}}{%
\pgfpathmoveto{\pgfqpoint{0.000000in}{0.000000in}}%
\pgfpathlineto{\pgfqpoint{-0.055556in}{0.000000in}}%
\pgfusepath{stroke,fill}%
}%
\begin{pgfscope}%
\pgfsys@transformshift{4.275000in}{1.373485in}%
\pgfsys@useobject{currentmarker}{}%
\end{pgfscope}%
\end{pgfscope}%
\begin{pgfscope}%
\pgftext[x=0.619444in,y=1.373485in,right,]{\rmfamily\fontsize{9.000000}{10.800000}\selectfont \(\displaystyle 1.0\)}%
\end{pgfscope}%
\begin{pgfscope}%
\pgfsetbuttcap%
\pgfsetroundjoin%
\definecolor{currentfill}{rgb}{0.000000,0.000000,0.000000}%
\pgfsetfillcolor{currentfill}%
\pgfsetlinewidth{0.501875pt}%
\definecolor{currentstroke}{rgb}{0.000000,0.000000,0.000000}%
\pgfsetstrokecolor{currentstroke}%
\pgfsetdash{}{0pt}%
\pgfsys@defobject{currentmarker}{\pgfqpoint{0.000000in}{0.000000in}}{\pgfqpoint{0.055556in}{0.000000in}}{%
\pgfpathmoveto{\pgfqpoint{0.000000in}{0.000000in}}%
\pgfpathlineto{\pgfqpoint{0.055556in}{0.000000in}}%
\pgfusepath{stroke,fill}%
}%
\begin{pgfscope}%
\pgfsys@transformshift{0.675000in}{1.533576in}%
\pgfsys@useobject{currentmarker}{}%
\end{pgfscope}%
\end{pgfscope}%
\begin{pgfscope}%
\pgfsetbuttcap%
\pgfsetroundjoin%
\definecolor{currentfill}{rgb}{0.000000,0.000000,0.000000}%
\pgfsetfillcolor{currentfill}%
\pgfsetlinewidth{0.501875pt}%
\definecolor{currentstroke}{rgb}{0.000000,0.000000,0.000000}%
\pgfsetstrokecolor{currentstroke}%
\pgfsetdash{}{0pt}%
\pgfsys@defobject{currentmarker}{\pgfqpoint{-0.055556in}{0.000000in}}{\pgfqpoint{0.000000in}{0.000000in}}{%
\pgfpathmoveto{\pgfqpoint{0.000000in}{0.000000in}}%
\pgfpathlineto{\pgfqpoint{-0.055556in}{0.000000in}}%
\pgfusepath{stroke,fill}%
}%
\begin{pgfscope}%
\pgfsys@transformshift{4.275000in}{1.533576in}%
\pgfsys@useobject{currentmarker}{}%
\end{pgfscope}%
\end{pgfscope}%
\begin{pgfscope}%
\pgftext[x=0.619444in,y=1.533576in,right,]{\rmfamily\fontsize{9.000000}{10.800000}\selectfont \(\displaystyle 1.5\)}%
\end{pgfscope}%
\begin{pgfscope}%
\pgftext[x=0.285920in,y=1.053302in,,bottom,rotate=90.000000]{\rmfamily\fontsize{9.000000}{10.800000}\selectfont Spannung (V)}%
\end{pgfscope}%
\begin{pgfscope}%
\pgftext[x=2.475000in,y=1.651048in,,base]{\rmfamily\fontsize{11.000000}{13.200000}\selectfont Moduliertes Signal}%
\end{pgfscope}%
\end{pgfpicture}%
\makeatother%
\endgroup%

    \caption{%
        \emph{Frequency-shift  keying}: Oben   sind  die   zu  \"ubertragenden
        digitalen  Daten  als  \code{1}  und \code{0}  abgebildet,  unten  das
        zugeh\"orige Verhalten des  modulierten Signals.%
    }
    \label{fig:fsk:concept}
\end{figure}


\textbf{Amplitude-shift  keying}: Die  ASK (Amplitudenumtastung  auf  Deutsch)
benutzt statt verschiedenen Frequenzen unterschiedliche Amplituden, um Symbole
zu  codieren.  \fref{fig:ask:concept}  stellt  das  grundlegende  Konzept  des
Verfahrens schematisch dar.

Unsere Umsetzung w\"urde einen Transistor  benutzen, um jeweils ein Solarmodul
bei  einer Tr\"agerfrequenz  von  etwa \SI{10}{\kilo\hertz}  kurzzuschliessen.
Dadurch bricht die Spannung auf  einem Strang von seriell geschalteten Modulen
kurz ein, was als Signal  codiert und ausgewertet werden kann. Grunds\"atzlich
handelt  es sich  bei  diesem Prinzip  um  eine ASK  mit  den zwei  Amplituden
\SI{0}{\volt} und  dem Betrag des Spannungsabfalls  durch Kurzschliessen eines
Moduls.
\todo{Zahlen in Ticks auf Achsen weglassen?}

\begin{figure}[h!tb]
    \centering
    %% Creator: Matplotlib, PGF backend
%%
%% To include the figure in your LaTeX document, write
%%   \input{<filename>.pgf}
%%
%% Make sure the required packages are loaded in your preamble
%%   \usepackage{pgf}
%%
%% Figures using additional raster images can only be included by \input if
%% they are in the same directory as the main LaTeX file. For loading figures
%% from other directories you can use the `import` package
%%   \usepackage{import}
%% and then include the figures with
%%   \import{<path to file>}{<filename>.pgf}
%%
%% Matplotlib used the following preamble
%%   \usepackage{fontspec}
%%   \setmainfont{Bitstream Vera Serif}
%%   \setsansfont{Bitstream Vera Sans}
%%   \setmonofont{Bitstream Vera Sans Mono}
%%
\begingroup%
\makeatletter%
\begin{pgfpicture}%
\pgfpathrectangle{\pgfpointorigin}{\pgfqpoint{4.500000in}{5.000000in}}%
\pgfusepath{use as bounding box, clip}%
\begin{pgfscope}%
\pgfsetbuttcap%
\pgfsetmiterjoin%
\pgfsetlinewidth{0.000000pt}%
\definecolor{currentstroke}{rgb}{0.000000,0.000000,0.000000}%
\pgfsetstrokecolor{currentstroke}%
\pgfsetstrokeopacity{0.000000}%
\pgfsetdash{}{0pt}%
\pgfpathmoveto{\pgfqpoint{0.000000in}{0.000000in}}%
\pgfpathlineto{\pgfqpoint{4.500000in}{0.000000in}}%
\pgfpathlineto{\pgfqpoint{4.500000in}{5.000000in}}%
\pgfpathlineto{\pgfqpoint{0.000000in}{5.000000in}}%
\pgfpathclose%
\pgfusepath{}%
\end{pgfscope}%
\begin{pgfscope}%
\pgfsetbuttcap%
\pgfsetmiterjoin%
\pgfsetlinewidth{0.000000pt}%
\definecolor{currentstroke}{rgb}{0.000000,0.000000,0.000000}%
\pgfsetstrokecolor{currentstroke}%
\pgfsetstrokeopacity{0.000000}%
\pgfsetdash{}{0pt}%
\pgfpathmoveto{\pgfqpoint{0.675000in}{3.819767in}}%
\pgfpathlineto{\pgfqpoint{4.275000in}{3.819767in}}%
\pgfpathlineto{\pgfqpoint{4.275000in}{4.750000in}}%
\pgfpathlineto{\pgfqpoint{0.675000in}{4.750000in}}%
\pgfpathclose%
\pgfusepath{}%
\end{pgfscope}%
\begin{pgfscope}%
\pgfpathrectangle{\pgfqpoint{0.675000in}{3.819767in}}{\pgfqpoint{3.600000in}{0.930233in}} %
\pgfusepath{clip}%
\pgfsetrectcap%
\pgfsetroundjoin%
\pgfsetlinewidth{0.501875pt}%
\definecolor{currentstroke}{rgb}{0.000000,0.000000,1.000000}%
\pgfsetstrokecolor{currentstroke}%
\pgfsetdash{}{0pt}%
\pgfpathmoveto{\pgfqpoint{0.675000in}{3.897287in}}%
\pgfpathlineto{\pgfqpoint{1.531798in}{3.897287in}}%
\pgfusepath{stroke}%
\end{pgfscope}%
\begin{pgfscope}%
\pgfpathrectangle{\pgfqpoint{0.675000in}{3.819767in}}{\pgfqpoint{3.600000in}{0.930233in}} %
\pgfusepath{clip}%
\pgfsetrectcap%
\pgfsetroundjoin%
\pgfsetlinewidth{0.501875pt}%
\definecolor{currentstroke}{rgb}{0.501961,0.501961,0.501961}%
\pgfsetstrokecolor{currentstroke}%
\pgfsetdash{}{0pt}%
\pgfpathmoveto{\pgfqpoint{1.531798in}{3.897287in}}%
\pgfpathlineto{\pgfqpoint{1.531798in}{4.672481in}}%
\pgfusepath{stroke}%
\end{pgfscope}%
\begin{pgfscope}%
\pgfpathrectangle{\pgfqpoint{0.675000in}{3.819767in}}{\pgfqpoint{3.600000in}{0.930233in}} %
\pgfusepath{clip}%
\pgfsetrectcap%
\pgfsetroundjoin%
\pgfsetlinewidth{0.501875pt}%
\definecolor{currentstroke}{rgb}{1.000000,0.000000,1.000000}%
\pgfsetstrokecolor{currentstroke}%
\pgfsetdash{}{0pt}%
\pgfpathmoveto{\pgfqpoint{1.531798in}{4.672481in}}%
\pgfpathlineto{\pgfqpoint{2.388596in}{4.672481in}}%
\pgfusepath{stroke}%
\end{pgfscope}%
\begin{pgfscope}%
\pgfpathrectangle{\pgfqpoint{0.675000in}{3.819767in}}{\pgfqpoint{3.600000in}{0.930233in}} %
\pgfusepath{clip}%
\pgfsetrectcap%
\pgfsetroundjoin%
\pgfsetlinewidth{0.501875pt}%
\definecolor{currentstroke}{rgb}{0.501961,0.501961,0.501961}%
\pgfsetstrokecolor{currentstroke}%
\pgfsetdash{}{0pt}%
\pgfpathmoveto{\pgfqpoint{2.388596in}{4.672481in}}%
\pgfpathlineto{\pgfqpoint{2.388596in}{3.897287in}}%
\pgfusepath{stroke}%
\end{pgfscope}%
\begin{pgfscope}%
\pgfpathrectangle{\pgfqpoint{0.675000in}{3.819767in}}{\pgfqpoint{3.600000in}{0.930233in}} %
\pgfusepath{clip}%
\pgfsetrectcap%
\pgfsetroundjoin%
\pgfsetlinewidth{0.501875pt}%
\definecolor{currentstroke}{rgb}{0.000000,0.000000,1.000000}%
\pgfsetstrokecolor{currentstroke}%
\pgfsetdash{}{0pt}%
\pgfpathmoveto{\pgfqpoint{2.388596in}{3.897287in}}%
\pgfpathlineto{\pgfqpoint{3.245394in}{3.897287in}}%
\pgfusepath{stroke}%
\end{pgfscope}%
\begin{pgfscope}%
\pgfpathrectangle{\pgfqpoint{0.675000in}{3.819767in}}{\pgfqpoint{3.600000in}{0.930233in}} %
\pgfusepath{clip}%
\pgfsetrectcap%
\pgfsetroundjoin%
\pgfsetlinewidth{0.501875pt}%
\definecolor{currentstroke}{rgb}{0.501961,0.501961,0.501961}%
\pgfsetstrokecolor{currentstroke}%
\pgfsetdash{}{0pt}%
\pgfpathmoveto{\pgfqpoint{3.245394in}{3.897287in}}%
\pgfpathlineto{\pgfqpoint{3.245394in}{4.672481in}}%
\pgfusepath{stroke}%
\end{pgfscope}%
\begin{pgfscope}%
\pgfpathrectangle{\pgfqpoint{0.675000in}{3.819767in}}{\pgfqpoint{3.600000in}{0.930233in}} %
\pgfusepath{clip}%
\pgfsetrectcap%
\pgfsetroundjoin%
\pgfsetlinewidth{0.501875pt}%
\definecolor{currentstroke}{rgb}{1.000000,0.000000,1.000000}%
\pgfsetstrokecolor{currentstroke}%
\pgfsetdash{}{0pt}%
\pgfpathmoveto{\pgfqpoint{3.245394in}{4.672481in}}%
\pgfpathlineto{\pgfqpoint{4.102192in}{4.672481in}}%
\pgfusepath{stroke}%
\end{pgfscope}%
\begin{pgfscope}%
\pgfsetrectcap%
\pgfsetmiterjoin%
\pgfsetlinewidth{0.501875pt}%
\definecolor{currentstroke}{rgb}{0.000000,0.000000,0.000000}%
\pgfsetstrokecolor{currentstroke}%
\pgfsetdash{}{0pt}%
\pgfpathmoveto{\pgfqpoint{0.675000in}{3.819767in}}%
\pgfpathlineto{\pgfqpoint{0.675000in}{4.750000in}}%
\pgfusepath{stroke}%
\end{pgfscope}%
\begin{pgfscope}%
\pgfsetrectcap%
\pgfsetmiterjoin%
\pgfsetlinewidth{0.501875pt}%
\definecolor{currentstroke}{rgb}{0.000000,0.000000,0.000000}%
\pgfsetstrokecolor{currentstroke}%
\pgfsetdash{}{0pt}%
\pgfpathmoveto{\pgfqpoint{0.675000in}{3.819767in}}%
\pgfpathlineto{\pgfqpoint{4.275000in}{3.819767in}}%
\pgfusepath{stroke}%
\end{pgfscope}%
\begin{pgfscope}%
\pgfsetrectcap%
\pgfsetmiterjoin%
\pgfsetlinewidth{0.501875pt}%
\definecolor{currentstroke}{rgb}{0.000000,0.000000,0.000000}%
\pgfsetstrokecolor{currentstroke}%
\pgfsetdash{}{0pt}%
\pgfpathmoveto{\pgfqpoint{0.675000in}{4.750000in}}%
\pgfpathlineto{\pgfqpoint{4.275000in}{4.750000in}}%
\pgfusepath{stroke}%
\end{pgfscope}%
\begin{pgfscope}%
\pgfsetrectcap%
\pgfsetmiterjoin%
\pgfsetlinewidth{0.501875pt}%
\definecolor{currentstroke}{rgb}{0.000000,0.000000,0.000000}%
\pgfsetstrokecolor{currentstroke}%
\pgfsetdash{}{0pt}%
\pgfpathmoveto{\pgfqpoint{4.275000in}{3.819767in}}%
\pgfpathlineto{\pgfqpoint{4.275000in}{4.750000in}}%
\pgfusepath{stroke}%
\end{pgfscope}%
\begin{pgfscope}%
\pgfsetbuttcap%
\pgfsetroundjoin%
\definecolor{currentfill}{rgb}{0.000000,0.000000,0.000000}%
\pgfsetfillcolor{currentfill}%
\pgfsetlinewidth{0.501875pt}%
\definecolor{currentstroke}{rgb}{0.000000,0.000000,0.000000}%
\pgfsetstrokecolor{currentstroke}%
\pgfsetdash{}{0pt}%
\pgfsys@defobject{currentmarker}{\pgfqpoint{0.000000in}{0.000000in}}{\pgfqpoint{0.000000in}{0.055556in}}{%
\pgfpathmoveto{\pgfqpoint{0.000000in}{0.000000in}}%
\pgfpathlineto{\pgfqpoint{0.000000in}{0.055556in}}%
\pgfusepath{stroke,fill}%
}%
\begin{pgfscope}%
\pgfsys@transformshift{0.675000in}{3.819767in}%
\pgfsys@useobject{currentmarker}{}%
\end{pgfscope}%
\end{pgfscope}%
\begin{pgfscope}%
\pgfsetbuttcap%
\pgfsetroundjoin%
\definecolor{currentfill}{rgb}{0.000000,0.000000,0.000000}%
\pgfsetfillcolor{currentfill}%
\pgfsetlinewidth{0.501875pt}%
\definecolor{currentstroke}{rgb}{0.000000,0.000000,0.000000}%
\pgfsetstrokecolor{currentstroke}%
\pgfsetdash{}{0pt}%
\pgfsys@defobject{currentmarker}{\pgfqpoint{0.000000in}{-0.055556in}}{\pgfqpoint{0.000000in}{0.000000in}}{%
\pgfpathmoveto{\pgfqpoint{0.000000in}{0.000000in}}%
\pgfpathlineto{\pgfqpoint{0.000000in}{-0.055556in}}%
\pgfusepath{stroke,fill}%
}%
\begin{pgfscope}%
\pgfsys@transformshift{0.675000in}{4.750000in}%
\pgfsys@useobject{currentmarker}{}%
\end{pgfscope}%
\end{pgfscope}%
\begin{pgfscope}%
\pgftext[x=0.675000in,y=3.764212in,,top]{\rmfamily\fontsize{9.000000}{10.800000}\selectfont \(\displaystyle 0.0000\)}%
\end{pgfscope}%
\begin{pgfscope}%
\pgfsetbuttcap%
\pgfsetroundjoin%
\definecolor{currentfill}{rgb}{0.000000,0.000000,0.000000}%
\pgfsetfillcolor{currentfill}%
\pgfsetlinewidth{0.501875pt}%
\definecolor{currentstroke}{rgb}{0.000000,0.000000,0.000000}%
\pgfsetstrokecolor{currentstroke}%
\pgfsetdash{}{0pt}%
\pgfsys@defobject{currentmarker}{\pgfqpoint{0.000000in}{0.000000in}}{\pgfqpoint{0.000000in}{0.055556in}}{%
\pgfpathmoveto{\pgfqpoint{0.000000in}{0.000000in}}%
\pgfpathlineto{\pgfqpoint{0.000000in}{0.055556in}}%
\pgfusepath{stroke,fill}%
}%
\begin{pgfscope}%
\pgfsys@transformshift{1.125000in}{3.819767in}%
\pgfsys@useobject{currentmarker}{}%
\end{pgfscope}%
\end{pgfscope}%
\begin{pgfscope}%
\pgfsetbuttcap%
\pgfsetroundjoin%
\definecolor{currentfill}{rgb}{0.000000,0.000000,0.000000}%
\pgfsetfillcolor{currentfill}%
\pgfsetlinewidth{0.501875pt}%
\definecolor{currentstroke}{rgb}{0.000000,0.000000,0.000000}%
\pgfsetstrokecolor{currentstroke}%
\pgfsetdash{}{0pt}%
\pgfsys@defobject{currentmarker}{\pgfqpoint{0.000000in}{-0.055556in}}{\pgfqpoint{0.000000in}{0.000000in}}{%
\pgfpathmoveto{\pgfqpoint{0.000000in}{0.000000in}}%
\pgfpathlineto{\pgfqpoint{0.000000in}{-0.055556in}}%
\pgfusepath{stroke,fill}%
}%
\begin{pgfscope}%
\pgfsys@transformshift{1.125000in}{4.750000in}%
\pgfsys@useobject{currentmarker}{}%
\end{pgfscope}%
\end{pgfscope}%
\begin{pgfscope}%
\pgftext[x=1.125000in,y=3.764212in,,top]{\rmfamily\fontsize{9.000000}{10.800000}\selectfont \(\displaystyle 0.0002\)}%
\end{pgfscope}%
\begin{pgfscope}%
\pgfsetbuttcap%
\pgfsetroundjoin%
\definecolor{currentfill}{rgb}{0.000000,0.000000,0.000000}%
\pgfsetfillcolor{currentfill}%
\pgfsetlinewidth{0.501875pt}%
\definecolor{currentstroke}{rgb}{0.000000,0.000000,0.000000}%
\pgfsetstrokecolor{currentstroke}%
\pgfsetdash{}{0pt}%
\pgfsys@defobject{currentmarker}{\pgfqpoint{0.000000in}{0.000000in}}{\pgfqpoint{0.000000in}{0.055556in}}{%
\pgfpathmoveto{\pgfqpoint{0.000000in}{0.000000in}}%
\pgfpathlineto{\pgfqpoint{0.000000in}{0.055556in}}%
\pgfusepath{stroke,fill}%
}%
\begin{pgfscope}%
\pgfsys@transformshift{1.575000in}{3.819767in}%
\pgfsys@useobject{currentmarker}{}%
\end{pgfscope}%
\end{pgfscope}%
\begin{pgfscope}%
\pgfsetbuttcap%
\pgfsetroundjoin%
\definecolor{currentfill}{rgb}{0.000000,0.000000,0.000000}%
\pgfsetfillcolor{currentfill}%
\pgfsetlinewidth{0.501875pt}%
\definecolor{currentstroke}{rgb}{0.000000,0.000000,0.000000}%
\pgfsetstrokecolor{currentstroke}%
\pgfsetdash{}{0pt}%
\pgfsys@defobject{currentmarker}{\pgfqpoint{0.000000in}{-0.055556in}}{\pgfqpoint{0.000000in}{0.000000in}}{%
\pgfpathmoveto{\pgfqpoint{0.000000in}{0.000000in}}%
\pgfpathlineto{\pgfqpoint{0.000000in}{-0.055556in}}%
\pgfusepath{stroke,fill}%
}%
\begin{pgfscope}%
\pgfsys@transformshift{1.575000in}{4.750000in}%
\pgfsys@useobject{currentmarker}{}%
\end{pgfscope}%
\end{pgfscope}%
\begin{pgfscope}%
\pgftext[x=1.575000in,y=3.764212in,,top]{\rmfamily\fontsize{9.000000}{10.800000}\selectfont \(\displaystyle 0.0004\)}%
\end{pgfscope}%
\begin{pgfscope}%
\pgfsetbuttcap%
\pgfsetroundjoin%
\definecolor{currentfill}{rgb}{0.000000,0.000000,0.000000}%
\pgfsetfillcolor{currentfill}%
\pgfsetlinewidth{0.501875pt}%
\definecolor{currentstroke}{rgb}{0.000000,0.000000,0.000000}%
\pgfsetstrokecolor{currentstroke}%
\pgfsetdash{}{0pt}%
\pgfsys@defobject{currentmarker}{\pgfqpoint{0.000000in}{0.000000in}}{\pgfqpoint{0.000000in}{0.055556in}}{%
\pgfpathmoveto{\pgfqpoint{0.000000in}{0.000000in}}%
\pgfpathlineto{\pgfqpoint{0.000000in}{0.055556in}}%
\pgfusepath{stroke,fill}%
}%
\begin{pgfscope}%
\pgfsys@transformshift{2.025000in}{3.819767in}%
\pgfsys@useobject{currentmarker}{}%
\end{pgfscope}%
\end{pgfscope}%
\begin{pgfscope}%
\pgfsetbuttcap%
\pgfsetroundjoin%
\definecolor{currentfill}{rgb}{0.000000,0.000000,0.000000}%
\pgfsetfillcolor{currentfill}%
\pgfsetlinewidth{0.501875pt}%
\definecolor{currentstroke}{rgb}{0.000000,0.000000,0.000000}%
\pgfsetstrokecolor{currentstroke}%
\pgfsetdash{}{0pt}%
\pgfsys@defobject{currentmarker}{\pgfqpoint{0.000000in}{-0.055556in}}{\pgfqpoint{0.000000in}{0.000000in}}{%
\pgfpathmoveto{\pgfqpoint{0.000000in}{0.000000in}}%
\pgfpathlineto{\pgfqpoint{0.000000in}{-0.055556in}}%
\pgfusepath{stroke,fill}%
}%
\begin{pgfscope}%
\pgfsys@transformshift{2.025000in}{4.750000in}%
\pgfsys@useobject{currentmarker}{}%
\end{pgfscope}%
\end{pgfscope}%
\begin{pgfscope}%
\pgftext[x=2.025000in,y=3.764212in,,top]{\rmfamily\fontsize{9.000000}{10.800000}\selectfont \(\displaystyle 0.0006\)}%
\end{pgfscope}%
\begin{pgfscope}%
\pgfsetbuttcap%
\pgfsetroundjoin%
\definecolor{currentfill}{rgb}{0.000000,0.000000,0.000000}%
\pgfsetfillcolor{currentfill}%
\pgfsetlinewidth{0.501875pt}%
\definecolor{currentstroke}{rgb}{0.000000,0.000000,0.000000}%
\pgfsetstrokecolor{currentstroke}%
\pgfsetdash{}{0pt}%
\pgfsys@defobject{currentmarker}{\pgfqpoint{0.000000in}{0.000000in}}{\pgfqpoint{0.000000in}{0.055556in}}{%
\pgfpathmoveto{\pgfqpoint{0.000000in}{0.000000in}}%
\pgfpathlineto{\pgfqpoint{0.000000in}{0.055556in}}%
\pgfusepath{stroke,fill}%
}%
\begin{pgfscope}%
\pgfsys@transformshift{2.475000in}{3.819767in}%
\pgfsys@useobject{currentmarker}{}%
\end{pgfscope}%
\end{pgfscope}%
\begin{pgfscope}%
\pgfsetbuttcap%
\pgfsetroundjoin%
\definecolor{currentfill}{rgb}{0.000000,0.000000,0.000000}%
\pgfsetfillcolor{currentfill}%
\pgfsetlinewidth{0.501875pt}%
\definecolor{currentstroke}{rgb}{0.000000,0.000000,0.000000}%
\pgfsetstrokecolor{currentstroke}%
\pgfsetdash{}{0pt}%
\pgfsys@defobject{currentmarker}{\pgfqpoint{0.000000in}{-0.055556in}}{\pgfqpoint{0.000000in}{0.000000in}}{%
\pgfpathmoveto{\pgfqpoint{0.000000in}{0.000000in}}%
\pgfpathlineto{\pgfqpoint{0.000000in}{-0.055556in}}%
\pgfusepath{stroke,fill}%
}%
\begin{pgfscope}%
\pgfsys@transformshift{2.475000in}{4.750000in}%
\pgfsys@useobject{currentmarker}{}%
\end{pgfscope}%
\end{pgfscope}%
\begin{pgfscope}%
\pgftext[x=2.475000in,y=3.764212in,,top]{\rmfamily\fontsize{9.000000}{10.800000}\selectfont \(\displaystyle 0.0008\)}%
\end{pgfscope}%
\begin{pgfscope}%
\pgfsetbuttcap%
\pgfsetroundjoin%
\definecolor{currentfill}{rgb}{0.000000,0.000000,0.000000}%
\pgfsetfillcolor{currentfill}%
\pgfsetlinewidth{0.501875pt}%
\definecolor{currentstroke}{rgb}{0.000000,0.000000,0.000000}%
\pgfsetstrokecolor{currentstroke}%
\pgfsetdash{}{0pt}%
\pgfsys@defobject{currentmarker}{\pgfqpoint{0.000000in}{0.000000in}}{\pgfqpoint{0.000000in}{0.055556in}}{%
\pgfpathmoveto{\pgfqpoint{0.000000in}{0.000000in}}%
\pgfpathlineto{\pgfqpoint{0.000000in}{0.055556in}}%
\pgfusepath{stroke,fill}%
}%
\begin{pgfscope}%
\pgfsys@transformshift{2.925000in}{3.819767in}%
\pgfsys@useobject{currentmarker}{}%
\end{pgfscope}%
\end{pgfscope}%
\begin{pgfscope}%
\pgfsetbuttcap%
\pgfsetroundjoin%
\definecolor{currentfill}{rgb}{0.000000,0.000000,0.000000}%
\pgfsetfillcolor{currentfill}%
\pgfsetlinewidth{0.501875pt}%
\definecolor{currentstroke}{rgb}{0.000000,0.000000,0.000000}%
\pgfsetstrokecolor{currentstroke}%
\pgfsetdash{}{0pt}%
\pgfsys@defobject{currentmarker}{\pgfqpoint{0.000000in}{-0.055556in}}{\pgfqpoint{0.000000in}{0.000000in}}{%
\pgfpathmoveto{\pgfqpoint{0.000000in}{0.000000in}}%
\pgfpathlineto{\pgfqpoint{0.000000in}{-0.055556in}}%
\pgfusepath{stroke,fill}%
}%
\begin{pgfscope}%
\pgfsys@transformshift{2.925000in}{4.750000in}%
\pgfsys@useobject{currentmarker}{}%
\end{pgfscope}%
\end{pgfscope}%
\begin{pgfscope}%
\pgftext[x=2.925000in,y=3.764212in,,top]{\rmfamily\fontsize{9.000000}{10.800000}\selectfont \(\displaystyle 0.0010\)}%
\end{pgfscope}%
\begin{pgfscope}%
\pgfsetbuttcap%
\pgfsetroundjoin%
\definecolor{currentfill}{rgb}{0.000000,0.000000,0.000000}%
\pgfsetfillcolor{currentfill}%
\pgfsetlinewidth{0.501875pt}%
\definecolor{currentstroke}{rgb}{0.000000,0.000000,0.000000}%
\pgfsetstrokecolor{currentstroke}%
\pgfsetdash{}{0pt}%
\pgfsys@defobject{currentmarker}{\pgfqpoint{0.000000in}{0.000000in}}{\pgfqpoint{0.000000in}{0.055556in}}{%
\pgfpathmoveto{\pgfqpoint{0.000000in}{0.000000in}}%
\pgfpathlineto{\pgfqpoint{0.000000in}{0.055556in}}%
\pgfusepath{stroke,fill}%
}%
\begin{pgfscope}%
\pgfsys@transformshift{3.375000in}{3.819767in}%
\pgfsys@useobject{currentmarker}{}%
\end{pgfscope}%
\end{pgfscope}%
\begin{pgfscope}%
\pgfsetbuttcap%
\pgfsetroundjoin%
\definecolor{currentfill}{rgb}{0.000000,0.000000,0.000000}%
\pgfsetfillcolor{currentfill}%
\pgfsetlinewidth{0.501875pt}%
\definecolor{currentstroke}{rgb}{0.000000,0.000000,0.000000}%
\pgfsetstrokecolor{currentstroke}%
\pgfsetdash{}{0pt}%
\pgfsys@defobject{currentmarker}{\pgfqpoint{0.000000in}{-0.055556in}}{\pgfqpoint{0.000000in}{0.000000in}}{%
\pgfpathmoveto{\pgfqpoint{0.000000in}{0.000000in}}%
\pgfpathlineto{\pgfqpoint{0.000000in}{-0.055556in}}%
\pgfusepath{stroke,fill}%
}%
\begin{pgfscope}%
\pgfsys@transformshift{3.375000in}{4.750000in}%
\pgfsys@useobject{currentmarker}{}%
\end{pgfscope}%
\end{pgfscope}%
\begin{pgfscope}%
\pgftext[x=3.375000in,y=3.764212in,,top]{\rmfamily\fontsize{9.000000}{10.800000}\selectfont \(\displaystyle 0.0012\)}%
\end{pgfscope}%
\begin{pgfscope}%
\pgfsetbuttcap%
\pgfsetroundjoin%
\definecolor{currentfill}{rgb}{0.000000,0.000000,0.000000}%
\pgfsetfillcolor{currentfill}%
\pgfsetlinewidth{0.501875pt}%
\definecolor{currentstroke}{rgb}{0.000000,0.000000,0.000000}%
\pgfsetstrokecolor{currentstroke}%
\pgfsetdash{}{0pt}%
\pgfsys@defobject{currentmarker}{\pgfqpoint{0.000000in}{0.000000in}}{\pgfqpoint{0.000000in}{0.055556in}}{%
\pgfpathmoveto{\pgfqpoint{0.000000in}{0.000000in}}%
\pgfpathlineto{\pgfqpoint{0.000000in}{0.055556in}}%
\pgfusepath{stroke,fill}%
}%
\begin{pgfscope}%
\pgfsys@transformshift{3.825000in}{3.819767in}%
\pgfsys@useobject{currentmarker}{}%
\end{pgfscope}%
\end{pgfscope}%
\begin{pgfscope}%
\pgfsetbuttcap%
\pgfsetroundjoin%
\definecolor{currentfill}{rgb}{0.000000,0.000000,0.000000}%
\pgfsetfillcolor{currentfill}%
\pgfsetlinewidth{0.501875pt}%
\definecolor{currentstroke}{rgb}{0.000000,0.000000,0.000000}%
\pgfsetstrokecolor{currentstroke}%
\pgfsetdash{}{0pt}%
\pgfsys@defobject{currentmarker}{\pgfqpoint{0.000000in}{-0.055556in}}{\pgfqpoint{0.000000in}{0.000000in}}{%
\pgfpathmoveto{\pgfqpoint{0.000000in}{0.000000in}}%
\pgfpathlineto{\pgfqpoint{0.000000in}{-0.055556in}}%
\pgfusepath{stroke,fill}%
}%
\begin{pgfscope}%
\pgfsys@transformshift{3.825000in}{4.750000in}%
\pgfsys@useobject{currentmarker}{}%
\end{pgfscope}%
\end{pgfscope}%
\begin{pgfscope}%
\pgftext[x=3.825000in,y=3.764212in,,top]{\rmfamily\fontsize{9.000000}{10.800000}\selectfont \(\displaystyle 0.0014\)}%
\end{pgfscope}%
\begin{pgfscope}%
\pgfsetbuttcap%
\pgfsetroundjoin%
\definecolor{currentfill}{rgb}{0.000000,0.000000,0.000000}%
\pgfsetfillcolor{currentfill}%
\pgfsetlinewidth{0.501875pt}%
\definecolor{currentstroke}{rgb}{0.000000,0.000000,0.000000}%
\pgfsetstrokecolor{currentstroke}%
\pgfsetdash{}{0pt}%
\pgfsys@defobject{currentmarker}{\pgfqpoint{0.000000in}{0.000000in}}{\pgfqpoint{0.000000in}{0.055556in}}{%
\pgfpathmoveto{\pgfqpoint{0.000000in}{0.000000in}}%
\pgfpathlineto{\pgfqpoint{0.000000in}{0.055556in}}%
\pgfusepath{stroke,fill}%
}%
\begin{pgfscope}%
\pgfsys@transformshift{4.275000in}{3.819767in}%
\pgfsys@useobject{currentmarker}{}%
\end{pgfscope}%
\end{pgfscope}%
\begin{pgfscope}%
\pgfsetbuttcap%
\pgfsetroundjoin%
\definecolor{currentfill}{rgb}{0.000000,0.000000,0.000000}%
\pgfsetfillcolor{currentfill}%
\pgfsetlinewidth{0.501875pt}%
\definecolor{currentstroke}{rgb}{0.000000,0.000000,0.000000}%
\pgfsetstrokecolor{currentstroke}%
\pgfsetdash{}{0pt}%
\pgfsys@defobject{currentmarker}{\pgfqpoint{0.000000in}{-0.055556in}}{\pgfqpoint{0.000000in}{0.000000in}}{%
\pgfpathmoveto{\pgfqpoint{0.000000in}{0.000000in}}%
\pgfpathlineto{\pgfqpoint{0.000000in}{-0.055556in}}%
\pgfusepath{stroke,fill}%
}%
\begin{pgfscope}%
\pgfsys@transformshift{4.275000in}{4.750000in}%
\pgfsys@useobject{currentmarker}{}%
\end{pgfscope}%
\end{pgfscope}%
\begin{pgfscope}%
\pgftext[x=4.275000in,y=3.764212in,,top]{\rmfamily\fontsize{9.000000}{10.800000}\selectfont \(\displaystyle 0.0016\)}%
\end{pgfscope}%
\begin{pgfscope}%
\pgftext[x=2.475000in,y=3.573796in,,top]{\rmfamily\fontsize{9.000000}{10.800000}\selectfont Zeit (s)}%
\end{pgfscope}%
\begin{pgfscope}%
\pgfsetbuttcap%
\pgfsetroundjoin%
\definecolor{currentfill}{rgb}{0.000000,0.000000,0.000000}%
\pgfsetfillcolor{currentfill}%
\pgfsetlinewidth{0.501875pt}%
\definecolor{currentstroke}{rgb}{0.000000,0.000000,0.000000}%
\pgfsetstrokecolor{currentstroke}%
\pgfsetdash{}{0pt}%
\pgfsys@defobject{currentmarker}{\pgfqpoint{0.000000in}{0.000000in}}{\pgfqpoint{0.055556in}{0.000000in}}{%
\pgfpathmoveto{\pgfqpoint{0.000000in}{0.000000in}}%
\pgfpathlineto{\pgfqpoint{0.055556in}{0.000000in}}%
\pgfusepath{stroke,fill}%
}%
\begin{pgfscope}%
\pgfsys@transformshift{0.675000in}{3.897287in}%
\pgfsys@useobject{currentmarker}{}%
\end{pgfscope}%
\end{pgfscope}%
\begin{pgfscope}%
\pgfsetbuttcap%
\pgfsetroundjoin%
\definecolor{currentfill}{rgb}{0.000000,0.000000,0.000000}%
\pgfsetfillcolor{currentfill}%
\pgfsetlinewidth{0.501875pt}%
\definecolor{currentstroke}{rgb}{0.000000,0.000000,0.000000}%
\pgfsetstrokecolor{currentstroke}%
\pgfsetdash{}{0pt}%
\pgfsys@defobject{currentmarker}{\pgfqpoint{-0.055556in}{0.000000in}}{\pgfqpoint{0.000000in}{0.000000in}}{%
\pgfpathmoveto{\pgfqpoint{0.000000in}{0.000000in}}%
\pgfpathlineto{\pgfqpoint{-0.055556in}{0.000000in}}%
\pgfusepath{stroke,fill}%
}%
\begin{pgfscope}%
\pgfsys@transformshift{4.275000in}{3.897287in}%
\pgfsys@useobject{currentmarker}{}%
\end{pgfscope}%
\end{pgfscope}%
\begin{pgfscope}%
\pgftext[x=0.619444in,y=3.897287in,right,]{\rmfamily\fontsize{9.000000}{10.800000}\selectfont \(\displaystyle 0.0\)}%
\end{pgfscope}%
\begin{pgfscope}%
\pgfsetbuttcap%
\pgfsetroundjoin%
\definecolor{currentfill}{rgb}{0.000000,0.000000,0.000000}%
\pgfsetfillcolor{currentfill}%
\pgfsetlinewidth{0.501875pt}%
\definecolor{currentstroke}{rgb}{0.000000,0.000000,0.000000}%
\pgfsetstrokecolor{currentstroke}%
\pgfsetdash{}{0pt}%
\pgfsys@defobject{currentmarker}{\pgfqpoint{0.000000in}{0.000000in}}{\pgfqpoint{0.055556in}{0.000000in}}{%
\pgfpathmoveto{\pgfqpoint{0.000000in}{0.000000in}}%
\pgfpathlineto{\pgfqpoint{0.055556in}{0.000000in}}%
\pgfusepath{stroke,fill}%
}%
\begin{pgfscope}%
\pgfsys@transformshift{0.675000in}{4.052326in}%
\pgfsys@useobject{currentmarker}{}%
\end{pgfscope}%
\end{pgfscope}%
\begin{pgfscope}%
\pgfsetbuttcap%
\pgfsetroundjoin%
\definecolor{currentfill}{rgb}{0.000000,0.000000,0.000000}%
\pgfsetfillcolor{currentfill}%
\pgfsetlinewidth{0.501875pt}%
\definecolor{currentstroke}{rgb}{0.000000,0.000000,0.000000}%
\pgfsetstrokecolor{currentstroke}%
\pgfsetdash{}{0pt}%
\pgfsys@defobject{currentmarker}{\pgfqpoint{-0.055556in}{0.000000in}}{\pgfqpoint{0.000000in}{0.000000in}}{%
\pgfpathmoveto{\pgfqpoint{0.000000in}{0.000000in}}%
\pgfpathlineto{\pgfqpoint{-0.055556in}{0.000000in}}%
\pgfusepath{stroke,fill}%
}%
\begin{pgfscope}%
\pgfsys@transformshift{4.275000in}{4.052326in}%
\pgfsys@useobject{currentmarker}{}%
\end{pgfscope}%
\end{pgfscope}%
\begin{pgfscope}%
\pgftext[x=0.619444in,y=4.052326in,right,]{\rmfamily\fontsize{9.000000}{10.800000}\selectfont \(\displaystyle 0.2\)}%
\end{pgfscope}%
\begin{pgfscope}%
\pgfsetbuttcap%
\pgfsetroundjoin%
\definecolor{currentfill}{rgb}{0.000000,0.000000,0.000000}%
\pgfsetfillcolor{currentfill}%
\pgfsetlinewidth{0.501875pt}%
\definecolor{currentstroke}{rgb}{0.000000,0.000000,0.000000}%
\pgfsetstrokecolor{currentstroke}%
\pgfsetdash{}{0pt}%
\pgfsys@defobject{currentmarker}{\pgfqpoint{0.000000in}{0.000000in}}{\pgfqpoint{0.055556in}{0.000000in}}{%
\pgfpathmoveto{\pgfqpoint{0.000000in}{0.000000in}}%
\pgfpathlineto{\pgfqpoint{0.055556in}{0.000000in}}%
\pgfusepath{stroke,fill}%
}%
\begin{pgfscope}%
\pgfsys@transformshift{0.675000in}{4.207364in}%
\pgfsys@useobject{currentmarker}{}%
\end{pgfscope}%
\end{pgfscope}%
\begin{pgfscope}%
\pgfsetbuttcap%
\pgfsetroundjoin%
\definecolor{currentfill}{rgb}{0.000000,0.000000,0.000000}%
\pgfsetfillcolor{currentfill}%
\pgfsetlinewidth{0.501875pt}%
\definecolor{currentstroke}{rgb}{0.000000,0.000000,0.000000}%
\pgfsetstrokecolor{currentstroke}%
\pgfsetdash{}{0pt}%
\pgfsys@defobject{currentmarker}{\pgfqpoint{-0.055556in}{0.000000in}}{\pgfqpoint{0.000000in}{0.000000in}}{%
\pgfpathmoveto{\pgfqpoint{0.000000in}{0.000000in}}%
\pgfpathlineto{\pgfqpoint{-0.055556in}{0.000000in}}%
\pgfusepath{stroke,fill}%
}%
\begin{pgfscope}%
\pgfsys@transformshift{4.275000in}{4.207364in}%
\pgfsys@useobject{currentmarker}{}%
\end{pgfscope}%
\end{pgfscope}%
\begin{pgfscope}%
\pgftext[x=0.619444in,y=4.207364in,right,]{\rmfamily\fontsize{9.000000}{10.800000}\selectfont \(\displaystyle 0.4\)}%
\end{pgfscope}%
\begin{pgfscope}%
\pgfsetbuttcap%
\pgfsetroundjoin%
\definecolor{currentfill}{rgb}{0.000000,0.000000,0.000000}%
\pgfsetfillcolor{currentfill}%
\pgfsetlinewidth{0.501875pt}%
\definecolor{currentstroke}{rgb}{0.000000,0.000000,0.000000}%
\pgfsetstrokecolor{currentstroke}%
\pgfsetdash{}{0pt}%
\pgfsys@defobject{currentmarker}{\pgfqpoint{0.000000in}{0.000000in}}{\pgfqpoint{0.055556in}{0.000000in}}{%
\pgfpathmoveto{\pgfqpoint{0.000000in}{0.000000in}}%
\pgfpathlineto{\pgfqpoint{0.055556in}{0.000000in}}%
\pgfusepath{stroke,fill}%
}%
\begin{pgfscope}%
\pgfsys@transformshift{0.675000in}{4.362403in}%
\pgfsys@useobject{currentmarker}{}%
\end{pgfscope}%
\end{pgfscope}%
\begin{pgfscope}%
\pgfsetbuttcap%
\pgfsetroundjoin%
\definecolor{currentfill}{rgb}{0.000000,0.000000,0.000000}%
\pgfsetfillcolor{currentfill}%
\pgfsetlinewidth{0.501875pt}%
\definecolor{currentstroke}{rgb}{0.000000,0.000000,0.000000}%
\pgfsetstrokecolor{currentstroke}%
\pgfsetdash{}{0pt}%
\pgfsys@defobject{currentmarker}{\pgfqpoint{-0.055556in}{0.000000in}}{\pgfqpoint{0.000000in}{0.000000in}}{%
\pgfpathmoveto{\pgfqpoint{0.000000in}{0.000000in}}%
\pgfpathlineto{\pgfqpoint{-0.055556in}{0.000000in}}%
\pgfusepath{stroke,fill}%
}%
\begin{pgfscope}%
\pgfsys@transformshift{4.275000in}{4.362403in}%
\pgfsys@useobject{currentmarker}{}%
\end{pgfscope}%
\end{pgfscope}%
\begin{pgfscope}%
\pgftext[x=0.619444in,y=4.362403in,right,]{\rmfamily\fontsize{9.000000}{10.800000}\selectfont \(\displaystyle 0.6\)}%
\end{pgfscope}%
\begin{pgfscope}%
\pgfsetbuttcap%
\pgfsetroundjoin%
\definecolor{currentfill}{rgb}{0.000000,0.000000,0.000000}%
\pgfsetfillcolor{currentfill}%
\pgfsetlinewidth{0.501875pt}%
\definecolor{currentstroke}{rgb}{0.000000,0.000000,0.000000}%
\pgfsetstrokecolor{currentstroke}%
\pgfsetdash{}{0pt}%
\pgfsys@defobject{currentmarker}{\pgfqpoint{0.000000in}{0.000000in}}{\pgfqpoint{0.055556in}{0.000000in}}{%
\pgfpathmoveto{\pgfqpoint{0.000000in}{0.000000in}}%
\pgfpathlineto{\pgfqpoint{0.055556in}{0.000000in}}%
\pgfusepath{stroke,fill}%
}%
\begin{pgfscope}%
\pgfsys@transformshift{0.675000in}{4.517442in}%
\pgfsys@useobject{currentmarker}{}%
\end{pgfscope}%
\end{pgfscope}%
\begin{pgfscope}%
\pgfsetbuttcap%
\pgfsetroundjoin%
\definecolor{currentfill}{rgb}{0.000000,0.000000,0.000000}%
\pgfsetfillcolor{currentfill}%
\pgfsetlinewidth{0.501875pt}%
\definecolor{currentstroke}{rgb}{0.000000,0.000000,0.000000}%
\pgfsetstrokecolor{currentstroke}%
\pgfsetdash{}{0pt}%
\pgfsys@defobject{currentmarker}{\pgfqpoint{-0.055556in}{0.000000in}}{\pgfqpoint{0.000000in}{0.000000in}}{%
\pgfpathmoveto{\pgfqpoint{0.000000in}{0.000000in}}%
\pgfpathlineto{\pgfqpoint{-0.055556in}{0.000000in}}%
\pgfusepath{stroke,fill}%
}%
\begin{pgfscope}%
\pgfsys@transformshift{4.275000in}{4.517442in}%
\pgfsys@useobject{currentmarker}{}%
\end{pgfscope}%
\end{pgfscope}%
\begin{pgfscope}%
\pgftext[x=0.619444in,y=4.517442in,right,]{\rmfamily\fontsize{9.000000}{10.800000}\selectfont \(\displaystyle 0.8\)}%
\end{pgfscope}%
\begin{pgfscope}%
\pgfsetbuttcap%
\pgfsetroundjoin%
\definecolor{currentfill}{rgb}{0.000000,0.000000,0.000000}%
\pgfsetfillcolor{currentfill}%
\pgfsetlinewidth{0.501875pt}%
\definecolor{currentstroke}{rgb}{0.000000,0.000000,0.000000}%
\pgfsetstrokecolor{currentstroke}%
\pgfsetdash{}{0pt}%
\pgfsys@defobject{currentmarker}{\pgfqpoint{0.000000in}{0.000000in}}{\pgfqpoint{0.055556in}{0.000000in}}{%
\pgfpathmoveto{\pgfqpoint{0.000000in}{0.000000in}}%
\pgfpathlineto{\pgfqpoint{0.055556in}{0.000000in}}%
\pgfusepath{stroke,fill}%
}%
\begin{pgfscope}%
\pgfsys@transformshift{0.675000in}{4.672481in}%
\pgfsys@useobject{currentmarker}{}%
\end{pgfscope}%
\end{pgfscope}%
\begin{pgfscope}%
\pgfsetbuttcap%
\pgfsetroundjoin%
\definecolor{currentfill}{rgb}{0.000000,0.000000,0.000000}%
\pgfsetfillcolor{currentfill}%
\pgfsetlinewidth{0.501875pt}%
\definecolor{currentstroke}{rgb}{0.000000,0.000000,0.000000}%
\pgfsetstrokecolor{currentstroke}%
\pgfsetdash{}{0pt}%
\pgfsys@defobject{currentmarker}{\pgfqpoint{-0.055556in}{0.000000in}}{\pgfqpoint{0.000000in}{0.000000in}}{%
\pgfpathmoveto{\pgfqpoint{0.000000in}{0.000000in}}%
\pgfpathlineto{\pgfqpoint{-0.055556in}{0.000000in}}%
\pgfusepath{stroke,fill}%
}%
\begin{pgfscope}%
\pgfsys@transformshift{4.275000in}{4.672481in}%
\pgfsys@useobject{currentmarker}{}%
\end{pgfscope}%
\end{pgfscope}%
\begin{pgfscope}%
\pgftext[x=0.619444in,y=4.672481in,right,]{\rmfamily\fontsize{9.000000}{10.800000}\selectfont \(\displaystyle 1.0\)}%
\end{pgfscope}%
\begin{pgfscope}%
\pgftext[x=0.385842in,y=4.284884in,,bottom,rotate=90.000000]{\rmfamily\fontsize{9.000000}{10.800000}\selectfont Symbol}%
\end{pgfscope}%
\begin{pgfscope}%
\pgftext[x=2.475000in,y=4.819444in,,base]{\rmfamily\fontsize{11.000000}{13.200000}\selectfont Daten}%
\end{pgfscope}%
\begin{pgfscope}%
\pgfsetbuttcap%
\pgfsetmiterjoin%
\pgfsetlinewidth{0.000000pt}%
\definecolor{currentstroke}{rgb}{0.000000,0.000000,0.000000}%
\pgfsetstrokecolor{currentstroke}%
\pgfsetstrokeopacity{0.000000}%
\pgfsetdash{}{0pt}%
\pgfpathmoveto{\pgfqpoint{0.675000in}{2.284884in}}%
\pgfpathlineto{\pgfqpoint{4.275000in}{2.284884in}}%
\pgfpathlineto{\pgfqpoint{4.275000in}{3.215116in}}%
\pgfpathlineto{\pgfqpoint{0.675000in}{3.215116in}}%
\pgfpathclose%
\pgfusepath{}%
\end{pgfscope}%
\begin{pgfscope}%
\pgfpathrectangle{\pgfqpoint{0.675000in}{2.284884in}}{\pgfqpoint{3.600000in}{0.930233in}} %
\pgfusepath{clip}%
\pgfsetrectcap%
\pgfsetroundjoin%
\pgfsetlinewidth{0.501875pt}%
\definecolor{currentstroke}{rgb}{0.000000,0.000000,1.000000}%
\pgfsetstrokecolor{currentstroke}%
\pgfsetdash{}{0pt}%
\pgfpathmoveto{\pgfqpoint{0.675000in}{2.750000in}}%
\pgfpathlineto{\pgfqpoint{0.695584in}{2.790012in}}%
\pgfpathlineto{\pgfqpoint{0.705876in}{2.805451in}}%
\pgfpathlineto{\pgfqpoint{0.714452in}{2.814532in}}%
\pgfpathlineto{\pgfqpoint{0.721313in}{2.818890in}}%
\pgfpathlineto{\pgfqpoint{0.727317in}{2.820426in}}%
\pgfpathlineto{\pgfqpoint{0.733321in}{2.819783in}}%
\pgfpathlineto{\pgfqpoint{0.739324in}{2.816982in}}%
\pgfpathlineto{\pgfqpoint{0.746185in}{2.811251in}}%
\pgfpathlineto{\pgfqpoint{0.753904in}{2.801864in}}%
\pgfpathlineto{\pgfqpoint{0.763339in}{2.786855in}}%
\pgfpathlineto{\pgfqpoint{0.777061in}{2.760378in}}%
\pgfpathlineto{\pgfqpoint{0.803648in}{2.708363in}}%
\pgfpathlineto{\pgfqpoint{0.813940in}{2.693341in}}%
\pgfpathlineto{\pgfqpoint{0.822517in}{2.684693in}}%
\pgfpathlineto{\pgfqpoint{0.829378in}{2.680717in}}%
\pgfpathlineto{\pgfqpoint{0.835382in}{2.679530in}}%
\pgfpathlineto{\pgfqpoint{0.841385in}{2.680523in}}%
\pgfpathlineto{\pgfqpoint{0.847389in}{2.683665in}}%
\pgfpathlineto{\pgfqpoint{0.854250in}{2.689760in}}%
\pgfpathlineto{\pgfqpoint{0.862827in}{2.700760in}}%
\pgfpathlineto{\pgfqpoint{0.873118in}{2.717977in}}%
\pgfpathlineto{\pgfqpoint{0.889414in}{2.750443in}}%
\pgfpathlineto{\pgfqpoint{0.909140in}{2.788910in}}%
\pgfpathlineto{\pgfqpoint{0.919432in}{2.804620in}}%
\pgfpathlineto{\pgfqpoint{0.928008in}{2.813986in}}%
\pgfpathlineto{\pgfqpoint{0.934870in}{2.818598in}}%
\pgfpathlineto{\pgfqpoint{0.940873in}{2.820365in}}%
\pgfpathlineto{\pgfqpoint{0.946877in}{2.819956in}}%
\pgfpathlineto{\pgfqpoint{0.952880in}{2.817383in}}%
\pgfpathlineto{\pgfqpoint{0.959742in}{2.811898in}}%
\pgfpathlineto{\pgfqpoint{0.967461in}{2.802755in}}%
\pgfpathlineto{\pgfqpoint{0.976895in}{2.787981in}}%
\pgfpathlineto{\pgfqpoint{0.990617in}{2.761691in}}%
\pgfpathlineto{\pgfqpoint{1.018062in}{2.708006in}}%
\pgfpathlineto{\pgfqpoint{1.028354in}{2.693078in}}%
\pgfpathlineto{\pgfqpoint{1.036931in}{2.684527in}}%
\pgfpathlineto{\pgfqpoint{1.043792in}{2.680637in}}%
\pgfpathlineto{\pgfqpoint{1.049796in}{2.679528in}}%
\pgfpathlineto{\pgfqpoint{1.055799in}{2.680598in}}%
\pgfpathlineto{\pgfqpoint{1.061803in}{2.683816in}}%
\pgfpathlineto{\pgfqpoint{1.068664in}{2.689991in}}%
\pgfpathlineto{\pgfqpoint{1.077241in}{2.701078in}}%
\pgfpathlineto{\pgfqpoint{1.087532in}{2.718372in}}%
\pgfpathlineto{\pgfqpoint{1.103828in}{2.750886in}}%
\pgfpathlineto{\pgfqpoint{1.123554in}{2.789279in}}%
\pgfpathlineto{\pgfqpoint{1.133846in}{2.804899in}}%
\pgfpathlineto{\pgfqpoint{1.142422in}{2.814171in}}%
\pgfpathlineto{\pgfqpoint{1.149284in}{2.818698in}}%
\pgfpathlineto{\pgfqpoint{1.155287in}{2.820388in}}%
\pgfpathlineto{\pgfqpoint{1.161291in}{2.819901in}}%
\pgfpathlineto{\pgfqpoint{1.167294in}{2.817252in}}%
\pgfpathlineto{\pgfqpoint{1.174156in}{2.811685in}}%
\pgfpathlineto{\pgfqpoint{1.181874in}{2.802460in}}%
\pgfpathlineto{\pgfqpoint{1.191309in}{2.787607in}}%
\pgfpathlineto{\pgfqpoint{1.205031in}{2.761254in}}%
\pgfpathlineto{\pgfqpoint{1.232476in}{2.707651in}}%
\pgfpathlineto{\pgfqpoint{1.242768in}{2.692818in}}%
\pgfpathlineto{\pgfqpoint{1.250487in}{2.685031in}}%
\pgfpathlineto{\pgfqpoint{1.257348in}{2.680885in}}%
\pgfpathlineto{\pgfqpoint{1.263352in}{2.679543in}}%
\pgfpathlineto{\pgfqpoint{1.269355in}{2.680380in}}%
\pgfpathlineto{\pgfqpoint{1.275359in}{2.683371in}}%
\pgfpathlineto{\pgfqpoint{1.282220in}{2.689304in}}%
\pgfpathlineto{\pgfqpoint{1.290797in}{2.700129in}}%
\pgfpathlineto{\pgfqpoint{1.301089in}{2.717190in}}%
\pgfpathlineto{\pgfqpoint{1.316526in}{2.747784in}}%
\pgfpathlineto{\pgfqpoint{1.337968in}{2.789646in}}%
\pgfpathlineto{\pgfqpoint{1.348260in}{2.805176in}}%
\pgfpathlineto{\pgfqpoint{1.356836in}{2.814352in}}%
\pgfpathlineto{\pgfqpoint{1.363697in}{2.818796in}}%
\pgfpathlineto{\pgfqpoint{1.369701in}{2.820409in}}%
\pgfpathlineto{\pgfqpoint{1.375705in}{2.819844in}}%
\pgfpathlineto{\pgfqpoint{1.381708in}{2.817118in}}%
\pgfpathlineto{\pgfqpoint{1.388570in}{2.811469in}}%
\pgfpathlineto{\pgfqpoint{1.396288in}{2.802163in}}%
\pgfpathlineto{\pgfqpoint{1.405723in}{2.787232in}}%
\pgfpathlineto{\pgfqpoint{1.419445in}{2.760816in}}%
\pgfpathlineto{\pgfqpoint{1.446032in}{2.708721in}}%
\pgfpathlineto{\pgfqpoint{1.456324in}{2.693605in}}%
\pgfpathlineto{\pgfqpoint{1.464901in}{2.684860in}}%
\pgfpathlineto{\pgfqpoint{1.471762in}{2.680799in}}%
\pgfpathlineto{\pgfqpoint{1.477766in}{2.679535in}}%
\pgfpathlineto{\pgfqpoint{1.483769in}{2.680450in}}%
\pgfpathlineto{\pgfqpoint{1.489773in}{2.683517in}}%
\pgfpathlineto{\pgfqpoint{1.496634in}{2.689531in}}%
\pgfpathlineto{\pgfqpoint{1.505211in}{2.700444in}}%
\pgfpathlineto{\pgfqpoint{1.515503in}{2.717582in}}%
\pgfpathlineto{\pgfqpoint{1.531798in}{2.750000in}}%
\pgfpathlineto{\pgfqpoint{1.531798in}{2.750000in}}%
\pgfusepath{stroke}%
\end{pgfscope}%
\begin{pgfscope}%
\pgfpathrectangle{\pgfqpoint{0.675000in}{2.284884in}}{\pgfqpoint{3.600000in}{0.930233in}} %
\pgfusepath{clip}%
\pgfsetrectcap%
\pgfsetroundjoin%
\pgfsetlinewidth{0.501875pt}%
\definecolor{currentstroke}{rgb}{1.000000,0.000000,1.000000}%
\pgfsetstrokecolor{currentstroke}%
\pgfsetdash{}{0pt}%
\pgfpathmoveto{\pgfqpoint{1.531798in}{2.750000in}}%
\pgfpathlineto{\pgfqpoint{1.551524in}{2.981238in}}%
\pgfpathlineto{\pgfqpoint{1.561816in}{3.076036in}}%
\pgfpathlineto{\pgfqpoint{1.569535in}{3.128155in}}%
\pgfpathlineto{\pgfqpoint{1.575538in}{3.155449in}}%
\pgfpathlineto{\pgfqpoint{1.579827in}{3.167300in}}%
\pgfpathlineto{\pgfqpoint{1.583257in}{3.172038in}}%
\pgfpathlineto{\pgfqpoint{1.585830in}{3.172791in}}%
\pgfpathlineto{\pgfqpoint{1.588403in}{3.171136in}}%
\pgfpathlineto{\pgfqpoint{1.591834in}{3.165203in}}%
\pgfpathlineto{\pgfqpoint{1.596122in}{3.151891in}}%
\pgfpathlineto{\pgfqpoint{1.601268in}{3.127559in}}%
\pgfpathlineto{\pgfqpoint{1.608129in}{3.081883in}}%
\pgfpathlineto{\pgfqpoint{1.616706in}{3.006216in}}%
\pgfpathlineto{\pgfqpoint{1.627855in}{2.884577in}}%
\pgfpathlineto{\pgfqpoint{1.665592in}{2.451718in}}%
\pgfpathlineto{\pgfqpoint{1.674169in}{2.386504in}}%
\pgfpathlineto{\pgfqpoint{1.681030in}{2.350660in}}%
\pgfpathlineto{\pgfqpoint{1.686176in}{2.334302in}}%
\pgfpathlineto{\pgfqpoint{1.689607in}{2.328634in}}%
\pgfpathlineto{\pgfqpoint{1.692180in}{2.327180in}}%
\pgfpathlineto{\pgfqpoint{1.694753in}{2.328133in}}%
\pgfpathlineto{\pgfqpoint{1.697326in}{2.331488in}}%
\pgfpathlineto{\pgfqpoint{1.700756in}{2.339663in}}%
\pgfpathlineto{\pgfqpoint{1.705044in}{2.355706in}}%
\pgfpathlineto{\pgfqpoint{1.711048in}{2.388558in}}%
\pgfpathlineto{\pgfqpoint{1.717909in}{2.439719in}}%
\pgfpathlineto{\pgfqpoint{1.726486in}{2.520993in}}%
\pgfpathlineto{\pgfqpoint{1.738493in}{2.657671in}}%
\pgfpathlineto{\pgfqpoint{1.768511in}{3.009378in}}%
\pgfpathlineto{\pgfqpoint{1.777945in}{3.090746in}}%
\pgfpathlineto{\pgfqpoint{1.784806in}{3.133916in}}%
\pgfpathlineto{\pgfqpoint{1.790810in}{3.159021in}}%
\pgfpathlineto{\pgfqpoint{1.795098in}{3.169237in}}%
\pgfpathlineto{\pgfqpoint{1.798529in}{3.172644in}}%
\pgfpathlineto{\pgfqpoint{1.801102in}{3.172393in}}%
\pgfpathlineto{\pgfqpoint{1.803675in}{3.169738in}}%
\pgfpathlineto{\pgfqpoint{1.807105in}{3.162484in}}%
\pgfpathlineto{\pgfqpoint{1.811394in}{3.147560in}}%
\pgfpathlineto{\pgfqpoint{1.816540in}{3.121386in}}%
\pgfpathlineto{\pgfqpoint{1.823401in}{3.073481in}}%
\pgfpathlineto{\pgfqpoint{1.831977in}{2.995513in}}%
\pgfpathlineto{\pgfqpoint{1.843127in}{2.871907in}}%
\pgfpathlineto{\pgfqpoint{1.878291in}{2.465197in}}%
\pgfpathlineto{\pgfqpoint{1.886867in}{2.396363in}}%
\pgfpathlineto{\pgfqpoint{1.893729in}{2.357164in}}%
\pgfpathlineto{\pgfqpoint{1.898875in}{2.338109in}}%
\pgfpathlineto{\pgfqpoint{1.903163in}{2.329373in}}%
\pgfpathlineto{\pgfqpoint{1.906594in}{2.327168in}}%
\pgfpathlineto{\pgfqpoint{1.909166in}{2.328321in}}%
\pgfpathlineto{\pgfqpoint{1.912597in}{2.333591in}}%
\pgfpathlineto{\pgfqpoint{1.916885in}{2.346093in}}%
\pgfpathlineto{\pgfqpoint{1.922031in}{2.369495in}}%
\pgfpathlineto{\pgfqpoint{1.928035in}{2.407687in}}%
\pgfpathlineto{\pgfqpoint{1.935754in}{2.472145in}}%
\pgfpathlineto{\pgfqpoint{1.946046in}{2.579477in}}%
\pgfpathlineto{\pgfqpoint{1.963199in}{2.787183in}}%
\pgfpathlineto{\pgfqpoint{1.980352in}{2.985673in}}%
\pgfpathlineto{\pgfqpoint{1.990644in}{3.079396in}}%
\pgfpathlineto{\pgfqpoint{1.998363in}{3.130505in}}%
\pgfpathlineto{\pgfqpoint{2.004366in}{3.156926in}}%
\pgfpathlineto{\pgfqpoint{2.008655in}{3.168124in}}%
\pgfpathlineto{\pgfqpoint{2.012085in}{3.172331in}}%
\pgfpathlineto{\pgfqpoint{2.014658in}{3.172682in}}%
\pgfpathlineto{\pgfqpoint{2.017231in}{3.170627in}}%
\pgfpathlineto{\pgfqpoint{2.020662in}{3.164165in}}%
\pgfpathlineto{\pgfqpoint{2.024950in}{3.150206in}}%
\pgfpathlineto{\pgfqpoint{2.030096in}{3.125134in}}%
\pgfpathlineto{\pgfqpoint{2.036957in}{3.078561in}}%
\pgfpathlineto{\pgfqpoint{2.045534in}{3.001965in}}%
\pgfpathlineto{\pgfqpoint{2.056683in}{2.879524in}}%
\pgfpathlineto{\pgfqpoint{2.093562in}{2.455511in}}%
\pgfpathlineto{\pgfqpoint{2.102139in}{2.389250in}}%
\pgfpathlineto{\pgfqpoint{2.109000in}{2.352440in}}%
\pgfpathlineto{\pgfqpoint{2.114146in}{2.335308in}}%
\pgfpathlineto{\pgfqpoint{2.118434in}{2.328225in}}%
\pgfpathlineto{\pgfqpoint{2.121007in}{2.327172in}}%
\pgfpathlineto{\pgfqpoint{2.123580in}{2.328526in}}%
\pgfpathlineto{\pgfqpoint{2.127011in}{2.334061in}}%
\pgfpathlineto{\pgfqpoint{2.131299in}{2.346887in}}%
\pgfpathlineto{\pgfqpoint{2.136445in}{2.370662in}}%
\pgfpathlineto{\pgfqpoint{2.142449in}{2.409254in}}%
\pgfpathlineto{\pgfqpoint{2.150168in}{2.474155in}}%
\pgfpathlineto{\pgfqpoint{2.160460in}{2.581914in}}%
\pgfpathlineto{\pgfqpoint{2.178470in}{2.800408in}}%
\pgfpathlineto{\pgfqpoint{2.194766in}{2.987876in}}%
\pgfpathlineto{\pgfqpoint{2.204200in}{3.074336in}}%
\pgfpathlineto{\pgfqpoint{2.211919in}{3.126958in}}%
\pgfpathlineto{\pgfqpoint{2.217923in}{3.154686in}}%
\pgfpathlineto{\pgfqpoint{2.223068in}{3.168512in}}%
\pgfpathlineto{\pgfqpoint{2.226499in}{3.172452in}}%
\pgfpathlineto{\pgfqpoint{2.229072in}{3.172602in}}%
\pgfpathlineto{\pgfqpoint{2.231645in}{3.170347in}}%
\pgfpathlineto{\pgfqpoint{2.235076in}{3.163621in}}%
\pgfpathlineto{\pgfqpoint{2.239364in}{3.149340in}}%
\pgfpathlineto{\pgfqpoint{2.244510in}{3.123900in}}%
\pgfpathlineto{\pgfqpoint{2.251371in}{3.076881in}}%
\pgfpathlineto{\pgfqpoint{2.259948in}{2.999824in}}%
\pgfpathlineto{\pgfqpoint{2.271097in}{2.876990in}}%
\pgfpathlineto{\pgfqpoint{2.307119in}{2.461288in}}%
\pgfpathlineto{\pgfqpoint{2.315695in}{2.393476in}}%
\pgfpathlineto{\pgfqpoint{2.322557in}{2.355228in}}%
\pgfpathlineto{\pgfqpoint{2.327702in}{2.336940in}}%
\pgfpathlineto{\pgfqpoint{2.331991in}{2.328864in}}%
\pgfpathlineto{\pgfqpoint{2.334564in}{2.327209in}}%
\pgfpathlineto{\pgfqpoint{2.337137in}{2.327962in}}%
\pgfpathlineto{\pgfqpoint{2.339710in}{2.331117in}}%
\pgfpathlineto{\pgfqpoint{2.343140in}{2.339029in}}%
\pgfpathlineto{\pgfqpoint{2.347429in}{2.354753in}}%
\pgfpathlineto{\pgfqpoint{2.353432in}{2.387185in}}%
\pgfpathlineto{\pgfqpoint{2.360293in}{2.437918in}}%
\pgfpathlineto{\pgfqpoint{2.368870in}{2.518762in}}%
\pgfpathlineto{\pgfqpoint{2.380877in}{2.655078in}}%
\pgfpathlineto{\pgfqpoint{2.388596in}{2.750000in}}%
\pgfpathlineto{\pgfqpoint{2.388596in}{2.750000in}}%
\pgfusepath{stroke}%
\end{pgfscope}%
\begin{pgfscope}%
\pgfpathrectangle{\pgfqpoint{0.675000in}{2.284884in}}{\pgfqpoint{3.600000in}{0.930233in}} %
\pgfusepath{clip}%
\pgfsetrectcap%
\pgfsetroundjoin%
\pgfsetlinewidth{0.501875pt}%
\definecolor{currentstroke}{rgb}{0.000000,0.000000,1.000000}%
\pgfsetstrokecolor{currentstroke}%
\pgfsetdash{}{0pt}%
\pgfpathmoveto{\pgfqpoint{2.388596in}{2.750000in}}%
\pgfpathlineto{\pgfqpoint{2.409180in}{2.790012in}}%
\pgfpathlineto{\pgfqpoint{2.419472in}{2.805451in}}%
\pgfpathlineto{\pgfqpoint{2.428048in}{2.814532in}}%
\pgfpathlineto{\pgfqpoint{2.434909in}{2.818890in}}%
\pgfpathlineto{\pgfqpoint{2.440913in}{2.820426in}}%
\pgfpathlineto{\pgfqpoint{2.446917in}{2.819783in}}%
\pgfpathlineto{\pgfqpoint{2.452920in}{2.816982in}}%
\pgfpathlineto{\pgfqpoint{2.459781in}{2.811251in}}%
\pgfpathlineto{\pgfqpoint{2.467500in}{2.801864in}}%
\pgfpathlineto{\pgfqpoint{2.476935in}{2.786855in}}%
\pgfpathlineto{\pgfqpoint{2.490657in}{2.760378in}}%
\pgfpathlineto{\pgfqpoint{2.517244in}{2.708363in}}%
\pgfpathlineto{\pgfqpoint{2.527536in}{2.693341in}}%
\pgfpathlineto{\pgfqpoint{2.536113in}{2.684693in}}%
\pgfpathlineto{\pgfqpoint{2.542974in}{2.680717in}}%
\pgfpathlineto{\pgfqpoint{2.548978in}{2.679530in}}%
\pgfpathlineto{\pgfqpoint{2.554981in}{2.680523in}}%
\pgfpathlineto{\pgfqpoint{2.560985in}{2.683665in}}%
\pgfpathlineto{\pgfqpoint{2.567846in}{2.689760in}}%
\pgfpathlineto{\pgfqpoint{2.576423in}{2.700760in}}%
\pgfpathlineto{\pgfqpoint{2.586714in}{2.717977in}}%
\pgfpathlineto{\pgfqpoint{2.603010in}{2.750443in}}%
\pgfpathlineto{\pgfqpoint{2.622736in}{2.788910in}}%
\pgfpathlineto{\pgfqpoint{2.633028in}{2.804620in}}%
\pgfpathlineto{\pgfqpoint{2.641604in}{2.813986in}}%
\pgfpathlineto{\pgfqpoint{2.648466in}{2.818598in}}%
\pgfpathlineto{\pgfqpoint{2.654469in}{2.820365in}}%
\pgfpathlineto{\pgfqpoint{2.660473in}{2.819956in}}%
\pgfpathlineto{\pgfqpoint{2.666476in}{2.817383in}}%
\pgfpathlineto{\pgfqpoint{2.673338in}{2.811898in}}%
\pgfpathlineto{\pgfqpoint{2.681057in}{2.802755in}}%
\pgfpathlineto{\pgfqpoint{2.690491in}{2.787981in}}%
\pgfpathlineto{\pgfqpoint{2.704213in}{2.761691in}}%
\pgfpathlineto{\pgfqpoint{2.731658in}{2.708006in}}%
\pgfpathlineto{\pgfqpoint{2.741950in}{2.693078in}}%
\pgfpathlineto{\pgfqpoint{2.750527in}{2.684527in}}%
\pgfpathlineto{\pgfqpoint{2.757388in}{2.680637in}}%
\pgfpathlineto{\pgfqpoint{2.763392in}{2.679528in}}%
\pgfpathlineto{\pgfqpoint{2.769395in}{2.680598in}}%
\pgfpathlineto{\pgfqpoint{2.775399in}{2.683816in}}%
\pgfpathlineto{\pgfqpoint{2.782260in}{2.689991in}}%
\pgfpathlineto{\pgfqpoint{2.790836in}{2.701078in}}%
\pgfpathlineto{\pgfqpoint{2.801128in}{2.718372in}}%
\pgfpathlineto{\pgfqpoint{2.817424in}{2.750886in}}%
\pgfpathlineto{\pgfqpoint{2.837150in}{2.789279in}}%
\pgfpathlineto{\pgfqpoint{2.847442in}{2.804899in}}%
\pgfpathlineto{\pgfqpoint{2.856018in}{2.814171in}}%
\pgfpathlineto{\pgfqpoint{2.862880in}{2.818698in}}%
\pgfpathlineto{\pgfqpoint{2.868883in}{2.820388in}}%
\pgfpathlineto{\pgfqpoint{2.874887in}{2.819901in}}%
\pgfpathlineto{\pgfqpoint{2.880890in}{2.817252in}}%
\pgfpathlineto{\pgfqpoint{2.887752in}{2.811685in}}%
\pgfpathlineto{\pgfqpoint{2.895470in}{2.802460in}}%
\pgfpathlineto{\pgfqpoint{2.904905in}{2.787607in}}%
\pgfpathlineto{\pgfqpoint{2.918627in}{2.761254in}}%
\pgfpathlineto{\pgfqpoint{2.946072in}{2.707651in}}%
\pgfpathlineto{\pgfqpoint{2.956364in}{2.692818in}}%
\pgfpathlineto{\pgfqpoint{2.964083in}{2.685031in}}%
\pgfpathlineto{\pgfqpoint{2.970944in}{2.680885in}}%
\pgfpathlineto{\pgfqpoint{2.976948in}{2.679543in}}%
\pgfpathlineto{\pgfqpoint{2.982951in}{2.680380in}}%
\pgfpathlineto{\pgfqpoint{2.988955in}{2.683371in}}%
\pgfpathlineto{\pgfqpoint{2.995816in}{2.689304in}}%
\pgfpathlineto{\pgfqpoint{3.004393in}{2.700129in}}%
\pgfpathlineto{\pgfqpoint{3.014685in}{2.717190in}}%
\pgfpathlineto{\pgfqpoint{3.030122in}{2.747784in}}%
\pgfpathlineto{\pgfqpoint{3.051564in}{2.789646in}}%
\pgfpathlineto{\pgfqpoint{3.061856in}{2.805176in}}%
\pgfpathlineto{\pgfqpoint{3.070432in}{2.814352in}}%
\pgfpathlineto{\pgfqpoint{3.077293in}{2.818796in}}%
\pgfpathlineto{\pgfqpoint{3.083297in}{2.820409in}}%
\pgfpathlineto{\pgfqpoint{3.089301in}{2.819844in}}%
\pgfpathlineto{\pgfqpoint{3.095304in}{2.817118in}}%
\pgfpathlineto{\pgfqpoint{3.102165in}{2.811469in}}%
\pgfpathlineto{\pgfqpoint{3.109884in}{2.802163in}}%
\pgfpathlineto{\pgfqpoint{3.119319in}{2.787232in}}%
\pgfpathlineto{\pgfqpoint{3.133041in}{2.760816in}}%
\pgfpathlineto{\pgfqpoint{3.159628in}{2.708721in}}%
\pgfpathlineto{\pgfqpoint{3.169920in}{2.693605in}}%
\pgfpathlineto{\pgfqpoint{3.178497in}{2.684860in}}%
\pgfpathlineto{\pgfqpoint{3.185358in}{2.680799in}}%
\pgfpathlineto{\pgfqpoint{3.191362in}{2.679535in}}%
\pgfpathlineto{\pgfqpoint{3.197365in}{2.680450in}}%
\pgfpathlineto{\pgfqpoint{3.203369in}{2.683517in}}%
\pgfpathlineto{\pgfqpoint{3.210230in}{2.689531in}}%
\pgfpathlineto{\pgfqpoint{3.218807in}{2.700444in}}%
\pgfpathlineto{\pgfqpoint{3.229099in}{2.717582in}}%
\pgfpathlineto{\pgfqpoint{3.245394in}{2.750000in}}%
\pgfpathlineto{\pgfqpoint{3.245394in}{2.750000in}}%
\pgfusepath{stroke}%
\end{pgfscope}%
\begin{pgfscope}%
\pgfpathrectangle{\pgfqpoint{0.675000in}{2.284884in}}{\pgfqpoint{3.600000in}{0.930233in}} %
\pgfusepath{clip}%
\pgfsetrectcap%
\pgfsetroundjoin%
\pgfsetlinewidth{0.501875pt}%
\definecolor{currentstroke}{rgb}{1.000000,0.000000,1.000000}%
\pgfsetstrokecolor{currentstroke}%
\pgfsetdash{}{0pt}%
\pgfpathmoveto{\pgfqpoint{3.245394in}{2.750000in}}%
\pgfpathlineto{\pgfqpoint{3.265120in}{2.981238in}}%
\pgfpathlineto{\pgfqpoint{3.275412in}{3.076036in}}%
\pgfpathlineto{\pgfqpoint{3.283131in}{3.128155in}}%
\pgfpathlineto{\pgfqpoint{3.289134in}{3.155449in}}%
\pgfpathlineto{\pgfqpoint{3.293423in}{3.167300in}}%
\pgfpathlineto{\pgfqpoint{3.296853in}{3.172038in}}%
\pgfpathlineto{\pgfqpoint{3.299426in}{3.172791in}}%
\pgfpathlineto{\pgfqpoint{3.301999in}{3.171136in}}%
\pgfpathlineto{\pgfqpoint{3.305430in}{3.165203in}}%
\pgfpathlineto{\pgfqpoint{3.309718in}{3.151891in}}%
\pgfpathlineto{\pgfqpoint{3.314864in}{3.127559in}}%
\pgfpathlineto{\pgfqpoint{3.321725in}{3.081883in}}%
\pgfpathlineto{\pgfqpoint{3.330302in}{3.006216in}}%
\pgfpathlineto{\pgfqpoint{3.341451in}{2.884577in}}%
\pgfpathlineto{\pgfqpoint{3.379188in}{2.451718in}}%
\pgfpathlineto{\pgfqpoint{3.387765in}{2.386504in}}%
\pgfpathlineto{\pgfqpoint{3.394626in}{2.350660in}}%
\pgfpathlineto{\pgfqpoint{3.399772in}{2.334302in}}%
\pgfpathlineto{\pgfqpoint{3.403203in}{2.328634in}}%
\pgfpathlineto{\pgfqpoint{3.405776in}{2.327180in}}%
\pgfpathlineto{\pgfqpoint{3.408349in}{2.328133in}}%
\pgfpathlineto{\pgfqpoint{3.410922in}{2.331488in}}%
\pgfpathlineto{\pgfqpoint{3.414352in}{2.339663in}}%
\pgfpathlineto{\pgfqpoint{3.418640in}{2.355706in}}%
\pgfpathlineto{\pgfqpoint{3.424644in}{2.388558in}}%
\pgfpathlineto{\pgfqpoint{3.431505in}{2.439719in}}%
\pgfpathlineto{\pgfqpoint{3.440082in}{2.520993in}}%
\pgfpathlineto{\pgfqpoint{3.452089in}{2.657671in}}%
\pgfpathlineto{\pgfqpoint{3.482107in}{3.009378in}}%
\pgfpathlineto{\pgfqpoint{3.491541in}{3.090746in}}%
\pgfpathlineto{\pgfqpoint{3.498402in}{3.133916in}}%
\pgfpathlineto{\pgfqpoint{3.504406in}{3.159021in}}%
\pgfpathlineto{\pgfqpoint{3.508694in}{3.169237in}}%
\pgfpathlineto{\pgfqpoint{3.512125in}{3.172644in}}%
\pgfpathlineto{\pgfqpoint{3.514698in}{3.172393in}}%
\pgfpathlineto{\pgfqpoint{3.517271in}{3.169738in}}%
\pgfpathlineto{\pgfqpoint{3.520701in}{3.162484in}}%
\pgfpathlineto{\pgfqpoint{3.524990in}{3.147560in}}%
\pgfpathlineto{\pgfqpoint{3.530136in}{3.121386in}}%
\pgfpathlineto{\pgfqpoint{3.536997in}{3.073481in}}%
\pgfpathlineto{\pgfqpoint{3.545573in}{2.995513in}}%
\pgfpathlineto{\pgfqpoint{3.556723in}{2.871907in}}%
\pgfpathlineto{\pgfqpoint{3.591887in}{2.465197in}}%
\pgfpathlineto{\pgfqpoint{3.600463in}{2.396363in}}%
\pgfpathlineto{\pgfqpoint{3.607325in}{2.357164in}}%
\pgfpathlineto{\pgfqpoint{3.612471in}{2.338109in}}%
\pgfpathlineto{\pgfqpoint{3.616759in}{2.329373in}}%
\pgfpathlineto{\pgfqpoint{3.620190in}{2.327168in}}%
\pgfpathlineto{\pgfqpoint{3.622762in}{2.328321in}}%
\pgfpathlineto{\pgfqpoint{3.626193in}{2.333591in}}%
\pgfpathlineto{\pgfqpoint{3.630481in}{2.346093in}}%
\pgfpathlineto{\pgfqpoint{3.635627in}{2.369495in}}%
\pgfpathlineto{\pgfqpoint{3.641631in}{2.407687in}}%
\pgfpathlineto{\pgfqpoint{3.649350in}{2.472145in}}%
\pgfpathlineto{\pgfqpoint{3.659642in}{2.579477in}}%
\pgfpathlineto{\pgfqpoint{3.676795in}{2.787183in}}%
\pgfpathlineto{\pgfqpoint{3.693948in}{2.985673in}}%
\pgfpathlineto{\pgfqpoint{3.704240in}{3.079396in}}%
\pgfpathlineto{\pgfqpoint{3.711959in}{3.130505in}}%
\pgfpathlineto{\pgfqpoint{3.717962in}{3.156926in}}%
\pgfpathlineto{\pgfqpoint{3.722251in}{3.168124in}}%
\pgfpathlineto{\pgfqpoint{3.725681in}{3.172331in}}%
\pgfpathlineto{\pgfqpoint{3.728254in}{3.172682in}}%
\pgfpathlineto{\pgfqpoint{3.730827in}{3.170627in}}%
\pgfpathlineto{\pgfqpoint{3.734258in}{3.164165in}}%
\pgfpathlineto{\pgfqpoint{3.738546in}{3.150206in}}%
\pgfpathlineto{\pgfqpoint{3.743692in}{3.125134in}}%
\pgfpathlineto{\pgfqpoint{3.750553in}{3.078561in}}%
\pgfpathlineto{\pgfqpoint{3.759130in}{3.001965in}}%
\pgfpathlineto{\pgfqpoint{3.770279in}{2.879524in}}%
\pgfpathlineto{\pgfqpoint{3.807158in}{2.455511in}}%
\pgfpathlineto{\pgfqpoint{3.815735in}{2.389250in}}%
\pgfpathlineto{\pgfqpoint{3.822596in}{2.352440in}}%
\pgfpathlineto{\pgfqpoint{3.827742in}{2.335308in}}%
\pgfpathlineto{\pgfqpoint{3.832030in}{2.328225in}}%
\pgfpathlineto{\pgfqpoint{3.834603in}{2.327172in}}%
\pgfpathlineto{\pgfqpoint{3.837176in}{2.328526in}}%
\pgfpathlineto{\pgfqpoint{3.840607in}{2.334061in}}%
\pgfpathlineto{\pgfqpoint{3.844895in}{2.346887in}}%
\pgfpathlineto{\pgfqpoint{3.850041in}{2.370662in}}%
\pgfpathlineto{\pgfqpoint{3.856045in}{2.409254in}}%
\pgfpathlineto{\pgfqpoint{3.863764in}{2.474155in}}%
\pgfpathlineto{\pgfqpoint{3.874056in}{2.581914in}}%
\pgfpathlineto{\pgfqpoint{3.892066in}{2.800408in}}%
\pgfpathlineto{\pgfqpoint{3.908362in}{2.987876in}}%
\pgfpathlineto{\pgfqpoint{3.917796in}{3.074336in}}%
\pgfpathlineto{\pgfqpoint{3.925515in}{3.126958in}}%
\pgfpathlineto{\pgfqpoint{3.931519in}{3.154686in}}%
\pgfpathlineto{\pgfqpoint{3.936664in}{3.168512in}}%
\pgfpathlineto{\pgfqpoint{3.940095in}{3.172452in}}%
\pgfpathlineto{\pgfqpoint{3.942668in}{3.172602in}}%
\pgfpathlineto{\pgfqpoint{3.945241in}{3.170347in}}%
\pgfpathlineto{\pgfqpoint{3.948672in}{3.163621in}}%
\pgfpathlineto{\pgfqpoint{3.952960in}{3.149340in}}%
\pgfpathlineto{\pgfqpoint{3.958106in}{3.123900in}}%
\pgfpathlineto{\pgfqpoint{3.964967in}{3.076881in}}%
\pgfpathlineto{\pgfqpoint{3.973544in}{2.999824in}}%
\pgfpathlineto{\pgfqpoint{3.984693in}{2.876990in}}%
\pgfpathlineto{\pgfqpoint{4.020715in}{2.461288in}}%
\pgfpathlineto{\pgfqpoint{4.029291in}{2.393476in}}%
\pgfpathlineto{\pgfqpoint{4.036153in}{2.355228in}}%
\pgfpathlineto{\pgfqpoint{4.041298in}{2.336940in}}%
\pgfpathlineto{\pgfqpoint{4.045587in}{2.328864in}}%
\pgfpathlineto{\pgfqpoint{4.048160in}{2.327209in}}%
\pgfpathlineto{\pgfqpoint{4.050733in}{2.327962in}}%
\pgfpathlineto{\pgfqpoint{4.053306in}{2.331117in}}%
\pgfpathlineto{\pgfqpoint{4.056736in}{2.339029in}}%
\pgfpathlineto{\pgfqpoint{4.061025in}{2.354753in}}%
\pgfpathlineto{\pgfqpoint{4.067028in}{2.387185in}}%
\pgfpathlineto{\pgfqpoint{4.073889in}{2.437918in}}%
\pgfpathlineto{\pgfqpoint{4.082466in}{2.518762in}}%
\pgfpathlineto{\pgfqpoint{4.094473in}{2.655078in}}%
\pgfpathlineto{\pgfqpoint{4.102192in}{2.750000in}}%
\pgfpathlineto{\pgfqpoint{4.102192in}{2.750000in}}%
\pgfusepath{stroke}%
\end{pgfscope}%
\begin{pgfscope}%
\pgfsetrectcap%
\pgfsetmiterjoin%
\pgfsetlinewidth{0.501875pt}%
\definecolor{currentstroke}{rgb}{0.000000,0.000000,0.000000}%
\pgfsetstrokecolor{currentstroke}%
\pgfsetdash{}{0pt}%
\pgfpathmoveto{\pgfqpoint{0.675000in}{2.284884in}}%
\pgfpathlineto{\pgfqpoint{0.675000in}{3.215116in}}%
\pgfusepath{stroke}%
\end{pgfscope}%
\begin{pgfscope}%
\pgfsetrectcap%
\pgfsetmiterjoin%
\pgfsetlinewidth{0.501875pt}%
\definecolor{currentstroke}{rgb}{0.000000,0.000000,0.000000}%
\pgfsetstrokecolor{currentstroke}%
\pgfsetdash{}{0pt}%
\pgfpathmoveto{\pgfqpoint{0.675000in}{2.284884in}}%
\pgfpathlineto{\pgfqpoint{4.275000in}{2.284884in}}%
\pgfusepath{stroke}%
\end{pgfscope}%
\begin{pgfscope}%
\pgfsetrectcap%
\pgfsetmiterjoin%
\pgfsetlinewidth{0.501875pt}%
\definecolor{currentstroke}{rgb}{0.000000,0.000000,0.000000}%
\pgfsetstrokecolor{currentstroke}%
\pgfsetdash{}{0pt}%
\pgfpathmoveto{\pgfqpoint{0.675000in}{3.215116in}}%
\pgfpathlineto{\pgfqpoint{4.275000in}{3.215116in}}%
\pgfusepath{stroke}%
\end{pgfscope}%
\begin{pgfscope}%
\pgfsetrectcap%
\pgfsetmiterjoin%
\pgfsetlinewidth{0.501875pt}%
\definecolor{currentstroke}{rgb}{0.000000,0.000000,0.000000}%
\pgfsetstrokecolor{currentstroke}%
\pgfsetdash{}{0pt}%
\pgfpathmoveto{\pgfqpoint{4.275000in}{2.284884in}}%
\pgfpathlineto{\pgfqpoint{4.275000in}{3.215116in}}%
\pgfusepath{stroke}%
\end{pgfscope}%
\begin{pgfscope}%
\pgfsetbuttcap%
\pgfsetroundjoin%
\definecolor{currentfill}{rgb}{0.000000,0.000000,0.000000}%
\pgfsetfillcolor{currentfill}%
\pgfsetlinewidth{0.501875pt}%
\definecolor{currentstroke}{rgb}{0.000000,0.000000,0.000000}%
\pgfsetstrokecolor{currentstroke}%
\pgfsetdash{}{0pt}%
\pgfsys@defobject{currentmarker}{\pgfqpoint{0.000000in}{0.000000in}}{\pgfqpoint{0.000000in}{0.055556in}}{%
\pgfpathmoveto{\pgfqpoint{0.000000in}{0.000000in}}%
\pgfpathlineto{\pgfqpoint{0.000000in}{0.055556in}}%
\pgfusepath{stroke,fill}%
}%
\begin{pgfscope}%
\pgfsys@transformshift{0.675000in}{2.284884in}%
\pgfsys@useobject{currentmarker}{}%
\end{pgfscope}%
\end{pgfscope}%
\begin{pgfscope}%
\pgfsetbuttcap%
\pgfsetroundjoin%
\definecolor{currentfill}{rgb}{0.000000,0.000000,0.000000}%
\pgfsetfillcolor{currentfill}%
\pgfsetlinewidth{0.501875pt}%
\definecolor{currentstroke}{rgb}{0.000000,0.000000,0.000000}%
\pgfsetstrokecolor{currentstroke}%
\pgfsetdash{}{0pt}%
\pgfsys@defobject{currentmarker}{\pgfqpoint{0.000000in}{-0.055556in}}{\pgfqpoint{0.000000in}{0.000000in}}{%
\pgfpathmoveto{\pgfqpoint{0.000000in}{0.000000in}}%
\pgfpathlineto{\pgfqpoint{0.000000in}{-0.055556in}}%
\pgfusepath{stroke,fill}%
}%
\begin{pgfscope}%
\pgfsys@transformshift{0.675000in}{3.215116in}%
\pgfsys@useobject{currentmarker}{}%
\end{pgfscope}%
\end{pgfscope}%
\begin{pgfscope}%
\pgftext[x=0.675000in,y=2.229328in,,top]{\rmfamily\fontsize{9.000000}{10.800000}\selectfont \(\displaystyle 0.0000\)}%
\end{pgfscope}%
\begin{pgfscope}%
\pgfsetbuttcap%
\pgfsetroundjoin%
\definecolor{currentfill}{rgb}{0.000000,0.000000,0.000000}%
\pgfsetfillcolor{currentfill}%
\pgfsetlinewidth{0.501875pt}%
\definecolor{currentstroke}{rgb}{0.000000,0.000000,0.000000}%
\pgfsetstrokecolor{currentstroke}%
\pgfsetdash{}{0pt}%
\pgfsys@defobject{currentmarker}{\pgfqpoint{0.000000in}{0.000000in}}{\pgfqpoint{0.000000in}{0.055556in}}{%
\pgfpathmoveto{\pgfqpoint{0.000000in}{0.000000in}}%
\pgfpathlineto{\pgfqpoint{0.000000in}{0.055556in}}%
\pgfusepath{stroke,fill}%
}%
\begin{pgfscope}%
\pgfsys@transformshift{1.125000in}{2.284884in}%
\pgfsys@useobject{currentmarker}{}%
\end{pgfscope}%
\end{pgfscope}%
\begin{pgfscope}%
\pgfsetbuttcap%
\pgfsetroundjoin%
\definecolor{currentfill}{rgb}{0.000000,0.000000,0.000000}%
\pgfsetfillcolor{currentfill}%
\pgfsetlinewidth{0.501875pt}%
\definecolor{currentstroke}{rgb}{0.000000,0.000000,0.000000}%
\pgfsetstrokecolor{currentstroke}%
\pgfsetdash{}{0pt}%
\pgfsys@defobject{currentmarker}{\pgfqpoint{0.000000in}{-0.055556in}}{\pgfqpoint{0.000000in}{0.000000in}}{%
\pgfpathmoveto{\pgfqpoint{0.000000in}{0.000000in}}%
\pgfpathlineto{\pgfqpoint{0.000000in}{-0.055556in}}%
\pgfusepath{stroke,fill}%
}%
\begin{pgfscope}%
\pgfsys@transformshift{1.125000in}{3.215116in}%
\pgfsys@useobject{currentmarker}{}%
\end{pgfscope}%
\end{pgfscope}%
\begin{pgfscope}%
\pgftext[x=1.125000in,y=2.229328in,,top]{\rmfamily\fontsize{9.000000}{10.800000}\selectfont \(\displaystyle 0.0002\)}%
\end{pgfscope}%
\begin{pgfscope}%
\pgfsetbuttcap%
\pgfsetroundjoin%
\definecolor{currentfill}{rgb}{0.000000,0.000000,0.000000}%
\pgfsetfillcolor{currentfill}%
\pgfsetlinewidth{0.501875pt}%
\definecolor{currentstroke}{rgb}{0.000000,0.000000,0.000000}%
\pgfsetstrokecolor{currentstroke}%
\pgfsetdash{}{0pt}%
\pgfsys@defobject{currentmarker}{\pgfqpoint{0.000000in}{0.000000in}}{\pgfqpoint{0.000000in}{0.055556in}}{%
\pgfpathmoveto{\pgfqpoint{0.000000in}{0.000000in}}%
\pgfpathlineto{\pgfqpoint{0.000000in}{0.055556in}}%
\pgfusepath{stroke,fill}%
}%
\begin{pgfscope}%
\pgfsys@transformshift{1.575000in}{2.284884in}%
\pgfsys@useobject{currentmarker}{}%
\end{pgfscope}%
\end{pgfscope}%
\begin{pgfscope}%
\pgfsetbuttcap%
\pgfsetroundjoin%
\definecolor{currentfill}{rgb}{0.000000,0.000000,0.000000}%
\pgfsetfillcolor{currentfill}%
\pgfsetlinewidth{0.501875pt}%
\definecolor{currentstroke}{rgb}{0.000000,0.000000,0.000000}%
\pgfsetstrokecolor{currentstroke}%
\pgfsetdash{}{0pt}%
\pgfsys@defobject{currentmarker}{\pgfqpoint{0.000000in}{-0.055556in}}{\pgfqpoint{0.000000in}{0.000000in}}{%
\pgfpathmoveto{\pgfqpoint{0.000000in}{0.000000in}}%
\pgfpathlineto{\pgfqpoint{0.000000in}{-0.055556in}}%
\pgfusepath{stroke,fill}%
}%
\begin{pgfscope}%
\pgfsys@transformshift{1.575000in}{3.215116in}%
\pgfsys@useobject{currentmarker}{}%
\end{pgfscope}%
\end{pgfscope}%
\begin{pgfscope}%
\pgftext[x=1.575000in,y=2.229328in,,top]{\rmfamily\fontsize{9.000000}{10.800000}\selectfont \(\displaystyle 0.0004\)}%
\end{pgfscope}%
\begin{pgfscope}%
\pgfsetbuttcap%
\pgfsetroundjoin%
\definecolor{currentfill}{rgb}{0.000000,0.000000,0.000000}%
\pgfsetfillcolor{currentfill}%
\pgfsetlinewidth{0.501875pt}%
\definecolor{currentstroke}{rgb}{0.000000,0.000000,0.000000}%
\pgfsetstrokecolor{currentstroke}%
\pgfsetdash{}{0pt}%
\pgfsys@defobject{currentmarker}{\pgfqpoint{0.000000in}{0.000000in}}{\pgfqpoint{0.000000in}{0.055556in}}{%
\pgfpathmoveto{\pgfqpoint{0.000000in}{0.000000in}}%
\pgfpathlineto{\pgfqpoint{0.000000in}{0.055556in}}%
\pgfusepath{stroke,fill}%
}%
\begin{pgfscope}%
\pgfsys@transformshift{2.025000in}{2.284884in}%
\pgfsys@useobject{currentmarker}{}%
\end{pgfscope}%
\end{pgfscope}%
\begin{pgfscope}%
\pgfsetbuttcap%
\pgfsetroundjoin%
\definecolor{currentfill}{rgb}{0.000000,0.000000,0.000000}%
\pgfsetfillcolor{currentfill}%
\pgfsetlinewidth{0.501875pt}%
\definecolor{currentstroke}{rgb}{0.000000,0.000000,0.000000}%
\pgfsetstrokecolor{currentstroke}%
\pgfsetdash{}{0pt}%
\pgfsys@defobject{currentmarker}{\pgfqpoint{0.000000in}{-0.055556in}}{\pgfqpoint{0.000000in}{0.000000in}}{%
\pgfpathmoveto{\pgfqpoint{0.000000in}{0.000000in}}%
\pgfpathlineto{\pgfqpoint{0.000000in}{-0.055556in}}%
\pgfusepath{stroke,fill}%
}%
\begin{pgfscope}%
\pgfsys@transformshift{2.025000in}{3.215116in}%
\pgfsys@useobject{currentmarker}{}%
\end{pgfscope}%
\end{pgfscope}%
\begin{pgfscope}%
\pgftext[x=2.025000in,y=2.229328in,,top]{\rmfamily\fontsize{9.000000}{10.800000}\selectfont \(\displaystyle 0.0006\)}%
\end{pgfscope}%
\begin{pgfscope}%
\pgfsetbuttcap%
\pgfsetroundjoin%
\definecolor{currentfill}{rgb}{0.000000,0.000000,0.000000}%
\pgfsetfillcolor{currentfill}%
\pgfsetlinewidth{0.501875pt}%
\definecolor{currentstroke}{rgb}{0.000000,0.000000,0.000000}%
\pgfsetstrokecolor{currentstroke}%
\pgfsetdash{}{0pt}%
\pgfsys@defobject{currentmarker}{\pgfqpoint{0.000000in}{0.000000in}}{\pgfqpoint{0.000000in}{0.055556in}}{%
\pgfpathmoveto{\pgfqpoint{0.000000in}{0.000000in}}%
\pgfpathlineto{\pgfqpoint{0.000000in}{0.055556in}}%
\pgfusepath{stroke,fill}%
}%
\begin{pgfscope}%
\pgfsys@transformshift{2.475000in}{2.284884in}%
\pgfsys@useobject{currentmarker}{}%
\end{pgfscope}%
\end{pgfscope}%
\begin{pgfscope}%
\pgfsetbuttcap%
\pgfsetroundjoin%
\definecolor{currentfill}{rgb}{0.000000,0.000000,0.000000}%
\pgfsetfillcolor{currentfill}%
\pgfsetlinewidth{0.501875pt}%
\definecolor{currentstroke}{rgb}{0.000000,0.000000,0.000000}%
\pgfsetstrokecolor{currentstroke}%
\pgfsetdash{}{0pt}%
\pgfsys@defobject{currentmarker}{\pgfqpoint{0.000000in}{-0.055556in}}{\pgfqpoint{0.000000in}{0.000000in}}{%
\pgfpathmoveto{\pgfqpoint{0.000000in}{0.000000in}}%
\pgfpathlineto{\pgfqpoint{0.000000in}{-0.055556in}}%
\pgfusepath{stroke,fill}%
}%
\begin{pgfscope}%
\pgfsys@transformshift{2.475000in}{3.215116in}%
\pgfsys@useobject{currentmarker}{}%
\end{pgfscope}%
\end{pgfscope}%
\begin{pgfscope}%
\pgftext[x=2.475000in,y=2.229328in,,top]{\rmfamily\fontsize{9.000000}{10.800000}\selectfont \(\displaystyle 0.0008\)}%
\end{pgfscope}%
\begin{pgfscope}%
\pgfsetbuttcap%
\pgfsetroundjoin%
\definecolor{currentfill}{rgb}{0.000000,0.000000,0.000000}%
\pgfsetfillcolor{currentfill}%
\pgfsetlinewidth{0.501875pt}%
\definecolor{currentstroke}{rgb}{0.000000,0.000000,0.000000}%
\pgfsetstrokecolor{currentstroke}%
\pgfsetdash{}{0pt}%
\pgfsys@defobject{currentmarker}{\pgfqpoint{0.000000in}{0.000000in}}{\pgfqpoint{0.000000in}{0.055556in}}{%
\pgfpathmoveto{\pgfqpoint{0.000000in}{0.000000in}}%
\pgfpathlineto{\pgfqpoint{0.000000in}{0.055556in}}%
\pgfusepath{stroke,fill}%
}%
\begin{pgfscope}%
\pgfsys@transformshift{2.925000in}{2.284884in}%
\pgfsys@useobject{currentmarker}{}%
\end{pgfscope}%
\end{pgfscope}%
\begin{pgfscope}%
\pgfsetbuttcap%
\pgfsetroundjoin%
\definecolor{currentfill}{rgb}{0.000000,0.000000,0.000000}%
\pgfsetfillcolor{currentfill}%
\pgfsetlinewidth{0.501875pt}%
\definecolor{currentstroke}{rgb}{0.000000,0.000000,0.000000}%
\pgfsetstrokecolor{currentstroke}%
\pgfsetdash{}{0pt}%
\pgfsys@defobject{currentmarker}{\pgfqpoint{0.000000in}{-0.055556in}}{\pgfqpoint{0.000000in}{0.000000in}}{%
\pgfpathmoveto{\pgfqpoint{0.000000in}{0.000000in}}%
\pgfpathlineto{\pgfqpoint{0.000000in}{-0.055556in}}%
\pgfusepath{stroke,fill}%
}%
\begin{pgfscope}%
\pgfsys@transformshift{2.925000in}{3.215116in}%
\pgfsys@useobject{currentmarker}{}%
\end{pgfscope}%
\end{pgfscope}%
\begin{pgfscope}%
\pgftext[x=2.925000in,y=2.229328in,,top]{\rmfamily\fontsize{9.000000}{10.800000}\selectfont \(\displaystyle 0.0010\)}%
\end{pgfscope}%
\begin{pgfscope}%
\pgfsetbuttcap%
\pgfsetroundjoin%
\definecolor{currentfill}{rgb}{0.000000,0.000000,0.000000}%
\pgfsetfillcolor{currentfill}%
\pgfsetlinewidth{0.501875pt}%
\definecolor{currentstroke}{rgb}{0.000000,0.000000,0.000000}%
\pgfsetstrokecolor{currentstroke}%
\pgfsetdash{}{0pt}%
\pgfsys@defobject{currentmarker}{\pgfqpoint{0.000000in}{0.000000in}}{\pgfqpoint{0.000000in}{0.055556in}}{%
\pgfpathmoveto{\pgfqpoint{0.000000in}{0.000000in}}%
\pgfpathlineto{\pgfqpoint{0.000000in}{0.055556in}}%
\pgfusepath{stroke,fill}%
}%
\begin{pgfscope}%
\pgfsys@transformshift{3.375000in}{2.284884in}%
\pgfsys@useobject{currentmarker}{}%
\end{pgfscope}%
\end{pgfscope}%
\begin{pgfscope}%
\pgfsetbuttcap%
\pgfsetroundjoin%
\definecolor{currentfill}{rgb}{0.000000,0.000000,0.000000}%
\pgfsetfillcolor{currentfill}%
\pgfsetlinewidth{0.501875pt}%
\definecolor{currentstroke}{rgb}{0.000000,0.000000,0.000000}%
\pgfsetstrokecolor{currentstroke}%
\pgfsetdash{}{0pt}%
\pgfsys@defobject{currentmarker}{\pgfqpoint{0.000000in}{-0.055556in}}{\pgfqpoint{0.000000in}{0.000000in}}{%
\pgfpathmoveto{\pgfqpoint{0.000000in}{0.000000in}}%
\pgfpathlineto{\pgfqpoint{0.000000in}{-0.055556in}}%
\pgfusepath{stroke,fill}%
}%
\begin{pgfscope}%
\pgfsys@transformshift{3.375000in}{3.215116in}%
\pgfsys@useobject{currentmarker}{}%
\end{pgfscope}%
\end{pgfscope}%
\begin{pgfscope}%
\pgftext[x=3.375000in,y=2.229328in,,top]{\rmfamily\fontsize{9.000000}{10.800000}\selectfont \(\displaystyle 0.0012\)}%
\end{pgfscope}%
\begin{pgfscope}%
\pgfsetbuttcap%
\pgfsetroundjoin%
\definecolor{currentfill}{rgb}{0.000000,0.000000,0.000000}%
\pgfsetfillcolor{currentfill}%
\pgfsetlinewidth{0.501875pt}%
\definecolor{currentstroke}{rgb}{0.000000,0.000000,0.000000}%
\pgfsetstrokecolor{currentstroke}%
\pgfsetdash{}{0pt}%
\pgfsys@defobject{currentmarker}{\pgfqpoint{0.000000in}{0.000000in}}{\pgfqpoint{0.000000in}{0.055556in}}{%
\pgfpathmoveto{\pgfqpoint{0.000000in}{0.000000in}}%
\pgfpathlineto{\pgfqpoint{0.000000in}{0.055556in}}%
\pgfusepath{stroke,fill}%
}%
\begin{pgfscope}%
\pgfsys@transformshift{3.825000in}{2.284884in}%
\pgfsys@useobject{currentmarker}{}%
\end{pgfscope}%
\end{pgfscope}%
\begin{pgfscope}%
\pgfsetbuttcap%
\pgfsetroundjoin%
\definecolor{currentfill}{rgb}{0.000000,0.000000,0.000000}%
\pgfsetfillcolor{currentfill}%
\pgfsetlinewidth{0.501875pt}%
\definecolor{currentstroke}{rgb}{0.000000,0.000000,0.000000}%
\pgfsetstrokecolor{currentstroke}%
\pgfsetdash{}{0pt}%
\pgfsys@defobject{currentmarker}{\pgfqpoint{0.000000in}{-0.055556in}}{\pgfqpoint{0.000000in}{0.000000in}}{%
\pgfpathmoveto{\pgfqpoint{0.000000in}{0.000000in}}%
\pgfpathlineto{\pgfqpoint{0.000000in}{-0.055556in}}%
\pgfusepath{stroke,fill}%
}%
\begin{pgfscope}%
\pgfsys@transformshift{3.825000in}{3.215116in}%
\pgfsys@useobject{currentmarker}{}%
\end{pgfscope}%
\end{pgfscope}%
\begin{pgfscope}%
\pgftext[x=3.825000in,y=2.229328in,,top]{\rmfamily\fontsize{9.000000}{10.800000}\selectfont \(\displaystyle 0.0014\)}%
\end{pgfscope}%
\begin{pgfscope}%
\pgfsetbuttcap%
\pgfsetroundjoin%
\definecolor{currentfill}{rgb}{0.000000,0.000000,0.000000}%
\pgfsetfillcolor{currentfill}%
\pgfsetlinewidth{0.501875pt}%
\definecolor{currentstroke}{rgb}{0.000000,0.000000,0.000000}%
\pgfsetstrokecolor{currentstroke}%
\pgfsetdash{}{0pt}%
\pgfsys@defobject{currentmarker}{\pgfqpoint{0.000000in}{0.000000in}}{\pgfqpoint{0.000000in}{0.055556in}}{%
\pgfpathmoveto{\pgfqpoint{0.000000in}{0.000000in}}%
\pgfpathlineto{\pgfqpoint{0.000000in}{0.055556in}}%
\pgfusepath{stroke,fill}%
}%
\begin{pgfscope}%
\pgfsys@transformshift{4.275000in}{2.284884in}%
\pgfsys@useobject{currentmarker}{}%
\end{pgfscope}%
\end{pgfscope}%
\begin{pgfscope}%
\pgfsetbuttcap%
\pgfsetroundjoin%
\definecolor{currentfill}{rgb}{0.000000,0.000000,0.000000}%
\pgfsetfillcolor{currentfill}%
\pgfsetlinewidth{0.501875pt}%
\definecolor{currentstroke}{rgb}{0.000000,0.000000,0.000000}%
\pgfsetstrokecolor{currentstroke}%
\pgfsetdash{}{0pt}%
\pgfsys@defobject{currentmarker}{\pgfqpoint{0.000000in}{-0.055556in}}{\pgfqpoint{0.000000in}{0.000000in}}{%
\pgfpathmoveto{\pgfqpoint{0.000000in}{0.000000in}}%
\pgfpathlineto{\pgfqpoint{0.000000in}{-0.055556in}}%
\pgfusepath{stroke,fill}%
}%
\begin{pgfscope}%
\pgfsys@transformshift{4.275000in}{3.215116in}%
\pgfsys@useobject{currentmarker}{}%
\end{pgfscope}%
\end{pgfscope}%
\begin{pgfscope}%
\pgftext[x=4.275000in,y=2.229328in,,top]{\rmfamily\fontsize{9.000000}{10.800000}\selectfont \(\displaystyle 0.0016\)}%
\end{pgfscope}%
\begin{pgfscope}%
\pgftext[x=2.475000in,y=2.038912in,,top]{\rmfamily\fontsize{9.000000}{10.800000}\selectfont Zeit (s)}%
\end{pgfscope}%
\begin{pgfscope}%
\pgfsetbuttcap%
\pgfsetroundjoin%
\definecolor{currentfill}{rgb}{0.000000,0.000000,0.000000}%
\pgfsetfillcolor{currentfill}%
\pgfsetlinewidth{0.501875pt}%
\definecolor{currentstroke}{rgb}{0.000000,0.000000,0.000000}%
\pgfsetstrokecolor{currentstroke}%
\pgfsetdash{}{0pt}%
\pgfsys@defobject{currentmarker}{\pgfqpoint{0.000000in}{0.000000in}}{\pgfqpoint{0.055556in}{0.000000in}}{%
\pgfpathmoveto{\pgfqpoint{0.000000in}{0.000000in}}%
\pgfpathlineto{\pgfqpoint{0.055556in}{0.000000in}}%
\pgfusepath{stroke,fill}%
}%
\begin{pgfscope}%
\pgfsys@transformshift{0.675000in}{2.327167in}%
\pgfsys@useobject{currentmarker}{}%
\end{pgfscope}%
\end{pgfscope}%
\begin{pgfscope}%
\pgfsetbuttcap%
\pgfsetroundjoin%
\definecolor{currentfill}{rgb}{0.000000,0.000000,0.000000}%
\pgfsetfillcolor{currentfill}%
\pgfsetlinewidth{0.501875pt}%
\definecolor{currentstroke}{rgb}{0.000000,0.000000,0.000000}%
\pgfsetstrokecolor{currentstroke}%
\pgfsetdash{}{0pt}%
\pgfsys@defobject{currentmarker}{\pgfqpoint{-0.055556in}{0.000000in}}{\pgfqpoint{0.000000in}{0.000000in}}{%
\pgfpathmoveto{\pgfqpoint{0.000000in}{0.000000in}}%
\pgfpathlineto{\pgfqpoint{-0.055556in}{0.000000in}}%
\pgfusepath{stroke,fill}%
}%
\begin{pgfscope}%
\pgfsys@transformshift{4.275000in}{2.327167in}%
\pgfsys@useobject{currentmarker}{}%
\end{pgfscope}%
\end{pgfscope}%
\begin{pgfscope}%
\pgftext[x=0.619444in,y=2.327167in,right,]{\rmfamily\fontsize{9.000000}{10.800000}\selectfont \(\displaystyle -30\)}%
\end{pgfscope}%
\begin{pgfscope}%
\pgfsetbuttcap%
\pgfsetroundjoin%
\definecolor{currentfill}{rgb}{0.000000,0.000000,0.000000}%
\pgfsetfillcolor{currentfill}%
\pgfsetlinewidth{0.501875pt}%
\definecolor{currentstroke}{rgb}{0.000000,0.000000,0.000000}%
\pgfsetstrokecolor{currentstroke}%
\pgfsetdash{}{0pt}%
\pgfsys@defobject{currentmarker}{\pgfqpoint{0.000000in}{0.000000in}}{\pgfqpoint{0.055556in}{0.000000in}}{%
\pgfpathmoveto{\pgfqpoint{0.000000in}{0.000000in}}%
\pgfpathlineto{\pgfqpoint{0.055556in}{0.000000in}}%
\pgfusepath{stroke,fill}%
}%
\begin{pgfscope}%
\pgfsys@transformshift{0.675000in}{2.468111in}%
\pgfsys@useobject{currentmarker}{}%
\end{pgfscope}%
\end{pgfscope}%
\begin{pgfscope}%
\pgfsetbuttcap%
\pgfsetroundjoin%
\definecolor{currentfill}{rgb}{0.000000,0.000000,0.000000}%
\pgfsetfillcolor{currentfill}%
\pgfsetlinewidth{0.501875pt}%
\definecolor{currentstroke}{rgb}{0.000000,0.000000,0.000000}%
\pgfsetstrokecolor{currentstroke}%
\pgfsetdash{}{0pt}%
\pgfsys@defobject{currentmarker}{\pgfqpoint{-0.055556in}{0.000000in}}{\pgfqpoint{0.000000in}{0.000000in}}{%
\pgfpathmoveto{\pgfqpoint{0.000000in}{0.000000in}}%
\pgfpathlineto{\pgfqpoint{-0.055556in}{0.000000in}}%
\pgfusepath{stroke,fill}%
}%
\begin{pgfscope}%
\pgfsys@transformshift{4.275000in}{2.468111in}%
\pgfsys@useobject{currentmarker}{}%
\end{pgfscope}%
\end{pgfscope}%
\begin{pgfscope}%
\pgftext[x=0.619444in,y=2.468111in,right,]{\rmfamily\fontsize{9.000000}{10.800000}\selectfont \(\displaystyle -20\)}%
\end{pgfscope}%
\begin{pgfscope}%
\pgfsetbuttcap%
\pgfsetroundjoin%
\definecolor{currentfill}{rgb}{0.000000,0.000000,0.000000}%
\pgfsetfillcolor{currentfill}%
\pgfsetlinewidth{0.501875pt}%
\definecolor{currentstroke}{rgb}{0.000000,0.000000,0.000000}%
\pgfsetstrokecolor{currentstroke}%
\pgfsetdash{}{0pt}%
\pgfsys@defobject{currentmarker}{\pgfqpoint{0.000000in}{0.000000in}}{\pgfqpoint{0.055556in}{0.000000in}}{%
\pgfpathmoveto{\pgfqpoint{0.000000in}{0.000000in}}%
\pgfpathlineto{\pgfqpoint{0.055556in}{0.000000in}}%
\pgfusepath{stroke,fill}%
}%
\begin{pgfscope}%
\pgfsys@transformshift{0.675000in}{2.609056in}%
\pgfsys@useobject{currentmarker}{}%
\end{pgfscope}%
\end{pgfscope}%
\begin{pgfscope}%
\pgfsetbuttcap%
\pgfsetroundjoin%
\definecolor{currentfill}{rgb}{0.000000,0.000000,0.000000}%
\pgfsetfillcolor{currentfill}%
\pgfsetlinewidth{0.501875pt}%
\definecolor{currentstroke}{rgb}{0.000000,0.000000,0.000000}%
\pgfsetstrokecolor{currentstroke}%
\pgfsetdash{}{0pt}%
\pgfsys@defobject{currentmarker}{\pgfqpoint{-0.055556in}{0.000000in}}{\pgfqpoint{0.000000in}{0.000000in}}{%
\pgfpathmoveto{\pgfqpoint{0.000000in}{0.000000in}}%
\pgfpathlineto{\pgfqpoint{-0.055556in}{0.000000in}}%
\pgfusepath{stroke,fill}%
}%
\begin{pgfscope}%
\pgfsys@transformshift{4.275000in}{2.609056in}%
\pgfsys@useobject{currentmarker}{}%
\end{pgfscope}%
\end{pgfscope}%
\begin{pgfscope}%
\pgftext[x=0.619444in,y=2.609056in,right,]{\rmfamily\fontsize{9.000000}{10.800000}\selectfont \(\displaystyle -10\)}%
\end{pgfscope}%
\begin{pgfscope}%
\pgfsetbuttcap%
\pgfsetroundjoin%
\definecolor{currentfill}{rgb}{0.000000,0.000000,0.000000}%
\pgfsetfillcolor{currentfill}%
\pgfsetlinewidth{0.501875pt}%
\definecolor{currentstroke}{rgb}{0.000000,0.000000,0.000000}%
\pgfsetstrokecolor{currentstroke}%
\pgfsetdash{}{0pt}%
\pgfsys@defobject{currentmarker}{\pgfqpoint{0.000000in}{0.000000in}}{\pgfqpoint{0.055556in}{0.000000in}}{%
\pgfpathmoveto{\pgfqpoint{0.000000in}{0.000000in}}%
\pgfpathlineto{\pgfqpoint{0.055556in}{0.000000in}}%
\pgfusepath{stroke,fill}%
}%
\begin{pgfscope}%
\pgfsys@transformshift{0.675000in}{2.750000in}%
\pgfsys@useobject{currentmarker}{}%
\end{pgfscope}%
\end{pgfscope}%
\begin{pgfscope}%
\pgfsetbuttcap%
\pgfsetroundjoin%
\definecolor{currentfill}{rgb}{0.000000,0.000000,0.000000}%
\pgfsetfillcolor{currentfill}%
\pgfsetlinewidth{0.501875pt}%
\definecolor{currentstroke}{rgb}{0.000000,0.000000,0.000000}%
\pgfsetstrokecolor{currentstroke}%
\pgfsetdash{}{0pt}%
\pgfsys@defobject{currentmarker}{\pgfqpoint{-0.055556in}{0.000000in}}{\pgfqpoint{0.000000in}{0.000000in}}{%
\pgfpathmoveto{\pgfqpoint{0.000000in}{0.000000in}}%
\pgfpathlineto{\pgfqpoint{-0.055556in}{0.000000in}}%
\pgfusepath{stroke,fill}%
}%
\begin{pgfscope}%
\pgfsys@transformshift{4.275000in}{2.750000in}%
\pgfsys@useobject{currentmarker}{}%
\end{pgfscope}%
\end{pgfscope}%
\begin{pgfscope}%
\pgftext[x=0.619444in,y=2.750000in,right,]{\rmfamily\fontsize{9.000000}{10.800000}\selectfont \(\displaystyle 0\)}%
\end{pgfscope}%
\begin{pgfscope}%
\pgfsetbuttcap%
\pgfsetroundjoin%
\definecolor{currentfill}{rgb}{0.000000,0.000000,0.000000}%
\pgfsetfillcolor{currentfill}%
\pgfsetlinewidth{0.501875pt}%
\definecolor{currentstroke}{rgb}{0.000000,0.000000,0.000000}%
\pgfsetstrokecolor{currentstroke}%
\pgfsetdash{}{0pt}%
\pgfsys@defobject{currentmarker}{\pgfqpoint{0.000000in}{0.000000in}}{\pgfqpoint{0.055556in}{0.000000in}}{%
\pgfpathmoveto{\pgfqpoint{0.000000in}{0.000000in}}%
\pgfpathlineto{\pgfqpoint{0.055556in}{0.000000in}}%
\pgfusepath{stroke,fill}%
}%
\begin{pgfscope}%
\pgfsys@transformshift{0.675000in}{2.890944in}%
\pgfsys@useobject{currentmarker}{}%
\end{pgfscope}%
\end{pgfscope}%
\begin{pgfscope}%
\pgfsetbuttcap%
\pgfsetroundjoin%
\definecolor{currentfill}{rgb}{0.000000,0.000000,0.000000}%
\pgfsetfillcolor{currentfill}%
\pgfsetlinewidth{0.501875pt}%
\definecolor{currentstroke}{rgb}{0.000000,0.000000,0.000000}%
\pgfsetstrokecolor{currentstroke}%
\pgfsetdash{}{0pt}%
\pgfsys@defobject{currentmarker}{\pgfqpoint{-0.055556in}{0.000000in}}{\pgfqpoint{0.000000in}{0.000000in}}{%
\pgfpathmoveto{\pgfqpoint{0.000000in}{0.000000in}}%
\pgfpathlineto{\pgfqpoint{-0.055556in}{0.000000in}}%
\pgfusepath{stroke,fill}%
}%
\begin{pgfscope}%
\pgfsys@transformshift{4.275000in}{2.890944in}%
\pgfsys@useobject{currentmarker}{}%
\end{pgfscope}%
\end{pgfscope}%
\begin{pgfscope}%
\pgftext[x=0.619444in,y=2.890944in,right,]{\rmfamily\fontsize{9.000000}{10.800000}\selectfont \(\displaystyle 10\)}%
\end{pgfscope}%
\begin{pgfscope}%
\pgfsetbuttcap%
\pgfsetroundjoin%
\definecolor{currentfill}{rgb}{0.000000,0.000000,0.000000}%
\pgfsetfillcolor{currentfill}%
\pgfsetlinewidth{0.501875pt}%
\definecolor{currentstroke}{rgb}{0.000000,0.000000,0.000000}%
\pgfsetstrokecolor{currentstroke}%
\pgfsetdash{}{0pt}%
\pgfsys@defobject{currentmarker}{\pgfqpoint{0.000000in}{0.000000in}}{\pgfqpoint{0.055556in}{0.000000in}}{%
\pgfpathmoveto{\pgfqpoint{0.000000in}{0.000000in}}%
\pgfpathlineto{\pgfqpoint{0.055556in}{0.000000in}}%
\pgfusepath{stroke,fill}%
}%
\begin{pgfscope}%
\pgfsys@transformshift{0.675000in}{3.031889in}%
\pgfsys@useobject{currentmarker}{}%
\end{pgfscope}%
\end{pgfscope}%
\begin{pgfscope}%
\pgfsetbuttcap%
\pgfsetroundjoin%
\definecolor{currentfill}{rgb}{0.000000,0.000000,0.000000}%
\pgfsetfillcolor{currentfill}%
\pgfsetlinewidth{0.501875pt}%
\definecolor{currentstroke}{rgb}{0.000000,0.000000,0.000000}%
\pgfsetstrokecolor{currentstroke}%
\pgfsetdash{}{0pt}%
\pgfsys@defobject{currentmarker}{\pgfqpoint{-0.055556in}{0.000000in}}{\pgfqpoint{0.000000in}{0.000000in}}{%
\pgfpathmoveto{\pgfqpoint{0.000000in}{0.000000in}}%
\pgfpathlineto{\pgfqpoint{-0.055556in}{0.000000in}}%
\pgfusepath{stroke,fill}%
}%
\begin{pgfscope}%
\pgfsys@transformshift{4.275000in}{3.031889in}%
\pgfsys@useobject{currentmarker}{}%
\end{pgfscope}%
\end{pgfscope}%
\begin{pgfscope}%
\pgftext[x=0.619444in,y=3.031889in,right,]{\rmfamily\fontsize{9.000000}{10.800000}\selectfont \(\displaystyle 20\)}%
\end{pgfscope}%
\begin{pgfscope}%
\pgfsetbuttcap%
\pgfsetroundjoin%
\definecolor{currentfill}{rgb}{0.000000,0.000000,0.000000}%
\pgfsetfillcolor{currentfill}%
\pgfsetlinewidth{0.501875pt}%
\definecolor{currentstroke}{rgb}{0.000000,0.000000,0.000000}%
\pgfsetstrokecolor{currentstroke}%
\pgfsetdash{}{0pt}%
\pgfsys@defobject{currentmarker}{\pgfqpoint{0.000000in}{0.000000in}}{\pgfqpoint{0.055556in}{0.000000in}}{%
\pgfpathmoveto{\pgfqpoint{0.000000in}{0.000000in}}%
\pgfpathlineto{\pgfqpoint{0.055556in}{0.000000in}}%
\pgfusepath{stroke,fill}%
}%
\begin{pgfscope}%
\pgfsys@transformshift{0.675000in}{3.172833in}%
\pgfsys@useobject{currentmarker}{}%
\end{pgfscope}%
\end{pgfscope}%
\begin{pgfscope}%
\pgfsetbuttcap%
\pgfsetroundjoin%
\definecolor{currentfill}{rgb}{0.000000,0.000000,0.000000}%
\pgfsetfillcolor{currentfill}%
\pgfsetlinewidth{0.501875pt}%
\definecolor{currentstroke}{rgb}{0.000000,0.000000,0.000000}%
\pgfsetstrokecolor{currentstroke}%
\pgfsetdash{}{0pt}%
\pgfsys@defobject{currentmarker}{\pgfqpoint{-0.055556in}{0.000000in}}{\pgfqpoint{0.000000in}{0.000000in}}{%
\pgfpathmoveto{\pgfqpoint{0.000000in}{0.000000in}}%
\pgfpathlineto{\pgfqpoint{-0.055556in}{0.000000in}}%
\pgfusepath{stroke,fill}%
}%
\begin{pgfscope}%
\pgfsys@transformshift{4.275000in}{3.172833in}%
\pgfsys@useobject{currentmarker}{}%
\end{pgfscope}%
\end{pgfscope}%
\begin{pgfscope}%
\pgftext[x=0.619444in,y=3.172833in,right,]{\rmfamily\fontsize{9.000000}{10.800000}\selectfont \(\displaystyle 30\)}%
\end{pgfscope}%
\begin{pgfscope}%
\pgftext[x=0.321606in,y=2.750000in,,bottom,rotate=90.000000]{\rmfamily\fontsize{9.000000}{10.800000}\selectfont Spannung (V)}%
\end{pgfscope}%
\begin{pgfscope}%
\pgftext[x=2.475000in,y=3.284561in,,base]{\rmfamily\fontsize{11.000000}{13.200000}\selectfont Moduliertes Signal}%
\end{pgfscope}%
\begin{pgfscope}%
\pgfsetbuttcap%
\pgfsetmiterjoin%
\pgfsetlinewidth{0.000000pt}%
\definecolor{currentstroke}{rgb}{0.000000,0.000000,0.000000}%
\pgfsetstrokecolor{currentstroke}%
\pgfsetstrokeopacity{0.000000}%
\pgfsetdash{}{0pt}%
\pgfpathmoveto{\pgfqpoint{0.675000in}{0.750000in}}%
\pgfpathlineto{\pgfqpoint{4.275000in}{0.750000in}}%
\pgfpathlineto{\pgfqpoint{4.275000in}{1.680233in}}%
\pgfpathlineto{\pgfqpoint{0.675000in}{1.680233in}}%
\pgfpathclose%
\pgfusepath{}%
\end{pgfscope}%
\begin{pgfscope}%
\pgfpathrectangle{\pgfqpoint{0.675000in}{0.750000in}}{\pgfqpoint{3.600000in}{0.930233in}} %
\pgfusepath{clip}%
\pgfsetrectcap%
\pgfsetroundjoin%
\pgfsetlinewidth{0.501875pt}%
\definecolor{currentstroke}{rgb}{0.000000,0.000000,1.000000}%
\pgfsetstrokecolor{currentstroke}%
\pgfsetdash{}{0pt}%
\pgfpathmoveto{\pgfqpoint{0.675000in}{1.563953in}}%
\pgfpathlineto{\pgfqpoint{1.531798in}{1.563953in}}%
\pgfusepath{stroke}%
\end{pgfscope}%
\begin{pgfscope}%
\pgfpathrectangle{\pgfqpoint{0.675000in}{0.750000in}}{\pgfqpoint{3.600000in}{0.930233in}} %
\pgfusepath{clip}%
\pgfsetrectcap%
\pgfsetroundjoin%
\pgfsetlinewidth{0.501875pt}%
\definecolor{currentstroke}{rgb}{1.000000,0.000000,1.000000}%
\pgfsetstrokecolor{currentstroke}%
\pgfsetdash{}{0pt}%
\pgfpathmoveto{\pgfqpoint{1.531798in}{1.563953in}}%
\pgfpathlineto{\pgfqpoint{1.531798in}{0.866279in}}%
\pgfpathlineto{\pgfqpoint{1.638898in}{0.866279in}}%
\pgfusepath{stroke}%
\end{pgfscope}%
\begin{pgfscope}%
\pgfpathrectangle{\pgfqpoint{0.675000in}{0.750000in}}{\pgfqpoint{3.600000in}{0.930233in}} %
\pgfusepath{clip}%
\pgfsetrectcap%
\pgfsetroundjoin%
\pgfsetlinewidth{0.501875pt}%
\definecolor{currentstroke}{rgb}{1.000000,0.000000,1.000000}%
\pgfsetstrokecolor{currentstroke}%
\pgfsetdash{}{0pt}%
\pgfpathmoveto{\pgfqpoint{1.638898in}{0.866279in}}%
\pgfpathlineto{\pgfqpoint{1.638898in}{1.563953in}}%
\pgfpathlineto{\pgfqpoint{1.745997in}{1.563953in}}%
\pgfusepath{stroke}%
\end{pgfscope}%
\begin{pgfscope}%
\pgfpathrectangle{\pgfqpoint{0.675000in}{0.750000in}}{\pgfqpoint{3.600000in}{0.930233in}} %
\pgfusepath{clip}%
\pgfsetrectcap%
\pgfsetroundjoin%
\pgfsetlinewidth{0.501875pt}%
\definecolor{currentstroke}{rgb}{1.000000,0.000000,1.000000}%
\pgfsetstrokecolor{currentstroke}%
\pgfsetdash{}{0pt}%
\pgfpathmoveto{\pgfqpoint{1.745997in}{1.563953in}}%
\pgfpathlineto{\pgfqpoint{1.745997in}{0.866279in}}%
\pgfpathlineto{\pgfqpoint{1.853097in}{0.866279in}}%
\pgfusepath{stroke}%
\end{pgfscope}%
\begin{pgfscope}%
\pgfpathrectangle{\pgfqpoint{0.675000in}{0.750000in}}{\pgfqpoint{3.600000in}{0.930233in}} %
\pgfusepath{clip}%
\pgfsetrectcap%
\pgfsetroundjoin%
\pgfsetlinewidth{0.501875pt}%
\definecolor{currentstroke}{rgb}{1.000000,0.000000,1.000000}%
\pgfsetstrokecolor{currentstroke}%
\pgfsetdash{}{0pt}%
\pgfpathmoveto{\pgfqpoint{1.853097in}{0.866279in}}%
\pgfpathlineto{\pgfqpoint{1.853097in}{1.563953in}}%
\pgfpathlineto{\pgfqpoint{1.960197in}{1.563953in}}%
\pgfusepath{stroke}%
\end{pgfscope}%
\begin{pgfscope}%
\pgfpathrectangle{\pgfqpoint{0.675000in}{0.750000in}}{\pgfqpoint{3.600000in}{0.930233in}} %
\pgfusepath{clip}%
\pgfsetrectcap%
\pgfsetroundjoin%
\pgfsetlinewidth{0.501875pt}%
\definecolor{currentstroke}{rgb}{1.000000,0.000000,1.000000}%
\pgfsetstrokecolor{currentstroke}%
\pgfsetdash{}{0pt}%
\pgfpathmoveto{\pgfqpoint{1.960197in}{1.563953in}}%
\pgfpathlineto{\pgfqpoint{1.960197in}{0.866279in}}%
\pgfpathlineto{\pgfqpoint{2.067297in}{0.866279in}}%
\pgfusepath{stroke}%
\end{pgfscope}%
\begin{pgfscope}%
\pgfpathrectangle{\pgfqpoint{0.675000in}{0.750000in}}{\pgfqpoint{3.600000in}{0.930233in}} %
\pgfusepath{clip}%
\pgfsetrectcap%
\pgfsetroundjoin%
\pgfsetlinewidth{0.501875pt}%
\definecolor{currentstroke}{rgb}{1.000000,0.000000,1.000000}%
\pgfsetstrokecolor{currentstroke}%
\pgfsetdash{}{0pt}%
\pgfpathmoveto{\pgfqpoint{2.067297in}{0.866279in}}%
\pgfpathlineto{\pgfqpoint{2.067297in}{1.563953in}}%
\pgfpathlineto{\pgfqpoint{2.174396in}{1.563953in}}%
\pgfusepath{stroke}%
\end{pgfscope}%
\begin{pgfscope}%
\pgfpathrectangle{\pgfqpoint{0.675000in}{0.750000in}}{\pgfqpoint{3.600000in}{0.930233in}} %
\pgfusepath{clip}%
\pgfsetrectcap%
\pgfsetroundjoin%
\pgfsetlinewidth{0.501875pt}%
\definecolor{currentstroke}{rgb}{1.000000,0.000000,1.000000}%
\pgfsetstrokecolor{currentstroke}%
\pgfsetdash{}{0pt}%
\pgfpathmoveto{\pgfqpoint{2.174396in}{1.563953in}}%
\pgfpathlineto{\pgfqpoint{2.174396in}{0.866279in}}%
\pgfpathlineto{\pgfqpoint{2.281496in}{0.866279in}}%
\pgfusepath{stroke}%
\end{pgfscope}%
\begin{pgfscope}%
\pgfpathrectangle{\pgfqpoint{0.675000in}{0.750000in}}{\pgfqpoint{3.600000in}{0.930233in}} %
\pgfusepath{clip}%
\pgfsetrectcap%
\pgfsetroundjoin%
\pgfsetlinewidth{0.501875pt}%
\definecolor{currentstroke}{rgb}{1.000000,0.000000,1.000000}%
\pgfsetstrokecolor{currentstroke}%
\pgfsetdash{}{0pt}%
\pgfpathmoveto{\pgfqpoint{2.281496in}{0.866279in}}%
\pgfpathlineto{\pgfqpoint{2.281496in}{1.563953in}}%
\pgfpathlineto{\pgfqpoint{2.388596in}{1.563953in}}%
\pgfusepath{stroke}%
\end{pgfscope}%
\begin{pgfscope}%
\pgfpathrectangle{\pgfqpoint{0.675000in}{0.750000in}}{\pgfqpoint{3.600000in}{0.930233in}} %
\pgfusepath{clip}%
\pgfsetrectcap%
\pgfsetroundjoin%
\pgfsetlinewidth{0.501875pt}%
\definecolor{currentstroke}{rgb}{0.000000,0.000000,1.000000}%
\pgfsetstrokecolor{currentstroke}%
\pgfsetdash{}{0pt}%
\pgfpathmoveto{\pgfqpoint{2.388596in}{1.563953in}}%
\pgfpathlineto{\pgfqpoint{3.245394in}{1.563953in}}%
\pgfusepath{stroke}%
\end{pgfscope}%
\begin{pgfscope}%
\pgfpathrectangle{\pgfqpoint{0.675000in}{0.750000in}}{\pgfqpoint{3.600000in}{0.930233in}} %
\pgfusepath{clip}%
\pgfsetrectcap%
\pgfsetroundjoin%
\pgfsetlinewidth{0.501875pt}%
\definecolor{currentstroke}{rgb}{1.000000,0.000000,1.000000}%
\pgfsetstrokecolor{currentstroke}%
\pgfsetdash{}{0pt}%
\pgfpathmoveto{\pgfqpoint{3.245394in}{1.563953in}}%
\pgfpathlineto{\pgfqpoint{3.245394in}{0.866279in}}%
\pgfpathlineto{\pgfqpoint{3.352494in}{0.866279in}}%
\pgfusepath{stroke}%
\end{pgfscope}%
\begin{pgfscope}%
\pgfpathrectangle{\pgfqpoint{0.675000in}{0.750000in}}{\pgfqpoint{3.600000in}{0.930233in}} %
\pgfusepath{clip}%
\pgfsetrectcap%
\pgfsetroundjoin%
\pgfsetlinewidth{0.501875pt}%
\definecolor{currentstroke}{rgb}{1.000000,0.000000,1.000000}%
\pgfsetstrokecolor{currentstroke}%
\pgfsetdash{}{0pt}%
\pgfpathmoveto{\pgfqpoint{3.352494in}{0.866279in}}%
\pgfpathlineto{\pgfqpoint{3.352494in}{1.563953in}}%
\pgfpathlineto{\pgfqpoint{3.459593in}{1.563953in}}%
\pgfusepath{stroke}%
\end{pgfscope}%
\begin{pgfscope}%
\pgfpathrectangle{\pgfqpoint{0.675000in}{0.750000in}}{\pgfqpoint{3.600000in}{0.930233in}} %
\pgfusepath{clip}%
\pgfsetrectcap%
\pgfsetroundjoin%
\pgfsetlinewidth{0.501875pt}%
\definecolor{currentstroke}{rgb}{1.000000,0.000000,1.000000}%
\pgfsetstrokecolor{currentstroke}%
\pgfsetdash{}{0pt}%
\pgfpathmoveto{\pgfqpoint{3.459593in}{1.563953in}}%
\pgfpathlineto{\pgfqpoint{3.459593in}{0.866279in}}%
\pgfpathlineto{\pgfqpoint{3.566693in}{0.866279in}}%
\pgfusepath{stroke}%
\end{pgfscope}%
\begin{pgfscope}%
\pgfpathrectangle{\pgfqpoint{0.675000in}{0.750000in}}{\pgfqpoint{3.600000in}{0.930233in}} %
\pgfusepath{clip}%
\pgfsetrectcap%
\pgfsetroundjoin%
\pgfsetlinewidth{0.501875pt}%
\definecolor{currentstroke}{rgb}{1.000000,0.000000,1.000000}%
\pgfsetstrokecolor{currentstroke}%
\pgfsetdash{}{0pt}%
\pgfpathmoveto{\pgfqpoint{3.566693in}{0.866279in}}%
\pgfpathlineto{\pgfqpoint{3.566693in}{1.563953in}}%
\pgfpathlineto{\pgfqpoint{3.673793in}{1.563953in}}%
\pgfusepath{stroke}%
\end{pgfscope}%
\begin{pgfscope}%
\pgfpathrectangle{\pgfqpoint{0.675000in}{0.750000in}}{\pgfqpoint{3.600000in}{0.930233in}} %
\pgfusepath{clip}%
\pgfsetrectcap%
\pgfsetroundjoin%
\pgfsetlinewidth{0.501875pt}%
\definecolor{currentstroke}{rgb}{1.000000,0.000000,1.000000}%
\pgfsetstrokecolor{currentstroke}%
\pgfsetdash{}{0pt}%
\pgfpathmoveto{\pgfqpoint{3.673793in}{1.563953in}}%
\pgfpathlineto{\pgfqpoint{3.673793in}{0.866279in}}%
\pgfpathlineto{\pgfqpoint{3.780893in}{0.866279in}}%
\pgfusepath{stroke}%
\end{pgfscope}%
\begin{pgfscope}%
\pgfpathrectangle{\pgfqpoint{0.675000in}{0.750000in}}{\pgfqpoint{3.600000in}{0.930233in}} %
\pgfusepath{clip}%
\pgfsetrectcap%
\pgfsetroundjoin%
\pgfsetlinewidth{0.501875pt}%
\definecolor{currentstroke}{rgb}{1.000000,0.000000,1.000000}%
\pgfsetstrokecolor{currentstroke}%
\pgfsetdash{}{0pt}%
\pgfpathmoveto{\pgfqpoint{3.780893in}{0.866279in}}%
\pgfpathlineto{\pgfqpoint{3.780893in}{1.563953in}}%
\pgfpathlineto{\pgfqpoint{3.887992in}{1.563953in}}%
\pgfusepath{stroke}%
\end{pgfscope}%
\begin{pgfscope}%
\pgfpathrectangle{\pgfqpoint{0.675000in}{0.750000in}}{\pgfqpoint{3.600000in}{0.930233in}} %
\pgfusepath{clip}%
\pgfsetrectcap%
\pgfsetroundjoin%
\pgfsetlinewidth{0.501875pt}%
\definecolor{currentstroke}{rgb}{1.000000,0.000000,1.000000}%
\pgfsetstrokecolor{currentstroke}%
\pgfsetdash{}{0pt}%
\pgfpathmoveto{\pgfqpoint{3.887992in}{1.563953in}}%
\pgfpathlineto{\pgfqpoint{3.887992in}{0.866279in}}%
\pgfpathlineto{\pgfqpoint{3.995092in}{0.866279in}}%
\pgfusepath{stroke}%
\end{pgfscope}%
\begin{pgfscope}%
\pgfpathrectangle{\pgfqpoint{0.675000in}{0.750000in}}{\pgfqpoint{3.600000in}{0.930233in}} %
\pgfusepath{clip}%
\pgfsetrectcap%
\pgfsetroundjoin%
\pgfsetlinewidth{0.501875pt}%
\definecolor{currentstroke}{rgb}{1.000000,0.000000,1.000000}%
\pgfsetstrokecolor{currentstroke}%
\pgfsetdash{}{0pt}%
\pgfpathmoveto{\pgfqpoint{3.995092in}{0.866279in}}%
\pgfpathlineto{\pgfqpoint{3.995092in}{1.563953in}}%
\pgfpathlineto{\pgfqpoint{4.102192in}{1.563953in}}%
\pgfusepath{stroke}%
\end{pgfscope}%
\begin{pgfscope}%
\pgfsetrectcap%
\pgfsetmiterjoin%
\pgfsetlinewidth{0.501875pt}%
\definecolor{currentstroke}{rgb}{0.000000,0.000000,0.000000}%
\pgfsetstrokecolor{currentstroke}%
\pgfsetdash{}{0pt}%
\pgfpathmoveto{\pgfqpoint{0.675000in}{0.750000in}}%
\pgfpathlineto{\pgfqpoint{0.675000in}{1.680233in}}%
\pgfusepath{stroke}%
\end{pgfscope}%
\begin{pgfscope}%
\pgfsetrectcap%
\pgfsetmiterjoin%
\pgfsetlinewidth{0.501875pt}%
\definecolor{currentstroke}{rgb}{0.000000,0.000000,0.000000}%
\pgfsetstrokecolor{currentstroke}%
\pgfsetdash{}{0pt}%
\pgfpathmoveto{\pgfqpoint{0.675000in}{0.750000in}}%
\pgfpathlineto{\pgfqpoint{4.275000in}{0.750000in}}%
\pgfusepath{stroke}%
\end{pgfscope}%
\begin{pgfscope}%
\pgfsetrectcap%
\pgfsetmiterjoin%
\pgfsetlinewidth{0.501875pt}%
\definecolor{currentstroke}{rgb}{0.000000,0.000000,0.000000}%
\pgfsetstrokecolor{currentstroke}%
\pgfsetdash{}{0pt}%
\pgfpathmoveto{\pgfqpoint{0.675000in}{1.680233in}}%
\pgfpathlineto{\pgfqpoint{4.275000in}{1.680233in}}%
\pgfusepath{stroke}%
\end{pgfscope}%
\begin{pgfscope}%
\pgfsetrectcap%
\pgfsetmiterjoin%
\pgfsetlinewidth{0.501875pt}%
\definecolor{currentstroke}{rgb}{0.000000,0.000000,0.000000}%
\pgfsetstrokecolor{currentstroke}%
\pgfsetdash{}{0pt}%
\pgfpathmoveto{\pgfqpoint{4.275000in}{0.750000in}}%
\pgfpathlineto{\pgfqpoint{4.275000in}{1.680233in}}%
\pgfusepath{stroke}%
\end{pgfscope}%
\begin{pgfscope}%
\pgfsetbuttcap%
\pgfsetroundjoin%
\definecolor{currentfill}{rgb}{0.000000,0.000000,0.000000}%
\pgfsetfillcolor{currentfill}%
\pgfsetlinewidth{0.501875pt}%
\definecolor{currentstroke}{rgb}{0.000000,0.000000,0.000000}%
\pgfsetstrokecolor{currentstroke}%
\pgfsetdash{}{0pt}%
\pgfsys@defobject{currentmarker}{\pgfqpoint{0.000000in}{0.000000in}}{\pgfqpoint{0.000000in}{0.055556in}}{%
\pgfpathmoveto{\pgfqpoint{0.000000in}{0.000000in}}%
\pgfpathlineto{\pgfqpoint{0.000000in}{0.055556in}}%
\pgfusepath{stroke,fill}%
}%
\begin{pgfscope}%
\pgfsys@transformshift{0.675000in}{0.750000in}%
\pgfsys@useobject{currentmarker}{}%
\end{pgfscope}%
\end{pgfscope}%
\begin{pgfscope}%
\pgfsetbuttcap%
\pgfsetroundjoin%
\definecolor{currentfill}{rgb}{0.000000,0.000000,0.000000}%
\pgfsetfillcolor{currentfill}%
\pgfsetlinewidth{0.501875pt}%
\definecolor{currentstroke}{rgb}{0.000000,0.000000,0.000000}%
\pgfsetstrokecolor{currentstroke}%
\pgfsetdash{}{0pt}%
\pgfsys@defobject{currentmarker}{\pgfqpoint{0.000000in}{-0.055556in}}{\pgfqpoint{0.000000in}{0.000000in}}{%
\pgfpathmoveto{\pgfqpoint{0.000000in}{0.000000in}}%
\pgfpathlineto{\pgfqpoint{0.000000in}{-0.055556in}}%
\pgfusepath{stroke,fill}%
}%
\begin{pgfscope}%
\pgfsys@transformshift{0.675000in}{1.680233in}%
\pgfsys@useobject{currentmarker}{}%
\end{pgfscope}%
\end{pgfscope}%
\begin{pgfscope}%
\pgftext[x=0.675000in,y=0.694444in,,top]{\rmfamily\fontsize{9.000000}{10.800000}\selectfont \(\displaystyle 0.0000\)}%
\end{pgfscope}%
\begin{pgfscope}%
\pgfsetbuttcap%
\pgfsetroundjoin%
\definecolor{currentfill}{rgb}{0.000000,0.000000,0.000000}%
\pgfsetfillcolor{currentfill}%
\pgfsetlinewidth{0.501875pt}%
\definecolor{currentstroke}{rgb}{0.000000,0.000000,0.000000}%
\pgfsetstrokecolor{currentstroke}%
\pgfsetdash{}{0pt}%
\pgfsys@defobject{currentmarker}{\pgfqpoint{0.000000in}{0.000000in}}{\pgfqpoint{0.000000in}{0.055556in}}{%
\pgfpathmoveto{\pgfqpoint{0.000000in}{0.000000in}}%
\pgfpathlineto{\pgfqpoint{0.000000in}{0.055556in}}%
\pgfusepath{stroke,fill}%
}%
\begin{pgfscope}%
\pgfsys@transformshift{1.125000in}{0.750000in}%
\pgfsys@useobject{currentmarker}{}%
\end{pgfscope}%
\end{pgfscope}%
\begin{pgfscope}%
\pgfsetbuttcap%
\pgfsetroundjoin%
\definecolor{currentfill}{rgb}{0.000000,0.000000,0.000000}%
\pgfsetfillcolor{currentfill}%
\pgfsetlinewidth{0.501875pt}%
\definecolor{currentstroke}{rgb}{0.000000,0.000000,0.000000}%
\pgfsetstrokecolor{currentstroke}%
\pgfsetdash{}{0pt}%
\pgfsys@defobject{currentmarker}{\pgfqpoint{0.000000in}{-0.055556in}}{\pgfqpoint{0.000000in}{0.000000in}}{%
\pgfpathmoveto{\pgfqpoint{0.000000in}{0.000000in}}%
\pgfpathlineto{\pgfqpoint{0.000000in}{-0.055556in}}%
\pgfusepath{stroke,fill}%
}%
\begin{pgfscope}%
\pgfsys@transformshift{1.125000in}{1.680233in}%
\pgfsys@useobject{currentmarker}{}%
\end{pgfscope}%
\end{pgfscope}%
\begin{pgfscope}%
\pgftext[x=1.125000in,y=0.694444in,,top]{\rmfamily\fontsize{9.000000}{10.800000}\selectfont \(\displaystyle 0.0002\)}%
\end{pgfscope}%
\begin{pgfscope}%
\pgfsetbuttcap%
\pgfsetroundjoin%
\definecolor{currentfill}{rgb}{0.000000,0.000000,0.000000}%
\pgfsetfillcolor{currentfill}%
\pgfsetlinewidth{0.501875pt}%
\definecolor{currentstroke}{rgb}{0.000000,0.000000,0.000000}%
\pgfsetstrokecolor{currentstroke}%
\pgfsetdash{}{0pt}%
\pgfsys@defobject{currentmarker}{\pgfqpoint{0.000000in}{0.000000in}}{\pgfqpoint{0.000000in}{0.055556in}}{%
\pgfpathmoveto{\pgfqpoint{0.000000in}{0.000000in}}%
\pgfpathlineto{\pgfqpoint{0.000000in}{0.055556in}}%
\pgfusepath{stroke,fill}%
}%
\begin{pgfscope}%
\pgfsys@transformshift{1.575000in}{0.750000in}%
\pgfsys@useobject{currentmarker}{}%
\end{pgfscope}%
\end{pgfscope}%
\begin{pgfscope}%
\pgfsetbuttcap%
\pgfsetroundjoin%
\definecolor{currentfill}{rgb}{0.000000,0.000000,0.000000}%
\pgfsetfillcolor{currentfill}%
\pgfsetlinewidth{0.501875pt}%
\definecolor{currentstroke}{rgb}{0.000000,0.000000,0.000000}%
\pgfsetstrokecolor{currentstroke}%
\pgfsetdash{}{0pt}%
\pgfsys@defobject{currentmarker}{\pgfqpoint{0.000000in}{-0.055556in}}{\pgfqpoint{0.000000in}{0.000000in}}{%
\pgfpathmoveto{\pgfqpoint{0.000000in}{0.000000in}}%
\pgfpathlineto{\pgfqpoint{0.000000in}{-0.055556in}}%
\pgfusepath{stroke,fill}%
}%
\begin{pgfscope}%
\pgfsys@transformshift{1.575000in}{1.680233in}%
\pgfsys@useobject{currentmarker}{}%
\end{pgfscope}%
\end{pgfscope}%
\begin{pgfscope}%
\pgftext[x=1.575000in,y=0.694444in,,top]{\rmfamily\fontsize{9.000000}{10.800000}\selectfont \(\displaystyle 0.0004\)}%
\end{pgfscope}%
\begin{pgfscope}%
\pgfsetbuttcap%
\pgfsetroundjoin%
\definecolor{currentfill}{rgb}{0.000000,0.000000,0.000000}%
\pgfsetfillcolor{currentfill}%
\pgfsetlinewidth{0.501875pt}%
\definecolor{currentstroke}{rgb}{0.000000,0.000000,0.000000}%
\pgfsetstrokecolor{currentstroke}%
\pgfsetdash{}{0pt}%
\pgfsys@defobject{currentmarker}{\pgfqpoint{0.000000in}{0.000000in}}{\pgfqpoint{0.000000in}{0.055556in}}{%
\pgfpathmoveto{\pgfqpoint{0.000000in}{0.000000in}}%
\pgfpathlineto{\pgfqpoint{0.000000in}{0.055556in}}%
\pgfusepath{stroke,fill}%
}%
\begin{pgfscope}%
\pgfsys@transformshift{2.025000in}{0.750000in}%
\pgfsys@useobject{currentmarker}{}%
\end{pgfscope}%
\end{pgfscope}%
\begin{pgfscope}%
\pgfsetbuttcap%
\pgfsetroundjoin%
\definecolor{currentfill}{rgb}{0.000000,0.000000,0.000000}%
\pgfsetfillcolor{currentfill}%
\pgfsetlinewidth{0.501875pt}%
\definecolor{currentstroke}{rgb}{0.000000,0.000000,0.000000}%
\pgfsetstrokecolor{currentstroke}%
\pgfsetdash{}{0pt}%
\pgfsys@defobject{currentmarker}{\pgfqpoint{0.000000in}{-0.055556in}}{\pgfqpoint{0.000000in}{0.000000in}}{%
\pgfpathmoveto{\pgfqpoint{0.000000in}{0.000000in}}%
\pgfpathlineto{\pgfqpoint{0.000000in}{-0.055556in}}%
\pgfusepath{stroke,fill}%
}%
\begin{pgfscope}%
\pgfsys@transformshift{2.025000in}{1.680233in}%
\pgfsys@useobject{currentmarker}{}%
\end{pgfscope}%
\end{pgfscope}%
\begin{pgfscope}%
\pgftext[x=2.025000in,y=0.694444in,,top]{\rmfamily\fontsize{9.000000}{10.800000}\selectfont \(\displaystyle 0.0006\)}%
\end{pgfscope}%
\begin{pgfscope}%
\pgfsetbuttcap%
\pgfsetroundjoin%
\definecolor{currentfill}{rgb}{0.000000,0.000000,0.000000}%
\pgfsetfillcolor{currentfill}%
\pgfsetlinewidth{0.501875pt}%
\definecolor{currentstroke}{rgb}{0.000000,0.000000,0.000000}%
\pgfsetstrokecolor{currentstroke}%
\pgfsetdash{}{0pt}%
\pgfsys@defobject{currentmarker}{\pgfqpoint{0.000000in}{0.000000in}}{\pgfqpoint{0.000000in}{0.055556in}}{%
\pgfpathmoveto{\pgfqpoint{0.000000in}{0.000000in}}%
\pgfpathlineto{\pgfqpoint{0.000000in}{0.055556in}}%
\pgfusepath{stroke,fill}%
}%
\begin{pgfscope}%
\pgfsys@transformshift{2.475000in}{0.750000in}%
\pgfsys@useobject{currentmarker}{}%
\end{pgfscope}%
\end{pgfscope}%
\begin{pgfscope}%
\pgfsetbuttcap%
\pgfsetroundjoin%
\definecolor{currentfill}{rgb}{0.000000,0.000000,0.000000}%
\pgfsetfillcolor{currentfill}%
\pgfsetlinewidth{0.501875pt}%
\definecolor{currentstroke}{rgb}{0.000000,0.000000,0.000000}%
\pgfsetstrokecolor{currentstroke}%
\pgfsetdash{}{0pt}%
\pgfsys@defobject{currentmarker}{\pgfqpoint{0.000000in}{-0.055556in}}{\pgfqpoint{0.000000in}{0.000000in}}{%
\pgfpathmoveto{\pgfqpoint{0.000000in}{0.000000in}}%
\pgfpathlineto{\pgfqpoint{0.000000in}{-0.055556in}}%
\pgfusepath{stroke,fill}%
}%
\begin{pgfscope}%
\pgfsys@transformshift{2.475000in}{1.680233in}%
\pgfsys@useobject{currentmarker}{}%
\end{pgfscope}%
\end{pgfscope}%
\begin{pgfscope}%
\pgftext[x=2.475000in,y=0.694444in,,top]{\rmfamily\fontsize{9.000000}{10.800000}\selectfont \(\displaystyle 0.0008\)}%
\end{pgfscope}%
\begin{pgfscope}%
\pgfsetbuttcap%
\pgfsetroundjoin%
\definecolor{currentfill}{rgb}{0.000000,0.000000,0.000000}%
\pgfsetfillcolor{currentfill}%
\pgfsetlinewidth{0.501875pt}%
\definecolor{currentstroke}{rgb}{0.000000,0.000000,0.000000}%
\pgfsetstrokecolor{currentstroke}%
\pgfsetdash{}{0pt}%
\pgfsys@defobject{currentmarker}{\pgfqpoint{0.000000in}{0.000000in}}{\pgfqpoint{0.000000in}{0.055556in}}{%
\pgfpathmoveto{\pgfqpoint{0.000000in}{0.000000in}}%
\pgfpathlineto{\pgfqpoint{0.000000in}{0.055556in}}%
\pgfusepath{stroke,fill}%
}%
\begin{pgfscope}%
\pgfsys@transformshift{2.925000in}{0.750000in}%
\pgfsys@useobject{currentmarker}{}%
\end{pgfscope}%
\end{pgfscope}%
\begin{pgfscope}%
\pgfsetbuttcap%
\pgfsetroundjoin%
\definecolor{currentfill}{rgb}{0.000000,0.000000,0.000000}%
\pgfsetfillcolor{currentfill}%
\pgfsetlinewidth{0.501875pt}%
\definecolor{currentstroke}{rgb}{0.000000,0.000000,0.000000}%
\pgfsetstrokecolor{currentstroke}%
\pgfsetdash{}{0pt}%
\pgfsys@defobject{currentmarker}{\pgfqpoint{0.000000in}{-0.055556in}}{\pgfqpoint{0.000000in}{0.000000in}}{%
\pgfpathmoveto{\pgfqpoint{0.000000in}{0.000000in}}%
\pgfpathlineto{\pgfqpoint{0.000000in}{-0.055556in}}%
\pgfusepath{stroke,fill}%
}%
\begin{pgfscope}%
\pgfsys@transformshift{2.925000in}{1.680233in}%
\pgfsys@useobject{currentmarker}{}%
\end{pgfscope}%
\end{pgfscope}%
\begin{pgfscope}%
\pgftext[x=2.925000in,y=0.694444in,,top]{\rmfamily\fontsize{9.000000}{10.800000}\selectfont \(\displaystyle 0.0010\)}%
\end{pgfscope}%
\begin{pgfscope}%
\pgfsetbuttcap%
\pgfsetroundjoin%
\definecolor{currentfill}{rgb}{0.000000,0.000000,0.000000}%
\pgfsetfillcolor{currentfill}%
\pgfsetlinewidth{0.501875pt}%
\definecolor{currentstroke}{rgb}{0.000000,0.000000,0.000000}%
\pgfsetstrokecolor{currentstroke}%
\pgfsetdash{}{0pt}%
\pgfsys@defobject{currentmarker}{\pgfqpoint{0.000000in}{0.000000in}}{\pgfqpoint{0.000000in}{0.055556in}}{%
\pgfpathmoveto{\pgfqpoint{0.000000in}{0.000000in}}%
\pgfpathlineto{\pgfqpoint{0.000000in}{0.055556in}}%
\pgfusepath{stroke,fill}%
}%
\begin{pgfscope}%
\pgfsys@transformshift{3.375000in}{0.750000in}%
\pgfsys@useobject{currentmarker}{}%
\end{pgfscope}%
\end{pgfscope}%
\begin{pgfscope}%
\pgfsetbuttcap%
\pgfsetroundjoin%
\definecolor{currentfill}{rgb}{0.000000,0.000000,0.000000}%
\pgfsetfillcolor{currentfill}%
\pgfsetlinewidth{0.501875pt}%
\definecolor{currentstroke}{rgb}{0.000000,0.000000,0.000000}%
\pgfsetstrokecolor{currentstroke}%
\pgfsetdash{}{0pt}%
\pgfsys@defobject{currentmarker}{\pgfqpoint{0.000000in}{-0.055556in}}{\pgfqpoint{0.000000in}{0.000000in}}{%
\pgfpathmoveto{\pgfqpoint{0.000000in}{0.000000in}}%
\pgfpathlineto{\pgfqpoint{0.000000in}{-0.055556in}}%
\pgfusepath{stroke,fill}%
}%
\begin{pgfscope}%
\pgfsys@transformshift{3.375000in}{1.680233in}%
\pgfsys@useobject{currentmarker}{}%
\end{pgfscope}%
\end{pgfscope}%
\begin{pgfscope}%
\pgftext[x=3.375000in,y=0.694444in,,top]{\rmfamily\fontsize{9.000000}{10.800000}\selectfont \(\displaystyle 0.0012\)}%
\end{pgfscope}%
\begin{pgfscope}%
\pgfsetbuttcap%
\pgfsetroundjoin%
\definecolor{currentfill}{rgb}{0.000000,0.000000,0.000000}%
\pgfsetfillcolor{currentfill}%
\pgfsetlinewidth{0.501875pt}%
\definecolor{currentstroke}{rgb}{0.000000,0.000000,0.000000}%
\pgfsetstrokecolor{currentstroke}%
\pgfsetdash{}{0pt}%
\pgfsys@defobject{currentmarker}{\pgfqpoint{0.000000in}{0.000000in}}{\pgfqpoint{0.000000in}{0.055556in}}{%
\pgfpathmoveto{\pgfqpoint{0.000000in}{0.000000in}}%
\pgfpathlineto{\pgfqpoint{0.000000in}{0.055556in}}%
\pgfusepath{stroke,fill}%
}%
\begin{pgfscope}%
\pgfsys@transformshift{3.825000in}{0.750000in}%
\pgfsys@useobject{currentmarker}{}%
\end{pgfscope}%
\end{pgfscope}%
\begin{pgfscope}%
\pgfsetbuttcap%
\pgfsetroundjoin%
\definecolor{currentfill}{rgb}{0.000000,0.000000,0.000000}%
\pgfsetfillcolor{currentfill}%
\pgfsetlinewidth{0.501875pt}%
\definecolor{currentstroke}{rgb}{0.000000,0.000000,0.000000}%
\pgfsetstrokecolor{currentstroke}%
\pgfsetdash{}{0pt}%
\pgfsys@defobject{currentmarker}{\pgfqpoint{0.000000in}{-0.055556in}}{\pgfqpoint{0.000000in}{0.000000in}}{%
\pgfpathmoveto{\pgfqpoint{0.000000in}{0.000000in}}%
\pgfpathlineto{\pgfqpoint{0.000000in}{-0.055556in}}%
\pgfusepath{stroke,fill}%
}%
\begin{pgfscope}%
\pgfsys@transformshift{3.825000in}{1.680233in}%
\pgfsys@useobject{currentmarker}{}%
\end{pgfscope}%
\end{pgfscope}%
\begin{pgfscope}%
\pgftext[x=3.825000in,y=0.694444in,,top]{\rmfamily\fontsize{9.000000}{10.800000}\selectfont \(\displaystyle 0.0014\)}%
\end{pgfscope}%
\begin{pgfscope}%
\pgfsetbuttcap%
\pgfsetroundjoin%
\definecolor{currentfill}{rgb}{0.000000,0.000000,0.000000}%
\pgfsetfillcolor{currentfill}%
\pgfsetlinewidth{0.501875pt}%
\definecolor{currentstroke}{rgb}{0.000000,0.000000,0.000000}%
\pgfsetstrokecolor{currentstroke}%
\pgfsetdash{}{0pt}%
\pgfsys@defobject{currentmarker}{\pgfqpoint{0.000000in}{0.000000in}}{\pgfqpoint{0.000000in}{0.055556in}}{%
\pgfpathmoveto{\pgfqpoint{0.000000in}{0.000000in}}%
\pgfpathlineto{\pgfqpoint{0.000000in}{0.055556in}}%
\pgfusepath{stroke,fill}%
}%
\begin{pgfscope}%
\pgfsys@transformshift{4.275000in}{0.750000in}%
\pgfsys@useobject{currentmarker}{}%
\end{pgfscope}%
\end{pgfscope}%
\begin{pgfscope}%
\pgfsetbuttcap%
\pgfsetroundjoin%
\definecolor{currentfill}{rgb}{0.000000,0.000000,0.000000}%
\pgfsetfillcolor{currentfill}%
\pgfsetlinewidth{0.501875pt}%
\definecolor{currentstroke}{rgb}{0.000000,0.000000,0.000000}%
\pgfsetstrokecolor{currentstroke}%
\pgfsetdash{}{0pt}%
\pgfsys@defobject{currentmarker}{\pgfqpoint{0.000000in}{-0.055556in}}{\pgfqpoint{0.000000in}{0.000000in}}{%
\pgfpathmoveto{\pgfqpoint{0.000000in}{0.000000in}}%
\pgfpathlineto{\pgfqpoint{0.000000in}{-0.055556in}}%
\pgfusepath{stroke,fill}%
}%
\begin{pgfscope}%
\pgfsys@transformshift{4.275000in}{1.680233in}%
\pgfsys@useobject{currentmarker}{}%
\end{pgfscope}%
\end{pgfscope}%
\begin{pgfscope}%
\pgftext[x=4.275000in,y=0.694444in,,top]{\rmfamily\fontsize{9.000000}{10.800000}\selectfont \(\displaystyle 0.0016\)}%
\end{pgfscope}%
\begin{pgfscope}%
\pgftext[x=2.475000in,y=0.504028in,,top]{\rmfamily\fontsize{9.000000}{10.800000}\selectfont Zeit (s)}%
\end{pgfscope}%
\begin{pgfscope}%
\pgfsetbuttcap%
\pgfsetroundjoin%
\definecolor{currentfill}{rgb}{0.000000,0.000000,0.000000}%
\pgfsetfillcolor{currentfill}%
\pgfsetlinewidth{0.501875pt}%
\definecolor{currentstroke}{rgb}{0.000000,0.000000,0.000000}%
\pgfsetstrokecolor{currentstroke}%
\pgfsetdash{}{0pt}%
\pgfsys@defobject{currentmarker}{\pgfqpoint{0.000000in}{0.000000in}}{\pgfqpoint{0.055556in}{0.000000in}}{%
\pgfpathmoveto{\pgfqpoint{0.000000in}{0.000000in}}%
\pgfpathlineto{\pgfqpoint{0.055556in}{0.000000in}}%
\pgfusepath{stroke,fill}%
}%
\begin{pgfscope}%
\pgfsys@transformshift{0.675000in}{0.750000in}%
\pgfsys@useobject{currentmarker}{}%
\end{pgfscope}%
\end{pgfscope}%
\begin{pgfscope}%
\pgfsetbuttcap%
\pgfsetroundjoin%
\definecolor{currentfill}{rgb}{0.000000,0.000000,0.000000}%
\pgfsetfillcolor{currentfill}%
\pgfsetlinewidth{0.501875pt}%
\definecolor{currentstroke}{rgb}{0.000000,0.000000,0.000000}%
\pgfsetstrokecolor{currentstroke}%
\pgfsetdash{}{0pt}%
\pgfsys@defobject{currentmarker}{\pgfqpoint{-0.055556in}{0.000000in}}{\pgfqpoint{0.000000in}{0.000000in}}{%
\pgfpathmoveto{\pgfqpoint{0.000000in}{0.000000in}}%
\pgfpathlineto{\pgfqpoint{-0.055556in}{0.000000in}}%
\pgfusepath{stroke,fill}%
}%
\begin{pgfscope}%
\pgfsys@transformshift{4.275000in}{0.750000in}%
\pgfsys@useobject{currentmarker}{}%
\end{pgfscope}%
\end{pgfscope}%
\begin{pgfscope}%
\pgftext[x=0.619444in,y=0.750000in,right,]{\rmfamily\fontsize{9.000000}{10.800000}\selectfont \(\displaystyle 890\)}%
\end{pgfscope}%
\begin{pgfscope}%
\pgfsetbuttcap%
\pgfsetroundjoin%
\definecolor{currentfill}{rgb}{0.000000,0.000000,0.000000}%
\pgfsetfillcolor{currentfill}%
\pgfsetlinewidth{0.501875pt}%
\definecolor{currentstroke}{rgb}{0.000000,0.000000,0.000000}%
\pgfsetstrokecolor{currentstroke}%
\pgfsetdash{}{0pt}%
\pgfsys@defobject{currentmarker}{\pgfqpoint{0.000000in}{0.000000in}}{\pgfqpoint{0.055556in}{0.000000in}}{%
\pgfpathmoveto{\pgfqpoint{0.000000in}{0.000000in}}%
\pgfpathlineto{\pgfqpoint{0.055556in}{0.000000in}}%
\pgfusepath{stroke,fill}%
}%
\begin{pgfscope}%
\pgfsys@transformshift{0.675000in}{0.866279in}%
\pgfsys@useobject{currentmarker}{}%
\end{pgfscope}%
\end{pgfscope}%
\begin{pgfscope}%
\pgfsetbuttcap%
\pgfsetroundjoin%
\definecolor{currentfill}{rgb}{0.000000,0.000000,0.000000}%
\pgfsetfillcolor{currentfill}%
\pgfsetlinewidth{0.501875pt}%
\definecolor{currentstroke}{rgb}{0.000000,0.000000,0.000000}%
\pgfsetstrokecolor{currentstroke}%
\pgfsetdash{}{0pt}%
\pgfsys@defobject{currentmarker}{\pgfqpoint{-0.055556in}{0.000000in}}{\pgfqpoint{0.000000in}{0.000000in}}{%
\pgfpathmoveto{\pgfqpoint{0.000000in}{0.000000in}}%
\pgfpathlineto{\pgfqpoint{-0.055556in}{0.000000in}}%
\pgfusepath{stroke,fill}%
}%
\begin{pgfscope}%
\pgfsys@transformshift{4.275000in}{0.866279in}%
\pgfsys@useobject{currentmarker}{}%
\end{pgfscope}%
\end{pgfscope}%
\begin{pgfscope}%
\pgftext[x=0.619444in,y=0.866279in,right,]{\rmfamily\fontsize{9.000000}{10.800000}\selectfont \(\displaystyle 900\)}%
\end{pgfscope}%
\begin{pgfscope}%
\pgfsetbuttcap%
\pgfsetroundjoin%
\definecolor{currentfill}{rgb}{0.000000,0.000000,0.000000}%
\pgfsetfillcolor{currentfill}%
\pgfsetlinewidth{0.501875pt}%
\definecolor{currentstroke}{rgb}{0.000000,0.000000,0.000000}%
\pgfsetstrokecolor{currentstroke}%
\pgfsetdash{}{0pt}%
\pgfsys@defobject{currentmarker}{\pgfqpoint{0.000000in}{0.000000in}}{\pgfqpoint{0.055556in}{0.000000in}}{%
\pgfpathmoveto{\pgfqpoint{0.000000in}{0.000000in}}%
\pgfpathlineto{\pgfqpoint{0.055556in}{0.000000in}}%
\pgfusepath{stroke,fill}%
}%
\begin{pgfscope}%
\pgfsys@transformshift{0.675000in}{0.982558in}%
\pgfsys@useobject{currentmarker}{}%
\end{pgfscope}%
\end{pgfscope}%
\begin{pgfscope}%
\pgfsetbuttcap%
\pgfsetroundjoin%
\definecolor{currentfill}{rgb}{0.000000,0.000000,0.000000}%
\pgfsetfillcolor{currentfill}%
\pgfsetlinewidth{0.501875pt}%
\definecolor{currentstroke}{rgb}{0.000000,0.000000,0.000000}%
\pgfsetstrokecolor{currentstroke}%
\pgfsetdash{}{0pt}%
\pgfsys@defobject{currentmarker}{\pgfqpoint{-0.055556in}{0.000000in}}{\pgfqpoint{0.000000in}{0.000000in}}{%
\pgfpathmoveto{\pgfqpoint{0.000000in}{0.000000in}}%
\pgfpathlineto{\pgfqpoint{-0.055556in}{0.000000in}}%
\pgfusepath{stroke,fill}%
}%
\begin{pgfscope}%
\pgfsys@transformshift{4.275000in}{0.982558in}%
\pgfsys@useobject{currentmarker}{}%
\end{pgfscope}%
\end{pgfscope}%
\begin{pgfscope}%
\pgftext[x=0.619444in,y=0.982558in,right,]{\rmfamily\fontsize{9.000000}{10.800000}\selectfont \(\displaystyle 910\)}%
\end{pgfscope}%
\begin{pgfscope}%
\pgfsetbuttcap%
\pgfsetroundjoin%
\definecolor{currentfill}{rgb}{0.000000,0.000000,0.000000}%
\pgfsetfillcolor{currentfill}%
\pgfsetlinewidth{0.501875pt}%
\definecolor{currentstroke}{rgb}{0.000000,0.000000,0.000000}%
\pgfsetstrokecolor{currentstroke}%
\pgfsetdash{}{0pt}%
\pgfsys@defobject{currentmarker}{\pgfqpoint{0.000000in}{0.000000in}}{\pgfqpoint{0.055556in}{0.000000in}}{%
\pgfpathmoveto{\pgfqpoint{0.000000in}{0.000000in}}%
\pgfpathlineto{\pgfqpoint{0.055556in}{0.000000in}}%
\pgfusepath{stroke,fill}%
}%
\begin{pgfscope}%
\pgfsys@transformshift{0.675000in}{1.098837in}%
\pgfsys@useobject{currentmarker}{}%
\end{pgfscope}%
\end{pgfscope}%
\begin{pgfscope}%
\pgfsetbuttcap%
\pgfsetroundjoin%
\definecolor{currentfill}{rgb}{0.000000,0.000000,0.000000}%
\pgfsetfillcolor{currentfill}%
\pgfsetlinewidth{0.501875pt}%
\definecolor{currentstroke}{rgb}{0.000000,0.000000,0.000000}%
\pgfsetstrokecolor{currentstroke}%
\pgfsetdash{}{0pt}%
\pgfsys@defobject{currentmarker}{\pgfqpoint{-0.055556in}{0.000000in}}{\pgfqpoint{0.000000in}{0.000000in}}{%
\pgfpathmoveto{\pgfqpoint{0.000000in}{0.000000in}}%
\pgfpathlineto{\pgfqpoint{-0.055556in}{0.000000in}}%
\pgfusepath{stroke,fill}%
}%
\begin{pgfscope}%
\pgfsys@transformshift{4.275000in}{1.098837in}%
\pgfsys@useobject{currentmarker}{}%
\end{pgfscope}%
\end{pgfscope}%
\begin{pgfscope}%
\pgftext[x=0.619444in,y=1.098837in,right,]{\rmfamily\fontsize{9.000000}{10.800000}\selectfont \(\displaystyle 920\)}%
\end{pgfscope}%
\begin{pgfscope}%
\pgfsetbuttcap%
\pgfsetroundjoin%
\definecolor{currentfill}{rgb}{0.000000,0.000000,0.000000}%
\pgfsetfillcolor{currentfill}%
\pgfsetlinewidth{0.501875pt}%
\definecolor{currentstroke}{rgb}{0.000000,0.000000,0.000000}%
\pgfsetstrokecolor{currentstroke}%
\pgfsetdash{}{0pt}%
\pgfsys@defobject{currentmarker}{\pgfqpoint{0.000000in}{0.000000in}}{\pgfqpoint{0.055556in}{0.000000in}}{%
\pgfpathmoveto{\pgfqpoint{0.000000in}{0.000000in}}%
\pgfpathlineto{\pgfqpoint{0.055556in}{0.000000in}}%
\pgfusepath{stroke,fill}%
}%
\begin{pgfscope}%
\pgfsys@transformshift{0.675000in}{1.215116in}%
\pgfsys@useobject{currentmarker}{}%
\end{pgfscope}%
\end{pgfscope}%
\begin{pgfscope}%
\pgfsetbuttcap%
\pgfsetroundjoin%
\definecolor{currentfill}{rgb}{0.000000,0.000000,0.000000}%
\pgfsetfillcolor{currentfill}%
\pgfsetlinewidth{0.501875pt}%
\definecolor{currentstroke}{rgb}{0.000000,0.000000,0.000000}%
\pgfsetstrokecolor{currentstroke}%
\pgfsetdash{}{0pt}%
\pgfsys@defobject{currentmarker}{\pgfqpoint{-0.055556in}{0.000000in}}{\pgfqpoint{0.000000in}{0.000000in}}{%
\pgfpathmoveto{\pgfqpoint{0.000000in}{0.000000in}}%
\pgfpathlineto{\pgfqpoint{-0.055556in}{0.000000in}}%
\pgfusepath{stroke,fill}%
}%
\begin{pgfscope}%
\pgfsys@transformshift{4.275000in}{1.215116in}%
\pgfsys@useobject{currentmarker}{}%
\end{pgfscope}%
\end{pgfscope}%
\begin{pgfscope}%
\pgftext[x=0.619444in,y=1.215116in,right,]{\rmfamily\fontsize{9.000000}{10.800000}\selectfont \(\displaystyle 930\)}%
\end{pgfscope}%
\begin{pgfscope}%
\pgfsetbuttcap%
\pgfsetroundjoin%
\definecolor{currentfill}{rgb}{0.000000,0.000000,0.000000}%
\pgfsetfillcolor{currentfill}%
\pgfsetlinewidth{0.501875pt}%
\definecolor{currentstroke}{rgb}{0.000000,0.000000,0.000000}%
\pgfsetstrokecolor{currentstroke}%
\pgfsetdash{}{0pt}%
\pgfsys@defobject{currentmarker}{\pgfqpoint{0.000000in}{0.000000in}}{\pgfqpoint{0.055556in}{0.000000in}}{%
\pgfpathmoveto{\pgfqpoint{0.000000in}{0.000000in}}%
\pgfpathlineto{\pgfqpoint{0.055556in}{0.000000in}}%
\pgfusepath{stroke,fill}%
}%
\begin{pgfscope}%
\pgfsys@transformshift{0.675000in}{1.331395in}%
\pgfsys@useobject{currentmarker}{}%
\end{pgfscope}%
\end{pgfscope}%
\begin{pgfscope}%
\pgfsetbuttcap%
\pgfsetroundjoin%
\definecolor{currentfill}{rgb}{0.000000,0.000000,0.000000}%
\pgfsetfillcolor{currentfill}%
\pgfsetlinewidth{0.501875pt}%
\definecolor{currentstroke}{rgb}{0.000000,0.000000,0.000000}%
\pgfsetstrokecolor{currentstroke}%
\pgfsetdash{}{0pt}%
\pgfsys@defobject{currentmarker}{\pgfqpoint{-0.055556in}{0.000000in}}{\pgfqpoint{0.000000in}{0.000000in}}{%
\pgfpathmoveto{\pgfqpoint{0.000000in}{0.000000in}}%
\pgfpathlineto{\pgfqpoint{-0.055556in}{0.000000in}}%
\pgfusepath{stroke,fill}%
}%
\begin{pgfscope}%
\pgfsys@transformshift{4.275000in}{1.331395in}%
\pgfsys@useobject{currentmarker}{}%
\end{pgfscope}%
\end{pgfscope}%
\begin{pgfscope}%
\pgftext[x=0.619444in,y=1.331395in,right,]{\rmfamily\fontsize{9.000000}{10.800000}\selectfont \(\displaystyle 940\)}%
\end{pgfscope}%
\begin{pgfscope}%
\pgfsetbuttcap%
\pgfsetroundjoin%
\definecolor{currentfill}{rgb}{0.000000,0.000000,0.000000}%
\pgfsetfillcolor{currentfill}%
\pgfsetlinewidth{0.501875pt}%
\definecolor{currentstroke}{rgb}{0.000000,0.000000,0.000000}%
\pgfsetstrokecolor{currentstroke}%
\pgfsetdash{}{0pt}%
\pgfsys@defobject{currentmarker}{\pgfqpoint{0.000000in}{0.000000in}}{\pgfqpoint{0.055556in}{0.000000in}}{%
\pgfpathmoveto{\pgfqpoint{0.000000in}{0.000000in}}%
\pgfpathlineto{\pgfqpoint{0.055556in}{0.000000in}}%
\pgfusepath{stroke,fill}%
}%
\begin{pgfscope}%
\pgfsys@transformshift{0.675000in}{1.447674in}%
\pgfsys@useobject{currentmarker}{}%
\end{pgfscope}%
\end{pgfscope}%
\begin{pgfscope}%
\pgfsetbuttcap%
\pgfsetroundjoin%
\definecolor{currentfill}{rgb}{0.000000,0.000000,0.000000}%
\pgfsetfillcolor{currentfill}%
\pgfsetlinewidth{0.501875pt}%
\definecolor{currentstroke}{rgb}{0.000000,0.000000,0.000000}%
\pgfsetstrokecolor{currentstroke}%
\pgfsetdash{}{0pt}%
\pgfsys@defobject{currentmarker}{\pgfqpoint{-0.055556in}{0.000000in}}{\pgfqpoint{0.000000in}{0.000000in}}{%
\pgfpathmoveto{\pgfqpoint{0.000000in}{0.000000in}}%
\pgfpathlineto{\pgfqpoint{-0.055556in}{0.000000in}}%
\pgfusepath{stroke,fill}%
}%
\begin{pgfscope}%
\pgfsys@transformshift{4.275000in}{1.447674in}%
\pgfsys@useobject{currentmarker}{}%
\end{pgfscope}%
\end{pgfscope}%
\begin{pgfscope}%
\pgftext[x=0.619444in,y=1.447674in,right,]{\rmfamily\fontsize{9.000000}{10.800000}\selectfont \(\displaystyle 950\)}%
\end{pgfscope}%
\begin{pgfscope}%
\pgfsetbuttcap%
\pgfsetroundjoin%
\definecolor{currentfill}{rgb}{0.000000,0.000000,0.000000}%
\pgfsetfillcolor{currentfill}%
\pgfsetlinewidth{0.501875pt}%
\definecolor{currentstroke}{rgb}{0.000000,0.000000,0.000000}%
\pgfsetstrokecolor{currentstroke}%
\pgfsetdash{}{0pt}%
\pgfsys@defobject{currentmarker}{\pgfqpoint{0.000000in}{0.000000in}}{\pgfqpoint{0.055556in}{0.000000in}}{%
\pgfpathmoveto{\pgfqpoint{0.000000in}{0.000000in}}%
\pgfpathlineto{\pgfqpoint{0.055556in}{0.000000in}}%
\pgfusepath{stroke,fill}%
}%
\begin{pgfscope}%
\pgfsys@transformshift{0.675000in}{1.563953in}%
\pgfsys@useobject{currentmarker}{}%
\end{pgfscope}%
\end{pgfscope}%
\begin{pgfscope}%
\pgfsetbuttcap%
\pgfsetroundjoin%
\definecolor{currentfill}{rgb}{0.000000,0.000000,0.000000}%
\pgfsetfillcolor{currentfill}%
\pgfsetlinewidth{0.501875pt}%
\definecolor{currentstroke}{rgb}{0.000000,0.000000,0.000000}%
\pgfsetstrokecolor{currentstroke}%
\pgfsetdash{}{0pt}%
\pgfsys@defobject{currentmarker}{\pgfqpoint{-0.055556in}{0.000000in}}{\pgfqpoint{0.000000in}{0.000000in}}{%
\pgfpathmoveto{\pgfqpoint{0.000000in}{0.000000in}}%
\pgfpathlineto{\pgfqpoint{-0.055556in}{0.000000in}}%
\pgfusepath{stroke,fill}%
}%
\begin{pgfscope}%
\pgfsys@transformshift{4.275000in}{1.563953in}%
\pgfsys@useobject{currentmarker}{}%
\end{pgfscope}%
\end{pgfscope}%
\begin{pgfscope}%
\pgftext[x=0.619444in,y=1.563953in,right,]{\rmfamily\fontsize{9.000000}{10.800000}\selectfont \(\displaystyle 960\)}%
\end{pgfscope}%
\begin{pgfscope}%
\pgfsetbuttcap%
\pgfsetroundjoin%
\definecolor{currentfill}{rgb}{0.000000,0.000000,0.000000}%
\pgfsetfillcolor{currentfill}%
\pgfsetlinewidth{0.501875pt}%
\definecolor{currentstroke}{rgb}{0.000000,0.000000,0.000000}%
\pgfsetstrokecolor{currentstroke}%
\pgfsetdash{}{0pt}%
\pgfsys@defobject{currentmarker}{\pgfqpoint{0.000000in}{0.000000in}}{\pgfqpoint{0.055556in}{0.000000in}}{%
\pgfpathmoveto{\pgfqpoint{0.000000in}{0.000000in}}%
\pgfpathlineto{\pgfqpoint{0.055556in}{0.000000in}}%
\pgfusepath{stroke,fill}%
}%
\begin{pgfscope}%
\pgfsys@transformshift{0.675000in}{1.680233in}%
\pgfsys@useobject{currentmarker}{}%
\end{pgfscope}%
\end{pgfscope}%
\begin{pgfscope}%
\pgfsetbuttcap%
\pgfsetroundjoin%
\definecolor{currentfill}{rgb}{0.000000,0.000000,0.000000}%
\pgfsetfillcolor{currentfill}%
\pgfsetlinewidth{0.501875pt}%
\definecolor{currentstroke}{rgb}{0.000000,0.000000,0.000000}%
\pgfsetstrokecolor{currentstroke}%
\pgfsetdash{}{0pt}%
\pgfsys@defobject{currentmarker}{\pgfqpoint{-0.055556in}{0.000000in}}{\pgfqpoint{0.000000in}{0.000000in}}{%
\pgfpathmoveto{\pgfqpoint{0.000000in}{0.000000in}}%
\pgfpathlineto{\pgfqpoint{-0.055556in}{0.000000in}}%
\pgfusepath{stroke,fill}%
}%
\begin{pgfscope}%
\pgfsys@transformshift{4.275000in}{1.680233in}%
\pgfsys@useobject{currentmarker}{}%
\end{pgfscope}%
\end{pgfscope}%
\begin{pgfscope}%
\pgftext[x=0.619444in,y=1.680233in,right,]{\rmfamily\fontsize{9.000000}{10.800000}\selectfont \(\displaystyle 970\)}%
\end{pgfscope}%
\begin{pgfscope}%
\pgftext[x=0.357293in,y=1.215116in,,bottom,rotate=90.000000]{\rmfamily\fontsize{9.000000}{10.800000}\selectfont Spannung (V)}%
\end{pgfscope}%
\begin{pgfscope}%
\pgftext[x=2.475000in,y=1.749677in,,base]{\rmfamily\fontsize{11.000000}{13.200000}\selectfont Moduliertes Signal, Kurzschluss \"uber Modul}%
\end{pgfscope}%
\end{pgfpicture}%
\makeatother%
\endgroup%

    \caption{%
        \emph{Amplitude-shift  keying}: Oben   sind  die   zu  \"ubertragenden
        digitalen  Daten als  \code{1} und  \code{0} abgebildet. Die  mittlere
        Abbildung stellt eine typische Umsetzung des Konzepts mit harmonischer
        Tr\"agerschwingung und zwei verschiedenen Amplituden dar.\protect\\
        Das  unterste   Signal  ist  eine  vereinfachte   Darstellung  unserer
        Variante.  Die  Gleichspannung an  der Leitung  ist etwas  weniger als
        \SI{1}{\kilo\volt}; beim Kurzschluss eines Moduls erfolgt eine Abfolge
        von Spannungseinbr\"uchen auf der Leitung, welche die Daten kodiert.%
    }
    \label{fig:ask:concept}
\end{figure}

