% **************************************************************************** %
\chapter{Software}
\label{chap:software}
% **************************************************************************** %

Die  Software   unseres  Systems   gliedert  sich   analog  zur   Hardware  in
zwei  prim\"are  Teile: Die   Firmware  des  Sensors  und   die  Software  des
Masters.

In  Abschnitt  \ref{sec:fw:dataFlow} wird  im  Folgenden  kurz dargelegt,  wie
diese beiden  Teilsysteme verkn\"upft sind. Anschliessend  erkl\"art Abschnitt
\ref{sec:firmware:sensor}  die Funktionsweise  der  Firmware des  Sensors. Die
Software des \Raspi~ ist  abschliessend in Abschnitt \ref{sec:software:master}
dokumentiert.


{\begin{a3pages}
\setlength{\parindentbak}{\parindent}
    \noindent\adjustbox{valign=t}{\begin{minipage}{115mm}
    % ---------------------------------------------------------------------------- %
    \section{Datenfluss}
    \label{sec:fw:dataFlow}
    % ---------------------------------------------------------------------------- %
    Abbildung  \ref{fig:dataFlow}   zeigt  den   Ablauf  der   Spannungs-  und
    Strommessung von den Messsonden bis zur Speicherung der Messergebnisse und
    der zugeh\"origen Metadaten in der Datenbank.

    \setlength{\parindent}{\parindentbak}  %   restore  paragraph  indentation
    Die   Spannung  wird   in  jedem   PV-Modul  von   einem  Sensor   mittels
    eines    analog/digital-Konverters   gemessen. Anschliessend    wird   ein
    laufender  Durchschnittswert  (siehe  Abschnitt  \ref{subs:Sensor},  Seite
    \pageref{subs:Sensor})  zusammen mit  der Serienummer  des Microchips  und
    einer  Checksumme in  ein  Datenpaket gepackt  und  \"uber die  DC-Leitung
    verschickt.

    Die   ON-OFF    Keying-Schaltung   benutzt   als   Referenz    den   Clock
    eines     separaten,     spannungsgesteuerten    Oszillators     (englisch
    \emph{voltage-controlled oscillator}, VCO).

    Im  \Master~  wird das  Datenpaket  entpackt  und auf  seine  Integrit\"at
    gepr\"uft.   Passt die  Pr\"ufsumme nicht  zu  den Daten,  wird das  Paket
    verworfen. Sind  die Daten  intakt  (bzw. wird  keine Diskrepanz  zwischen
    Pr\"ufsumme und  Daten festgestellt),  werden Spannung und  Serienummer im
    zugeh\"origen Table der Datenbank gespeichert. Zur chronologischen Ordnung
    der Daten wird  noch ein Timestamp abgelegt. Damit man  weiss, aus welchem
    Strang das Paket gekommen ist, wird  die Strang-Nummer noch in den Eintrag
    eingef\"ugt.

    Die  Messung der  Strang-Str\"ome  erfolgt direkt  vom  \Master~ aus;  die
    zugeh\"origen Messwerte werden in einem separaten Table abgelegt.

    Zur  Verwaltung  der  installierten  Sensoren  und  PV-Module  wird  f\"ur
    jeden  Strang  ein  Table  gef\"uhrt,  in  dem  die  Serienummern  der  in
    diesem Strang  installierten Sensoren (bzw. deren  Microchips) gespeichert
    sind. Die  Eintr\"age  in  diesen  Tables erfolgen  automatisch,  wenn  in
    einem ankommenden  Datenpaket eine bisher noch  nicht bekannte Serienummer
    detektiert wird.

    \vspace*{40mm}

    \hfill\adjustbox{valign=t}{\begin{minipage}{50mm}
        \figcaption
        [Datenfluss von Messungen bis Speicherung]
        {%
            Datenfluss von den Messungen der Modulspannungen und Strang-Str\"ome bis
            zur Speicherung in der Datenbank%
        }
        \label{fig:dataFlow}
    \end{minipage}}
\end{minipage}}
\hspace*{15mm}
\adjustbox{valign=t}{\begin{minipage}{215mm}
    \input{images/software/dataFlow.tex}
\end{minipage}}
\end{a3pages}}
