{\begin{a3pages}
% ---------------------------------------------------------------------------- %
\section{Firmware \Sensor}
\label{sec:firmware:sensor}
% ---------------------------------------------------------------------------- %

\setlength{\parindentbak}{\parindent}
    \noindent\adjustbox{valign=t}{\begin{minipage}{135mm}
Als  CPU auf  dem  Sensor agiert  wie im  Abschnitt~\ref{sec:hw:sensorplatine}
erw\"ahnt   ein  Atmel   Smart  ARM   Cortex  M0+.    Die  Firmware   ist  aus
Kompatibilit\"atsgr\"unden zur CPU in C geschrieben.

% ---------------------------------------------------------------------------- %
\subsection{Benutzte Bibliotheken}
% ---------------------------------------------------------------------------- %

Es wird komplett freie Software verwendet, weswegen zum Bauen der Binaries GNU
Makefiles benutzt werden.

    \setlength{\parindent}{\parindentbak} % restore paragraph indentation
Die  Firmware  ist  auf  dem Atmel  Software  Framework  (ASF)  aufgebaut. Das
ASF  f\"uhrt  einen   Hardware  Abstraction  Layer  (HAL)   ein,  welcher  die
Hardware-Bl\"ocke der einzelnen Atmel CPUs  in einfache Interfaces in Form von
C-Funtionen abstrahiert. Um die g\"angigen  ARM-Schnittstellen zu nutzen, wird
CMSIS im  ASF  verwendet. Das  ASF  darf  f\"ur  Atmel  Chips  ohne  weiteres
verwendet werden, solange die Copyright Bemerkungen nicht entfernt werden.

% ---------------------------------------------------------------------------- %
\subsection{Die Firmware}
% ---------------------------------------------------------------------------- %

Die Firmware besteht im Kern aus einer \code{main()}-Funktion, welche in einer
Endlosschleife   l\"auft. Abbildung~\ref{fig:sensor:firmware:mainloop}  zeigt,
wie  diese  Schleife  aufgebaut  ist.  Zudem  zeigt  sie  den  regelm\"assigen
\code{SysTick}-Interrupt, welcher den ADC ausliest und die LEDs toggelt.

% ---------------------------------------------------------------------------- %
\subsubsection{UART}
\label{subs:UART}
% ---------------------------------------------------------------------------- %

Die Endlosschleife  \"uberpr\"uft zuerst, ob  Daten \"uber die  UART empfangen
wurden, sprich Anweisungen vom Master oder Antworten von anderen Sensoren. Ist
das der Fall, so wird darauf reagiert.  Zuerst wird \"uberpr\"uft ob das Paket
auch f\"ur den Sensor bestimmt ist. Falls das  erste Byte 0 ist, ist das Paket
f\"ur den Master bestimmt. Falls es 1 ist, so ist das Paket f\"ur einen Sensor
bestimmt. Um  zu bestimmen,  ob  das  Paket f\"ur  sich  selbst bestimmt  ist,
pr\"uft  die Firmware,  ob  die  n\"achsten 4  empfangenen  Bytes der  eigenen
ID  entsprechen. Hier  ist wichtig,  dass  alle  Daten Little  Endian  codiert
sind.  Falls das Paket tats\"achlich f\"ur  den Sensor bestimmt ist, so wertet
er  nun den  Befehl  aus  und reagiert  darauf. Im  Normalfall bedeutet  dies,
dass  die  UART  die  gemittelten  Daten  der  letzten  64  Messwerte  an  den
Master  adressiert  und verschickt. Hierzu  wird  ein  Datenpaket, wie  es  in
Abbildung~\ref{fig:sensor:firmware:datenpaket} zu sehen ist, verschickt.

\end{minipage}}
\hspace*{45mm}
\adjustbox{valign=t}{\begin{minipage}{165mm}
    \vspace*{5em}
    \begin{tikzpicture}[%
        align=center,
        text=solarized-base02,
        draw=solarized-base02,
        %remember picture,
        %overlay,
    ]
    \small
    \sffamily
    %\begin{scope}[x={(0mm,0.95\textwidth)},y={(0mm,50mm)}]

    %\draw [help lines] (0,0) grid (10,-15);

    \begin{scope}[
        every node/.style = draw,
        terminal/.append style={
            rounded rectangle,
            fill=solarized-violet,
            text=solarized-base3,
            inner sep=2mm,
        }, % data packages
        sign/.style={
            inner sep=2mm,
            rounded corners=1mm,
            fill=solarized-magenta,
            text=solarized-base3,
        },         % custom signal style
        circ/.style={
            inner sep=2mm,
            rounded corners=1mm,
            double,
            fill=solarized-base02,
            draw=solarized-base02,
            text=solarized-base2,
        }, % circuitry
        proc/.style={
            inner sep=2mm,
            rounded corners=1mm,
            fill=solarized-cyan,
            text=solarized-base3,
        },       % process/activity
        dec/.style={
            inner sep=2mm,
            rounded corners=1mm,
            fill=solarized-base02,
            text=solarized-base3,
        },       % decision/activity
        stor/.style={
            fill=cyan!30
        },         % storage
        dbtable/.style={
            text=solarized-base3,
            draw=solarized-base3,
            rounded corners=1mm,
            inner sep=2mm,
        } % database tables
    ]
        \node (newUART) [
            dec,
            align=center,
            diamond,
            minimum width=33mm,
            minimum height=33mm,
        ] at (0,0) {neuer\\UART\\Request?};
        \node (init) [
            proc,
            above=of newUART,
            align=center,
        ] {Hardware\\initialisieren};
        \node (forMe) [
            dec,
            align=center,
            diamond,
            right=of newUART,
            minimum width=33mm,
            minimum height=33mm,
        ] {F\"ur mich?};
        \node (package) [
            proc,
            right=of forMe,
            align=center,
        ] {Datenpaket mit\\Moving Average\\ und Addresse\\erstellen};
        \node (crc) [
            proc,
            below=of package,
            align=center,
        ] {CRC brechnen\\und zu Daten\\hinzuf\"ugen};
        \node (send) [
            proc,
            left=of crc,
            align=center,
        ] {Datenpaket\\senden};
        \node (getADC) [
            proc,
            below=of send,
            align=center,
        ] {ADC-Messung\\ausf\"uhren};
        \node (movAvg) [
            proc,
            below right=of getADC,
            align=center,
        ] {Moving\\Average\\berechnen};
        \node (LED) [
            proc,
            below left=of movAvg,
            align=center,
        ] {LED\\blinken\\lassen};
        \node (systick) [
            proc,
            above left=of LED,
            align=center,
        ] {auf n\"achsten\\\texttt{systick} warten};
    \end{scope}

    \begin{scope}[
            rounded corners,
            every path/.append style={draw=solarized-base03,},
            >=latex',
    ]
        \draw[-latex] (init) -- (newUART);
        \draw[-latex] (newUART) -- node[midway,anchor=south] {ja} (forMe);
        \draw[-latex] (forMe) -- node[midway,anchor=south] {ja} (package);
        \draw[-latex] (package) -- (crc);
        \draw[-latex] (crc) -- (send);
        \draw[-latex] (forMe) edge[bend left] node[midway,anchor=north] {nein} (newUART);
        \draw[-latex] (send) edge[bend left] (newUART);
        \draw (newUART) edge[loop below]  node {nein} (newUART);

        \draw[-latex] (getADC) edge[bend left] (movAvg);
        \draw[-latex] (movAvg) edge[bend left] (LED);
        \draw[-latex] (LED) edge[bend left] (systick);
        \draw[-latex] (systick) edge[bend left] (getADC);

    \end{scope}

\end{tikzpicture}

    \figcaption{\code{main()}-Loop der Sensorfirmware}
    \label{fig:sensor:firmware:mainloop}
\end{minipage}}

\end{a3pages}}

\begin{figure*}[ht!]
  \centering
  \begin{bytefield}[bitwidth=2em]{8}
    \bitheader{0,7} \\
        \colorbitbox{gray!50}{black}{8}{Indikatorbyte}\\
    \begin{rightwordgroup}{\sffamily Adresse}
        \colorbitbox{solarized-base2}{solarized-base02}{8}{Adressbyte 0} \\
        \colorbitbox{solarized-base2}{solarized-base02}{8}{Adressbyte 1} \\
        \colorbitbox{solarized-base2}{solarized-base02}{8}{Adressbyte 2} \\
        \colorbitbox{solarized-base2}{solarized-base02}{8}{Adressbyte 3}
    \end{rightwordgroup} \\
    \begin{rightwordgroup}{\sffamily Messdaten}
        \colorbitbox{solarized-magenta}{solarized-base3}{8}{Spannungsbyte 0}\\
        \colorbitbox{solarized-magenta}{solarized-base3}{8}{Spannungsbyte 1}
    \end{rightwordgroup} \\
    \begin{rightwordgroup}{\sffamily CRC}
        \colorbitbox{solarized-blue}{solarized-base3}{8}{CRCbyte 0}\\
        \colorbitbox{solarized-blue}{solarized-base3}{8}{CRCbyte 1}\\
        \colorbitbox{solarized-blue}{solarized-base3}{8}{CRCbyte 2}\\
        \colorbitbox{solarized-blue}{solarized-base3}{8}{CRCbyte 3}
    \end{rightwordgroup} \\
  \end{bytefield}
  \caption{Aufbau eines Datenpakets}
  \label{fig:sensor:firmware:datenpaket}
\end{figure*}

Damit der  Empf\"anger eines  Pakets die  Integrit\"at der  Daten verifizieren
kann,  wird eine  Pr\"ufsumme (CRC)  der Daten  mitgeschickt.  Um  die CRC  zu
berechnen,  wird  zuerst  das Datenpaket  zusammengestellt. Davon  wird  dann
mithilfe des  ASF eine Pr\"ufsumme  erstellt und an das  bestehende Datenpaket
angeh\"angt. Abschliessend  wird  das   Datenpaket  per  UART  verschickt. Der
Vorgang  ist auch  in den  Abbildungen \ref{fig:sensor:firmware:mainloop}  und
\ref{fig:dataFlow} gezeigt.


% ---------------------------------------------------------------------------- %
\subsubsection{Sensor}
\label{subs:Sensor}
% ---------------------------------------------------------------------------- %

Als  Spannungssensor  dient  ein  Analog  Digital  Konverter  (ADC).   Daf\"ur
wird   mithilfe  des ASF   ein  12   Bit  ADC   ausgelesen  und   von  diesen
Werten  ein  laufender  Mittelwert  (\emph{Moving  Average})  erstellt.   Dies
wird   regelm\"assig   innerhalb   des  SysTick-Interrupts   erledigt.    Dann
wird    mithilfe    eines   sogenannten    \emph{Cascaded    Integrator–Comb
Filters}    \cite{ref:wiki:ccicFilter}    der    aktuelle    Moving    Average
berechnet. Formel~\ref{eq:hogenauer} zeigt die zugeh\"orige Methodik.

\begin{equation}\label{eq:hogenauer}
    \begin{split}
        y[n] &= \sum_{k=0}^{RM-1} x[n-k] \\
             &= y[n-1] + x[n] - x[n-RM].
    \end{split}
\end{equation}

\begin{conditions}
    R & Interpolationsrate \\
    M & Anzahl Samples pro Stufe \\
    N & Anzahl Stufen im Filter \\
\end{conditions}

% ---------------------------------------------------------------------------- %
\subsubsection{Statusanzeige}
\label{subs:Statusanzeige}
% ---------------------------------------------------------------------------- %

Damit  man  optisch verifizieren  kann,  ob  der Sensor  korrekt  funtioniert,
wird  im  SysTick-Interrupt  ein  Toggeln  der  LED  ausgef\"uhrt. Blinkt  die
LED  nicht mehr,  kann  man  daraus schliessen,  dass  der  Sensor nicht  mehr
korrekt  funktioniert  (siehe Abschnitt  \ref{subsec:hw:sensor:interface}  auf
Seite \pageref{subsec:hw:sensor:interface}).

% ---------------------------------------------------------------------------- %
\subsection{Open On Chip Debugger}
% ---------------------------------------------------------------------------- %

Zum Programmieren der  CPUs haben wir OpenOCD gew\"ahlt. Auch  hier ist wieder
anzumerken, dass es ein St\"uck freie Software ist. OpenOCD ist extrem leicht
zu erweitern und unterst\"utzt eine grosse Breite an Programmierschnittstellen
wie den STLinkv2 oder den Segger J-Link. Es werden Protokolle wie JTag und SWD
ohne spezielle Konfigurationen unterst\"utzt. Ebenfalls gibt es einen Reichtum
an Chips, die unterst\"utzt sind.

Da der  SAMD09 erst im  ersten Quartal 2016 auf  den Markt gekommen  ist, sind
sehr wenige Projekte  vorhanden, welche ihn bereits benutzt  haben.  Daher hat
OpenOCD  unseren Chip  zu Beginn  nicht unterst\"utzt. Der  bestehende Treiber
f\"ur Atmel-Chips ist deshalb von uns entsprechend erweitert worden.

Um   die   Kosten   tief   zu   halten,   haben   wir   einen   StLinkv2   als
Programmierschnittstelle verwendet. Auch  daf\"ur brauchte es einen  Patch des
Treibers,  da  der  STLinkv2  nur  32  Bit-Schreibbefehle  unterst\"utzt,  der
SAMD09  aber 16  Bit-Schreibbefehle  erwartet. Wir haben  deshalb einen  Patch
geschrieben,  welcher  16  Bit-Schreibbefehle  anstatt  32  Bit-Schreibbefehle
sendet.
